\documentclass[a4paper, 12pt]{book}
  
\usepackage{algorithm}
\usepackage{amsmath}
\usepackage{amssymb}
\usepackage{multicol} 
\usepackage{color, colortbl}
\usepackage{tikz}
\usepackage[bahasa]{babel}

\definecolor{Gray}{gray}{0.9}
\definecolor{LightCyan}{rgb}{0.70,1,1}
\definecolor{OliveGreen}{rgb}{0,0.6,0}

\setlength{\parindent}{0pt}
\newcommand{\norm}[1]{\left\lVert#1\right\rVert}
\newcommand*{\sectionformat}{\centering}

\begin{document}
\title{\textbf{Element - Element dari Geometri Euklid\\
Cuplikan Dalam Indonesia}}
\author{Euclid dari Alexandria}
\maketitle
%\author{Euclid of Alexandria}
%\begin{titlepage}
%\begin{center}
%{\huge \bfseries Element Element Geometri Euklid - Terjemahan Bahasa Indonesia}
%{\huge Element Element Geometri Euklid - Terjemahan Bahasa Indonesia}
%\end{center}
%\end{titlepage}
%comment comment comment
\setcounter{chapter}{1}
\chapter*{\centering \chaptername{ \thechapter}\\
Dasar - Dasar dari Geometri Bidang Datar Yang Melibatkan Garis Lurus}
\section*{\sectionformat }
Untuk menggambarkan sebuah garis lurus terhadap sebuah garis lurus tertentu yang melalui
sebuah titik tertentu.\\
Misalkan $A$
\end{document}

