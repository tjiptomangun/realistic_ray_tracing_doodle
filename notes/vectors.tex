\documentclass[a4paper, 12pt]{article}
  
\usepackage{algorithm}
\usepackage{amsmath}
\usepackage{amssymb}
\usepackage{multicol}
\setlength{\parindent}{0pt}

\title{Notes on Vectors}
%comment comment comment
\begin{document} 
\textbf{Augmented Matrix} is a matrix obtained by appending the columns of 
two given matrices.\\
Given the matrices A and B, where
$$
A =
\begin{bmatrix}
1 && 3 && 2 \\
2 && 0 && 1 \\
5 && 2 && 2
\end{bmatrix}
\quad
, B = 
\begin{bmatrix}
4 \\
3 \\
1
\end{bmatrix}
$$
\\
the augmented matrix $(A|B)$ is written as
$$
(A|B) = 
\begin{bmatrix}
1 & 3 & 2 & 4\\
2 & 0 & 1 & 3\\
5 & 2 & 2 & 1
\end{bmatrix}
$$
\\
Augmented Matrix can be used to represent sytem equations.
For example :
\begin{multicols}{2}
\begin{align*}
x_1 + x_2 + 2x_3 &= 9 \\
2x_1 + 4x_2 - 3x_3 &= 1 \\
3x_1 + 6x_2 - 5x_3 &= 0\\
\end{align*}
\break
\[ \left[ {\begin{array}{cccc}
1 & 1 & 2 & 9 \\
2 & 4 & -3 & 1 \\
3 & 6 & -5 & 0
\end{array}} \right] \]
\end{multicols}

The basic method for solving a linear system is to perform appropiate 
algebraic operations on the system that do not alter the solution set 
and that produce a succession of increasingly simple system, until a 
point is reached where it can be ascertained whether the system is 
consistent, and if so, what its solutions are. Typically, the algebraic
operations are as follows:
\begin{enumerate}
\item Multiply an equation through a a nonzero constant
\item Interchange two equations
\item Add a constant time one equation to another 
\end{enumerate}

Since the rows (horizontal lines) of an augmented matrix correspond
to the equations in the associated system, these three operations 
correspond to the following operation on the rows of the augmented
matrix 
\begin{enumerate}
\item Multiply a row through a a nonzero constant
\item Interchange two rows
\item Add a constant time one row to another 
\end{enumerate}

\begin{multicols}{2}
\begin{align*}
x + y + 2z &= 9 \\
2x + 4y - 3z &= 1 \\
3x + 6y - 5z  &= 0\\
\end{align*}
\break
\[ \left[ {\begin{array}{cccc}
1 & 1 & 2 & 9 \\
2 & 4 & -3 & 1 \\
3 & 6 & -5 & 0
\end{array}} \right] \]
\end{multicols}

\begin{multicols}{2} 
Add -2 times the first equation to all equation
\break
Add -2 times the first row to all row
\end{multicols}

\begin{multicols}{2}
\begin{align*}
-x  -y - 2z &= -9 \\
2y - 7z &= -17 \\
x + 4y - 9z  &= -18\\
\end{align*}
\break
\[ \left[ {\begin{array}{cccc}
-1 & -1 & -2 & -9 \\
0 & 2 & -7 & -17 \\
1 & 4 & -9 & -18 
\end{array}} \right] \]
\end{multicols}





\textbf{Inner Product}
The standard \textit{inner product} (also called \textit{dot product}) of two $n$-vectors is defined as the scalar
\begin{align*}
a^Tb = a_1b_1 + a_2b_2 + ... + a_nb_n
\end{align*}

\textbf{Linear combinations.}
if $a_1....a_m$ are n-vectors, and $\beta_1...\beta_m$ are scalars, the $n-vector$
\begin{align*}
\beta_1a_1+...+\beta_ma_m
\end{align*}
is called a \textit{linear combination} of vectors $a_1...a_n$. The scalars $\beta_1...\beta_m$
 are called the \textit{coefficients} of the lienar combination.
\\
\\
\textbf{The inner product function.} Suppose $a$ is an $n-vector$. We can define scalar-valued
function $f$ of $n-vectors$, given by
\begin{equation}
f(x) = a^Tx = a_1x_1 + a_2x_2+...+a_nx_n
\end{equation}
for any $n-vector x$. This function gives the inner product of its $n-vector$ argument $x$ with
some (fixed) $n-vector$.
\\
\\
\textbf{Superposition and linearity}. The inner product function $f$ defined above satisfies
property
\begin{align*}
f(\alpha x +\beta y) &= a^T(\alpha x + \beta y) \\
&= a^T(\alpha x) + a^T(\beta y) \\
&= \alpha(a^Tx) + \beta(a^Ty) \\
&= \alpha f(x) + \beta f(x)
\end{align*}
\\
for all $n-vectors x, y$, and all scalars $\alpha, \beta$. This property is called \textit{superposition}.
 A function that satisfies the superposition property is called \textit{linear}.
\\
\\
\textbf{Superposition equality} is thus
\begin{equation}
f(\alpha x + \beta y) = \alpha f(x) + \beta f(y)
\end{equation}
\\
A function  $ f: \mathbf{R}^n \to \mathbf{R} $ is linear if satisfies
\begin{itemize}
\item \textit{Homogenity.} For any $n$-vector $x$ and any scala $\alpha$, $f(\alpha x) = \alpha f(x)$
\item \textit{Additivity.} For any $n$-vector $x$ and $y$, $f(x + y) = f(x) + f(y)$
\end{itemize}

\textbf{Inner product representation of linear function}
A function is linear if it is defined as inner product of it's argument with some fixed vector.

$f(x) = a^Tx$ for all $x$.
$a^Tx$ is inner product representation of $f$
\\
\\
\textbf{Affine functions}. A linear function plus a constant is called an \textit{affine} function.
A function $f: \mathbf{R} ^n \to \mathbf{R}$ is affine if an only if it can be expressed as $f(x) = a^T x + b$

\end{document}

