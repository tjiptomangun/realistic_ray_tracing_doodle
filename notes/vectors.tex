\documentclass[a4paper, 12pt]{article}
  
\usepackage{algorithm}
\usepackage{amsmath}
\usepackage{amssymb}
\setlength{\parindent}{0pt}

\title{Notes on Vectors}

\begin{document} 
\textbf{Linear combinations.}
if $a1....am$ are n-vectors, and $\beta_1...\beta_m$ are scalars, the $n-vector$
\begin{align*}
\beta_1a_1+...+\beta_ma_m
\end{align*}
is called a \textit{linear combination} of vectors $a_1...a_n$. The scalars $\beta_1...\beta_m$
 are called the \textit{coefficients} of the lienar combination.
\\
\\
\textbf{The inner product function.} Suppose $a$ is an $n-vector$. We can defien scalar-valued
function $f$ of $n-vectors$, given by
\begin{equation}
f(x) = a^Tx = a_1x_1 + a_2x_2+...+a_nx_n
\end{equation}
for any $n-vector x$. This function gives the inner product of its $n-vector$ argument $x$ with
some (fixed) $n-vector$.
\\
\\
\textbf{Superposition and linearity}. The inner product function $f$ defined above satisfies
property
\begin{align*}
f(\alpha x +\beta y) &= a^T(\alpha x + \beta y) \\
&= a^T(\alpha x) + a^T(\beta y) \\
&= \alpha(a^Tx) + \beta(a^Ty) \\
&= \alpha f(x) + \beta f(x)
\end{align*}
\\
for all $n-vectors x, y$, and all scalars $\alpha, \beta$. This property is called \textit{superposition}.
 A function that satisfies the superposition property is called \textit{linear}.
\\
\\
\textbf{Superposition equality} is thus
\begin{equation}
f(\alpha x + \beta y) = \alpha f(x) + \beta f(y)
\end{equation}
\end{document}

