\documentclass[a4paper, 12pt]{article}
  
\usepackage{algorithm}
\usepackage{amsmath}
\usepackage{amssymb}
\usepackage{multicol} 
\usepackage{color, colortbl}

\definecolor{Gray}{gray}{0.9}
\definecolor{LightCyan}{rgb}{0.70,1,1}
\definecolor{OliveGreen}{rgb}{0,0.6,0}

\setlength{\parindent}{0pt}

\title{Notes on Vectors}
%comment comment comment
\begin{document} 
\textbf{Linear Equation}
is equation where $n$ variables $x_1, x_2, ..., x_n$  can be expressed in the form
$a_1x_1 + a_2x_2 + ... + a_nx_n = b $ \\
where $a_1, a_2, ..., a_nx_n$ and $b$ are constants, and $a's$  are not all zero.
\\
\\
\textbf{System of linear equations} or \textbf{linear system} is a finite test of 
linear equations.The variables are called \textbf{unknowns}
\\
\\
\textbf{Solution} of a linear system with $n$ unknonws $x_1, x_2, ... , x_n$ is 
a sequance of $n$ numbers  $s_1, s_2, ... s_n$ for which the substitution
$x_1 = s_1, x_2 = s_2, ..., x_n = s_n$
makes each equation a true statement.
\\
\\
\textit{Every system of linear equations has zero, one or infinitely many solutions.
There are no other possibilities.}
\\
\\
\textbf{Consistent} linear system is the one that has at least one solution.
They are two equations in this system,  one solution of infinitely many.
\\
\\
\textbf{Inconsistent} linear system is the one that has no solutions.
\\
\\ 
\textbf{Augmented Matrix} is a matrix obtained by appending the columns of 
two given matrices.\\
Given the matrices A and B, where
$$
A =
\begin{bmatrix}
1 && 3 && 2 \\
2 && 0 && 1 \\
5 && 2 && 2
\end{bmatrix}
\quad
, B = 
\begin{bmatrix}
4 \\
3 \\
1
\end{bmatrix}
$$
\\
the augmented matrix $(A|B)$ is written as
$$
(A|B) = 
\begin{bmatrix}
1 & 3 & 2 & 4\\
2 & 0 & 1 & 3\\
5 & 2 & 2 & 1
\end{bmatrix}
$$
\\
Augmented Matrix can be used to represent sytem equations.
For example :
\begin{multicols}{2}
\begin{align*}
x_1 + x_2 + 2x_3 &= 9 \\
2x_1 + 4x_2 - 3x_3 &= 1 \\
3x_1 + 6x_2 - 5x_3 &= 0\\
\end{align*}
\break
\[ \left[ {\begin{array}{cccc}
1 & 1 & 2 & 9 \\
2 & 4 & -3 & 1 \\
3 & 6 & -5 & 0
\end{array}} \right] \]
\end{multicols}

\textbf{Elementary Row Operations}
The basic method for solving a linear system is to perform appropiate 
algebraic operations on the system that do not alter the solution set 
and that produce a succession of increasingly simple system, until a 
point is reached where it can be ascertained whether the system is 
consistent, and if so, what its solutions are. Typically, the algebraic
operations are as follows:
\begin{enumerate}
\item Multiply an equation through by a nonzero constant
\item Interchange two equations
\item Add a constant times of one equation to another 
\end{enumerate}

Since the rows (horizontal lines) of an augmented matrix correspond
to the equations in the associated system, these three operations 
correspond to the following operation on the rows of the augmented
matrix 
\begin{enumerate}
\item Multiply a row through by a nonzero constant
\item Interchange two rows
\item Add a constant times of one row to another 
\end{enumerate}

\begin{multicols}{2}
\begin{align*}
x + y + 2z &= 9 \\
2x + 4y - 3z &= 1 \\
3x + 6y - 5z  &= 0\\
\end{align*}
\break
\[ \left[ {\begin{array}{cccc}
1 & 1 & 2 & 9 \\
2 & 4 & -3 & 1 \\
3 & 6 & -5 & 0
\end{array}} \right] \]
\end{multicols}

\begin{multicols}{2} 
Add -2 times the first equation to all equation
\break
Add -2 times the first row to all row
\end{multicols}

\begin{multicols}{2}
\begin{align*}
-x  -y - 2z &= -9 \\
2y - 7z &= -17 \\
x + 4y - 9z  &= -18\\
\end{align*}
\break
\[ \left[ {\begin{array}{cccc}
-1 & -1 & -2 & -9 \\
0 & 2 & -7 & -17 \\
1 & 4 & -9 & -18 
\end{array}} \right] \]
\end{multicols} 

\begin{multicols}{2} 
Add the first equation the third equation
\break
Add the first row to the third row
\end{multicols}

\begin{multicols}{2}
\begin{align*}
-x  -y - 2z &= -9 \\
2y - 7z &= -17 \\
3y - 11z  &= -27\\
\end{align*}
\break
\[ \left[ {\begin{array}{cccc}
-1 & -1 & -2 & -9 \\
0 & 2 & -7 & -17 \\
0 & 3 & -11 & -27 
\end{array}} \right] \]
\end{multicols} 

\begin{multicols}{2} 
Add -1 of the third equation the second equation
\break
Add -1 of the third row to the second row
\end{multicols} 

\begin{multicols}{2}
\begin{align*}
-x  -y - 2z &= -9 \\
-y + 4z &= 10 \\
3y - 11z  &= -27\\
\end{align*}
\break
\[ \left[ {\begin{array}{cccc}
-1 & -1 & -2 & -9 \\
0 & -1 & +4 & 10 \\
0 & 3 & -11 & -27 
\end{array}} \right] \]
\end{multicols} 

\begin{multicols}{2} 
Add -1 of the second equation the first equation
\break
Add -1 of the second row to the first row
\end{multicols} 

\begin{multicols}{2}
\begin{align*}
-x +  0y  - 6z &= -19 \\
-y + 4z &= 10 \\
3y - 11z  &= -27\\
\end{align*}
\break
\[ \left[ {\begin{array}{cccc}
-1 & 0 & -6 & -19 \\
0 & -1 & 4 & 10 \\
0 & 3 & -11 & -27 
\end{array}} \right] \]
\end{multicols} 

\begin{multicols}{2} 
Add 3 times of the second equation the third equation
\break
Add 3 times  of the second row to the third row
\end{multicols} 

\begin{multicols}{2}
\begin{align*}
-x +  0y  - 6z &= -19 \\
-y + 4z &= 10 \\
z  &= 3
\end{align*}
\break
\[ \left[ {\begin{array}{cccc}
-1 & 0 & -6 & -19 \\
0 & -1 & 4 & 10 \\
0 & 0 & 1  & 3 
\end{array}} \right] \]
\end{multicols} 

\begin{multicols}{2} 
Add 6 times of the third equation the first equation
\break
Add 6 times of the third row to the first row
\end{multicols} 

\begin{multicols}{2}
\begin{align*}
-x &= -1 \\
-y + 4z &= 10 \\
z  &= 3\\
\end{align*}
\break
\[ \left[ {\begin{array}{cccc}
-1 & 0 & 0  & -1 \\
0 & -1 & 4 & 10 \\
0 & 0 & 1  & 3 
\end{array}} \right] \]
\end{multicols} 

\begin{multicols}{2} 
Add -4 times of the third equation the second equation
\break
Add -4 times of the third row to the second row
\end{multicols}

\begin{multicols}{2}
\begin{align*}
-x &= -1 \\
-y &= -2 \\
z  &= 3\\
\end{align*}
\break
\[ \left[ {\begin{array}{cccc}
-1 & 0 & 0  & -1 \\
0 & -1 & 0 & -2 \\
0 & 0 & 1  & 3 
\end{array}} \right] \]
\end{multicols} 

\begin{multicols}{2} 
Multiply -1 times for each equation 1 and 2
\break
Multiply -1 times for each row 1 and 2
\end{multicols}


\begin{multicols}{2}
\begin{align*}
x &= 1 \\
y &= 2 \\
z  &= 3\\
\end{align*}
\break
\[ \left[ {\begin{array}{cccc}
1 & 0 & 0  & 1 \\
0 & 1 & 0 & 2 \\
0 & 0 & 1  & 3 
\end{array}} \right] \]
\end{multicols} 

\textbf{Echelon Forms and Reduced Echelon Forms}
\begin{enumerate}
\item If row does not consists of entirely zeroes, then the first nonzero 
      number in the row is a 1. This is called a \textit{leading 1}.
\item It there are any rows that consist entirely of zeroes, the they are 
      grouped together a the bottom of the matrix.
\item In any two succesive rows that do not consist entirely of zeros, the 
      leading 1 in the lower row occurs farther to the right the the 
      leading 1 in the higher row.
\item Each column that contains a leading 1 has zeroes everywhere else in
      the column.
\end{enumerate}
1 to 3 is property of  \textit{row echelon form}, meanwhile 1 to 4 is 
property of \textit{reduced row echelon form}.
\\
\\
Example of Reduced Echelon Form
$$
\begin{bmatrix}
1 && 0 && 0 && 4 \\
0 && 1 && 0 && 7 \\
0 && 0 && 1 && -1 \\
\end{bmatrix}
\quad
\begin{bmatrix}
1 && 0 && 0 \\
0 && 1 && 0 \\
0 && 0 && 1 \\
\end{bmatrix}
\quad
\begin{bmatrix}
0 && 1 && -2 && 0 && 1 \\
0 && 0 && 0 && 1 && 3  \\
0 && 0 && 0 && 0 && 0  \\
0 && 0 && 0 && 0 && 0  \\
\end{bmatrix}
$$

Example of Echelon Form but not Reduced
$$
\begin{bmatrix}
1 && 4 && -3 && 7 \\
0 && 1 && 6 && 2 \\
0 && 0 && 1 && 5
\end{bmatrix}
\quad
\begin{bmatrix}
1 && 1 && 0 \\
0 && 1 && 0 \\
0 && 0 && 0
\end{bmatrix}
\quad
\begin{bmatrix}
0 && 1 && 2 && 6 && 0 \\
0 && 0 && 1 && -1 && 0 \\
0 && 0 && 0 && 0 && 1
\end{bmatrix}
$$
\\
\\
\textbf{General Solutions}. If a linear system has infinitely many solutions,
then a set of parametric equations from which all solutions can be obtained
by assigning numerical values to the parameters is called 
\textbf{general solution} to the system.
\\
\\
\textbf{Elimination Procedure} is steps that can be used to reduce matrix to 
reduced echelon form
\\
\\
\textbf{Gauss-Jordan Elimination}.
\[ \left[ {\begin{array}{cccccc}
0 & 0 & -2  & 0 & 7 & 12 \\
2 & 4 & -10 & 6 & 12 & 28 \\
2 & 4 & -5 & 6 & -5 & -1
\end{array}} \right] \]

\begin{enumerate}
\item Locate the leftmost column that does not consist entirely zeros
\[ \left[ {\begin{array}{cccccc}
0 & 0 & -2  & 0 & 7 & 12 \\
\rowcolor{Gray}
2 & 4 & -10 & 6 & 12 & 28 \\
2 & 4 & -5 & 6 & -5 & -1
\end{array}} \right] \]
\item Interchange the top row with another row, if necessary, to bring 
a nonzeor entry to the top of the column found in 1
\[ \left[ {\begin{array}{cccccc}
\rowcolor{Gray}
2 & 4 & -10 & 6 & 12 & 28 \\
\rowcolor{Gray}
0 & 0 & -2  & 0 & 7 & 12 \\
2 & 4 & -5 & 6 & -5 & -1
\end{array}} \right] \]
\item If the entry that is now at the top of the column found in 1 is a,
multiply the first row by $1/a$ in order to introduce a leading 1.
\[ \left[ {\begin{array}{cccccc}
\rowcolor{Gray}
1 & 2 & -5 & 3 & 6 & 14 \\
0 & 0 & -2  & 0 & 7 & 12 \\
2 & 4 & -5 & 6 & -5 & -1
\end{array}} \right] \]
\item Add a suitable multiples of the top row to the rows below so that
all entries below the leading 1 becomes zeros
\[ \left[ {\begin{array}{cccccc}
1 & 2 & -5 & 3 & 6 & 14 \\
0 & 0 & -2  & 0 & 7 & 12 \\
\rowcolor{Gray}
0 & 0 & 5 & 0 & -17 & -29
\end{array}} \right] \]
\item Now cover the top row in the matrix and begina again with step 1
applied to the submatrix that remains. Continue in this way untuk the
entire matrix in row echelon form
\[ \left[ {\begin{array}{cccccc}
\rowcolor{LightCyan}
1 & 2 & -5 & 3 & 6 & 14 \\
0 & 0 & -2  & 0 & 7 & 12 \\
0 & 0 & 5 & 0 & -17 & -29
\end{array}} \right] \]

\[ \left[ {\begin{array}{cccccc}
\rowcolor{LightCyan}
1 & 2 & -5 & 3 & 6 & 14 \\
\rowcolor{Gray}
0 & 0 & -2  & 0 & 7 & 12 \\
0 & 0 & 5 & 0 & -17 & -29
\end{array}} \right] \]

\[ \left[ {\begin{array}{cccccc}
\rowcolor{LightCyan}
1 & 2 & -5 & 3 & 6 & 14 \\
\rowcolor{Gray}
0 & 0 & 1  & 0 & -7/2 & -6 \\
0 & 0 & 5 & 0 & -17 & -29
\end{array}} \right] \]

\[ \left[ {\begin{array}{cccccc}
\rowcolor{LightCyan}
1 & 2 & -5 & 3 & 6 & 14 \\
\rowcolor{LightCyan}
0 & 0 & 1  & 0 & -7/2 & -6 \\
\rowcolor{Gray}
0 & 0 & 0  & 0 & 1/2  & 1 
\end{array}} \right] \]

\[ \left[ {\begin{array}{cccccc}
\rowcolor{LightCyan}
1 & 2 & -5 & 3 & 6 & 14 \\
\rowcolor{LightCyan}
0 & 0 & 1  & 0 & -7/2 & -6 \\
\rowcolor{LightCyan}
0 & 0 & 0  & 0 & 1  & 2 
\end{array}} \right] \]

\item Beginning with the last nonzero row and working upward, add suitable
multiples of each row to the rows above to introduce zeros above the 
leading 1's

\[ \left[ {\begin{array}{cccccc}
1 & 2 & -5 & 3 & 6 & 14 \\
0 & 0 & 1  & 0 & -7/2 & -6 \\
0 & 0 & 0  & 0 & 1  & 2 
\end{array}} \right] \]



\[ \left[ {\begin{array}{cccccc}
1 & 2 & -5 & 3 & 6 & 14 \\
0 & 0 & 1  & 0 & -7/2 & -6 \\
\rowcolor{Gray}
0 & 0 & 0  & 0 & 1  & 2 
\end{array}} \right] \]



\[ \left[ {\begin{array}{cccccc}
\rowcolor{Gray}
1 & 2 & -5 & 3 & 0 & 2 \\
\rowcolor{Gray}
0 & 0 & 1  & 0 & 0 & 1\\
\rowcolor{LightCyan}
0 & 0 & 0  & 0 & 1  & 2 
\end{array}} \right] \]

\[ \left[ {\begin{array}{cccccc} 
1 & 2 & -5 & 3 & 0 & 2 \\ 
\rowcolor{Gray}
0 & 0 & 1  & 0 & 0 & 1\\
\rowcolor{LightCyan}
0 & 0 & 0  & 0 & 1  & 2 
\end{array}} \right] \]

\[ \left[ {\begin{array}{cccccc} 
\rowcolor{Gray}
1 & 2 & 0 & 3 & 0 & 7 \\ 
\rowcolor{LightCyan}
0 & 0 & 1  & 0 & 0 & 1\\
\rowcolor{LightCyan}
0 & 0 & 0  & 0 & 1  & 2 
\end{array}} \right] \]

\[ \left[ {\begin{array}{cccccc} 
\rowcolor{LightCyan}
1 & 2 & 0 & 3 & 0 & 7 \\ 
\rowcolor{LightCyan}
0 & 0 & 1  & 0 & 0 & 1\\
\rowcolor{LightCyan}
0 & 0 & 0  & 0 & 1  & 2 
\end{array}} \right] \]

\end{enumerate}

\textbf{Homogeneous Linear Systems} \\
A system of linear equations is said to be \textbf{\textit{homogeneous}} if 
the constant terms all zero; that is, the system has the form
\begin{eqnarray*}
a_{11}x_1  &+ a_{12}x_2 &+ ... + a_{1n}x_n = 0\\
a_{21}x_1  &+ a_{22}x_2 &+ ... + a_{2n}x_n = 0\\
...\\
a_{m1}x_1  &+ a_{m2}x_2 &+ ... + a_{mn}x_n = 0\\
\end{eqnarray*}

Every homogeneous system of linear equations is consistent because
all such systems have $x_1 = 0, x_2 = 0, x_n = 0$ as a solution.
This solution is called \textbf{\textit{trivial solution}}; if there
are other solutions, they are called \textbf{\textit{nontrivial solutions}}.

\begin{align*}
x_1 + 3x_2 - 2x_3 + 2x_5 &= 0 \\
2x_1 + 6x_2 - 5x_3 - 2x_4 + 4x_5 - 3x_6 &= 0\\
5x_3 + 10x_4 + 15x_6 &= 0\\
2x_1 + 6x_2 + 8x_4 + 4x_5 + 18x_6 &= 0 \\
\end{align*}

\[ \left[
\begin{array}{ccccccc} 
1 & 3 & -2 &  0 & 2 &  0 & 0\\
2 & 6 & -5 & -2 & 4 & -3 & 0\\
0 & 0 &  5 & 10 & 0 & 15 & 0\\
2 & 6 &  0 &  8 & 4 & 18 & 0\\ 
\end{array} 
\right] \] 


\[ \left[
\begin{array}{ccccccc} 
1  &  3  &  0 &  4  &  2 &  0 & 0\\
0  &  0  &  1 &  2  &  0 &  0 & 0\\
0  &  0  &  0 &  0  &  0 &  1 & 0\\ 
0  &  0  &  0 &   0 &  0 &  0 & 0\\
\end{array} 
\right] \] 

$
x_6 = 0 \\
x_3 = -2x_4 \\
x_4 = t \\
x_3 = -2t \\
x_1 = -3v -4t - 2w \\
x_2 = v
$
\\
\\
Note that the trivial solution results when $v = t = w = 0$.
\\
\\
\textbf{Leading Variables} Are those whose columns in the 
\textit{Reduced Row Echelon Form} contains leading $1$.
\\
\\
\textbf{Free Variables} Are those whose columns in the 
\textit{Reduced Row Echelon Form} do not contain leading $1$.
\\
\\
$x_1, x_3$ and $x_6$ are all \textit{Leading Variables}
\\
$x_2, x_4$ and $x_5$ are all \textit{Free Variables}
\\
\\
\textbf{Free Variable Theorem for Homogeneous System}. If a homogeneous has 
$n$ unknowns, and if the \textit{reduced row echelon form} of its 
\textit{augmented matrix} has r nonzero rows, then the system has $n - r$
free variables.
\\
\\
\textbf{Free Variables , Homogeneous System and Solutions}. A 
\textit{Homogeneous Linear System} with more unknowns that equations
has infinitely many solutions.
\\
\\
\textbf{Inner Product}
The standard \textit{inner product} (also called \textit{dot product}) of 
two $n$-vectors is defined as the scalar
\begin{align*}
a^Tb = a_1b_1 + a_2b_2 + ... + a_nb_n
\end{align*}
\\
\textbf{Linear combinations.}
if $a_1....a_m$ are n-vectors, and $\beta_1...\beta_m$ are scalars, the $n-vector$
\begin{align*}
\beta_1a_1+...+\beta_ma_m
\end{align*}
is called a \textit{linear combination} of vectors $a_1...a_n$. The scalars $\beta_1...\beta_m$
 are called the \textit{coefficients} of the lienar combination.
\\
\\
\textbf{The inner product function.} Suppose $a$ is an $n-vector$. We can define scalar-valued
function $f$ of $n-vectors$, given by
\begin{equation}
f(x) = a^Tx = a_1x_1 + a_2x_2+...+a_nx_n
\end{equation}
for any $n-vector x$. This function gives the inner product of its $n-vector$ argument $x$ with
some (fixed) $n-vector$.
\\
\\
\textbf{Superposition and linearity}. The inner product function $f$ defined above satisfies
property
\begin{align*}
f(\alpha x +\beta y) &= a^T(\alpha x + \beta y) \\
&= a^T(\alpha x) + a^T(\beta y) \\
&= \alpha(a^Tx) + \beta(a^Ty) \\
&= \alpha f(x) + \beta f(x)
\end{align*}
\\
for all $n-vectors x, y$, and all scalars $\alpha, \beta$. This property is called \textit{superposition}.
 A function that satisfies the superposition property is called \textit{linear}.
\\
\\
\textbf{Superposition equality} is thus
\begin{equation}
f(\alpha x + \beta y) = \alpha f(x) + \beta f(y)
\end{equation}
\\
A function  $ f: \mathbf{R}^n \to \mathbf{R} $ is linear if satisfies
\begin{itemize}
\item \textit{Homogenity.} For any $n$-vector $x$ and any scala $\alpha$, $f(\alpha x) = \alpha f(x)$
\item \textit{Additivity.} For any $n$-vector $x$ and $y$, $f(x + y) = f(x) + f(y)$
\end{itemize}

\textbf{Inner product representation of linear function}
A function is linear if it is defined as inner product of it's argument with some fixed vector.

$f(x) = a^Tx$ for all $x$.
$a^Tx$ is inner product representation of $f$
\\
\\
\textbf{Affine functions}. A linear function plus a constant is called an \textit{affine} function.
A function $f: \mathbf{R} ^n \to \mathbf{R}$ is affine if an only if it can be expressed as $f(x) = a^T x + b$

\end{document}
