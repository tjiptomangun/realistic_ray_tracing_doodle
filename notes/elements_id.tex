\documentclass[a4paper]{book}
  
\usepackage{algorithm}
\usepackage{amsmath}
\usepackage{amssymb}
\usepackage{multicol} 
\usepackage{color, colortbl}
\usepackage{tikz}
\usepackage[bahasa]{babel}
\usetikzlibrary{calc}

\hbadness=999999
\vbadness=999999

\definecolor{Gray}{gray}{0.9}
\definecolor{LightCyan}{rgb}{0.70,1,1}
\definecolor{OliveGreen}{rgb}{0,0.6,0}
\definecolor{ashgrey}{rgb}{0.7, 0.75, 0.71}
\definecolor{aliceblue}{rgb}{0.94, 0.97, 1.0} 
\usepackage{hyperref}
\hypersetup{
%    colorlinks=true, %set true if you want colored links
    linktoc=all,     %set to all if you want both sections and subsections linked
%    linkcolor=blue,  %choose some color if you want links to stand out
}

\makeatletter

%%\pgfmathdeclarefunction{newasin}{1}{%
%%    \begingroup%
%%        \expandafter\pgfmath@x#1pt\relax%
%%        \pgfmath@xa\pgfmath@x%
%%        \ifdim\pgfmath@x<0pt\relax%
%%            \pgfmath@x-\pgfmath@x%
%%        \fi%
%%        \pgfmath@x1000\pgfmath@x%
%%        \afterassignment\pgfmath@gobbletilpgfmath@%
%%        \expandafter\c@pgfmath@counta\the\pgfmath@x\relax\pgfmath@%
%%        \pgfmath@x-\csname pgfmath@acos@\the\c@pgfmath@counta\endcsname pt\relax%
%%        \advance\pgfmath@x90pt\relax%
%%        \ifdim\pgfmath@xa<0pt\relax%
%%            \pgfmath@x-\pgfmath@x%
%%        \fi%
%%        \pgfmath@returnone\pgfmath@x%
%%    \endgroup%
%%}
%%
%%\pgfmathdeclarefunction{newacos}{1}{%
%%    \begingroup%
%%        \expandafter\pgfmath@x#1pt\relax%
%%        \pgfmath@xa\pgfmath@x%
%%        \ifdim\pgfmath@x<0pt\relax%
%%            \pgfmath@x-\pgfmath@x%
%%        \fi%
%%        \pgfmath@x1000\pgfmath@x%
%%        \afterassignment\pgfmath@gobbletilpgfmath@%
%%        \expandafter\c@pgfmath@counta\the\pgfmath@x\relax\pgfmath@%
%%        \expandafter\pgfmath@x\csname pgfmath@acos@\the\c@pgfmath@counta\endcsname pt\relax%
%%        \ifdim\pgfmath@xa<0pt\relax%
%%            \pgfmath@x-\pgfmath@x%
%%            \advance\pgfmath@x by180pt\relax%
%%        \fi%
%%        \pgfmath@returnone\pgfmath@x%
%%    \endgroup%
%%}
%%
\newcommand{\norm}[1]{\left\lVert#1\right\rVert}

\begin{document}
\title{\textbf{Element - Element dari Geometri Euklid 
(Cuplikan Dalam Indonesia)}}
\author{Euclid dari Alexandria}
\maketitle
\tableofcontents
\pagenumbering{arabic}
\pagestyle{empty}
\setcounter{chapter}{1}
\chapter*{\centering \chaptername{ \thechapter} 
\addcontentsline{toc}{chapter}{Buku \thechapter}
Dasar - Dasar dari Geometri Bidang Datar Yang Melibatkan Garis Lurus}
%\chapter{Dasar - Dasar dari Geometri Bidang Datar Yang Melibatkan Garis Lurus}


\begin{center}
\addcontentsline{toc}{section}{Definisi - definisi}
\textbf{Definisi - definisi}
\end{center} 
\begin{enumerate}
\item Sebuah titik adalah yang darinya tidak ada bagian lagi.
\item Dan sebuah garis adalah panjang yang tidak memiliki lebar.
\item Dan ujung - ujung dari sebuah garis adalah titik - titik.
\item Sebuah garis lurus adalah garis yang terletak dengan rata dengan titik - titik padanya.
\item Dan sebuah bidang adalah yang memiliki panjang dan lebar saja.
\item Dan ujung - ujung sebuah bidang adalah garis - garis.
\item Dan bidang datar adalah bidang yang terletak dengan rata dengan garis - garis lurus padanya.
\item Dan sudut bidang datar adalah kemiringan garis - garis tersebut 
satu dengan lain, ketika dua garis dalam sebuah bidang datar bertemu satu dengan
yang lain, dan tidak terletak pada garis lurus.
\item Dan ketika garis - garis yang mengapit sudut tersebut adalah garis lurus maka sudut
tersebut disebut sebagai sudut \textit{rektilinier}.
\item Dan ketika sebuah garis lurus berdiri diatas garis lurus lain dan 
membentuk sudut - sudut yang bersebelahan yang sama besar satu dengan yang 
lain, masing - masing sudut yang sama besar tersebut adalah sudut siku - siku,
dan garis lurus yang pertama dikatakan tegak lurus terhadap garis lurus kedua
di mana dia berdiri.
\item Sebuah sudut tumpul adalah sudut yang lebih besar dari sudut siku - siku.
\item Sebuah sudut lancip adalah sudut yang lebih kecil dari sudut siku - siku.
\item Sebuah batas adalah ujung - ujung dari sesuatu.
\item Sebuah bangun adalah yang berada didalam batas atau batas - batas.
\item Sebuah lingkaran adalah bentuk bidang datar yang dikandung dalam sebuah 
sebuah garis [yang disebut keliling], (sedemikian rupa) sehingga semua 
garis - garis lurus menyebar kearah keliling dari satu titik di dalam bidang
tersebut sehingga garis - garis lurus tersebut sama panjang.
\item Dan titik tersebut dinamakan pusat lingkaran.
\item Dan diameter dari lingkaran adalah garis lurus yang melalui 
pusat lingkaran tadi dan berhenti di masing - masing arah pada
keliling lingkaran. Dan garis lurus seperti itu juga membagi lingkaran 
menjadi dua bagian sama besar.
\item Dan setengah lingkaran adalah bidang yang dikandung dari diameter dan 
keliling yang dipotong diameter. Dan pusat setengah lingkaran sama dengan 
pusat dari lingkaran.
\item Bangun - bangun rektilinier adalah bangun - bangun yang berada di dalam
garis - garis lurus: bangun segitiga adalah bangun yang berada di dalam tiga buah
garis lurus, bangun segiempat adalah bangun yang berada di dalam empat garis lurus dan 
segibanyak adalah bidang yang berada di dalam lebih dari empat garis lurus.
\item Dan pada bangun segitiga: sebuah segitiga sama sisi adalah yang memiliki tiga
sisi sama besar, segitiga sama kaki adalah yang memiliki dua sisi sama besar, dan 
segitiga adalah yang tidak memiliki sisi yang sama besar.
\item Dan selanjutnya pada bangun segitiga: segitiga siku - siku adalah yang memiliki
satu sudut siku - siku, segitiga tumpul adalah yang memiliki satu sudut tumpulm dan 
segitiga lancip adalah yang memiliki tiga sudut lancip.
\item Dan pada bangun segiempat: Bujur sangkar adalah yang segiempat yang
sudut - sudut nya siku - siku dan sisi - sisinya sama panjang, persegipanjang
adalah yang sudut - sudut nya siku dan tetapi tidak semua sisinya sama panjang, 
belah ketupat adalah yang memiliki sisi - sisi sama panjang tetapi
sudut - sudutnya tidak siku - siku, jajaran genjang adalah yang sudut - sudut 
dan sisi - sisi yang berhadapan sama besar tetapi tidak siku - siku dan tidak 
semua sisi sama panjang, dan persegiempat lainnya yang tidak dalam kategori tersebut
disebut dengan trapezium.
\item Garis - garis sejajar adalah garis - garis lurus yang jika berada dalam 
bidang yang sama dan dipanjangkan hingga tak terhingga ke masing - masing arah
tidak bertemu satu dengan yang lain.  
\end{enumerate}

\begin{center}
\addcontentsline{toc}{section}{Postulat - postulat}
\textbf{Postulat - postulat}
\end{center}
\begin{enumerate}
\item Anggap sudah didalilkan melalui dua titik dapat dibuat garis lurus.
\item Dan memanjangkan sebuah ruas garis dengan panjang tertentu tanpa terputus.
\item Dan dengan pusat dan jari -jari dapat dibuat sebuah lingkaran.
\item Dan bahwa semua sudut siku - siku sama besar.
\item Dan jika suatu garis lurus memotong dua buah garis lurus yang lain dan 
membentuk sudut - sudut interior pada sisi yang sama yang jumlah besar 
sudutnya lebih kecil dari dua  sudut siku - siku, maka kedua garis lurus yang 
dipotong tadi, jika dipanjangkan sampai tak terhingga, akan bertemu di sisi 
tersebut (sisi yang jumlah sudut interior pada sisi yang sama yang sama dengan 
dua sudut siku), dan tidak akan bertemu di sisi satunya.
\end{enumerate}

\begin{center}
\addcontentsline{toc}{section}{Gagasan - gagasan Umum}
\textbf{Gagasan - gagasan Umum}
\end{center}
\begin{enumerate}
\item Hal - hal yang nilai nya sama dengan satu hal lain di luar keduanya adalah sama satu dengan yang lain. 
\item Dan jika hal - hal yang sama masing - masing ditambahkan ke hal yang lain yang sama, maka hasil nya sama satu dengan yang lain. 
\item Dan jika hal - hal yang sama masing - masing dikurangi dengan hal yang lain yang sama, maka hasil nya sama satu dengan yang lain. 
\item Dan hal - hal yang berhimpit satu dengan yang lain adalah sama.
\item Dan keseluruhan lebih besar dari bagian nya.
\end{enumerate}

\stepcounter{section}
\section*{\centering Proposisi \thesection}
\addcontentsline{toc}{section}{Proposisi \thesection}
Untuk menggambarkan segitiga sama sisi pada sebuah ruas garis.
\begin{center}
\begin{tikzpicture}
\draw (-1.5, 0) circle (3);
\draw (1.5, 0) circle (3);
\draw (-4.5, 0) node [anchor=east]{D} (-1.5, 0) node [anchor=east]{A} (-1.5, 0) 
-- (1.5, 0) node[anchor=west]{B} (4.5, 0) node[anchor=west]{E} 
(0, 3.0)node[anchor=north]{C};
\draw (-1.5, 0) -- ++(60:3) -- ++(-60:3);
\end{tikzpicture}
\end{center}
Misalkan $AB$ adalah ruas garis.

Dibutuhkan untuk menggambarkan segitiga sama sisi pada ruas garis $AB$.

Misalkan $BCD$ adalah lingkaran dengan pusat $A$ dan jari jari $AB$ sudah 
digambarkan [Post. 3], dan lagi misalkan lingkaran $ACE$ dengan pusat $B$ 
dan jari- jari $BA$ sudah digambarkan [Post. 3]. Dan misalkan garis lurus 
$CA$ dan $CB$ dipertemukan dari titik $C$, dimana lingkaran - lingkaran 
tersebut saling memotong, masing - masing ke titik $A$ dan $B$ [Post. 1].  


Dan karena $A$ adalah pusat dari lingkaran $CDB$, maka $AC$ sama dengan $AB$
[Def. 1.15]. Dan karena titik $B$ adalah pusat dari lingkaran $CAE$, maka
$BC$ sama dengan $BA$ [Def. 1.15]. Tetapi $CA$ juga sudah dibuktikan sama 
dengan $AB$. Dengan demikian $CA$ dan $CB$ adalah sama dengan $AB$. Tetapi
hal - hal yang nilainya sama dengan satu hal lain di luar keduanya adalah sama
satu dengan yang lain [G.U. 1]. Dengan demikian, $CA$ juga sama dengan $CB$.  
Dengan demikian ketika garis lurus $CA$, $AB$ dan $BC$ sama satu dengan 
yang lain.

Dengan demikian, segitiga $ABC$ adalah sama sisi, dan dibangun dari
satu ruas gari $AB$. Yang adalah hal yang kita butuhkan.  

\stepcounter{section}
\section*{\centering Proposisi \thesection} 
\addcontentsline{toc}{section}{Proposisi \thesection}

Untuk menempatkan sebuah ruas garis yang sama dengan ruas garis tertentu
pada satu titik tertentu (sebagai salah satu ujungnya).

\begin{tikzpicture}[scale=.6, every node/.style={scale=.6}]
\draw [-] (0, 0) -- (2., 3) node [anchor=south west]{$C$};
\draw [-] (-2, -.0) -- (0, 0);
\draw (0, 0) node [anchor=west] {$B$}-- (120:2) node [anchor = south] {$D$}--+ (240:2) node [anchor=east] {$A$};
\draw (0, 0) -- (120:2) --+ (240:9)node [anchor = south east] {$E$};
\draw (0, 0) -- (-60:7) node [anchor = south west]{$F$};
\draw (0, 0) circle (3.606);
%\draw (-1, 1.73205) -- (240:8);
\draw (-1, 1.73205) circle (5.606) (-7, 0) node [anchor = west] {$K$};
\draw (-3.8, -3.8) node {$L$};
\draw (2.3, -3.4) node {$G$};
\draw (0.0, 4) node {$H$}; 
\end{tikzpicture} 

Misalkan $A$ adalah suatu titik tertentu, dan $BC$ adalah suatu ruas garis
tertentu. Dibutuhkan untuk menempatkan garis lurus di $A$ yang sama panjang
dengan garis lurus $BC$.

Sebab, misalkan ruas garis $AB$ sudah terhubung dari titik $A$ ke titik $B$
[Post. 1], dan misalkan segitiga sama sisi $DAB$ sudah digambarkan 
padanya [Prop 1.1]. Dan misalkan garis lurus $AE$ dan $BF$ sudah 
ditarik lurus dengan masing masing $DA$ dan $DB$ [Post. 2]. Dan
misalkan lingkaran $CGH$ dengan pusat $B$ dan jari - jari $BC$ sudah 
digambarkan [Post. 3], dan lagi misalkan lingkaran $GKL$ dengan pusat $D$
dan jari - jari $DG$ sudah digambarkan [Post. 3].  

Dengan demikian, karena titik $B$ adalah pusat dari lingkaran $CGH$, $BC$ sama 
dengan $BG$ [Def. 1.15].  Dan lagi karena titik $D$ adalah pusat dari 
lingkaran $GKL$, $DL$ sama dengan $DG$ [Def. 1.15]. Dan di dalam ini, $DA$
sama dengan $DB$. Dengan demikian sisa dari $AL$ sama dengan sisa $BG$
[G.U. 3]. Tetapi $BC$ juga sudah dibuktikan sama dengan $BG$. Tetapi hal - hal
yang nilainya sama dengan satu hal lain di luar keduanya adalah sama satu 
dengan yang lain [G.U. 1]. Dengan demikian AL juga sama dengan BC.

Dengan demikian, ruas garis AL sama dengan ruas garis tertentu
$BC$, yang terletak pada garis tertentu $A$. Hal yang kita butuhkan.  

Ilustrasi lain
\begin{center}
\begin{tikzpicture}
\draw [-, ashgrey] (0, 0.0) node [anchor = north]{$A$} -- (2, 0.0) node [anchor = west] {$B$};
\draw [-, dashed, ashgrey] (0, 0) -- (120:3) --++ (240:3)--++(-0:3); 
\draw [-] (-1.5, 2.6) node [anchor = south] {$C$}-- (.5, 2.6);
\draw [-, dashed, ashgrey] (-1.5, 2.6) circle(2);
\draw [-, dashed, ashgrey] (-3.0, 0.0) circle(5);
\draw [-, dashed, ashgrey] (-3.0, 0.0) --++(0:6);
\draw [-, dashed, ashgrey] (-3.0, 0.0) --++(60:6);
\end{tikzpicture} 
\end{center}

\stepcounter{section}
\section*{\centering Proposisi \thesection} 
\addcontentsline{toc}{section}{Proposisi \thesection}

Pada dua buah ruas garis yang tidak sama panjang, cara untuk
memotong dari ruas garis yang lebih besar ruas garis yang sama panjang dengan
ruas garis yang lebih pendek.

\begin{center}
\begin{tikzpicture}[scale=.9, every node/.style={scale=.9}]
\draw [-] (0, 0.0) node [anchor = north]{$A$} -- (3, 0.0) node [anchor = west] {$B$};
\draw [-, dashed, ashgrey] (0, 0)--(135 : 2.82842712474)--(195:2.82842712474)--(0, 0);
\draw [-, dashed, ashgrey] (0, 0)--(-45 : 5);
\draw [-] (-2, 2) -- (.5, 2);
\draw (-0.5, 2.9) node {$C$};
\draw [-, dashed, ashgrey] (-2, 2) circle(2.5);
\draw [-, dashed, ashgrey] (-2, 2) --++(75:2.5);
\draw [-, dashed, ashgrey] (0, 0)--(135 : 2.82842712474)--(195:2.82842712474)--(15: 5);
\draw [-, dashed, ashgrey] (-2.73205080756, -0.732050807567) circle(5.32842712474);
\draw [-] (0, 0.0) node [anchor = north]{$A$} -- (15:2.5) node [anchor = west] {$D$};
\draw [-] (0, 0.0) circle (2.5);
\draw (2.5, 0) node [anchor = north west] {$E$};
\draw (0, -2.5) node [anchor = north] {$F$};
\end{tikzpicture}
\end{center}

Misalkan $AB$ dan $C$ adalah dua buah garis tak sama panjang yang dimaksud,
dimana $AB$ adalah garis yang lebih panjang. Jadi dibutuhkan untuk memotong
dari ruas garis $AB$ sebuah ruas garis yang sama panjang dengan $C$.

Misalkan ruas garis $AD$, sama panjang dengan ruas garis $C$, sudah ditempatkan
pada titik $A$ sebagai salah satu ujungnya [Prop. 1.2]. Dan misalkan lingkaran
$DEF$ sudah digambarkan dengan titik $A$ dan jari - jari $AD$ [Post. 3].

Dan karena titik $A$ adalah pusat dari lingkaran $DEF$, $AE$ dengan demikian
sama dengan $AD$ [Def.1.14]. Tetapi $C$ juga sama panjang dengan $AD$. Dengan 
demikian $AE$ dan $C$ adalah sama dengan $AD$. Sehingga $AE$ juga sama dengan
$C$ [G.U.1].

Dengan demikian , untuk dua buah ruas garis yang tidak sama panjang $AB$ 
dan $C$, ruas garis $AE$ yang sama panjang dengan $C$ telah dipotong dari 
$AB$. Hal yang mana yang perlu kita lakukan.

\stepcounter{section}
\section*{\centering Proposisi \thesection} 
\addcontentsline{toc}{section}{Proposisi \thesection}

Jika pada dua buah segitiga, yang pertama  memiliki dua buah sisi yang sama 
panjang dengan dua buah sisi segitiga yang satunya, dan pada keduanya sudut
yang diapit oleh sisi - sisi yang sama panjang tadi adalah sama besar, 
maka alas (sisi ketiga) dari kedua segitiga tersebut adalah sama besar, dan kedua 
segitiga tersebut akan sama, dan sudut - sudut selebihnya yang menghadap dua 
sisi - sisi yang sama panjang yang pertama tadi adalah sama besar pada 
kedua segitiga.
\begin{center}
\begin{tikzpicture}[scale=.8, every node/.style={scale=.8}]
\draw(-5, -1) node[anchor=east] {$B$} -- (-1,  2) node[anchor=south]{A} 
     -- (0, -1) node[anchor=west]{$C$} -- (-5, -1); 
\draw(3, -1) node[anchor=east] {$E$} -- (7,  2) node[anchor=south]{D} 
     -- (8, -1) node[anchor=west]{$F$} -- (3, -1); 
\draw(3, -1) arc (235 : 305 : 4.35);
\end{tikzpicture}
\end{center}

Misalkan $ABC$ dan $DEF$ adalah dua buah segitiga yang memiliki 
sisi - sisi $AB$ dan $AC$ yang sama panjang dengan sisi - sisi $DE$ dan $DF$.
Yakni $AB$ dengan $DE$  dan $AC$ dengan $DF$. Dan misalkan sudut $BAC$ sama
besar dengan sudut $EDF$. Saya katakan bahwa sisi alas $BC$ juga sama panjang
dengan  sisi alas $EF$, dan segitiga $ABC$ juga akan sama dengan segitiga 
$DEF$, dan sudut sudut selebihnya yang menghadap sisi - sisi yang sama yang 
akan sama satu dengan yang lain. Yakni $ABC$ dengan $DEF$ dan $ACB$ dengan 
$DFE$.

Karena jika segitiga $ABC$ di tempatkan ke segitiga $DEF$, titik $A$ 
ditempatkan pada titik $D$, dan ruas garis $AB$ pada $DE$, maka
titik $B$ juga akan berhimpit dengan $E$ dengan alasan $AB$ sama panjang
dengan $DE$. Jadi karena $AB$ berhimpit dengan $DE$, maka ruas garis $AC$ 
juga akan berhimpit dengan $DF$, dengan alasan sudut $BAC$ sama besar 
dengan $DEF$. Jadi titik $C$ juga akan berhimpit dengan titik $F$, lagi
dikarenakan alasan $AC$ sama panjang dengan $DF$. Tapi titik $B$ juga tepat
berhimpit dengan titik $E$, sehingga sisi alas $BC$ akan berhimpit
dengan sisi alan $EF$. Karena jika $B$ berhimpitan dengan $E$, dan $C$ dengan
$F$ , dan sisi alas $BC$ tidak berhimpit dengan $EF$, maka dua buah ruas
garis akan meliputi daerah tersebut. Hal yang sangat mustahil [Post. 1].
Dengan demikian sisi alas $BC$ juga berhimpit dengan $EF$ dan sama panjang
[G.U.4], dan sudut - sudut selebihnya akan sama besar [G.U.4]. Yakni
$ABC$ sama dengan $DEF$ dan $ABC$ sama dengan $DFE$ [G.U.4].

Dengan demikian, jika dua buah segitiga memiliki dua buah sisi yang sama
panjang dan sebuah sudut yang diapit oleh sisi - sisi tadi adalah sama besar,
maka kedua nya juga akan memiliki sisi alas yang sama panjang, dan kedua
segitiga tersebut adalah sama, dan sudut - sudut selebihnya yang menghadap
sisi - sisi panjang tadi akan sama besar. Hal yang perlu dibuktikan.

\stepcounter{section}
\section*{\centering Proposisi \thesection} 
\addcontentsline{toc}{section}{Proposisi \thesection}

Pada segitiga sama sisi, sudut - sudut alas sama besar satu dengan yang lain, 
dan jika sisi - sisi yang sama panjang dipanjangkan lagi maka sudut - sudut 
yang ada di bawah sisi alas akan sama besar satu dengan yang lain.

\begin{center}
\begin{tikzpicture}
\draw [-, black] (-2, 0) node [anchor = north] {$D$} -- (0, 5) node [anchor = south]{$A$};
\draw [-, black] (2, 0) node [anchor = north] {$E$} -- (0, 5);
\draw [-, black] (-1.2, 2.0) node [anchor = east] {$B$} -- (1.2, 2.0) node [anchor = west]{$C$};
\draw [-, black] (-1.2, 2.0) -- (1.6, 1.0) node [anchor = west]{$G$};
\draw [-, black] (1.2, 2.0) -- (-1.6, 1.0) node [anchor = east]{$F$}; 
\end{tikzpicture}
\end{center}

Misalkan $ABC$ adalah segitiga sama sisi dengan sisi $AB$ sama panjang dengan 
sisi $AC$, dan misalkan ruas garis $BD$ and $CE$ sudah dipanjangkan 
masing - masing dari $AB$ dan $AC$[Post. 2].  Saya katakan bahwa sudut $ABC$ 
sama dengan $ACB$ dan sudut $CBD$ sama dengan $BCE$.

Sebab misalkan titik $F$ sudah diambil secara acak pada $BD$, dan misalkan 
$AG$ sudah dipotong dari  $AE$, sama panjang dengan $AF$ [Prop. 1.3]. Dan 
misalkan garis - garis $FC$ dan $GB$ sudah dihubungkan [Post. 1].

Kenyataanya, karena $AF$ sama panjang dengan $AG$, dan $AB$ dengan $AC$, kedua
ruas garis $FA$, $AC$ sama panjang dengan kedua ruas garis $GA$, $AB$. Keduanya
juga mengapit sudut yang sama yakni $FAG$. Dengan demikian, sisi alas $FC$ sama 
dengan sisi alas $GB$, dan segitiga $AFC$ sama dengan segitiga $AGB$, dan sudut
- sudut nya yang dihadapan sisi - sisi yang sama panjang sama besar satu sama
lain [Prop. 1.4]. Yakni $ACF$ dengan $ABG$ dan $AFC$ dengan $AGB$. Dan karena
panjang $AF$ sama dengan $AG$ dan di dalamnya $AB$ sama dengan $AC$, maka panjang
ruas garis sisanya $BF$ sama dengan $CG$ [G.U.3]. Tetapi $FC$ juga sudah kita
tunjukkan sama dengan $GB$. Jadi kedua garis lurus $BF$, $FC$ adalah sama dengan
dua garis lurus $CG$, $GB$, masing - masing, dan sudut $BFC$ sama besar dengan 
sudut $CGB$, dan sisi alas ada $BC$ adalah sama. Dengan demikian 
segitiga $BFC$ akan sama dengan segitiga $CGB$ dan sudut - sudut sisanya yang
menghadap sisi - sisi yang sama besar akan sama besar dengan 
sudut - sudut sisanya yang bersesuaian [Prop 1.4]. Dengan demikian $FBC$ sama
besar dengan $GCB$ dan $BCF$ dengan $CBG$. Dengan demikian, karena sudut 
keseluruhan $ABG$ sudah ditunjukkan sama dengan sudut $ACF$, yang didalamnya 
$CBG$ sama dengan $BCF$, sisa pengurangan $ABC$ dengan demikian sama dengan 
sisa pengurangan $ACB$ [G.U. 3]. Dan keduanya adalah alas dari segitiga $ABC$.
Dan $FBC$ juga sudah ditunjukkan sama dengan $GCB$. Dan keduanya ada dibawah
sisi alas.

Dengan demikian, untuk segitiga sama sisi, sudut - sudut yang pada alas sama
besar satu dengan yang lain, dan jika sisi - sisi yang sama panjang 
masing - masing dipanjangkan maka sudut - sudut dibawah sisi alas akan sama 
besar satu dengan yang lain. Hal yang perlu dibuktikan.

\stepcounter{section}
\section*{\centering Proposisi \thesection} 
\addcontentsline{toc}{section}{Proposisi \thesection}

Jika sebuah segitiga memiliki dua buah sudut yang sama besar satu dengan yang 
lain, maka sisi - sisi yang menghadap sudut - sudut yang sama besar tersebut
akan sama panjang satu dengan yang lain.
\begin{center}
\begin{tikzpicture}
\draw[-](0, 0)node[anchor=east] {B} --++(0:4) node [anchor=west] {$C$} --++
(110:5.84760880033) node [anchor=south] {$A$} --++(250:5.84760880033);
\draw[-](4, 0)--++(127:4.477) node [anchor=south east] {$D$};
\end{tikzpicture}
\end{center} 

Misalkan $ABC$ adalah segitiga dengan sudut $ABC$ sama dengan $ACB$. Saya 
katakan bahwa sisi $AB$ juga sama panjang dengan sisi $AC$.

Sebab jika $AB$ tidak sama panjang dengan $AC$ maka salah satu daripadanya
akan lebih panjang. Misalkan $AB$ adalah sisi yang lebih panjang. Dan 
misalkan $DB$ sama panjang dengan sisi yang lebih pendek $AC$, sudah dipotong
dari sisi yang lebih panjang $AB$ [Prop. 1.3]. Dan misalkan $DC$ sudah
terhubung [Post.1].

Dengan demikian, karena $DB$ sama panjang dengan $AC$, dan $BC$ adalah milik
keduanya, masing - masing sisi - sisi $DB$, $BC$ sama panjang dengan sisi - sisi
$AC$, $CB$, dan sudut $DBC$ sama besar dengan sudut $ACB$.
Dengan demikian sisi alas $DC$ sama besar dengan sisi alas $AB$, dan segitiga
$DBC$ akan sama dengan segitiga $ACB$ [Prop. 1.4], yang lebih kecil sama
dengan yang lebih besar. Gagasan yang adalah mustahil [G.U. 5].
Dengan demikian keduanya sama.

Dengan demikian, jika sebuah segitiga memiliki dua buah sudut yang sama besar
satu dengan yang lain maka sisi - sisi yang menghadap sudut - sudut yang sama
besar akan juga sama panjang satu dengan yang lain. Hal yang perlu dibuktikan.

\stepcounter{section}
\section*{\centering Proposisi \thesection} 
\addcontentsline{toc}{section}{Proposisi \thesection}
Pada satu garis lurus , dua buah garis lurus yang lain yang masing - masing 
sama panjang dengan dua buah garis lurus yang lain lagi, tidak bisa dibuat 
bertemu pada titik yang berbeda di sisi yang sama dari garis lurus, tetapi
memiliki titik - titik ujung yang sama pada garis lurus tersebut.
\begin{center}
\begin{tikzpicture}
\draw[-] (-4, -4)node [anchor=east] {\textbf{A}} (-4, -4) -- (4, -4) 
node[anchor=west]{\textbf{B}} -- (2, 0) node [anchor=south west] {\textbf{D}}--
(-4, -4);
\draw[-] (4, -4) -- (0, 0.5) node [anchor=south] {\textbf{C}} -- (-4, -4);
\draw[-] (0, 0.5) -- (2, 0);
\end{tikzpicture} 
\end{center}
Karena, jika mungkin, misalkan kedua buah garis $AC$, $CB$, masing - masing 
sama panjang dengan dua buah garis yang lain $AD$, $DB$, sudah di buat berada
pada satu garis yang sama $AB$, dan bertemu di titik - titik $C$ dan $D$, pada
sisi yang sama dari $AB$, dan memiliki ujung - ujung yang sama pada $AB$. 
Sehingga $CA$ sama panjang dengan $DA$, dengan sama - sama memiliki titik $A$ 
sebagai ujungnya dan $CB$ sama panjang dengan $DB$ dengan, dengan sama - sama
memiliki titik $B$ sebagai ujungnya. Dan misalkan $CD$ sudah dipertemukan 
[Post. 1].  

Dengan demikian, karena $AC$ sama panjang dengan $AD$, sudut $ACD$ 
juga akan sama dengan $ADC$ [Prop 1.5]. Dengan demikian sudut $ADC$ 
(yang sama dengan sudut $ACD$ atau $DCA$)lebih 
besar dari sudut $DCB$ (atau $BCD$) 
[G.U.5]. Dengan demikian $CDB$ (yang lebih besar dari $ADC$ dan $ACD$ [G.U.5]) 
jauh lebih besar dari $DCB$ (atau $BCD$).
Dan lagi karena $CB$ sama panjang dengan $DB$, sudut $CDB$ juga sama besar
dengan sudut $DCB$ [Prop. 1.5]. Tetapi sudah ditunjukkan bahwa sudut terdahulu 
pertama juga jauh lebih besar dari yang kemudian. Hal yang tidak mungkin.

Dengan demikian, pada satu garis lurus , dua buah garis lurus yang lain yang masing - masing 
sama panjang dengan dua buah garis lurus yang lain lagi, tidak bisa dibuat 
bertemu pada titik yang berbeda di sisi yang sama dari garis lurus, tetapi
memiliki titik - titik ujung yang sama pada garis lurus tersebut.

\stepcounter{section}
\section*{\centering Proposisi \thesection} 
\addcontentsline{toc}{section}{Proposisi \thesection}
Jika pada dua buah segitiga,  segitiga yang pertama memiliki dua buah sisi 
yang sama panjang dengan segitiga yang kedua, dan juga alas kedua segitiga 
tersebut sama besar, maka pada kedua segitiga sudut - sudut 
yang diapit oleh sisi yang sama panjang tadi akan sama besar juga.

\begin{center}
\begin{tikzpicture}
\draw[-](-3, -1)node[anchor = east]{\text{B}}(-3, -1) -- 
(0, -.25)node[anchor=west]{\textbf{C}} -- (-2, 2.75) node[anchor=south]{\textbf{A}} 
-- (-3, -1);

\draw[-](3, -1)node[anchor = east]{\text{E}}(3, -1) -- 
(6, -.25)node[anchor=west]{\textbf{F}} -- (5, 3) node[anchor=south]{\textbf{G}} 
-- (3, -1);
\draw[-](6, -0.25) -- (4, 2.75) node[anchor=south]{\textbf{D}} -- (3, -1);
\end{tikzpicture}
\end{center}

Misalkan $ABC$ dan $DEF$ adalah dua buah segitiga dengan dua buah sisi 
$AB$ dan $BC$ yang sama panjang dengan dua sisi $DE$ dan $DF$. Yakni
$AB$ dengan $DE$, dan $AC$ dengan $DF$. Misalkan juga sisi alas $BC$ sama
panjang dengan $EF$. Saya katakan bahwa sudut $BAC$ juga sama besar dengan 
sudut $EDF$.

Karena jika segitiga $ABC$ diterapkan pada segitiga $DEF$, titik $B$ ditaruh
pada titik $E$, dan garis lurus $BC$ pada $EF$, dan titik $C$ akan bertemu
dengan titik $F$, dikarenakan $BC$ sama panjang $EF$. Jadi karena $BC$
berimpit dengan $EF$, sisi - sisi $BA$ dan $CA$ masing - masing juga akan 
berhimpit dengan $ED$ dan $EF$. Karena jika sisi $BC$  berimpit dengan $EF$, 
tetapi masing - masing sisi - sisi $AB$ dan $AC$  tidak berimpit dengan 
$ED$ dan $EF$, tetapi meleset seperti $EG$ dan $EF$ di gambar atas, maka 
pada garis yang sama kita akan membuat dua buah garis yang masing - masing
sama panjang panjang dengan dua buah garis lain dan bertemu pada titik yang
berbeda pada sisi yang sama tetapi bertemu pada ujung yang sama. Tetapi garis
seperti itu tidak bisa dibuat [Prop.1. 7]. Dengan demikian, sisi alas $BC$ 
diterapkan ke sisi alas $EF$, masing - masing sisi $BA$ dan sisi  $AC$ tidak 
bisa bertemu dengan $ED$ dan $EF$. Dengan demikian mereka harus bertemu.
Dengan demikian sudut $BAC$ juga akan bertemu dengan $EDF$ dan sama besar 
[G.U. 4]

Dengan demikian, jika dua buah segitiga masing - masing memiliki dua buah 
sisi sama panjang dengan sisi lain, dan keduanya memiliki alas 
sama panjang maka sudut yang diapit oleh sisi yang sama panjang tadi akan 
sama besar.

\stepcounter{section}
\section*{\centering Proposisi \thesection} 
\addcontentsline{toc}{section}{Proposisi \thesection}
Untuk membagi jadi dua bagian sama besar sudut rektilinier tertentu.
\begin{center}
\begin{tikzpicture}
\draw (0, 0) node [anchor = north] {B} -- (1, 3) node [anchor = east] {D} 
	-- (2, 6) node [anchor = south] {A};
\draw (4, 0) node [anchor = north] {C} -- (3, 3) node [anchor = west] {E} 
	-- (2, 6);
\draw (1, 3) -- (3, 3);
\draw [dashed, gray](1, 3) circle(2);
\draw [dashed, gray](3, 3) circle(2);
\draw (1, 3) -- ++(-60: 2) -- ++(60: 2);
\draw (2, {3 - 2*sin(60)}) node [anchor=north] {F} -- (2, 6);
\end{tikzpicture}
\end{center} 

Misalkan $BAC$ adalah sudut rektilinier. Jadi dibutuhkan untuk 
membaginya menjadi dua bagian sama besar.

Misalkan titik $D$ sudah diambil secara acak di sepanjang $AB$, dan misalkan
$AE$, sama panjang dengan $AD$ sudah dipotong dari $AC$ [Prop. 1. 3], dan 
misalkan $DE$ sudah dihubungkan. Dan misalkan segitiga sama sisi $DEF$ sudah
dibuat pada $DE$ [Prop. 1.1], dan misalkan $AF$ sudah dihubungkan. Saya 
katakan bahwa sudut $BAC$ sudah dibagi dua sama besar oleh garis lurus $AF$.

Dikarenakan $AD$ sama panjang dengan $AE$, dan $AF$ adalah milik keduanya, masing - masing
kedua garis lurus $DA$, $AF$ sama panjang dengan garis lurus $EA$, $AF$. Dan 
sisi alas $DF$ sama panjang dengan sisi alas $EF$. Dengan demikian, segitiga
$DAF$ sama dengan segitiga $EAF$ [Prop. 1.8].

Dengan demikian, sudut rectiliniear $BAC$ tadi sudah dibagi menjadi dua bagian
sama besar oleh garis lurus $AF$. Yang adalah hal yang ingin dilakukan.

\stepcounter{section}
\section*{\centering Proposisi \thesection} 
\addcontentsline{toc}{section}{Proposisi \thesection}
Untuk memotong satu buah garis dengan panjang tertentu menjadi dua buah bagian 
sama panjang.

\begin{center}
\begin{tikzpicture}
\draw(0, 0) node [anchor=east] {A} -- (4, 0) node [anchor=west] {B};
\draw(0, 0) [dashed, gray] circle(4); 
\draw(4, 0) [dashed, gray] circle(4); 
\draw (0, 0) -- (2, {4*sin(60)}) node [anchor=south] {C}-- (4, 0);
\draw [dashed, gray](0, 0) -- (2, {-4*sin(60)}) -- (4, 0);
\draw (2, {4*sin(60)}) -- (2, {-4*sin(60)});
\draw (2, 0) node[anchor=north east] {D};
\end{tikzpicture}
\end{center}

Misalkan $AB$ adalah sebuah ruas garis dengan panjang tertentu, jadi diperlukan
untuk memotong $AB$ menjadi dua bagian sama panjang.

Misalkan segitiga sama sisi $ABC$ sudah di buat dari $AB$ [Prop. 1.1] dan
misalkan sudut $ABC$ sudah dibagi menjadi dua bagian sama besar oleh garis 
$CD$ [Prop. 1.9]. Saya katakan bahwa garis lurus $AB$ sudah dipotong menjadi
dua bagian sama besar pada titik $D$.

Sebabnya adalah karena $AC$ sama dengan $CB$ dan $CD$ adalah milik keduanya, masing - masing
kedua garis lurus $AC$, $CD$ sama panjang dengan garis lurus $BC$, $CD$. Dan 
sudut $ACD$ sama besar dengan sudut $BCD$. Dengan demikian sisi alas $AD$ sama
panjang dengan sisi alas $BC$ [Prop. 1.4].

Dengan demikian ruas garis garis $AB$ sudah dipotong menjadi dua bagian panjang
pada titik $D$. Yang adalah hal yang dibutuhkan.

\stepcounter{section}
\section*{\centering Proposisi \thesection} 
\addcontentsline{toc}{section}{Proposisi \thesection}
Untuk membuat sebuah garis yang membentuk sudut siku (tegak lurus) terhadap satu
garis lurus tertentu dari satu titik tertentu pada garis lurus tertentu itu.
\begin{center}
\begin{tikzpicture}
\draw (0, 0) node [anchor=south] {A} -- (10, 0) node [anchor=south] {B};
\draw (5, 0) node [anchor=north] {C};
\draw (3, 0) node [anchor=north east] {D};
\draw [dashed, gray](5, 0) circle(2);
\draw (7, 0) node [anchor=north west] {E};
\draw [dashed, gray](7, 0) circle(4);
\draw [dashed, gray](3, 0) circle(4);
\draw (7, 0) --++(120:4) --++(-120:4);
\draw (5, 0) -- (5, {4*sin(60)}) node [anchor=south] {F};
\end{tikzpicture}
\end{center}

Misalkan $AB$ adalah garis lurus tertentu dan $C$ adalah titik tertentu yang
terletak padanya. Jadi perlu untuk menarik garis lurus dari $C$ yang tegak
lurus terhadap $AB$

Misalkan titik $D$ sudah dipilih dengan acak pada $AC$, dan misalkan $CE$ 
sudah dibuat sama panjang dengan dengan $CD$ [Prop. 1. 3], dan misalkan
segitiga sama sisi $FDE$ sudah dibuat pada $DE$ [Prop. 1.1], dan misalkan
$FC$ sudah dihubungkan. Saya katakan bahwa garis lurus $FC$ sudah ditarik
tegak lurus terhadap garis lurus $AB$ dari titik $C$ yang apa pada nya.

Disebabkan karena $DC$ sama panjang dengan $CE$, dan $CF$ adalah milik keduanya, 
masing - masing kedua garis lurus $DC$, $CF$ sama panjang dengan garis lurus
$EC$, $CF$. Dan sisi alas $DF$ sama panjang dengan $FE$. Dengan demikian
sudut $DCF$ sama besar dengan sudut $ECF$ [Prop. 1.8], dan keduanya 
bersebelahan. Tetapi jika sebuah garis lurus berdiri diatas garis lurus
yang lain dan membuat sudut bersebelahan yang sama besar satu dengan yang
lain, maka masing - masing sudut yang sama besar tersebut adalah sudut
siku - siku [Def. 1.10]. Dengan demikian sudut - sudut
$DCF$ dan $FCE$ adalah sudut siku - siku.

Dengan demikian garis lurus $CF$ sudah ditarik tegak lurus dengan
$AB$ pada titik $C$ yang ada padanya. Hal yang kita perlukan.


\stepcounter{section}
\section*{\centering Proposisi \thesection} 
\addcontentsline{toc}{section}{Proposisi \thesection}
Untuk menggambarkan satu garis lurus yang tegak lurus terhadap garis
lurus tak terhingga tertentu dari satu titik yang tidak berada pada
garis tertentu tersebut.  
\begin{center}
\begin{tikzpicture}
\draw (0, 0) node [anchor=south east] {A} -- (10, 0) node [anchor=south west]  {B};
\draw (5, 4) node [anchor=south] {C};
\draw (5, 4) circle(5);
%\draw (6.4, -.7) node [anchor=north west] {D} -- (6.6, -.8);
\draw (6.5, -.77) circle(.03) node [anchor=north west] {D};
\draw (2, 0) node [anchor=north]{G};
\draw (8, 0) node [anchor=north]{E};
\draw (5, 0) node [anchor=north]{H} -- (5, 4);
\draw (2, 0) -- (5, 4)--(8, 0);
\end{tikzpicture}
\end{center} 
Misalkan $AB$ adalah garis lurus tak terhingga tertentu dan $C$ adalah titik 
tertentu, yang tidak berada di ($AB$). Jadi diperlukan untuk menarik garis lurus
tegak lurus dari titik $C$ yang tidak berada di $AB$ terhadap garis tak 
terhingga $AB$.

Karena misalkan titik $D$ sudah diambil secara acak di sisi yang berbeda $AB$ dari
$C$, dan misalkan lingkaran $EFG$ sudah digambarkan dengan pusat $C$ dan 
jari - jari $CD$ [Post. 3], dan misalkan garis $EG$ sudah dibagi dua pada
titik $H$ [Prop. 1.10], dan misalkan garis - garis lurus $CG$, $CH$ dan $CE$
sudah dihubungkan. Saya katakan bahwa garis lurus $CH$ sudah ditarik tegak 
lurus terhadap garis lurus tak terhingga $AB$ dari titik $C$, yang tidak 
berada pada $AB$. 

Disebabkan karena $GH$ sama panjang dengan $HE$ dan $HC$ adalah milik keduanya, 
masing - masing kedua garis lurus $GH$, $HC$ sama panjang dengan kedua garis 
lurus  $EH$, $HC$ dan sisi alas $CG$ sama panjang dengan sisi alas $CE$. 
Dengan demikian sudut $GHC$ sama besar dengan sudut $EHC$ [Prop. 1.8], dan 
keduanya  bersebelahan. Tetapi jika sebuah garis lurus berdiri di atas
sebuah garis lurus yang lain membuat sudut - sudut yang bersebelahan 
sama besar, maka masing - masing dari sudut yang sama besar tersebut adalah
sudut siku - siku, dan garis lurus yang pertama disebut tegak lurus terhadap
garis tempatnya berdiri [Def. 1.10].

Dengan demikian garis lurus $CH$ sudah ditarik tegak lurus terhadap garis
lurus tak terhingga $AB$ dari titik tertentu $C$ yang tidak berada pada
$AB$. Yang adalah hal yang diperlukan.


\stepcounter{section}
\section*{\centering Proposisi \thesection} 
\addcontentsline{toc}{section}{Proposisi \thesection}
Jika sebuah garis lurus berdiri pada sebuah garis lain membentuk sudut - sudut, maka
sudut - sudut yang dibentuk adalah dua buah sudut siku - siku , atau sudut sudut yang
jumlahnya sama dengan dua buah sudut siku - siku.
\begin{center}
\begin{tikzpicture}
\draw [-, black](-2,-2) -- (2, -2);
\draw (-2, -2.5) node {D} (0, -2.5) node {B} (2, -2.5) node {C};
\draw [-, black](-0,-2) -- (0, 2); 
\draw [-, black](-0,-2) -- (3, 2);
\draw (0, 2.5) node {E} (3, 2.5) node {A};
\end{tikzpicture}
\end{center}

Sebab misalkan garis lurus $AB$ berdiri pada garis lurus $CD$ membentuk 
sudut - sudut $CBA$ dan $ABD$. Saya katakan bahwa sudut - sudut $CBA$ dan
$ABD$ adalah masing - masing sudut siku - siku atau besar keduanya adalah 
sejumlah dua buah sudut siku - siku.

Pada kenyataannya, jika $CBA$ sama dengan $ABD$ maka keduanya adalah dua buah 
sudut siku - siku [Def 1.10]. Jika tidak, asumsikan $BE$ sudah ditarik secara
tegak lurus dari titik $B$ pada garis lurus $CD$ [Prop. 1.11]. Dengan demikian
$CBE$ dan $EBD$ adalah dua sudut siku - siku. Dan karena besar sudut $CBE$ sama 
dengan jumlah kedua sudut $CBA$ dan $ABE$, misalkan $EBD$ sudah ditambahkan pada 
keduanya. Dengan demikian jumlah sudut - sudut $CBE$ dan $EBD$ sama dengan
jumlah ketiga sudut $CBA$, $ABE$ dan $EBD$ [G.U. 2]. Dan lagi, karena $DBA$
sama dengan jumlah kedua sudut $DBE$ dan $ABE$, misalkan $ABC$ sudah 
ditambahkan pada keduanya. Dengan demikian jumlah sudut sudut $DBA$  dan $ABC$
sama dengan jumlah ketiga sudut $DBE$, $EBA$ dan $ABC$ [G.U. 2]. Tetapi jumlah
dari $CBE$ dan $EBD$ sudah ditunjukkan sama dengan jumlah tiga sudut. Dan hal -
hal yang sama terhadap hal yang sama adalah sama satu dengan yang lain. Dengan
demikian jumlah dari $CBE$ dan $EBD$ juga sama dengan jumlah dari $DBA$ dan 
$ABC$. Tetapi jumlah dari $CBA$dan $ABD$ adalah dua buah sudut siku - siku. 
Dengan demikian jumlah dari $ABD$ dan $ABC$ adalah juga sama dengan dua buah
sudut siku - siku.

Dengan demikian, jika sebuah garis lurus berdiri di atas garis lurus yang lain
membentuk sudut, pasti akan membentuk dua buah sudut siku - siku atau dua buah
sudut yang berjumlah dua buah sudut siku - siku. Yang adalah hal yang ingin
kita tunjukkan.

\stepcounter{section}
\section*{\centering Proposisi \thesection} 
\addcontentsline{toc}{section}{Proposisi \thesection}

Jika dua buah garis lurus yang tidak berada pada sisi yang sama, membentuk
sudut sudut yang berdampingan yang jumlahnya adalah dua buah sudut siku - siku
dengan garis lurus lain pada satu titik di atasnya, maka kedua garis lurus tersebut 
akan lurus satu dengan yang lain.

\begin{center}
\begin{tikzpicture}
\draw (0, 0) node [anchor=north] {C} -- (5, 0) node [anchor=north] {B} --
	(10, 0) node [anchor=north]{D};
\draw (5, 0) -- (0, 5) node [anchor=south] {A};
\draw (5, 0) -- (10,5) node [anchor=south] {E};
\end{tikzpicture}
\end{center}

Karena misalkan kedua garis lurus $BC$ dan $BD$, tidak berada pada sisi yang sama,
membentuk sudut - sudut berdampingan $ABC$ dan $ABD$ yang jumlahnya sama dengan
dua buah sudut siku - siku dengan garis lurus $AB$, pada titik $B$ pada nya, Saya
katakan garis $BD$ adalah lurus terhadap $CB$.

Karena jika $BD$ tidak lurus terhadap $BC$ maka misalkan $BE$ lurus
terhadap $BC$. 

Dengan demikian, karena garis lurus $AB$ berdiri pada garis lurus $CBE$, maka 
jumlah sudut - sudut $ABC$ dan $ABE$ dengan demikian sama dengan dua buah 
sudut siku - siku [Prop. 1.13]. Tetapi jumlah dari $ABC$ dan $ABD$ juga sama
dengan dua buah sudut siku - siku. Dengan demikian jumlah sudut - sudut $CBA$
dan $ABE$ sama dengan jumlah kedua sudut $CBA$ dan $ABD$ [G.U. 1]. Misalkan 
sudut $CBA$ sudah dikurangkan dari keduanya. Dengan demikian sisa pertama 
yakni $ABE$ sama dengan sisa kedua yakni $ABD$
 [G.U. 3]. Yang lebih kecil sama
dengan yang lebih besar. Hal yang mustahil. Dengan demikian $BE$ tidak lurus pada
$CB$, dengan cara yang sama, kita bisa juga buktikan bahwa garis - garis lain
juga tidak kecuali $BD$. Dengan demikian $CB$ lurus terhadap $BD$.

Dengan demikian, jika dua buah garis lurus, tidak berada pada sisi yang sama, 
membentuk sudut berdampingan yang jumlahnya adalah sama dengan dua buah sudut
siku - siku dengan garis lurus lain, pada satu titik padanya, maka kedua
garis lurus tersebut adalah lurus satu terhadap yang lain. Yang adalah hal yang
perlu kita tunjukkan.  

\stepcounter{section}
\section*{\centering Proposisi \thesection} 
\addcontentsline{toc}{section}{Proposisi \thesection}
Jika dua buah garis lurus saling memotong satu dengan yang lain maka sudut - 
sudut yang bertolak belakang  sama besar satu dengan yang lain.
\begin{center}
\begin{tikzpicture}
\draw (-1, 2) node [anchor=east] {D} -- (7, 2) node [anchor=west] {C};
\draw (3, 2) node [anchor=south] {E};
\draw (6, 0) node [anchor=west] {B} -- (0, 4) node [anchor=east] {A};
\end{tikzpicture}
\end{center}
Karena misalkan dua buah garis lurus $AB$ dan $CD$ saling memotong satu dengan
yang lain di titik $E$. Saya katakan sudut $AEC$ sama besar dengan dengan 
sudut $DEB$, dan sudut $CEB$ sama besar dengan sudut $AED$.

Karena garis lurus $AE$ berdiri di atas garis lurus $CD$, membentuk sudut
$CEA$ dan $AED$, jumlah sudut $CEA$ dan $AED$ dengan demikian sama dengan
dua buah sudut siku - siku [Prop. 1.13]. Dan lagi, karena garis lurus $DE$ 
berdiri di garis lurus $AB$, membentuk sudut $AED$ dan $DEB$, jumlah sudut 
$AED$ dan $DEB$ dengan demikian sama dengan dua sudut siku - siku
[Prop. 1.13]. Tapi jumlah $CEA$ dan $AED$ juga ditunjukkan sama dengan dua 
buah sudut siku - siku. Dengan demikian, jumlah $CEA$ dan $AED$ adalah sama dengan
jumlah $AED$ dan $DEB$ [G.U. 1]. Misalkan $AED$ telah dikurangkan dari keduanya. 
Jadi, sudut sisanya $CEA$ adalah sama dengan sudut sisanya $BED$ [G.U. 3]. 
Demikian pula, bisa diperlihatkan bahwa $CEB$ dan $DEA$ juga sama.

Jadi, jika dua garis lurus memotong satu sama lain maka sudut - sudut yang 
bertolak belakang sama besar dengan yang lain. Hal yang perlu kita tunjukkan.

\stepcounter{section}
\section*{\centering Proposisi \thesection} 
\addcontentsline{toc}{section}{Proposisi \thesection}
Untuk segitiga, jika salah satu sisi diperpanjang, besar sudut 
exterior adalah lebih besar dari sudut interior yang tidak bersebelahan.

\begin{center}
\begin{tikzpicture}
\draw (0, 0) node [anchor=east] {B} -- ++(60:4) node [anchor=south]{A}
 -- ++ (290:3.7) node [anchor=south west]{C}
 -- ++ (290:2.3) node[anchor=north]{G};
\draw (0, 0) -- (6, 0) node [anchor=west] {D};
\draw (0, 0) -- ++(60:4) 
 -- ++ (290:1.85) node[anchor=east] {E} 
	-- ++ (33.3:3.15) node [anchor=south] {F} -- ++ (239.8:4.);
\draw (0, 0) -- ++(60:4) 
 -- ++ (290:1.85)-- ++ (213.4:3.15); 
\draw [gray, dashed]({4*cos(60)}, {4*sin(60)}) 
	-- ({(4*cos(60)- 3.7*cos(290-180))/2}, 0) 
	-- ++({290-(180 - (60 + (180-110)))/2}: {4*sin(60)/sin(85)})
	-- ({4*cos(60) + (4*sin(60)/sin(70))*cos(70)}, 0); 
\end{tikzpicture}
\end{center}
Misalkan $ABC$ adalah segitiga, dan misalkan salah satu sisinya $BC$ telah 
diperpanjang sampai $D$.  Saya katakan bahwa sudut eksterior $ACD$ lebih besar 
dari sudut interior yang tidak bersebelahan $CBA$ dan $BAC$.

Misalkan garis lurus $AC$ dipotong menjadi dua pada titik $E$ [Prop. 1.10]. 
Dan $BE$ sudah dipertemukan dan misalkan garis itu diperpanjang sebagai 
garis lurus ke titik $F$. 
Dan misalkan $EF$ dibuat sama panjang dengan $BE$ [Prop. 1.3], 
dan misalkan $FC$ telah dipertemukan, 
dan misalkan $AC$ telah ditarik sampai ke (titik) G.

Dengan demikian, karena $AE$ sama dengan $EC$, dan $BE$ sama dengan $EF$,
masing - masing kedua garis lurus $AE$, $EB$ sama panjang dengan kedua 
garis lurus $CE$, $EF$. Juga, sudut $AEB$ sama dengan sudut $FEC$, karena
keduanya adalah sudut -sudut yang bertolak belakang [Prop. 1.15]. 
Dengan demikian, sisi alas $AB$ sama dengan sisi alas $FC$, dan segitiga 
$ABE$ sama dengan segitiga $FEC$, dan sudut-sudut selebihnya yang berhadapan
dengan sisi-sisi yang sama panjang adalah sama panjang adalah sama besar
pada kedua segitiga[Prop. 1.4]. Dengan demikian, $BAE$ sama dengan $ECF$. 
Tetapi $ECD$ lebih besar dari $ECF$. Dengan demikian, $ACD$ lebih besar 
dari $BAE$. 
Demikian pula, dengan memotong $BC$ menjadi dua, dapat ditunjukkan 
bahwa $BCG$ - yaitu, $ACD$ - juga lebih besar dari $ABC$.  

Dengan demikian untuk segitiga, jika salah satu sisi diperpanjang, 
besar sudut eksterior adalah lebih besar dari sudut interior yang 
tidak bersebelahan.  

\stepcounter{section}
\section*{\centering Proposisi \thesection} 
\addcontentsline{toc}{section}{Proposisi \thesection}
Untuk semua segitiga, jumlah dua buah sudut manapun lebih kecil dari
dua jumlah dua buah sudut siku - siku.
\begin{center}
\begin{tikzpicture}
\draw (0, 0) node [anchor=north]{B} -- (4, 0) node [anchor=north]{C} -- (6, 0) node [anchor=north] {D};
\draw (0, 0) --++ (120:4) node [anchor=south]{A} -- (4, 0);
\end{tikzpicture}
\end{center}

Misalkan $ABC$ adalah sebuah tiga. Saya katakan bahwa jumlah  dua buah 
sudut manapun segitiga $ABC$ lebih kecil dari jumlah dua buah sudut siku - siku.

Misalkan $BC$ sudah diperpanjang sampai ke $D$.

Dan karena sudut $ACD$ adalah sudut eksterior segitiga $ABC$, maka sudut 
tersebut lebih besar dari sudut interior yang tidak bersebelahan 
$ABC$ [Prop. 1.16]. 
Misalkan $ACB$ sudah ditambahkan pada keduanya. Dengan demikian jumlah 
sudut - sudut $ACD$ dan $ACB$ lebih besar dari jumlah sudut - sudut $ABC$ dan 
$BCA$. Tetapi jumlah $ACD$ dan $ACB$  lebih besar dari jumlah sudut -sudut
$ACD$ dan $ACB$ adalah sama dengan dua buah sudut siku - siku [Prop.1.13].
Dengan demikian jumlah dari $ABC$ dan $BCA$ lebih kecil dari dua buah sudut 
siku - siku, dan lebih jauh jumlah $CAB$ dan $ABC$ lebih kecil dari dua
buah sudut siku - siku.

Dengan demikian, untuk semua segitiga, jumlah dari dua buah sudut manapun
lebih kecil dari dua buah sudut siku - siku. Hal yang perlu dibuktikan.


\stepcounter{section}
\section*{\centering Proposisi \thesection} 
\addcontentsline{toc}{section}{Proposisi \thesection}
Untuk semua segitiga, sisi yang lebih panjang menghadap sudut yang lebih
besar.

\begin{center}
\begin{tikzpicture}
\draw (0, 0) node[anchor=north]{B} --++ (120:4) node [anchor=south] {A} 
	-- ++ (330:{4*cos(30)/cos(60)}) node[anchor=west] {C} -- (0, 0);
\draw (0, 0)  --++ (120:4) -- ++ (330:4)node [anchor=west] {D}  -- (0, 0); 
\end{tikzpicture}
\end{center}

Misalkan $ABC$ adalah segitiga dengan sisi $AC$ lebih besar dari $AB$. Saya
katakan sudut $ABC$ juga lebih besar dari $BCA$.

Disebabkan karena $AC$ lebih besar dari $AB$, misalkan $AD$ dibuat sama
dengan $AB$ [Prop 1.3], dan misalkan $BD$ sudah terhubung.

Dan karena sudut $ADB$ adalah sudut exterior dari segitiga $BCD$, maka ia
akan lebih besar dari sudut interior yang tidak bersebelahan $DCB$ [Prop 1.16].
Tetapi $ADB$ sama dengan $ABD$, karena sisi $AB$ juga sama panjang dengan 
sisi $AD$ [Prop 1.15]. Dengan demikian $ABD$ juga lebih besar dari $ACB$.
Dengan demikian $ABC$ jauh lebih besar dari $ACB$.

Dengan demikian, pada setiap segitiga, sisi yang lebih besar menghadap sudut
yang lebih besar. Yang adalah hal yang perlu dibuktikan.

\stepcounter{section}
\section*{\centering Proposisi \thesection} 
\addcontentsline{toc}{section}{Proposisi \thesection}
Pada semua segitiga, sudut yang lebih besar menghadap sisi yang lebih panjang.
\begin{center}
\begin{tikzpicture}
\draw(0, 0) node[anchor=east] {B} -- (2, 2) node[anchor=south] {A} -- (2, -6) node[anchor=north] {C}-- (0,0);
\end{tikzpicture}
\end{center} 

Misalkan $ABC$ adalah segitiga dengan sudut $ABC$ lebih besar dari $BCA$. 
Saya katakan sisi $AC$ juga lebih besar dari $AB$.

Karena jika tidak, $AC$ pasti sama dengan atau lebih kecil dari $AB$. Kenyataannya
$AC$ tidak sama dengan $AB$. Karena dengan demikian sudut $ABC$ seharusnya sama 
dengan $ACB$ [Prop 1.5]. Tetapi tidak demikian. Dengan demikian $AC$ tidak 
sama dengan $AB$. Juga $AC$ tidak lebih kecil dari $AB$. Karena dengan
demikian sudut $ABC$ akan lebih kecil dari sudut $ACB$ [Prop 1.18]. Tetapi
tidak demikian kenyataannya. Dengan demikian $AC$ tidak lebih kecul dari 
$AB$. Tetapi juga telah ditunjukkan $AC$ tidak sama dengan $AB$. Dengan demikian
$AC$ lebih besar dari $AB$.

Dengan demikian , pada semua segitga, sudut yang lebih besar menghadap sisi yang
lebih panjang. Hal yang perlu dibuktikan.

\stepcounter{section}
\section*{\centering Proposisi \thesection}
\addcontentsline{toc}{section}{Proposisi \thesection}
Pada semua segitiga, jumlah dua buah sisinya lebih besar dari sisi selebihnya.

\begin{center}
%\begin{tikzpicture}
\begin{tikzpicture}[scale=.8, every node/.style={scale=.8}]
\draw (0, 0) node [anchor=north east] {B} --++ (60: 3) node [anchor=east] {A} 
	-- ++(-30:5.2) node [anchor=north west] {C} --++(180: 6.01);
\draw (0, 0) node [anchor=north east] {B} --++ (60: 8.2) node [anchor=south] {D}
	-- ++(285.05: 7.359);
\draw [gray, dashed] (0, 0) -- (240:6.01) --++ (30:10.3);
\draw [gray, dashed] ({sqrt(3*3 + 5.2*5.2)}, 0) -- ++(-30:5.2)
	-- (0, 0);
\end{tikzpicture}
\end{center}

Misalkan $ABC$ adalah segitiga. Saya katakan pada segitiga $ABC$, jumlah dua
buah sisi lebih besar dari sisi selebihnya. Sehinga jumlah $BA$ dan $AC$ lebih
besar dari $BC$ ,jumlah dari $AB$ dan $BC$ lebih besar dari $AC$, dan jumlah
dari $BC$ dan $CA$ lebih besar dari $AB$.

Sebab misalkan $BA$ sudah ditarik sampai titik $D$, dan misalkan $AD$ sudah dibuat 
sama dengan $CA$ [Prop. 1.3], dan misalkan $DC$ sudah terhubung.

Dengan demikian, karena $DA$ sama dengan $AC$, sudut $ADC$ juga sama dengan 
$ACD$ [Prop. 1.5]. Dengan demikian $BCD$ lebih besar dari $ACD$ [G.U 5]. 
Dan karena $DCB$ adalah segitiga dengan sudut $BCD$ lebih besar dari $BDC$, 
and sudut yang lebih besar menghadap sisi yang lebih besar [Prop. 1.19], $DB$
dengan demikian lebih besar dari $BC$. Tetapi $DA$ sama panjang dengan $AC$.
Dengan demikian, jumlah dari sisi $BA$ dan $AC$ lebih besar dari besar sisi
$BC$. Dengan cara yang sama kita bisa buktikan jumlah $AB$ dan $BC$ lebih besar
dari $CA$, dan jumlah dari sisi - sisi $BC$ dan $CA$ lebih besar dari $AB$.  

Sebab itu, pada semua segitiga, jumlah dua buah sisinya lebih besar dari sisi 
selebihnya. Hal yang perlu ditunjukkan.  

\stepcounter{section}
\section*{\centering Proposisi \thesection}
\addcontentsline{toc}{section}{Proposisi \thesection}
Jika dari ujung - ujung satu sisi segitiga dua buah garis lurus ditarik dan 
bertemu di dalam segitiga tersebut, jumlah kedua garis lurus tadi akan lebih
kecil dari jumlah dua sisi segitiga lainnya, tetapi akan membentuk sudut
yang lebih besar.

\begin{center}
\begin{tikzpicture}
%\begin{tikzpicture}[scale=.8, every node/.style={scale=.8}]
\draw (0, 0) node [anchor = north east] {B} --++ (65:4) node [anchor=south]{A}
	--++(-30:7.25) node [anchor=north west]{C} --++ (180:7.975);
\draw (0, 0) --++ (50:4.05) node [anchor=south west] {E};
\draw (7.975, 0) --++ (160:6.5) node [anchor=east]{D};
\end{tikzpicture}
\end{center}

Sebab misalkan dua buah garis - lurus $BD$ dan $DC$ sudah ditarik dari
satu sisi $DC$ dari segitiga $ABC$, masing - masing dari titik - titik ujung
$B$ dan $C$. Saya katakan $BD$ dan $DC$ lebih kecil dari jumlah sisi - sisi
selebihnya segitiga yaitu $BA$ dan $AC$, tetapi membentuk sudut 
$BDC$ yang lebih besar  dari $BAC$.

Sebab misalkan $BD$ sudah ditarik melalui sampai ke $E$. Dan karena pada semua
segitiga jumlah dari dua buah sisi lebih besar dari sisi selebihnya 
[Prop. 1.20], pada segitiga $ABE$ jumlah dua sisi $AB$ dan $AE$ dengan demikian
lebih besar dari $BE$. Misalkan $EC$ sudah ditambahkan pada keduanya. Dengan 
demikian jumlah dari $BA$ dan $AC$ lebih besar dari jumlah $BE$ dan $EC$. Dan 
lagi, karena pada segitiga $CED$ jumlah dua buah sisi $CE$ dan $ED$ lebih
besar dari $CD$, misalkan $DB$ sudah ditambahkan pada keduanya. Dengan demikian
jumlah dari $CE$ dan $EB$ lebih besar dari jumlah $CD$ dan $DB$. Tetapi
sudah ditunjukkan jumlah $BA$ dan $AC$ lebih besar dari jumlah $BE$ dan $EC$.
Dengan demikian jumlah $BA$ dan $AC$ jauh lebih besar dari jumlah $BD$ 
dan $DC$.

Lagi, karena pada semua segitiga sudut exterior lebih besar dari sudut - sudut
interior yang tidak bersebelahan [Prop. 1.16], pada segitiga $CDE$ sudut exterior
$BDC$ dengan demikian lebih besar dari $CED$. Karenanya, dengan alasan yang
sama sudut exterior $CEB$ dari segitiga $ABE$ juga lebih
besar dari $BAC$. Tetapi $BDC$ sudah ditunjukkan lebih besar dari $CEB$. Dengan
demikian $BDC$ jauh lebih besar dari $BAC$.
 
Dengan demikian, jika dari ujung - ujung satu sisi segitiga dua buah 
garis lurus ditarik dan bertemu di dalam segitiga tersebut, jumlah kedua 
garis lurus tadi akan lebih kecil dari jumlah dua sisi segitiga lainnya, 
tetapi akan membentuk sudut
yang lebih besar. Hal yang perlu kita tunjukkan.


\stepcounter{section}
\section*{\centering Proposisi \thesection}
\addcontentsline{toc}{section}{Proposisi \thesection}
Untuk membuat sebuah segitiga dari tiga buah garis lurus dimana panjang
sisi - sisi segitiga sama dengan ketiga garis lurus. Perlu dipastikan bahwa
jumlah dua buah garis mana saja lebih besar dari selebihnya, dengan 
perhitungan pada fakta pada semua segitiga, jumlah dari dua buah sisi
selalu lebih besar dari sisi selebinya [Prop. 1.20].
\begin{center}
%\begin{tikzpicture}
\begin{tikzpicture}[scale=.8, every node/.style={scale=.8}]
\draw (0, 5) node [anchor=east] {A}-- (5, 5);
\draw (0, 4) node [anchor=east] {B}-- (4, 4);
\draw (0, 3) node [anchor=east] {C}-- (3, 3);
\draw (0, -5) node[anchor=east] {D} --++ (0:5) node[anchor=north]{F} 
	   --++ (0:4) node[anchor=north]{G} --++ (0:3) node [anchor=north west]{H}
	   --++ (0:2) node[anchor=west]{E};
\draw (5, -5) circle(5);
\draw (9, -5) circle(3);
\draw (9, -5) --++(90:3) node[anchor=south]{K} --++({180 + asin(3/5)}:5);
\draw (9, -8) node[anchor=north]{L}; 
\end{tikzpicture}
\end{center}

Misalkan $A$, $B$, $C$ adalah tiga buah garis lurus, yang dimana panjang
dua buah garis mana saja lebih lebih besar dari garis sisanya. Dengan demikian
panjang $A$ dan $B$ lebih besar dari $C$, panjang $A$ dan $C$ lebih besar dari
$B$, juga panjang $B$ dan $C$ lebih besar dari $A$. Jadi diperlukan untuk
membuat segitiga dengan sisi - sisi sama dengan $A$, $B$ dan $C$.

Misalkan garis $DE$ sudah dibuat, berujung di $D$ dan tak terhingga di arah $E$.
Dan misalkan $DF$ sudah dibuat sama dengan $A$, dan $FG$ sama dengan $B$, dan 
$GH$ sama dengan $C$ [Prop. 1.3]. Dan misalkan lingkaran $DKL$ sudah di ditarik
dengan pusat pusat $F$ dan jari - jari $FD$. Dan lagi, lingkaran $KLH$ sudah
ditarik dengan pusat $G$ dan jari - jari $GH$. Dan misalkan $KF$ dan $KG$ sudah
dihubungkan. Saya katakan segitiga $KFG$ sudah dibuat dengan tiga buah
garis yang sama panjang dengan $A$, $B$ dan $C$.

Disebabkan titik $F$ adalah pusat lingkaran $DKL$, $FD$ sama dengan $FK$. 
Tetapi, $FD$ sama dengan $A$. Dengan demikian, $KF$ juga sama dengan $A$. Lagi, 
karena titik $G$ adalah pusat dari lingkaran $LKH$, $GH$ sama dengan $GK$, $GH$
juga sama dengan $GK$. Tetapi, $GH$ sama dengan $C$. Dengan demikian $GH$ sama
degnan $C$. $FG$ juga sama dengan $B$. Dengan demikian, masing - masing 
ketiga garis lurus $KF$, $FG$ dan $GK$ sama dengan  $A$, $B$ dan $C$.

Dengan demikian, segitiga $KFG$ sudah dibuat dari tiga buah garis lurus $KF$, 
$FG$ dan $GK$, yang masing - masing sama panjang dengan tiga garis lurus 
$A$, $B$ dan $C$. Hal yang perlu kita lakukan.


Contoh lain

\begin{center}
%\begin{tikzpicture}
\begin{tikzpicture}[scale=.5, every node/.style={scale=.5}]
\draw (0, -5)[gray, dashed] node[anchor=east] {D} --++ (0:3) node[anchor=north]{F} 
	   --++ (0:4) node[anchor=north]{G} --++ (0:5) node [anchor=north west]{H}
	   --++ (0:2) node[anchor=west]{E};
\draw (3, -5)[gray, dashed] circle(3);
\draw (7, -5)[gray, dashed] circle(5);
\draw (3, -5)[gray, dashed] --++(90:3) node[anchor=south]{K} --++({- acos(4/5)}:5);
\draw (3, -8)[gray, dashed] node[anchor=north]{L}; 
\end{tikzpicture}
\end{center}


\stepcounter{section}
\section*{\centering Proposisi \thesection}
\addcontentsline{toc}{section}{Proposisi \thesection}
Untuk membentuk sebuah sudut rektilinier yang sama besar dengan sudut 
rektilinier tertentu pada satu titik pada satu garis lurus tertentu.

\begin{center}
\begin{tikzpicture}
\draw (0, 0) node[anchor=east] {C} --++ (20:3) node[anchor=south]{D} --++(20:1);
\draw (0, 0) --++ (-25:3.5) node[anchor=north]{E} --++(-25:1);
\draw (0, 0) --++ (20:3) --++(278:2.5); 
\draw (0, -6) node [anchor=east] {A} --++ (0:3.5) node[anchor=north] {G}
	--++ (0:2.5) node [anchor=north] {H} --++ (0:1) node[anchor=north] {B}; 
\draw (0, -6)[gray, dashed] circle(3.0);
\draw (3.5, -6)[gray, dashed] circle(2.5);
\draw (0, -6) --++(44.5:3) node[anchor=south]{F}--++(303:2.5);
\end{tikzpicture}
\end{center}

Misalkan $AB$ adalah satu garis lurus tertentu, $A$ adalah satu titik pada 
garis tersebut, dan $DCE$ adalah sudut rektilinier tertentu. Jadi diperlukan
untuk membuat sudut rektilinier yang sama besar dengan sudut rectiliniear
$DCE$ pada titik tertentu $A$ pada garis lurus tertentu $AB$.

Misalkan titik $D$ dan $E$ sudah diambil secara acak pada masing - masing
pada garis lurus $CD$ dan $CE$, dan misalkan $DE$ sudah dihubungkan. 
Dan misalkan segitga $AFG$ sudah di buat dari tiga buah garis lurus yang 
masing - masing sama panjang dengan $CD$, $DE$ dan $CE$, sehingga $CD$ sama
panjang dengan $AF$, $CE$ sama panjang dengan $AG$ dan lebih jauh $DE$ 
sama panjang dengan $FG$ [Prop. 1.22].

Dengan demikian karena kedua garis lurus $DC$ dan $CE$ sama dengan kedua
garis lurus $FA$ dan $AG$, dan sisi alas $DE$ sama panjang dengan sisi
alas $FG$, sudut $DCE$ dengan demikian sama besar dengan 
sudut $FAG$ [Prop. 1.8].

Dengan demikian, sudut rektilinier $FAG$ sama besar dengan sudut rektilinier
$DCE$, sudah dibuat pada titik $A$ pada garis lurus $AB$. Yang adalah hal
yang perlu dilakukan.

\stepcounter{section}
\section*{\centering Proposisi \thesection}
\addcontentsline{toc}{section}{Proposisi \thesection}
Jika pada dua buah segitiga , yang pertama memiliki dua buah sisi yang sama 
panjang  dengan dua buah sisi pada segitiga yang kedua, tetapi pada yang pertama 
sisi - sisi tadi mengapit sudut yang lebih besar daripada sudut yang diapit oleh 
kedua sisi tadi pada segitiga yang kedua , maka segitiga yang pertama juga akan
 memiliki sisi alas yang lebih panjang yang kedua.
\begin{center}
\begin{tikzpicture}
\draw [dashed, gray](0, 0) --++(305:1)circle(.01);
\draw [dashed, gray](0, 0) --++(260:1)circle(.01) --++ (12.7:0.75);
\draw (0, 0) --++(305:4) node[anchor=north]{B};
\draw (0, 0) node [anchor=south] {A} --++(260:4.5) node[anchor=north]{C} --++ (20.70:3.29);

\draw [dashed, gray](6, 0) --++(305:1)circle(.01);
\draw [dashed, gray](6, 0) --++(260:1)circle(.01) --++ (12.7:0.75);
\draw [dashed, gray](6, 0) circle(1);
\draw [dashed, gray](6, 0) --++(305:1) circle(.75);
\draw (6, 0) --++(305:4) node[anchor=north]{E};
\draw (6, 0) node [anchor=south] {D} --++(280:4.5) node[anchor=north]{F} --++ (37.5:1.91);
\draw (6, 0) --++(260:4.5) node[anchor=north]{G}-++ (20.70:3.29);
\draw (6, 0) --++(260:4.5) --++ (0:1.57);
\end{tikzpicture}
\end{center} 

Misalkan $ABC$ dan $DEF$ adalah dua buah segitiga dengan masing - masing 
dua sisi $AB$ dan $AC$ sama panjang dengan dua sisi $DE$ dan $DF$. Yakni
$AB$ dengan $DE$, dan $AC$ terhadap $DF$. Dan misalkan juga sudut $A$ 
lebih besar dari sudut $D$. Saya katakan sisi alas $BC$ juga lebih besar
dari sisi alas $EF$.

Disebabkan karena sudut $BAC$ lebih besar daripada sudut $EDF$, misalkan
sudut $EDG$, sama besar dengan sudut $BAC$, sudah dibuat pada titik $D$
pada titik $DE$ [Prop. 1.23]. Dan misalkan $DG$ dibuat sama panjang
dengan salah satu $AC$ atau $DF$ [Prop. 1.3]. Dan misalkan $EF$ dan 
$FG$ sudah dihubungkan.

Dengan demikian, karena $AB$ sama dengan $DE$ dan $AC$ terhadap $DG$,
masing - masing garis lurus $BA$, $AC$ sama panjang dengan  dengan
kedua garis lurus $ED$, $DG$. Juga sudut $BAC$ sama besar dengan sudut
$EDG$. Dengan demikian sisi alas $BC$ sama panjang dengan sisi alas
$EG$ [Prop. 1.4]. Dan lagi karena $DF$ sama panjang dengan $DG$, sudut
$DFG$ sama besar dengan sudut $DGF$ [Prop. 1.5]. Dengan demikian $DFG$
lebih besar dari $EGF$ (Karena $DFG$ sama dengan $DGF$, dan $EGF$
adalah bagian dari $DGF$ [G.U. 1.5]). Dengan demikian $EFG$ jauh lebih
besar dari dari $EGF$ (karena $DFG$ bagian dari $EFG$ [G.U. 1.5]). Dan
karena segitiga $EFG$ memiliki sudut $EFG$ lebih besar dari sudut
$EGF$ dan sudut yang lebih besar menghadap sisi yang lebih besar 
[Prop. 1.19], sisi $EG$ dengan demikian juga lebih besar dari
sisi $EF$. Tetapi $EG$ sama dengan $BC$. Dengan demikian $BC$ juga
lebih besar dari $EF$.

Dengan demikian, jika pada dua buah segitiga , yang pertama memiliki 
dua buah sisi yang sama panjang  dengan dua buah sisi pada segitiga yang 
kedua, tetapi pada yang pertama sisi - sisi tadi mengapit sudut yang 
lebih besar daripada sudut yang diapit oleh kedua sisi tadi pada segitiga 
yang kedua , maka segitiga yang pertama juga akan memiliki sisi alas 
yang lebih panjang yang kedua. Yang adalah hal yang perlu ditunjukkan.

\stepcounter{section}
\section*{\centering Proposisi \thesection}
\addcontentsline{toc}{section}{Proposisi \thesection}
Jika dua buah segitiga , yang pertama memiliki dua buah sisi yang
sama panjang dengan dua buah sisi pada segitiga yang kedua, tetapi
yang pertama memiliki sisi alas yang lebih panjang dari yang kedua, 
maka yang pertama juga memiliki sudut yang diapit oleh dua sisi tadi
lebih besar daripada sudut yang diapit oleh sisi yang bersesuaian 
pada segitiga yang kedua.
\begin{center}
\begin{tikzpicture}
\draw(0, 0) node[anchor=south]{A} --++ (230:4)node [anchor=east]{B} --++ (10:5) node[anchor=west]{C} --++ (137:3.225);
\draw(0, -5) node[anchor=south]{D} --++ (230:4)node [anchor=east]{E} --++ (-2.5:2.99) node[anchor=west]{F} --++ (97.5:3.225);
\end{tikzpicture}
\end{center}

Misalkan $ABC$ dan $DEF$ adalah dua buah segitiga memiliki sisi - sisi
$AB$ dan $AC$ sama panjang dengan dua buah sisi $DE$ dan $DEF$, yaitu
$AB$ sama dengan $DE$ dan $AC$ dengan $DF$. Dan misalkan sisi alas $BC$
lebih besar dari sisi alas $EF$. Saya katakan bahwa sudut $BAC$ juga lebih
besar daripada sudut $EDF$.

Karena jika tidak, $BAC$ pasti sama besar atau lebih kecil dari $EDF$. Pada 
kenyatannyannya, $BAC$ tidak sama dengan $EDF$. Karena jika benar maka sisi
alas $BC$ juga kan sama dengan $EF$ [Prop. 1.4]. Tetapi tidak. Dengan demikian
sudut $BAC$ tidak sama dengan $EDF$. Tidak juga bahwa $BAC $lebih kecil 
daripada $EDF$. Karena jika demikian, sisi alas $BC$ juga akan lebih
kecil daripada $EF$ [Prop. 1.24]. Tetapi itu tidak benar. Dengan demikian, 
sudut $BAC$ tidak lebih kecil daripada $EDF$. Tetapi juga ditunjukkan $BAC$
tidak sama dengan $EDF$. Dengan demikian $BAC$ lebih besar dari $EDF$.

Dengan demikian, jika dua buah segitiga , yang pertama memiliki dua buah sisi 
yang sama panjang dengan dua buah sisi pada segitiga yang kedua, tetapi
yang pertama memiliki sisi alas yang lebih panjang dari yang kedua, 
maka yang pertama juga memiliki sudut yang diapit oleh dua sisi tadi
lebih besar daripada sudut yang diapit oleh sisi yang bersesuaian 
pada segitiga yang kedua. Hal yang perlu ditunjukkan.

\stepcounter{section}
\section*{\centering Proposisi \thesection}
\addcontentsline{toc}{section}{Proposisi \thesection}
Jika dua buah segitiga yang pertama memiliki dua buah sudut yang sama besar
dengan yang segitiga yang kedua, dan memiliki satu sisi yang sama panjang dengan
satu sisi pada segitiga yang dua, bisa sisi yang berada pada dua sudut tadi atau
menghadap salah satu sudut tadi, maka sisi - sisi selebihnya pada yang 
pertama akan sama panjang dengan sisi - sisi selebihnya pada yang kedua, dan 
sudut selebihnya pada yang pertama akan sama besar dengan sudut selebihnya pada
yang kedua.
\begin{center}
\begin{tikzpicture}
\draw (0, 0) node [anchor=east]{B} --++(70:4) node [anchor=south]{A}
	--++(-45:5.32) node[anchor=west]{C} --++(180:5.135);
\draw (0, 0) --++(70:3) node [anchor=east]{G} --++ (-34.53:4.98);
\draw (0, 0) --++(70:4) --++(-55:4.59) node[anchor=north]{H};

\draw (6, 3) node [anchor=east]{E} --++(70:4) node [anchor=south]{D}
	--++(-45:5.32) node[anchor=west]{F} --++(180:5.135); 
\end{tikzpicture}
\end{center}


Misalkan $ABC$ dan $DEF$ adalah dua buah segitiga dengan masing - masing dua 
sudut $ABC$ dan $BCA$ sama dengan dua sudut $DEF$ dan $EFD$.  Yaitu $ABC$ sama
besar dengan $DEF$ dan $BCA$ terhadap $EFD$. Dan misalkan kedua nya juga 
memiliki satu sisi yang sama panjang. Pertama sisi yang berada pada sudut - sudut
yang berpasangan sama tadi. Yakni $BC$ sama panjang dengan $EF$. Saya katakan bahwa
sisi - sisi selebihnya pada keduanya juga memiliki sisi bersesuaian yang sama 
panjang pada segitiga satunya. Yakni $AB$ sama panjang dengan $DE$ dan $AC$ 
terhadap $DF$.  Dan sudut selebihnya juga akan sama besar pada keduanya. 
Yakni $BAC$ sama dengan $EDF$.

Karena jika $AB$ tidak sama dengan $DE$ maka yang satu adalah lebih besar. Misalkan
$AB$ yang lebih besar, dan misalkan $BG$ dibuat sama dengan $DE$ [Prop. 1.3], dan
misalkan $GC$ sudah dipertemukan.

Dengan demikian, karena $GB$ sama panjang dengan $DE$, dan $BC$ sama panjang 
dengan $EF$ dan masing - masing kedua garis lurus $GB$, $BC$ sama panjang dengan 
kedua garis lurus $DE$, $EF$. Dan sudut $GBC$ sama besar dengan sudut  $DEF$.
Dengan demikian sisi alas $GC$ sama panjang dengan sisi alas $DF$, dan segitiga
$GBC$ sama dengan segitiga  $DEF$ dan sudut - sudut selebihnya yang menghadap
sisi sisi yang sama panjang akan sama besar dengan sudut selebihnya yang 
bersesuaian [Prop. 1.4]. Dengan demikian $GCB$ sama besar dengan $DFE$. Tetapi
$DFE$ diasumsikan sama besar dengan $ACB$. Dengan demikian $BCG$ juga sama
besar dengan $BCA$, yang lebih kecil sama besar dengan yang lebih besar 
(Karena sisi yang lebih besar akan menghadap sudut yang lebih 
besar [Prop 1.19], penerjemah).
Hal yang benar - benar mustahil [G.U. 5]. Dengan demikian $AB$ tidak tidak sama dengan
$DE$. Dengan demikian keduanya sama. Dan $BC$ juga sama dengan $EF$. Sehingga
masing - masing kedua garis lurus $AB$, $BC$ sama panjang dengan dua garis 
lurus $DE$, $EF$. Dan sudut $ABC$ sama besar dengan sudut $DEF$. Dengan demikian
sisi alas $AC$ sama panjang dengan sisi alas $DF$, dan sudut selebihnya $BAC$ 
sama besar dengan sudut selebih nya $EDF$ [Prop. 1.4].

Lagi, misalkan sisi - sisi yang menghadap sudut - sudut yang sama besar adalah
sama panjang, sebagai contoh misalkan $AB$ sama panjang dengan $DE$. Lagi saya
katakan sisi -sisi dari $ABC$ selebihnya akan sama panjang dengan sisi - sisi
 selebihnya dari $DEF$. Yakni $AC$ sama panjang dengan $DF$ dan $BC$ dengan 
$EF$. Lebih lanjut lagi, sudut selebihnya $BAC$ sama besar dengan sudut 
selebihnya $EDF$.

Karena jika $BC$ tidak sama panjang dengan $EF$ maka salah satu pasti lebih besar.
Misalkan $BC$ adalah yang lebih besar. 
Misalkan $BH$ sudah dibuat sama panjang dengan $EF$ [Prop.1.3], 
dan misalkan $AH$ sudah dihubungkan. Dan karena $BH$ sama
panjang dengan $EF$ dan $AB$ terhadap $DE$, masing - masing kedua garis lurus 
$AB$, $BH$ sama panjang dengan kedua garis lurus $DE$, $EF$. Dan sudut yang
diapit keduanya sama besar. Dengan demikian sisi alas $AH$ sama panjang dengan 
$DF$, dan segitiga $ABH$ sama dengan segitiga $DEF$, dan sudut selebihnya yang
menghadap sisi - sisi yang sama panjang sama besar dengan sudut bersesuaian 
selebihnya [Prop. 1.4]. Dengan demikian sudut $BHA$ sama besar dengan sudut
$EFD$. Tetapi sudut $EFD$ sama besar dengan sudut $BCA$. Dengan demikian pada 
segitiga $AHC$, sudut eksterior $BHA$ sama besar dengan sudut interior yang tidak
bersebelahan $BCA$. Hal yang benar - benar mustahil [Prop. 1.16]. Dengan demikian 
$BC$ tidak tidak sama dengan $EF$. Dengan demikian sama. Dan $AB$ juga sama
panjang dengan $DE$. Dengan demikian masing - masing kedua garis lurus 
$AB$, $BC$ sama panjang dengan kedua garis lurus $DE$, $EF$. Dan keduanya mengapit
sudut yang sama besar. Dengan demikian sisi alas $AC$ sama besar dengan sisi 
alas $DF$ dan segitiga $ABC$ sama dengan segitiga $DEF$, dan sudut selebihnya
$BAC$ sama besar dengan sudut selebihnya $DEF$ [Prop. 1.4].

Dengan demikian jika dua buah segitiga yang pertama memiliki dua buah sudut yang sama besar
dengan yang segitiga yang kedua, dan memiliki satu sisi yang sama panjang dengan
satu sisi pada segitiga yang dua, bisa sisi yang berada pada dua sudut tadi atau
menghadap salah satu sudut tadi, maka sisi - sisi selebihnya pada yang 
pertama akan sama panjang dengan sisi - sisi selebihnya pada yang kedua, dan 
sudut selebihnya pada yang pertama akan sama besar dengan sudut selebihnya pada
yang kedua. Hal yang perlu ditunjukkan.  

\stepcounter{section}
\section*{\centering Proposisi \thesection}
\addcontentsline{toc}{section}{Proposisi \thesection}
Jika sebuah garis lurus memotong dua buah garis lurus lain dan membuat sudut - 
sudut yang berseberangan sama besar  satu dengan yang lain, kedua garis lurus 
tersebut sejajar satu dengan yang lain.
\begin{center}
\begin{tikzpicture}
\draw (0, 3) node [anchor=south]{A} -- 
	(6, 3) node [anchor=south]{B} -- (9, 2) node[anchor=west]{G};
\draw (0, 1) node [anchor=north]{C} --
	(6, 1) node [anchor=north]{D} -- (9, 2);
\draw (4, 4) -- (1, 0);
\draw (3, 3) node [anchor=south]{E};
\draw (2, 1) node [anchor=north]{F};
\end{tikzpicture}
\end{center} 
Misalkan garis lurus $EF$, jatuh memotong dua buah garis lurus $AB$ dan $CD$, 
membentuk sudut - sudut berseberangan $AEF$ dan $EFD$ sama besar satu dengan 
yang lain. Saya katakan garis lurus $AB$ dan $AC$ adalah sejajar.

Karena jika tidak, jika diperpanjang, $AB$ dan $CD$ pasti akan bertemu; apakah
di arah $B$ dan $D$, atau pada arah $A$ dan $C$ [Def. 1.23]. Misalkan keduanya
sudah diperpanjang, dan misalkan misalkan keduanya bertemu pada arah 
$B$ dan $D$ pada titik $G$. Sehingga, untuk segitiga $GEF$, sudut eksterior
$AEF$ sama besar dengan sudut interior tak bersebelahan $EFG$. Hal yang tidak
mungkin [Prop. 1.16]. Dengan demikian, jika diperpanjang $AB$ dan $CD$ tidak
akan bertemua pada arah $B$ dan $D$. Hal yang sama bisa ditunjukkan tidak 
akan ada pertemuan keduanya pada arah $A$ dan $C$. Tetapi garis - garis lurus
yang tidak bertemu di arah manapun adalah sejajar [Def. 1.23]. Dengan 
demikian $AB$ dan $CD$ sejajar.

Dengan demikian, jika sebuah garis lurus memotong dua garis lurus yang lain dan
membentuk sudut yang berseberangan yang sama besar satu dengan yang lain, maka
kedua garis tersebut adalah sejajar. Hal yang perlu ditunjukkan.

\stepcounter{section}
\section*{\centering Proposisi \thesection}
\addcontentsline{toc}{section}{Proposisi \thesection}
Jika sebuah garis lurus memotong dua garis lurus yang lain dan membentuk
sudut eksterior yang sama besar dengan sudut interior yang tidak besebelahan
yang sehadap, atau membentuk jumlah dua sudut interior pada sisi 
yang sama dengan dua buah sudut siku - siku, maka kedua garis lurus
tadi sejajar satu dengan yang lain.
\begin{center}
\begin{tikzpicture}
\draw (0, 3) node [anchor=south]{A} -- 
	(6, 3) node [anchor=south]{B};
\draw (0, 1) node [anchor=north]{C} --
	(6, 1) node [anchor=north]{D};
\draw (1, 4) node [anchor=south]{E} -- (4, 0) node [anchor=north]{F};
\draw (2, 3) node [anchor=south]{G};
\draw (3.5, 1) node [anchor=south]{H};
\end{tikzpicture}
\end{center} 
Karena misalkan $EF$ memotong garis lurus $AB$ dan $CD$, membentuk sudut 
eksterior $EGB$ yang sama besar dengan sudut interior yang tidak bersebelahan
$GHD$, atau jumlah sudut - sudut interior yang sehadap $BGH$ 
dan $GHD$, sama dengan dua buah sudut siku - siku. Saya katakan $AB$ sejajar 
dengan $CD$.

Disebabkan , untuk kasus pertama, $EGB$ sama besar dengan $GHD$, tetapi $EGB$
juga sama besar dengan $AGH$ [Prop. 1.15], $AGH$ dengan demikian juga sama
besar dengan $GHD$. Dan keduanya adalah sudut yang berseberangan. Dengan
demikian $AB$ sejajar dengan $CD$ [Prop.1.27].

Lagi, karena (pada kasus kedua), jumlah besar sudut $BGH$ dan $GHD$ sama 
dengan dua buah sudut siku - siku, dan jumlah $AGH$ dan $BGH$ dengan demikian
sama dengan dua buah sudut siku - siku  [Prop. 1.13], jumlah $AGH$ dan
dan $BGH$dengan demikian sama dengan jumlah $BGH$ dan $GHD$. Misalkan $BGH$
sudah dikurangkan dari keduanya. Dengan demikian sisa $AGH$ sama denga
sisa $GHD$. Dan keduanya adalah sudut yang berseberangan, dengan demikian
$AB$ sejajar dengan $CD$ [Prop. 1.27].

Dengan demikian, jika sebuah garis lurus memotong dua garis lurus yang lain 
dan membentuk sudut eksterior yang sama besar dengan sudut interior yang tidak 
besebelahan yang sehadap, atau membentuk jumlah dua sudut interior pada sisi 
yang sama dengan dua buah sudut siku - siku, maka kedua garis lurus
tadi sejajar satu dengan yang lain. Yang merupakan hal yang perlu ditunjukkan.


\stepcounter{section}
\section*{\centering Proposisi \thesection} 
\addcontentsline{toc}{section}{Proposisi \thesection}
Sebuah ruas garis yang jatuh memotong ruas - ruas garis yang sejajar akan membuat
sudut - sudut berseberangan sama besar, sudut eksterior sama besar dengan sudut
interior yang tidak bersebelahan, dan jumlah sudut - sudut interior yang sehadap
sama dengan dua buah sudut siku - siku.

\begin{center}
\begin{tikzpicture}
\draw(-3, 4) node[anchor=south]{E} --++ (330:1.6)  node [anchor=west]{G} --++ (330:4) node[anchor=west]{H} --++(330:1.6) node[anchor=west]{F};
\draw (-4, 3) node[anchor=east]{A}(-4, 3) -- (4, 3) node [anchor=west]{B};
\draw (-4, 1) node[anchor=east]{C}(-4, 1) -- (4, 1) node [anchor=west]{D};
\end{tikzpicture} 
\end{center}

Misalkan garis lurus $EF$ jatuh memotong garis - garis sejajar $AB$ dan $CD$. 
Saya katakan bahwa $EF$ membuat sudut - sudut yang berseberangan
$AGH$ dan $GHD$ sama besar, sudut eksterior $EGB$ sama besar dengan sudut interior
yang tak bersebelahan $GHD$, dan jumlah dari sudut - sudut interior yang sehadap
$BGH$ dan $GHD$, sama dengan dua buah sudut siku - siku.

Karena jika $AGH$ tidak sama besar dengan $GHD$ maka salah satunya pasti
lebih besar. Misalkan $AGH$ adalah yang lebih besar. Misalkan $BGH$ 
ditambahkan kepada keduanya. Maka jumlah dari $AGH$ dan $BGH$ akan 
lebih besar dari jumlah dari $BGH$ dan $GHD$. Tetapi besar dari $AGH$
dan $BGD$ harus sama dengan dua buah sudut siku - siku [1.13]. Jadi
jumlah $BGH$ dan $GHD$ juga lebih kecil dari dua buah sudut siku - siku.
Tetapi garis - garis lurus yang dipanjangkan sampai tak terhingga dari
sudut - sudut interior yang lebih kecil dari dua buah sudut siku - siku 
akan bertemu [Post 5]. Jadi, $AB$ dan $CD$ jika dipanjangkan sampai tak
terhingga dari sisi yang sudut interior sehadap lebih kecil dari dua sudut
siku. Tetapi keduanya tidak bertemu dengan mengacu pada asumsi keduanya
adalah garis sejajar [Def. 1.23]. Dengan demikian $AGH$ bukan tidak sama
besar dengan $GHD$. Keduanya sama besar. Tetapi $AGH$ sama besar dengan
$EGB$ [Prop. 1.15]. Dan $EGB$ dengan demikian sama besar dengan $GHD$.
Misalkan $BGH$ sudah ditambahkan kepada keduanya. Dengan demikian jumlah
$EGB$ dan $BGH$ sama dengan jumlah $BGH$ dan $GHD$. Tetapi jumlah
$EGB$ dan $BGH$ sama dengan dua sudut siku [Prop. 1.13]. Dengan demikian
jumlah $BGH$ dan $GHD$ juga sama dengan dua buah sudut siku - siku.

Dengan demikian, sebuah ruas garis yang jatuh memotong ruas - ruas garis 
yang sejajar akan membuat sudut - sudut berseberangan sama besar, sudut 
eksterior sama besar dengan sudut interior yang tidak bersebelahan, dan jumlah 
sudut - sudut interior yang sehadap sama dengan dua buah sudut siku - siku.
Hal yang perlu ditunjukkan.

\stepcounter{section}
\section*{\centering Proposisi \thesection} 
\addcontentsline{toc}{section}{Proposisi \thesection}
Garis - garis lurus yang sejajar terhadap satu garus lurus yang sama
juga sejajar satu dengan yang lain.  
\begin{center}
\begin{tikzpicture}
\draw (0, 0) node [anchor = east] {A} --++ (0:6) node [anchor = west] {B};
\draw (0, -2) node [anchor = east] {E} --++ (0:6) node [anchor = west] {F};
\draw (0, -4) node [anchor = east] {C} --++ (0:6) node [anchor = west] {D};
\draw (4, 1)--++(250:1) node[anchor = south east]{G} 
	--++(250:2.1) node[anchor = south east]{H}
	--++(250:2.2) node[anchor = south east]{K}
	--++(250:1);
\end{tikzpicture}
\end{center}

Misalkan masing - masing garis lurus $AC$ dan $CD$ sejajar dengan $EF$. Saya
katakan $AB$ juga sejajar dengan $CD$.

Sebab misalkan garis lusur $GK$ jatuh memotong ($AB$, $CD$, dan $EF$).

Dan karena garis lurus $GK$ sudah jatuh memotong garis - garis lurus sejajar
$AB$ dan $EF$, sudut $AGK$ dengan demikian sama dengan sudut $GHF$ 
[Prop. 1.29]. Lagi, karena garis lurus $GK$ memotong garis - garis sejajar
$EF$ dan $CD$, sudut $GHF$ sama besar dengan $GKD$. [Prop. 1.29]. Tetapi $AGK$
sudah ditunjukkan juga sama dengan $GHF$. Dengan demikian $AGK$ juga
sama besar dengan $GKD$. Dan keduanya adalah sudut yang berseberangan.
Dengan demikian $AB$ sejajar dengan $CD$ [Prop. 1.27].

Dengan demikian, garis - garis lurus yang sejajar terhadap satu garus lurus 
yang sama juga sejajar satu dengan yang lain. Hal yang perlu ditunjukkan.

\stepcounter{section}
\section*{\centering Proposisi \thesection} 
\addcontentsline{toc}{section}{Proposisi \thesection}
Untuk menggambarkan sebuah ruas garis yang sejajar terhadap sebuah ruas 
garis tertentu yang melalui sebuah titik tertentu.

\begin{tikzpicture}
\draw(2, 2) node[anchor=south]{A} (2, 2) -- (-2, -2) node[anchor=north]{D};
\draw (-4, 2) node[anchor=east]{E}(-4, 2) -- (4, 2) node [anchor=west]{F};
\draw (-4, -2) node[anchor=east]{B}(-4, -2) -- (4, -2) node [anchor=west]{C};
\end{tikzpicture}

Misalkan $A$ adalah sebuah titik tertentu, dan $BC$ adalah sebuah garis tertentu. Jadi
yang dimaksudkan adalah meggambarkan satu garis yang sejajar dengan garis $BC$ 
dan melalui titik $A$.

Misalkan titik $D$ adalah titik yang berada di $BC$, and $AD$ terhubung. 
Dan misalkan (sudut) $DAE$, yang sama besar dengan sudut $ADC$, telah dibuat 
pada ruas garis $SA$ dengan titik A berada diujungnya [1.23]. Dan misalkan garis
lurus $AF$ sudah dipanjangkan dari $EA$.

Dan karena ruas garis $AD$, jatuh memotong kedua ruas garis $BC$ dan $EF$, membuat 
sudut - sudut yang berseberangan $EAD$ dan $ADC$ sama besar, $EAF$ dengan demikian
sejajar dengan $BC$ [1.27]

Dengan demikian, garis lurus $EAF$ sudah digambar sejajar dengan garis lurus $BC$
melalui titik $A$. Yang adalah hal yang perlu kita lakukan.

\stepcounter{section}
\section*{\centering Proposisi \thesection} 
\addcontentsline{toc}{section}{Proposisi \thesection}

Pada semua segitiga, jika salah satu sisi diperpanjang, maka sudut exterior 
sama dengan jumlah dua sudut interior yang tidak bersebelahan, dan jumlah 
dari tiga sudut interior sama dengan dua sudut siku - siku.

\begin{center}
\begin{tikzpicture}
\draw (0, 0) node [anchor=north]{B} --++ (60:3) node [anchor=south] {A} 
	--++ (300:3) node [anchor=north] {C} --++ (180:{2*3*cos(60)});
\draw ({2*3*cos(60)}, 0) --++ (0: 2) node[anchor=north]{D};
\draw [gray, dashed] (0, 0) --++ (60:.5) --++(300:.5) --++(180:{2*.5*cos(60)});
\draw [gray, dashed] ({2*3*cos(60)}, 0) circle(.5);
\draw [gray, dashed] ({2*3*cos(60)}, 0) circle({2*.5*cos(60)});
\draw [gray, dashed] ({2*3*cos(60) + .5 }, 0) circle(.5);
\draw [gray, dashed] ({2*3*cos(60)}, 0) --++ (60:.5) --++ (300:.5);
\draw ({2*3*cos(60)}, 0) --++ (60:3) node[anchor=south]{E}; 
\end{tikzpicture}
\end{center}

Misalkan $ABC$ adalah sebuah segitiga, dan misalkan salah satu dari sisi - 
sisinya $BC$ sudah dipanjangkan sampai ke $D$. Saya katakan bahwa sudut 
eksterior $ACD$ sama dengan jumlah dua sudut yang tidak bersebelahan $CAB$ dan
$ABC$, dan jumlah dari tiga sudut interior segitiga $ABC$, $BCA$, $CAB$, 
sama dengan dua jumlah dua buah sudut siku - siku.

Sebab misalkan $CE$ sudah digambarkan melalui titik $C$ sejajar dengan garis
lurus $AB$ [Prop. 1.31]

Dan karena $AB$ sejajar dengan $CE$, dan $AC$ sudah jatuh memotong keduanya 
sudut yang berseberangan $BAC$ dan $ACE$ sama besar satu dengan yang lain
[Prop. 1.29]. Lagi, karena $AB$ sejajar dengan $CE$, dan garis lurus $BD$
memotong keduanya, sudut eksterior $ECD$ sama dengan sudut interior yang 
tidak berisian $ABC$ [Prop. 1.29]. Tetapi $ACE$ sudah ditunjukkan sama besar
dengan $BAC$. Dengan demikian, jumlah keseluruhan sudut $ACD$ sama dengan
jumlah dua sudut interior yang tidak bersebelahan $BAC$ dan $ABC$.

Misalkan $ABC$ sudah ditambahkan pada keduanya. Dengan demikian jumlah
$ACD$ dan $ACB$ sama dengan jumlah ketiga sudut $ABC$, $BCA$ dan $CAB$. Tapi
jumlah $ACD$ dan $ABC$ sama dengan dua sudut siku - siku [Prop. 1.13].
Dengan demikian jumlah $ACB$ dan $CBA$ dan $CAB$ juga sama dengan jumlah
dua buah sudut siku - siku.

Dengan demikian, pada semua segitiga, jika salah satu sisi dipanjangkan, sudut
eksterion sama dengan jumlah sudut - sudut interior yang tidak bersisian, dan 
jumlah ketiga sudut interior sama dengan jumlah dua buah sudut siku - siku.
Hal yang perlu ditunjukkan.

\stepcounter{section}
\section*{\centering Proposisi \thesection} 
\addcontentsline{toc}{section}{Proposisi \thesection} 
Garis - garis lurus yang menghubungkan garis - garis lurus yang sama panjang dan 
sejajar pada sisi - sisi yang sama adalah juga saling sama panjang dan sejajar.
\begin{center}
\begin{tikzpicture}
\draw (0, 0) node [anchor=north]{D} --++ (0:5) node [anchor=north]{C}
	--++ (80:3) node[anchor=south]{A} --++(180:5) node [anchor=south]{B}
	--++ (260:3);
\draw ({3*cos(80)}, {3*sin(80)}) -- (5, 0);
\end{tikzpicture}
\end{center}

Misalkan $AB$ dan $CD$ adalah garis - garis yang sama panjang dan sejajar, dan
misalkan garis - garis lurus $AC$ dan $BD$ menghubungkannya pada sisi yang sama.
Saya katakan bahwa $AC$ dan $BD$ juga sama panjang dan sejajar.

Misalkan $BC$ sudah dihubungkan. Dan karena $AB$ sejajar dengan $CD$, dan $BC$ 
sudah memotong keduanya, sudut yang berseberangan $ABC$ dan $BCD$ juga sama
besar satu dengan yang lain [Prop. 1.29]. Dan karena $AB$ juga sama dengan 
$CD$, dan $BC$ adalah milik keduanya untuk keduanya, kedua garis lurus $AB$, $BC$ sama
panjang dengan kedua garis lurus $DC$, $CB$. Dan sudut $ABC$ sama dengan sudut
$BCD$. Dengan demikian sisi alas $AC$ sama panjang dengan sisi alas $BD$, dan 
segitiga $ABC$ sama dengan segitiga $DCB$, dan sudut - sudut yang selebihnya 
yang menghadap sisi - sisi yang sama sama dengan sudut - sudut selebihnya yang 
berhubungan. Dengan demikian, segitiga $ACB$ sama dengan segitiga $CDB$. Juga
karena garis lurus $BC$ yang jatuh memotong kedua garis $AC$ dan $BD$ sudah
membentuk sudut yang berseberangan ($ACB$ dan $CBD$) sama besar satu dengan 
yang lain, $AC$ dengan demikian sejajar dengan $BD$ [Prop. 1.27]. Dan $AC$ 
juga sudah ditunjukkan sama panjang dengan $BD$.

Dengan demikian garis - garis lurus yang menghubungkan gari - garis lurus 
yang sama panjang dan sejajar pada sisi - sisi yang sama adalah juga saling
 sama panjang dan sejajar.  

\stepcounter{section}
\section*{\centering Proposisi \thesection} 
\addcontentsline{toc}{section}{Proposisi \thesection} 
Pada bidang - bidang jajaran genjang, sisi - sisi dan sudut - sudut yang 
saling berhadapan sama besar satu dengan yang lain, dan sebuah diagonal 
membaginya jadi dua bagian yang sama besar.  

\begin{tikzpicture}
\draw (-3, 2) node[anchor=east]{A} -- (4, 2) node [anchor=west]{B};
\draw (-4, -2) node[anchor=east]{C}(-4, -2) -- (3, -2) node [anchor=west]{D};
\draw (-4, -2) -- (-3, 2);
\draw (3, -2) -- (4, 2);
\draw (-4, -2) -- (4, 2); 
\end{tikzpicture}

Misalkan $ABCD$ ada sebuah jajaran genjang, dan $BC$ adalah diagonalnya. 
Saya katakan bahwa untuk jajaran genjang $ABCD$, sisi sisi yang berhadapan 
dan sudut - sudut yang berhadapan sama besar satu dengan yang lain, 
dan diagonal $BC$ membaginya menjadi dua bagian yang sama besar.

Karena $AB$ sejajar dengan $CD$, dan ruas garis $BC$ jatuh memotong keduanya, 
sudut - sudut yang berseberangan $ABC$ dan $BCD$ sama besar satu sama lain 
[Prop. 1.29]. Lagi karena $AC$ sejajar dengan $BD$, dan $BC$ memotong keduanya, 
sudut - sudut yang berseberangan $ACB$ dan $CBD$ sama besar satu 
dengan yang lain [Prop, 1.29]. Dengan demikian $ABC$ dan $BCD$ adalah dua buah 
segitiga yang memiliki kedua sudut masing - masing $ABC$ dan $BCA$ sama besar 
dengan kedua sudut $BCD$ dan $CDB$, dan satu sisi sama besar, yaitu $BC$, yang
diapit sudut - sudut  tadi yang milik keduanya.
Dengan demikian keduanya akan memiliki satu sisi - sisi yang berhubungan yang saling
sama besar dan sudut sisanya sama besar dengan sudut sisa nya di segitiga yang satu
[Prop. 1.26].  Dengan demikian, sisi $AB$ sama besar dengan $CD$ , dan sisi 
$AC$ dengan $BD$. Lebih lanjut, sudut $BAC$ sama besar
dengan $CDB$. Dan karena $ABC$ sama besar dengan $BCD$, dan $CBD$ dengan $ACB$, 
keseluruhan sudut segitiga $ABD$ dengan demikian sama dengan keseluruhan  sudut 
segitiga $ACD$.

Dengan demikian, pada bidang jajaran genjang, sisi - sisi yang berhadapan dan
sudut - sudut yang berhadapan sama besar satu dengan yang lain.

Dan, saya juga katakan bahwa sebuah diagonal memotong nya menjadi dua bagian 
sama besar. Sebab karena $AB$ sama panjang $CD$ dan $BC$ milik bersama, masing - masing 
kedua garis lurus $AB$, $BC$ sama panjang dengan kedua garis lurus $DC$, $CB$. 
Dan sudut $ABC$ sama besar dengan sudut $BCD$. Dengan demikian sisi alas $AC$ 
juga sama panjang dengan $DB$, dan segitiga $ABC$ sama dengan segitiga $BCD$ 
[Prop. 1.4].

Dengan demikian, diaganoal $BC$ memotong jajaran genjang $ACDB$ menjadi dua 
bagian sama besar. Yang adalah hal yang perlu kita tunjukkan.

\stepcounter{section}
\section*{\centering Proposisi \thesection} 
\addcontentsline{toc}{section}{Proposisi \thesection} 
Jajaran - jajaran genjang yang memiliki alas yang sama dan berada diantara dua
garis - garis sejajar yang sama , sama besar satu dengan yang lain.  

\begin{center}
%\begin{tikzpicture}
\begin{tikzpicture}[scale=.75, every node/.style={scale=.8}]
\draw (-8, 2) node[anchor=south]{A}(-8, 2) -- (-5, 2) 
node [anchor=south]{D} (-5, 2) -- (2, 2) 
node [anchor=south]{E} (2, 2) -- (5, 2) node[anchor=south]{F};
\draw (-6, -2) node[anchor=north]{B}(-6, -2) 
	-- (-3, -2) node [anchor=north]{C};
\draw (-6, -2) -- (-8, 2);
\draw (-3, -2) -- (-5, 2);
\draw (-6, -2) -- (2, 2);
\draw (-3, -2) -- (5, 2);
\draw (-3.75, -0.5) node[anchor=west]{G};
\end{tikzpicture} 
\end{center} 
Misalkan $ABCD$ dan $EBCF$ adalah bidang - bidang jajaran genjang yang berada
pada  alas yang sama yaitu $BC$, dan berada diantara dua buah garis sejajar 
$AF$ dan $BC$. Saya katakan bahwa $ABCD$ sama dengan bidang jajaran genjang 
$EBCF$.

Karena $ABCD$ adalah sebuah jajaran genjang, $AD$ sama besar dengan $BC$ 
[1.34].  Dengan dengan alasan - alasan yang sama, $EF$ juga sama dengan 
$BC$.  Dengan demikian $AD$ juga sama dengan $EF$. Dan $DE$ ada pada keduanya.
Dengan demikian keseluruhan garis lurus $AE$ sama panjang dengan keseluruhan
garis lurus $DF$. Dan $AB$ juga sama panjang dengan $DC$. Dengan demikian 
kedua garis lurus $EA$, $AB$ sama panjang dengan kedua garis lurus $FD$, $DC$.
Dan sudut $FDC$ sama besar dengan sudut $EAB$, sudut eksterior sama besar 
dengan interior [Prop. 1.29](atau[Prop. 1.28]). Dengan demikian sisi alas $EB$ 
sama panjang dengan sisi alas $FC$, dan segitiga $EAB$ sama dengan segitiga 
$DFC$ [Prop. 1.4]. Misalkan $DGE$ telah dikurangi dari keduanya. Dengan demikian
trapesium selebihnya $ABCGD$ sama dengan trapesium sisanya $EGCF$. Misalkan
segitiga $GBC$ sudah ditambahkan pada keduanya. Dengan demikian, jajaran 
genjang $ABCD$ sama besar dengan jajaran gencang $EBCF$.

Dengan demikian, jajaran - jajaran genjang yang memiliki alas yang sama dan 
berada diantara dua garis - garis sejajar yang sama , sama besarnya satu 
dengan yang lain.  Yang adalah hal yang perlu ditunjukkan.

\stepcounter{section}
\section*{\centering Proposisi \thesection} 
\addcontentsline{toc}{section}{Proposisi \thesection} 
Jajaran - jajaran genjang, yang berada pada alas yang sama panjang dan berada
diantara garis - garis sejajar yang sama adalah sama besar satu dengan yang 
lain.
\begin{center}
\begin{tikzpicture}
\draw (0, 0) node [anchor = north] {B} -- (1, 4) node [anchor = south] {A} 
	--  (3, 4) node [anchor = south] {D} -- (2, 0) node [anchor = north] {C}
	-- (0, 0); 
\draw (8, 0) node [anchor = north] {F} -- (10, 0) node [anchor = north] {G} 
	-- (9, 4) node [anchor = south] {H} -- (7, 4) node [anchor = south] {E}
	-- (8, 0);
\draw (0, 0) -- (7, 4);
\draw (2, 0) -- (9, 4);
\draw (1, 4) -- (9, 4);
\draw (0, 0) -- (10, 0); 
\end{tikzpicture}
\end{center}
Misalkan $ABCD$ dan $EFGH$ adalah bidang - bidang jajaran genjang yang memiliki
sisi alas $BC$ dan $FG$ yang sama panjang, dan berada diantara dua buah garis
sejajar yang sama $AH$ dan $BG$. Saya katakan bahwa jajaran genjang $ABCD$ sama
besar dengan jajaran genjang $EFGH$.

Karena misalkan $BE$ dan $CH$ sudah dihubungkan. Dan karena $BC$ sama panjang 
dengan $FG$, tetapi $FG$ sama panjang dengan $EH$ [Prop. 1.34], $BC$ dengan 
demikian sama besar dengan $EH$. Dan keduanya juga sejajar, dan $EB$ dan $HC$
menghubungkan keduanya. Tetapi garis - garis lurus yang menghubungkan pada sisi 
yang sama garis - garis lurus yang sejajar dan sama panjang adalah juga sejajar
dan sama panjang satu dengan lain [Prop. 1.33]. [dengan demikian, $EB$ dan $HC$
juga adalah sejajar dan sama panjang]. Dengan demikian, $EBCH$ adalah jajaran 
genjang [Prop. 1.34], dan sama besar dengan $ABCD$. Karena ia memiliki sisi
alas yang  sama, $BC$, (dengan $ABCD$), dan  ada diantara garis -  garis 
sejajar yang sama (dengan $ABCD$) [Prop. 1.35]. Jadi, untuk alasan - alsan yang
sama, $EFGH$ juga sama dengan jajaran genjang $EBCH$ [Prop. 1.34]. Jadi jajaran
genjang $ABCD$ sama besar dengan $EFGH$.

Jadi, jajaran - jajaran genjang, yang berada pada alas yang sama panjang dan
 berada diantara garis - garis sejajar yang sama adalah sama besar satu 
dengan yang lain.

\stepcounter{section}
\section*{\centering Proposisi \thesection} 
\addcontentsline{toc}{section}{Proposisi \thesection} 
Segitiga - segitiga yang memiliki sisi alas yang sama dan berada diantara
garis - garis sejajar yang sama adalah sama besar satu dengan yang lain.
\begin{center}
\begin{tikzpicture}[scale=.75, every node/.style={scale=.8}]
\draw (0, 0) node[anchor=north] {B} -- (3, 0) node[anchor=north]{C};
\draw (0, 0) -- (1, 4) node [anchor = south] {A} -- (3, 0);
\draw (0, 0) -- (7, 4) node [anchor = south] {D} -- (3, 0);
\draw (0, 0) -- (-2, 4) node[anchor = south] {E};
\draw (3, 0) -- (10, 4) node [anchor = south] {F};
\draw (-2, 4) -- (10, 4);
\end{tikzpicture}
\end{center}
Misalkan $ABC$ dan $DBC$ adalah dua buah segitiga yang memiliki sisi alas yang
sama, $BC$, dan berada diantara dua garis sejajar yang sama $AD$ dan $BC$.
Saya katakan segitiga $ABC$ sama dengan segitiga $DBC$.

Misalkan $AD$ sudah dipanjangkan kedua arah $E$ dan $F$, misalkan garis lurus
$BE$ sudah ditarik melalui $B$ sejajar dengan $CA$ [Prop. 1.31], dan 
misalkan garis lurus $CF$ sudah ditarik melalui $C$ sejajar dengan $BD$ 
[Prop. 1.31]. Dengan demikian, kedua  $EBCA$ dan $DBCF$ adalah jajaran genjang, 
dan sama besar. Karena keduanya memiliki sisi alas yang sama, $BC$, dan berada
diantara garis - garis sejajar yang sama $BC$ dan $EF$ [Prop. 1.35]. Dan
segitiga $ABC$ adalah setengah dari jajaran genjang $EBCA$. Karena
diagonal $AB$ memotong jajaran genjang $EBCA$ menjadi dua bagian yang sama 
besar [Prop. 1.34]. Dan segitiga $DBC$ adalah setengah dari jajaran genjang 
$DBCF$.
Karena diagonal $DC$ memotong jajaran genjang $DBCF$ menjadi dua bagian yang 
sama besar. [Dan setengah dari satu hal adalah sama besar satu dengan 
yang lain].

Dengan demikian, segitiga - segitiga yang memiliki sisi alas yang sama dan 
berada diantara garis - garis sejajar yang sama adalah sama besar satu dengan 
yang lain. Hal yang perlu ditunjukkan.

\stepcounter{section}
\section*{\centering Proposisi \thesection} 
\addcontentsline{toc}{section}{Proposisi \thesection} 
Segitiga - segitiga yang berada pada sisi alas yang sama besar dan berada 
diantara garis - garis sejajar yang sama adalah sama besar satu dengan 
yang lain.  
\begin{center}
\begin{tikzpicture}[scale=.9, every node/.style={scale=.9}]
\draw(0, 0) node [anchor=north] {B} -- (3, 0) node [anchor=north] {C}
	-- (4, 4) node [anchor=south] {A} -- (0, 0) 
	-- (1, 4) node [anchor=south]{G} -- (4, 4);

\draw (12, 0) node [anchor = north]{F} -- (9, 0) node [anchor=north]{E} 
	-- (6, 4) node [anchor = south]{D} -- (12, 0) 
	-- (9, 4) node [anchor=south]{H} --(6, 4);

\draw (3, 0) -- (9, 0);
\draw (4, 4) -- (6, 4);
\end{tikzpicture}
\end{center}
Misalkan $ABC$ dan $DEF$ adalah segitiga - segitiga yang berada sisi alas yang
sama besar $BC$ dan $EF$, dan berada diantara garis - garis sejajar yang sama
$BF$ dan $AD$. Saya katakan segitiga $ABC$ sama besar dengan segitiga $DEF$.

Sebab misalkan $AD$ sudah dipanjangkan kedua arah $G$ dan $H$, dan misalkan 
garis lurus $BG$ sudah ditarik dari $B$ sejajar $CA$ [Prop. 1.31], dan misalkan
garis lurus $HF$ sudah ditarik dari $F$ sejajar $DE$ [Prop. 1.31]. Dengan 
demikian, $GBCA$ dan $DEFH$ masing - masing jajaran genjang. Dan $GBCA$ sama 
besar dengan $DEFH$. Karena keduanya berada pada sisi alas yang sama besar 
$BC$ dan $EF$, dan berada diantara garis - garis sejajar yang sama $BF$ dan 
$GH$ [Prop. 1.36]. Dan segitiga $ABC$ adalah setengah dari jajaran genjang 
$GBCA$. Karena diagonal $AB$ membaginya menjadi dua bagian sama bear 
[Prop. 1.34]. Dan segitiga $FED$  adalah setengah jajaran genjang $DEFH$.
Karena diagonal $DF$ memotongnya menjadi dua bagian sama besar. [Dan setengah
dari hal - hal yang sama besar adalah sama besar satu dengan yang lain]. 
Dengan demikian, segitiga $ABC$ sama besar dengan segitiga $DEF$.

Dengan demikian, segitiga - segitiga yang sama besar, dan berada memiliki 
sisi alas yang sama
dan menghadap sisi yang sama, adalah berada di antara garis - garis sejajar 
yang sama.

\stepcounter{section}
\section*{\centering Proposisi \thesection} 
\addcontentsline{toc}{section}{Proposisi \thesection} 
Segitiga - segitiga yang sama besar yang berada pada sisi alas yang sama,
dan yang berada pada sisi yang sama, juga berada diantara garis - garis sejajar
yang sama.
\begin{center}
\begin{tikzpicture}
\draw (0, 0) node [anchor = north east] {B} -- (.5, 4) node [anchor = south east]  {A}
	-- (6, 0) node [anchor = north west] {C} -- (0, 0);
\draw (.5,4) -- (5.5, 4) node [anchor = south west]{D} -- (6,0);
\draw (5.5, 4) -- (0, 0);
\draw (5.5, 4) --++ ({atan(4/5.5)}: -.75) node [anchor = north west] {C} -- (.5, 4);
\draw (5.5, 4) --++ ({atan(4/5.5)}: -.75) -- (6, 0);
\end{tikzpicture}
\end{center}
Misalkan $ABC$ dan $DBC$ adalah segitiga - segitiga yang sama besar yang berada 
pada sisi alas yang sama $BC$, dan berada pada sisi yang sama. Saya katakan 
bahwa keduanya berada diantara garis - garis sejajar yang sama.

Karena misalkan $AD$ sudah dihubungkan. Saya katakan $AD$ dan $BC$ adalah 
sejajar.

Karena jika tidak, misalkan $AE$ sudah ditarik melalu titik $A$ sejajar dengan 
garis lurus $BC$ [Prop. 1.31], dan misalkan $EC$ sudah dihubungkan. Dengan 
demikian, segitiga $ABC$ sama besar dengan segitiga $EBC$. Karena keduanya
berada pada alas yang sama $BC$, dan berada diantara garis - garis sejajar
yang sama [Prop. 1.37]. Tetapi $ABC$ sama besar dengan $DBC$. Dengan demikian
,$DBC$ sama besar dengan $EBC$, yang lebih besar sama besar dengan yang lebih
kecil. Hal yang benar - benar mustahil. Dengan demikian $AE$ tidak sejajar 
dengan  $BC$. Dengan cara yang sama bisa kita tunjukkan tidak ada garis lurus
lain selain $AD$. Dengan demikian $AD$ sejajar dengan $BC$.

Dengan demikian, segitiga- segitiga yang besar yang berada pada alas yang sama , 
dan berada pada sisi yang sama, juga adalah berada diantara garis - garis sejajar
yang sama. Hal yang perlu ditunjukkan.

\stepcounter{section}
\section*{\centering Proposisi \thesection} 
\addcontentsline{toc}{section}{Proposisi \thesection} 
Segitiga- segitiga yang sama besar, dan memiliki sisi alas yang sama besar, dan 
berada pada sisi yang sama, juga berada diantara garis - garis sejajar yang
sama.
\begin{center}
\begin{tikzpicture}
\draw(0, 0)node[anchor=north east] {B} -- (.5, 4) node [anchor=south east]{A}
	-- (5, 0) node [anchor=north]{C} -- (0, 0);
\draw(5, 0) -- (10, 0) node[anchor=north west]{E} 
	--(9.5, 4) node [anchor=south west]{D} --(5, 0);
\draw (.5, 4) --(9.5, 4);
\draw (5, 0) --++ ({atan(4/4.5)}:5.2) node [anchor = south east] {F}-- (.5, 4);
\draw (5, 0) --++ ({atan(4/4.5)}:5.2) -- (10, 0);
\end{tikzpicture}
\end{center}
Misalkan $ABC$ dan $CDE$ adalah segitiga - segitiga yang sama besar dan 
memiliki sisi alas yang sama besar masing - masing $BC$ dan $CE$, dan 
berada pada sisi yang sama (dari $BE$). Saya katakan bahwa mereka juga berada 
diantara garis - garis sejajar yang sama.

Sebab misalkan $AD$ sudah terhubung. Saya katakan bahwa $AD$ sejajar 
dengan $BE$.

Karena jika tidak, misalkan $AF$ sudah ditarik melalui $A$ sejajar 
dengan $BE$ [Prop. 1.31]. Misalkan $FE$ sudah terhubung. Dengan demikian,
segitiga $ABC$ sama besar dengan segitiga $FCE$. Karena keduanya berada
pada sisi alas yang sama besar, $BC$ dan $CE$ dan berada garis - garis
sejajar yang sama, $BE$ dan $AF$ [Prop. 1.38]. Tetapi segitiga $ABC$
sama besar dengan  segitiga $DCE$. Dengan demikian segitiga $DCE$ juga
sama besar dengan segitiga $FCE$. Yang lebih besar sama besar dengan yang
lebih kecil. Hal yang sangat tidak masuk akal. Dengan demikian $AF$ tidak
sejajar dengan $BE$. Dengan cara yang sama bisa kita tunjukkan tidak ada
garis lurus yang lain selain $AD$. Dengan demikian $AD$ sejajar dengan 
$BE$.

Dengan demikian, segitiga - segitiga yang sama besar yang berada pada
sisi alas yang sama besar, juga berada pada garis - garis sejajar yang
sama. Hal yang perlu ditunjukkan.


\stepcounter{section}
\section*{\centering Proposisi \thesection} 
\addcontentsline{toc}{section}{Proposisi \thesection} 
Jika sebuah jajaran genjang, memiliki sisi alas yang sama dengan sebuah
segitiga, dan berada di antara garis - garis sejajar yang sama, maka luas
jajaran genjang tersebut adalah dua kali luas segitiga tersebut.

\begin{center}
\begin{tikzpicture}
\draw (0, 0) node [anchor = north] {B} -- (5, 0) node [anchor = north] {C}
	-- (5.5, 4) node [anchor = south]{D} -- (0.5, 4) node [anchor = south] {A}
	-- (0, 0); 
\draw (0.5, 4) -- (5, 0) -- (9, 4) node [anchor = south] {E} -- (0, 0);
\draw (5.5, 4) -- (9, 4);
\end{tikzpicture}
\end{center}

Sebab misalkan jajaran genjang $ABCD$ memiliki sisi alas $BC$ yang sama dengan
sisi alas segitiga $EBC$, dan misalkan keduanya berada diantara garis - garis
sejajar yang sama $BC$ dan $AE$. Saya katakan luas jajaran genjang $ABCD$ 
adalah dua kali luas segitiga $BEC$.

Sebab misalkan $AC$ sudah terhubung. Sehingga segitiga $ABC$ sama besar
dengan segitiga $EBC$. Karena ia berada pada sisi alas yang sama, $BC$, dan 
berada diantara garis - garis sejajar yang sama, $BC$ dan $AE$, dengan $EBC$
[Prop. 1.37]. Tetapi, jajaran genjang $EBCD$ memiliki luas dua kali lebih
besar dari segitiga $ABC$. Karena diagonal $AC$ memotong jajaran genjang 
tersebut menjadi dua bagian yang sama luas [Prop. 1.34]. Dengan demikian 
luas jajaran genjang $ABCD$  juga dua kali luas segitiga $EBC$.

Dengan demikian, jika sebuah jajaran genjang memiliki sisa alas yang sama
dengan sebuah segitiga, dan berada diantara garis - garis sejejajar yang
sama, maka luas jajaran genjang tersebut adalah dua kali luas segitiga.
Hal yang perlu ditunjukkan.

\stepcounter{section}
\section*{\centering Proposisi \thesection} 
\addcontentsline{toc}{section}{Proposisi \thesection} 
Untuk membuat sebuah jajaran genjang yang sama besar dengan sebuah segitga
pada sebuah sudut rektilinear tertentu.
\begin{center}
\begin{tikzpicture}
\draw (0, 0) node [anchor = north] {B} -- (6, 0) node [anchor = north] {C}
	-- (0, 4) node [anchor = south] {A} -- (0, 0);
\draw (-2, 5) --++ (300:1) --++ (60:1);
\draw (-1.5, 5) node {D};
\draw (0, 4) -- (3, 0) node [anchor=north] {E} --++ (60: {4/cos(30)}) 
	node [anchor=south] {F} --++ (0:3) node [anchor=south] {G} -- (6, 0);
\draw (0, 4) --++ (0: 6);
\end{tikzpicture}
\end{center}
Misalkan $ABC$ adalah segitiga tertentu, dan $D$ adalah sebuah sudut 
rektilinear tertentu. Jadi diperlukan untuk membuat sebuah jajaran genjang
yang sama besar dengan segitiga $ABC$ pada sudut rektilinear $D$.

Misalkan $BC$ sudah dibagi menjadi dua pada $E$ [Prop. 1.10], dan misalkan
$AE$ sudah dihubungkan. Dan misalkan sudut $CEF$, sama besar dengan sudut
$D$, sudah dibuat pada titik $E$ pada garis lurus $EC$ [Prop. 1.23]. Dan
misalkan $AG$ sudah ditarik dari $A$ sejajar dengan $EC$ [Prop. 1.31], dan
misalkan $CG$ sudah digambarkan melalui $C$ sejajar dengan $EF$ [Prop. 1.31].
Dengan demikian, $FECG$ adalah jajaran genjang. Dan karena $BE$ sama panjang 
dengan $EC$, segitiga $ABE$ juga sama besar dengan segitiga $AEC$. Karena 
keduanya memiliki sisi alas yang sama panjang $BE$ dan $EC$, dan berada di 
antara garis - garis sejajar yang sama, $BC$ dan $AG$ [Prop. 1.38]. Dengan
demikian, segitiga $ABC$ dua kali lebih besar daripada segitiga $AEC$. Dan
jajaran genjang $FECG$ juga memiliki luas dua kali segitiga $AEC$. Karena dia
memiliki sisi alas yang sama dengan  segitiga $AEC$, dan berada di antara dua 
garis sejajar  dengan segitiga $AEC$ [Prop. 1.41]. Dengan demikian jajaran
genjang $FECG$ sama besar dengan segitiga $ABC$. $FECG$ juga memiliki sudut
$CEF$ yang sama besar dengan sudut $D$.

Dengan demikian, jajaran genjang $FECG$, sama besar dengan segitiga $ABC$, 
dan sudah ditarik pada sudut $CEF$, yang adalah sama besar dengan $D$. Yang 
adalah  hal yang perlu dilakukan.

\stepcounter{section}
\section*{\centering Proposisi \thesection} 
\addcontentsline{toc}{section}{Proposisi \thesection} 
Untuk setiap jajaran genjang, jajaran genjang - jajaran genjang komplementer
pada diagonalnya adalah sama besar satu dengan yang lain.

\begin{center}
\begin{tikzpicture}
\draw (0, 0) node [anchor=north] {B} -- (8, 0) node [anchor=north]{C}
	-- (9, 5) node[anchor=south] {D} -- (1, 5) node [anchor=south]{A}
	-- (0, 0);
\draw (2.5, 0) node [anchor=north] {G} -- (3.5, 5) node [anchor=south]{H};
\draw ({3.5*cos(atan(5))}, {3.5*sin(atan(5))}) node [anchor=east]{E} 
	--++(0: 8) node[anchor=west]{F};
\draw ({2.5 + 3.5*cos(atan(5))}, {3.5*sin(atan(5))}) node [anchor=south west]{K};
\draw (1, 5) -- (8, 0);
\end{tikzpicture}
\end{center}

Misalkan $ABCD$ adalah jajaran genjang, $AC$ adalah diagonalnya. Misalkan $EH$
dan $FG$ adalah jajaran genjang - jajaran genjang pada $AC$, dan $BK$ dan $KD$
adalah jajaran genjang - jajaran genjang komplementer pada $AC$. Saya katakan 
bahwa jajaran genjang komplementer $BK$ sama besar dengan  jajaran genjang 
komplementar $KD$.

Disebabkan karena $ABCD$ adalah jajaran genjang dan diagonalnya $AC$, segitiga
$ABC$ sama besar dengan segitiga $ACD$ [Prop. 1.34]. Lagi, karena $AEH$ adalah
 jajaran genjang dan diagonalnya $AK$, segitiga $AEK$ sama besar dengan 
segitiga $AHK$ [Prop. 1.34]. Jadi, dengan alasan yang sama. segitiga $KFC$ juga
sama besar dengan segitiga $KGC$. Dengan demikian, karena segitiga $AEK$ sama
besar dengan segitiga $AHK$, dan $KFC$ dengan $KGC$, jumlah segitiga $AEK$ 
ditambah $KGC$ sama besar dengan jumlah segitiga $AHK$ ditambah $KFC$. Dan
keseluruan segitiga $ABC$ sama besar dengan keseluruhan segitiga $ADC$. Dengan
demikian jajaran genjang sisa $BK$ sama besar dengan jajaran genjang  sisa 
$KD$.

Dengan demikian, untuk setiap bidang jajaran genjang, jajaran genjang - 
jajaran genjang komplementer pada diagonal adalah sama besar satu dengan yang
lain. Yang adalah hal yang perlu ditunjukkan.

\stepcounter{section}
\section*{\centering Proposisi \thesection} 
\addcontentsline{toc}{section}{Proposisi \thesection} 
Untuk menempatkan sebuah jajaran genjang yang sama besar dengan segitiga 
tertentu pada sebuah garis lurus tertentu dengan sudut rektilinear tertentu.
\begin{center}
\begin{tikzpicture}
\draw (0, 0) --++ (0:4) --++ (75:4) --(0, 0);
\draw (3.5, 1.5) node {C};
\draw (6, 3)--++(310:1) --++(240:1); %%110
\draw (6, 2.125) node {D};
\draw (2, -8) node [anchor=north]{A} --++ (80:2) node[anchor=north east]{B}
	--++ (80:4) node[anchor=south]{E} --++ (150:2) node [anchor=south]{F}
	--++ (260:4) node[anchor=east]{G} --++ (330:2) 
	--++ (330:4) node [anchor=west]{M} --++ (260:2) node [anchor=north]{L}
	--++ (150:4) --++ (150:2) node[anchor=north]{H} --++(80:2) --++ (260:2)
	-- ({2 + 2* sin(10)}, {-8 + 2*cos(10)}) --++(-30:4) 
	--++ (80:4) node[anchor=south]{K} --++ (150:4) --++(-30:4)
	-- ({2 + 2* sin(10)}, {-8 + 2*cos(10)});

\end{tikzpicture}
\end{center} 
Misalkan $AB$ adalah garis lurus tertentu, $C$ adalah segitiga tertentu, $D$
adalah sudut rektilinear tertentu. Jadi diperlukan untuk menempatkan sebuah
 jajaran genjang yang sama besar dengan segitiga $C$  pada garis lurus 
tertentu $AB$ pada sudut yang sama besar dengan sudut $D$.

Misalkan  jajaran genjang $BEFG$, sama besar dengan segitiga $C$, sudah 
dibuat pada sudut $EBG$ yang sama besar dengan sudut $D$ [Prop. 1.42]. Dan 
misalkan jajaran genjang tersebut sudah ditempatkan sehingga $BE$ rata pada
$AB$. Dan misalkan $FG$ sudah digambarkan melalui $H$, dan misalkan $AH$
sudah ditarik melalui $A$ sejajar dengan $BG$ atau $EF$ [Prop. 1.32], dan 
misalkan $HB$ sudah terhubung. Dan karena garis lurus $HF$ jatuh memotong
garis - garis sejajar $AH$ dan $EF$, jumlah dari sudut - sudut (interior)
$AHF$ dan $HFE$ dengan demikian sama dengan dua buah sudut siku - siku 
[Prop. 1.29].
Dengan demikian $BHF$ dan $GFE$ lebih kecil dari dua buah sudut siku - siku.
Dan garis - garis lurus dipanjangkan sampai terhingga dari sudut - sudut
interior yang lebih kecil dari dua buah sudut siku - siku akan bertemu 
[Post. 5]. Dengan demikian, karena dipanjangkan, $HB$ dan $FE$ akan bertemu.
Misalkan keduanya sudah dipanjangkan, dan misalkan keduanya bertemu di
$K$. Dan misalkan $KL$ sudah ditarik melalui titik $K$ sejajar dengan 
$EA$ atau $FH$ [Prop. 1.31]. Dan misalkan masing - masing $HA$ dan $GB$ 
sudah dipanjangkan sampai titik $L$ dan $M$. Dengan demikian
$HLFK$ adalah jajaran genjang, dan $HK$ diagonalnya. Dan $AG$ dan $ME$ 
adalah jajaran genjang, dan $LB$ dan $BF$ adalah jajaran genjang - jajaran
genjang komplementer pada (diagonal)$HK$. Dengan demikian, $LB$ sama besar 
dengan $BF$ 
[Prop. 1.43]. Tetapi $BF$ sama besar dengan segitiga $C$. Dengan demikian
$LB$ juga sama besar dengan $C$. Juga, karena sudut $GBE$ sama besar dengan
$ABM$ [Prop. 1.15], tetapi $GBE$ sama besar dengan $D$, $ABM$ dengan
demikian juga sama besar dengan sudut $D$.

Dengan demikian jajaran genjang $LB$, sama besar dengan segitiga $C$, sudah
ditempatkan pada garis lurus $AB$ pada sudut $ABM$, yang sama besar dengan 
$D$. Yang adalah hal yang perlu dilakukan.

\stepcounter{section}
\section*{\centering Proposisi \thesection} 
\addcontentsline{toc}{section}{Proposisi \thesection} 
Untuk membuat jajaran genjang yang sama besar dengan bidang rektilinear
tertentu dalam sudut rektilinear tertentu.
\begin{center}
\begin{tikzpicture} 

%\draw (0, -7) node [anchor=east]{A} --++ (-30:{5*sqrt(2)}) node [anchor=north] {B}
%	--++ ({-30+(180-45)}:5) node [anchor=south]{C}
%	--++ ({(-30+(180-45))+(180-90)}:5);

\draw (0, 9) node [anchor=east]{A} --++ (-30:{2*5*cos(75)}) node[anchor=north]{B}
	--++ ({-30 + (180-75)}: 5) node [anchor=south]{D}
	--++ ({-30 + (180-75) + (180 - 30)}:5);

\draw (0, 9) --++ (-30:{2*5*cos(75)}) 
	--++ ({-30 + (180-75) - 45}: {2.5/cos(45)}) node[anchor=west]{C}
	--++ ({-30 + (180-75) - 45 + 90}: {2.5/cos(45)});

\draw (8, 9) --++ (215:1) node [anchor=east]{E}--++ ({215+(180-60)}:1)
%%	--++ ({215 + (180-60) + (180-60)} : 1)
;

\draw (0, 0) node[anchor=north]{K} --++(0:{5*cos(75)}) node[anchor=north]{H}
	--++(60:5) node[anchor=south]{G} --++(180:{5*cos(75)}) node [anchor=south]{F}
	--++({180+(180-120)}:5);

\draw [gray, dashed]({5*cos(75)+ 5*cos(60)} , {5*sin(60)}) --++ (60:{(2.5/cos(45))*sin(45)}) 
	--++(180:2.5) --++ ({180+(180-120)}: 2.5)
	--++(0:{(2.5/cos(45))*sin(45)});

\draw [gray, dashed]({5*cos(75)+ 5*cos(60)} , {5*sin(60)}) --++ (60:{(2.5/cos(45))*sin(45)}) 
	--++(180:2.5) --++ ({180+(180-120)}: 2.5)
	--++ ({180+(180-120)}: 5) --++ (0:{2.5-5*cos(75)});

\draw [gray, dashed]({5*cos(75)+ 5*cos(60)} , {5*sin(60)}) --++ (60:{(2.5/cos(45))*sin(45)}) 
	--++(180:2.5) --++ ({180+(180-120)}: 2.5)
	--++ ({180+(180-120)}: 5) 
	--++ ({atan((5*sin(60))/(2.5 + 5*cos(60)))}:
		 {(5*sin(60) + 2.5*sin(60))/sin(atan((5*sin(60))/(2.5 + 5*cos(60))))})% (41:xxxx/sin(41))
	--++ ({180+60}:
		 {((5*sin(60) + 2.5*sin(60))/sin(atan((5*sin(60))/(2.5 + 5*cos(60)))))*sin(41)/sin(60)})%(240:(xxxx/sin(41)))*sin(41)/sin(60)
	--++ (180:
		 {((5*sin(60) + 2.5*sin(60))/sin(atan((5*sin(60))/(2.5 + 5*cos(60)))))*cos(41)
		 -
		 (((5*sin(60) + 2.5*sin(60))/sin(atan((5*sin(60))/(2.5 + 5*cos(60)))))*sin(41)/sin(60))*cos(60)%(240:(xxxx/sin(41)))*sin(41)/sin(60) 
		- 2.5
		 }
		 ); 
%41, 19, 120, 1, 41

\draw [gray, dashed]({5*cos(75)+ 5*cos(60)} , {5*sin(60)}) --++ (60:{(2.5/cos(45))*sin(45)}) 
	--++(180:2.5) --++ ({180+(180-120)}: 2.5)
	--++ ({180+(180-120)}: 5) 
	--++ ({atan((5*sin(60))/(2.5 + 5*cos(60)))}:
		 {(5*sin(60) + 2.5*sin(60))/sin(atan((5*sin(60))/(2.5 + 5*cos(60))))})
	--++ (180:1.2);

\draw ({5*cos(75)+ 5*cos(60)} , {5*sin(60)}) 
	--++ (0:
		 {((5*sin(60) + 2.5*sin(60))/sin(atan((5*sin(60))/(2.5 + 5*cos(60)))))*cos(41) 
		 -
		 (((5*sin(60) + 2.5*sin(60))/sin(atan((5*sin(60))/(2.5 + 5*cos(60)))))*sin(41)/sin(60))*cos(60)%(240:(xxxx/sin(41)))*sin(41)/sin(60) 
		- 2.5
		 }) node [anchor=west]{L}
	--++ (180+60:5) node [anchor=west]{M}
	--++ (180:
		 {((5*sin(60) + 2.5*sin(60))/sin(atan((5*sin(60))/(2.5 + 5*cos(60)))))*cos(41) 
		 -
		 (((5*sin(60) + 2.5*sin(60))/sin(atan((5*sin(60))/(2.5 + 5*cos(60)))))*sin(41)/sin(60))*cos(60)%(240:(xxxx/sin(41)))*sin(41)/sin(60) 
		- 2.5
		 }); 

	
\end{tikzpicture}
\end{center}
Misalkan $ABCD$ adalah bidang rektilinear tertentu, dan $E$ adalah sudut
rektilinear tertentu. Jadi perlu untuk membuat sebuah jajaran genjang yang
sama besar dengan bidang rektilinear $ABCD$ dengan sudut $E$.

Misalkan $DB$ sudah dihubungkan, dan misalkan jajaran genjang $FH$ sama 
besar dengan segitiga $ABD$, sudah dibentuk dengan sudut $HKF$ yang sama
besar dengan $E$ [Prop. 1.42]. Dan misalkan jajaran genjang $GM$, sama besar
dengan segitiga $DBC$, sudah diletakkan pada garis lurus $GH$ pada sudut
$GHM$, yang sama besar dengan $E$ [Prop. 1.44]. Dan karena sudut $E$ sama 
besar dengan sudut - sudut $HKF$ dan $GHM$, sudut $HKF$ dengan demikian
juga sama besar dengan sudut $GHM$. Misalkan $KHG$ sudah ditambahkan pada
keduanya. Dengan demikian jumlah dari $FKH$ dan $KHG$ sam dengan jumlah
$KHG$ dan $GHM$. Tetapi jumlah dari $FKH$ dan $KHG$ sama besar dengan dua
buah sudut siku - siku [Prop. 1.29]. Dengan demikian, jumlah $KHG$ dan $GHM$
juga sama besar dengan dua buah sudut siku - siku. Dengan demikian kedua 
garis lurus, $KH$ dan $HM$, tidak berada pada sisi yang sama, membentuk
sudut bersebelahan dengan garis lurus $GH$ pada arah $H$ pada nya, yang jumlah
nya juga sama besar dengan dua buah sudut siku - siku. Dengan demikian
$KH$ lurus terhadap $HM$ [Prop. 1.14]. Dan sebab garis lurus $HG$ jatuh
memotong garis - garis sejajar  $KM$ dan $FG$, sudut berseberangan $MHG$ dan
$HGF$ sama besar satu dengan yang lain [Prop. 1.29]. Misalkan $HGL$ sudah
ditambahkan pada keduanya. Dengan demikian jumlah $MHG$ dan $HGL$ sama besar
dengan jumlah $HGF$ dan $HGL$. Tetapi, jumlah dari $MHG$ dan $HGL$ sama
besar dengan dua buah sudut siku - siku [Prop. 1.29]. Dengan demikian
$HGF$ dan $HGL$ juga sama besar dengan dua buah sudut siku - siku. Dengan
demikian $FG$ lurus terhadap $GL$ [Prop. 1.14]. Dan karena $FK$ sama besar dan 
sejajar terhadap $HG$ [Prop. 1.34], tetapi juga $HG$ terhadap $ML$ 
[Prop. 1.34], $KF$ dengan demikian juga sama besar dan sejajar dengan
$ML$ [Prop. 1.30]. Dan garis lurus $KM$ dan $FL$ menghubungkan keduanya.
Dengan demikian, $KM$ dan $FL$ sama besar dan sejajar [Prop. 1.33]. 
Dengan demikian, $KFLM$ adalah jajaran genjang. Dan karena segitiga 
$ABD$ sama besar dengan jajaran genjang $FH$, dan $DBC$ terhadap
$GM$, keseluruhan bidang rektilinear $ABCD$ dengan demikian sama
besar dengan keseluruhan jajaran genjang $KFLM$.

Dengan demikian, jajaran genjang $KFLM$, sama besar dengan bidang
rektilinear $ABCD$, sudah dibuat pada sudut $FKM$, yang sama besar
dengan sudut $E$. Yang adalah hal yang perlu untuk dilakukan.

\stepcounter{section}
\section*{\centering Proposisi \thesection} 
\addcontentsline{toc}{section}{Proposisi \thesection} 
Untuk menggambarkan bujur sangkar pada satu garis lurus tertentu.
\begin{center}
\begin{tikzpicture}
\draw(0, 0) node [anchor=north west] {A} --++ (0:3) node[anchor=north]{B};
\draw(0, 0) --++ (90:6) node[anchor=south]{C};
\draw(0, 3) node [anchor=south west]{D} --++ (0:3) node [anchor=south]{C}
	--++(270:3);
\end{tikzpicture}
\end{center}
Misalkan $AB$ adalah garis lurus tertentu. Jadi diperlukan untuk menggambarkan
bujur sangkan pada pada garis lurus $AB$.

Misalkan $AC$ sudah ditarik dengan membentuk sudut siku - siku  pada garis 
lurus $AB$ pada titik $A$ [Prop. 1.11], dan misalkan $AD$ sudah dibuat sama 
panjang dengan $AB$ [Prop. 1.3]. Dan misalkan $DE$ sudah ditarik melalui 
titik $D$ sejajar dengan $AB$ [Prop. 1.31], dan misalkan $BE$ sudah ditarik 
melalui titik $B$ sejajar dengan $AD$ [Prop. 1.31]. Dengan demikian, $ADEB$
adalah jajaran genjang. Dengan demikian, $AB$ sama panjang dengan $DE$, dan 
$AD$ terhadap $BE$ [Prop. 1.34]. Tetapi, $AB$ sama panjang dengan $DE$.
Dengan demikian, keempat sisi $BA$, $AD$, $DE$ dan $EB$ sama panjang satu 
dengan lain. Dengan demikianm jajaran genjang $ADEB$ sama sisi. Sehingga saya
katakan juga bahwa jajaran genjang ini bersudut siku - siku. Disebabkan karena
garis lurus $AD$ memotong garis - garis sejajar $AB$ dan $DE$, jumlah 
dari sudut - sudut $BAD$ dan $ADE$ sama dengan dua buah sudut siku - siku.
Tetapi $BAD$ adalah sudut siku - siku. Dengan demikian, $ADE$ juga bersudut
siku - siku. Dan untuk bidang - bidang jajaran genjang, sudut - sudut dan 
sisi - sisi yang berhadapan sama besar satu dengan yang lain [Prop. 1.34].
Dengan demikian, masing = masing sudut yang berhadapan $ABE$ dan $BED$ 
adalah juga sudut siku - siku. Dengan demikian, $ADEB$ bersudut siku - siku.
Dan juga sudah ditunjukkan bahwa bersisi sama panjang.

Dengan demikian, $ADEB$ adalah bujur sangkar [Def. 1.22]. Dan digambarkan
pada garis lurus $AB$.

\stepcounter{section}
\section*{\centering Proposisi \thesection} 
\addcontentsline{toc}{section}{Proposisi \thesection} 
Pada segitiga siku - siku, besar bujur sangkar pada sisi yang menghadap sudut 
siku - siku sama besar dengan jumlah besar bujur jangkar - bujur sangkar pada 
sisi - sisi yang mengapit sudut siku - siku.
\begin{center}
\begin{tikzpicture}
\draw(0, 0) node [anchor=north east] {B} --++ ({atan(5/3)}:3) node[anchor=south] {A}
	--++ ({atan(5/3) - 90}: 5) node [anchor=north west]{C} --++(180:{sqrt(5*5 + 3*3)});
\draw(0, 0) --++(270: {sqrt(5*5 + 3*3)}) node [anchor=north]{D} 
	--++ (0:{sqrt(5*5 + 3*3)}) node[anchor=north]{E}
	--++ (90:{sqrt(5*5 + 3*3)});
\draw (0, 0) --++({atan(5/3) + 90} : 3) node[anchor=east]{F}
	--++ ({atan(5/3)}:3) node[anchor=south east]{G}
	--++ ({atan(5/3)+270}:3)
	--++ ({atan(5/3)}:5) node [anchor = south]{H}
	--++ ({atan(5/3) + 270}:5) node[anchor=west]{K}
	--++ ({atan(5/3) + 180}:5);
\draw ({3*cos(atan(5/3))}, {3*sin(atan(5/3)) }) 
	--++ (270: {3*sin(atan(5/3)) + sqrt(5*5 + 3*3)}) node [anchor=north]{L}
	--++ (180: {3*cos(atan(5/3))})
	-- ({3*cos(atan(5/3))}, {3*sin(atan(5/3))}) 
	%--++ (0:{((sqrt(5*5 + 3*3) - 3*cos(atan(5/3))))});
	--++ ({atan((sqrt(5*5 + 3*3) - 3*cos(atan(5/3)))/(3*sin(atan(5/3)) + sqrt(5*5 + 3*3))) - 90} :
	{sqrt(pow((sqrt(5*5 + 3*3) - 3*cos(atan(5/3))), 2.0) + pow((3*sin(atan(5/3)) + sqrt(5*5 + 3*3)), 2))});
\draw ({3*cos(atan(5/3) + 90)}, {3*sin(atan(5/3) + 90)}) 
	-- ({sqrt(pow(5.0, 2.0) + pow(3.0, 2.0))}, 0) 
	--++ ({atan(5/3)}:5) -- (0, 0);
\end{tikzpicture}
\end{center}
Misalkan $ABC$ adalah segitiga siku - siku dengan sudut $BAC$ adalah siku - 
siku.  Saya katakan bahwa besar bujur sangkar pada $BC$ sama dengan jumlah 
dari besar bujur sangkar pada $BA$ dan besar bujur sangkar pada $AC$.

Sebab misalkan, bujur sangkar $BDEC$ sudah digambarkan pada $BC$, dan bujur 
sangkar $GB$ pada $AB$ dan $HC$ pada $AC$ [Prop. 1.46]. Dan misalkan $AL$ 
sudah ditarik melalui titik $A$ sejajar dengan salah satu $BD$ atau $CE$ 
[Prop. 1.31]. Dan misalkan $AD$ dan $FC$ sudah dihubungkan. Dan karena masing
- masing sudut $BAC$ dan $BAG$ adalah sudut siku - siku, maka kedua garis 
lurus $AC$ dan $BG$, tidak berada pada sisi yang sama, membentuk sudut - sudut 
berdampingan dengan garis lurus $BA$ pada titik $A$, yang jumlahnya
adalah dua buah sudut siku - siku. Dengan demikian, $CA$ lurus pada $AG$ 
[Prop. 1.14]. Jadi, untuk alasan - alasan yang sama, $BA$ juga lurus pada $AH$.
Dan sebab sudut $DBC$ sama besar dengan $FBA$, karena keduanya adalah sudut
siku - siku, misalkan $ABC$ sudah ditambahkan pada keduanya. Dengan demikian, 
keseluruhan $DBA$ sama besar dengan keseluruhan sudut $FBC$. Dan karena $DB$
sama besar dengan $BC$, dan $FB$ terhadap $BA$, masing - masing kedua garis 
lurus $DB$, $BA$ sama panjang dengan kedua garis lurus $CB$, $BF$.Dan sudut
$DBA$ sama besar dengan sudut $FBC$, Dengan demikian, sisi alas $AD$ sama 
besar dengan sisi alas $FC$, dan segitiga $ABD$ sama besar dengan segitita
$FBC$ [Prop. 1.4]. Dan jajaran genjang $BL$ dua kali lebih besar dari segitiga
$ABD$. Karena keduanya memiliki sisi alas yang sama, $BD$, dan berada diantara
dua buah garis - garis sejajar yang sama $BD$ dan $AL$ [Prop. 1.41]. Dan
besar bujur sangkar $GB$ adalah dua kali besar segitiga $FBC$. Karena 
lagi keduanya memiliki alas yang sama, $FB$, dan berada diantara garis - garis
sejajar yang sama, $FB$ dan $GC$ [Prop. 1.41]. Dan dua hal yang masing - masing 
besar nya dua kali hal lain yang sama adalah sama besar satu dengan yang lain.
Dengan demikian besar jajaran genjang $BL$ juga sama besar dengan besar 
bujur sangkar $GB$. Juga, dengan cara yang sama, $AE$ dan $BK$ dihubungkan, 
jajaran genjang $CL$ dapat ditunjukkan sama besar dengan bujur sangkar
$HC$. Dengan demikian, keselurusan bujur sangkar $BDEC$ sama besar dengan
jumlah dua buah bujur sangkar $GB$ dan $HC$. Dan bujur sangkar $BDEC$ digambar
kan pada $BC$, dan masing - masing bujur sangkar $GB$ dan dan $HC$ pada
$BA$ dan $AC$. Dengan demikian, besar bujur sangkar pada sisi $BC$ sama besar 
dengan jumlah besar bujur sangkar pada sisi $BA$ dan besar bujur sangkar pada
sisi $AC$.

Dengan demikian, pada segitiga siku - siku, besar bujur sangkar pada sisi yang
menghadap sudut siku - siku sama besar dengan jumlah dari besar bujur sangkar
- bujur sangkar yang mengapit sudut siku - siku. Yang adalah hal yang perlu
kita tunjukkan.  

\stepcounter{section}
\section*{\centering Proposisi \thesection} 
\addcontentsline{toc}{section}{Proposisi \thesection} 

Jika besar bujur sangkar pada satu sisi sebuah segitiga sama besar dengan 
jumlah besar - besar bujur sangkar pada dua sisi selebihnya segitiga maka 
sudut yang diapit oleh dua sisi - sisi selebihnya tersebut adalah sudut
siku - siku.

\begin{center}
\begin{tikzpicture}
\draw(0, 0) node [anchor = north] {A} --++ ({(atan(-3/7))}:3) node [anchor=north] {B};
\draw(0, 0) --++({90 + atan(-3/7)}: 7) node [anchor=south] {C} --++ (270 : {sqrt(pow(3,2) + pow(7, 2)});
\draw(0, 0) --++ ({180 + atan(-3/7)}:3) node [anchor=north]{D} --++ 
	({atan(7/3) + atan(-3/7)} : {sqrt(3*3 + 7*7)});%%cotan = 1/tan, and cotan is x/y while tan is y/x
\end{tikzpicture}
\end{center}

Sebab misalkan besar bujur sangkar pada satu sisi, $BC$, dari segitiga
$ABC$ sama besar dengan jumlah besar bujur sangkar - bujur sangkar pada 
sisi - sisi $BA$ dan $AC$. Saya katakan bahwa $BAC$ adalah sudut siku - siku.

Sebab misalkan $AD$ sudah digambarkan dari titik $A$ dengan sudut siku
siku terhadap garis lurus $AC$ [Prop. 1.11], dan misalkan $AD$ sudah dibuat
sama panjang dengan $BA$ [Prop. 1.3], dan misalkan $DC$ sudah dihubungkan.
Dikarenakan $DA$ sama panjang dengan $AB$, besar bujur sangkar pada $DA$
dengan demikian sama dengan besar bujur sangkar pada $AB$. Misalkan bujur
sangkar $AC$ sudah ditambahkan pade keduanya. Dengan demikian jumlah 
besar bujur sangkar $DA$ dan bujur sangkar $AC$ sama besar dengan jumlah
besar bujur sangkar $BA$ dan $AC$. Tetapi, besar bujur sangkar pada $DC$
sama besar dengan jumlah besar bujur sangkar $DA$ dan $AC$. Karena $DAC$
adalah sudut siku - siku [Prop. 1.47]. Tetapi, besar bujur sangkar pada
$BC$ sama besar dengan besar bujur sangkar - bujur sangkar pada $BA$ dan
$AC$. Sebab itu sudah diasumsikan sebelumnya. Dengan demikian, bujur 
sangkar pada $DC$ sama besar dengan besar bujur sangkar pada $BC$. 
Sehingga sisi $DC$ sama panjang dengan sisi $BC$. Dan karena $DA$ sama
panjang dengan $AB$, dan $AC$ sama pada keduanya, kedua garis lurus 
$DA$, $AC$ sama panjang dengan kedua garis lurus $BA$, $AC$. Dan sisi 
alas $DC$ sama panjang dengan sisi alas $BC$. Dengan demikian, sudut
$DAC$ sama besar dengan sudut $BAC$ [Prop. 1.8]. Tetapi $DAC$ adalah 
sudut siku - siku. Dengan demikian $BAC$ juga adalah sudut siku - siku.

Dengan demikian, jika besar bujur sangkar pada satu sisi segitiga
sama besar dengan jumlah besar bujur sangkar - bujur sangkar pada
sisi - sisi selebihnya, maka besar sudut yang diapit oleh sisi - sisi 
selebihnya adalah sama dengan sudut siku - siku. Yang adalah hal yang 
perlu kita tunjukkan.


\stepcounter{chapter}
\chapter*{\centering \chaptername{ \thechapter} 
\addcontentsline{toc}{chapter}{Buku \thechapter}
Dasar - Dasar Aljabar Geometri}

\begin{center}
\addcontentsline{toc}{section}{Definisi - definisi}
\textbf{Definisi - definisi}
\end{center} 
\begin{enumerate}
\item Semua jajaran genjang rektangular (persegi panjang) dibentuk oleh
dua garis lurus garis lurus yang mengapit sudut siku - siku.
\begin{center}
\begin{tikzpicture}
\draw(0, 0) node[anchor=east]{A} --++ (0: 4) node [anchor=west]{B}
	--++(270:5) node[anchor=west]{C} --++(180:4) node [anchor=east]{D}
	--++(90:5);
\draw(4, -2) node[anchor=west]{$AB$ dan $BC$ adalah garis tersebut};
\draw(4, -3) node[anchor=west]{$ABC$ adalah sudut tersebut};
\end{tikzpicture}
\end{center}

\item Dan pada semua bidang jajaran genjang, misalkan diambil satu buah 
jajaran
genjang internal yang diagonalnya bagian dari diagonal eksternal bersama 
dengan dua komplemen nya , disebut dengan gnomon.  

\begin{center}
\begin{tikzpicture}
\draw (0, 0) node [anchor=north] {H} --++ (0: 3) node [anchor=north] {N} 
	--++ (0:3) node [anchor=north] {G} --++ (110:4) node [anchor=west] {O}
	--++ (110:4) node [anchor=south] {F} --++ (180:3) node [anchor=south] {K};
\draw [gray, dashed] ({3 + 8*cos(110)}, {8*sin(110)}) --++ (180:3) node [anchor=south]{E}
	--++ ({110+180} : 4) ;
%	--++ (180:3) node [anchor=south] {E} --++ ({110+180}:4) node [anchor=east] {M}
%	--++ ({110+180}:4);
\draw (0, 0) --++ (0: 3) --++ (110: 4) node [anchor=south west]{L} --++ (110:1) node[anchor=west]{P} --++(110:3);
\draw (0, 0) --++ (110:4) node [anchor=east] {M} --++(0:2) node [anchor=north]{R}--++(0:4);
\draw (0, 0) --++ (0:6) --++ ({180 + (atan(8*sin(110)/(8*cos(110) - 6)))}: {(sqrt(pow(8*sin(110),2) + pow(-8*cos(110)+6,2))) * .42})
	node [anchor=west] {Q}
	--++ ({180 + (atan(8*sin(110)/(8*cos(110) - 6)))}: {(sqrt(pow(8*sin(110),2) + pow(-8*cos(110)+6,2))) * .08});
\draw [gray, dashed] ({3 + 4*(cos(110))}, {4*sin(110})  
	--++ ({180 + (atan(8*sin(110)/(8*cos(110) - 6)))}: {(sqrt(pow(8*sin(110),2) + pow(-8*cos(110)+6,2))) * .5});
\draw [gray, <->] ({3 + 4.5*(cos(110))}, {4.5*sin(110})  arc (110:-180:.5); 
\draw (6, 7) node  {$P$, $Q$, $R$ adalah $gnomon$};
\end{tikzpicture}
\end{center}

\end{enumerate} 

\stepcounter{section}
\section*{\centering Proposisi \thesection} 
\addcontentsline{toc}{section}{Proposisi \thesection} 
Jika ada dua buah garis lurus, dan salah satunya dipotong - potong menjadi 
beberapa potongan, maka besar persegi panjang yang dibentuk oleh kedua garis lurus
 sama besar dengan jumlah dari besar persegi panjang - persegi panjang yang dibentuk
 garis yang tidak dipotong dengan potongan - potongan garis yang dipotong.


\begin{center}
\begin{tikzpicture}
\draw (2, 6) node [anchor=east]{A} --++ (0:3);
\draw (0, 4) node [anchor=south]{B} --++ (0:3) node [anchor=south]{D} 
	--++ (270:3) --++ (90:3)
	--++ (0: 1) node [anchor=south]{E} 
	--++ (270:3) --++ (90:3)
	--++ (0:1.5) node [anchor=south]{C}
	--++ (270:3) node [anchor=north] {H} --++ (180:1.5) node [anchor=north]{L}
	--++ (180:1) node [anchor=north] {K} --++ (180:3) node [anchor=east]{G}
	--++ (270:.5) node [anchor=north] {F} --++ (90:3.5);
\end{tikzpicture}
\end{center}

Misalkan $A$ dan $BC$ adalah dua buah garis lurus, dan misalkan $BC$ dipotong,
pada sembarang titik $D$ dan $E$. Saya katakan bahwa besar persegi panjang yang 
dibentuk $A$ dan $BC$ sama dengan besar persegi panjang yang dibentuk $A$ 
dan $BD$, ditambah $A$ dan $DE$ dan terakhir ditambah $A$ dan $EC$.

Sebab misalkan $BF$ sudah ditarik di titik $B$ siku - siku dengan $BC$ 
[Prop. 1.11], dan misalkan $BG$ sudah dibuat sama panjang dengan $A$ 
[Prop. 1.3], dan misalkan $GH$ sudah ditarik melalui titik $G$, sejajar dengan
$BC$ [Prop. 1.31], dan misalkan $DK$, $EL$ dan $CH$ sudah ditarik masing 
- masing melalui titik $D$, $E$ dan $C$, sejajar terhadap $BG$ [Prop. 1.31].  

Jadi persegi panjang $BH$ sama besar dengan persegi panjang $BK$, $DL$, dan 
$EH$. Dan $BH$ adalah persegi panjang yang dibentuk $A$ dan $BC$. Karena ia
dibentuk oleh $BG$ dan $BC$, dan $BG$ sama panjang dengan $A$. Dan $BK$ adalah
adalah persegi panjang yang dibentuk oleh $A$ dan $BD$. Karena dia dibentuk oleh
$GB$ dan $BD$, dan $BG$ sama panjang dengan $A$. Dan $DL$ adalah persegi panjang 
yang dibentuk oleh $A$ dan $DE$. Karena $DK$, seperti $BG$ [Prop. 1.34]. sama 
panjang dengan $A$. Dan dengan cara yang sama,  $EH$ adalah persegi panjang
yang dibentuk oleh $A$ dan $EC$. Dengan demikian, persegi panjang yang
dibentuk oleh $A$ dan $BC$ sama besar dengan persegi panjang yang dibentuk
$A$ dan $BD$ ditambah $A$ dan $DE$ dan akhirnya ditambah $A$ dan $EC$. 

Dengan demikian, jika dua buah garis lurus, satu nya dipotong menjadi beberapa
potongan, maka besar persegi panjang yang dibentuk oleh kedua garis itu, sama
besar dengan jumlah besar dari persegi panjang - persegi panjang yang dibentuk 
garis yang tidak dipotong dengan potongan - potongan garis yang dipotong.
Yang adalah hal yang perlu ditunjukkuan.\footnote {Proposisi ini adalah
bentuk geometri dari identitas aljabar $a(b + c + d +...) = ab +ac + ad + ...$}

\stepcounter{section}
\section*{\centering Proposisi \thesection} 
\addcontentsline{toc}{section}{Proposisi \thesection} 

Jika sebuah garis lurus dipotong disembarang titik, maka jumlah besar persegi
 panjang yang dibentuk oleh garis lurus utuh dengan potongan 
- potongannya, sama besar dengan dengan besar bujur sangkar yang dibentuk pada garis 
lurus tersebut secara utuh. 
\begin{center}
\begin{tikzpicture}
\draw (0, 0) node [anchor=north]{D} --++ (90:4) node [anchor=south]{A} 
	--++ (0:4) node [anchor=south]{B} --++ (270:4) node [anchor=north]{E}
	--++ (180:4);
\draw (1, 0) node [anchor=north] {F} --++ (90:4) node [anchor=south] {C};
\end{tikzpicture}
\end{center}
Sebab misalkan garis lurus $AB$ sudah dipotong disembarang titik $C$. Saya
katakan bahwa besar persegi panjang yang dibentuk $AB$ dan $BC$ ditambah besar persegi
panjang yang dibentuk $AB$ dan $AC$, sama dengan besar bujur sangkar yang dibentuk 
pada $AB$.

Sebab misalkan bujur sangkar $ADEB$ sudah digambarkan pada $AB$ [Prop. 1.46],
dan misalkan $CF$ sudah digambarkan melalui $C$ sejajar sejajar dengan 
$AD$ atau $BE$ [Prop. 1.31].

Sehingga bujur sangkar $AE$ sama besar dengan persegi panjang - persegi panjang 
$AF$ dan $CE$. Dan $AE$ adalah bujur sangkar pada $AB$. Dan $AF$ adalah persegi
panjang yang dibentuk garis lurus $BA$ dan $AC$. Karena dia dibentuk oleh $DA$
dan $AC$, dan $DA$ sama panjang dengan $AB$. Dan $CE$ adalah persegi panjang 
yang dibentuk oleh $AB$ dan $BC$. Karena $BE$ sama panjang dengan $AB$. Dengan
demikian, besar persegi panjang yang dibentuk oleh $BA$ dan $AC$, ditambah dengan besar
persegi panjang yang dibentuk $AB$ dan $BC$, sama besar dengan besar bujur sangkar pada
$AB$.

Dengan demikian, jika sebuah garis lurus dipotong secara acak, maka jumlah besar
persegi panjang yang dibentuk oleh keseluruhan garis lurus dengan potongan -
porongannya sama besar dengan besar bujur sangkar pada garis tersebut secara 
keseluruhan. Yang adalah hal yang perlu ditunjukkan.\footnote{Proposisi
ini adalah bentuk geometri dari identitas aljabar $ab + ac = a^2 jika
a  = b + c$}
 

\stepcounter{section}
\section*{\centering Proposisi \thesection} 
\addcontentsline{toc}{section}{Proposisi \thesection} 
Jika sebuah garis lurus dipotong secara acak, maka besar persegi panjang yang dibentuk
oleh garis lurus secara utuh dengan salah satu potongannya, adalah sama dengan besar
dengan jumlah besar persegi panjang yang dibentuk oleh kedua potongan ditambah
dengan besar bujur sangkar dari potongan yang pertama disebut (yaitu yang membentuk 
persegi panjang dengan garis utuh).
\begin{center}
\begin{tikzpicture}
\draw (0, 5) node [anchor=south] {A} --++ (0:3) node [anchor=south] {C} 
	--++ (0: 5) node[anchor=south] {B} --++ (270:5) node[anchor=north] {E}
	--++ (180:5) node[anchor=north] {D} --++ (180:3) node[anchor=north]{F}
	--++ (90:5);
\draw (3, 5) --++(270:5);
\end{tikzpicture}
\end{center}
Misalkan garis lurus $AB$ sudah dipotong secara acak pada titik $C$. Saya katakan 
persegi panjang yang dibentuk $AB$ dan $BC$ sama besar dengan persegi panjang 
yang dibentuk $AC$ dan $BC$, ditambah bujur sangkar pada $BC$.

Sebab misalkan bujur sangkar $CDEB$ sudah digambarkan pada $CB$ [Prop. 1.46], 
dan misalkan $ED$ sudah ditarik melalui $F$, dan misalkan $AF$ sudah 
ditarik dari $A$, sejajar dengan $CD$ atu $BE$ [Prop.1.31]. Sehingga besar
persegi panjang $AE$ sama besar dengan persegi panjang $AD$ dan besar 
bujur sangkar $CE$. Dan $AE$ adalah persegi panjang yang dibentuk oleh
$AB$ dan $BC$, Karena dia dibentuk oleh $AB$ dan $BE$,dan $BE$ sama panjang
dengan $BC$. Dan $AD$ adalah persegi panjang yang oleh $AC$ dan $CB$. Karena
$DC$ sama panjang dengan $CB$. Dan $DB$ adalah bujur sangkar pada $CB$. Dengan
demikian, besar persegi panjang yang dibentuk oleh $AB$ dan $BC$ sama besar 
dengan besar persegi panjang yang dibentuk $AC$ dan $CB$ ditambah besar bujur
sangkar pada $BC$.

Dengan demikian, jika sebuah garis lurus dipotong pada satu titik secara acak, 
maka besar persegi panjang yang dibentuk oleh keseluruhan garis dengan salah satu
potongannya sama besar dengan jumlah besar persegi panjang yang dibentuk oleh 
kedua potongan ditambah dengan besar bujur sangkar dari potongan yang 
disebutkan sebelumnya (yakni yang membentuk persegi panjang dengan garis utuh). 
Yang adalah hal yang perlu ditunjukkan.\footnote{Proposisi ini adalah bentuk 
geometri dari identitas aljabar $(a+b)a = ab + a^2$}

\stepcounter{section}
\section*{\centering Proposisi \thesection} 
\addcontentsline{toc}{section}{Proposisi \thesection} 
Jika sebuah garis lurus dipotong secara acak, maka besar bujur sangkar yang dibentuk
oleh garis tersebut ketika belum dipotong sama besar dengan jumlah besar 
bujur sangkar - bujur sangkar
yang dibentuk oleh potongan - potongannya dan dua kali besar persegi panjang 
yang dibentuk oleh potongan - potongan tersebut.

\begin{center}
\begin{tikzpicture}
\draw (0, 0) node [anchor=north] {D} --++ (0:5) node [anchor=north]{F} 
	--++ (0:3) node [anchor=north] {E} --++ (90:5) node[anchor=west] {K} 
	--++ (90:3) node [anchor=south] {B} --++ (180:3) node[anchor=south]{C}
	--++ (180:5) node[anchor=south] {A} --++ (270:3) node[anchor=east]{H}
	--++(0:5) node[anchor=north west]{G} --++ (0:3) --++(90:3)
    --++(225:{8*sqrt(2)}) -- ++ (90:5) --++ (0:5) --++ (90:3) --++ (270:8);
\end{tikzpicture}
\end{center}

Sebab misalkan garis lurus $AB$ sudah dipotong secara acak, pada titik $C$, Saya
katakan bahwa bujur sangkar pada $AB$ sama besar dengan jumlah dari 
bujur sangkar - bujur sangkar pada $AC$ dan $CB$, dan dua kali besar persegi panjang
yang dibentuk $AC$ dan $CB$.

Sebab misalkan bujur sangkar $ADEB$ sudah digambarkan pada $AB$ [Prop. 1.46], dan
misalkan $BD$ sudah dihubungkan, dan misalkan $CF$ sudah ditarik dari $C$ sejajar
dengan $AD$ atau $EB$ [Prop.1.31], dan misalkan $HK$ sudah ditarik melalui $G$, 
sejajar dengan $AB$ atau $DE$ [Prop. 1.31]. Dan karena $CF$ sejajar dengan $AD$, 
dan $DB$ sudah memotong keduanya, sudut eksterior $CGB$ sama besar dengan sudut
interior yang tidak bersebelahan $ADB$ [Prop.1.29]. Tapi, $ADB$ sama besar dengan
$ABD$, karena $BA$ sudah sama panjang dengan $AD$ [Prop.1.5]. Dengan demikian 
sudut $CGB$ juga sama besar dengan sudut $GBC$. Sehingga sisi $BC$ sama panjang
dengan sisi $CG$ [Prop. 1.6]. Tetapi $CB$ sama panjang dengan $GK$, dan $CG$ 
juga sama dengan $KB$ [Prop. 1.34]. Dengan demikian $GK$ juga sama panjang dengan 
$KB$. Maka $CGKB$ adalah sama sisi. Dengan demikian saya katakan bahwa dia juga
bersudut siku - siku. Karena $CG$ sejajar $BK$ dan garis lurus $CB$ memotong 
keduanya, sudut - sudut $KBC$ dan $CGB$ dengan demikian sama dengan dua buah sudut
siku - siku [Prop. 1.29]. Tetapi $KBC$ adalah siku - siku. Dengan demikian
$BCG$ juga adalah siku - siku. Dengan demikian sudut - sudut yang berhadapan
$CGK$ dan $GKB$ juga adalah sudut siku - siku [Prop. 1.34]. Dengan demikian
$CGKB$ adalah siku - siku. Dan juga sudah ditunjukkan bahwa dia juga sama sisi. 
Dengan dimikian dia adalah bujur sangkar. Dan dia berada pada $CB$. 
Juga dengan alasan - alasan yang sama $HF$ juga adalah bujur sangkar. Dan  dia 
berada pada $HG$, yang juga mengatakan ia ada pada $AC$ [Prop. 1.34]. Dengan 
demikian masing - masing $HF$ dan $KC$  berada pada $AC$ dan $CB$. Dan 
persegi panjang $AG$ sama besar dengan persegi panjang $GE$ [Prop. 1.43]. Dan 
$AG$ adalah persegi panjang yang dibentuk $AC$ dan $CB$. 
Karena $GC$ sama panjang dengan $CB$. Dengan demikian, $GE$ juga sama besar 
dengan persegi panjang yang dibentuk $AC$ dan $CB$. Dengan demikian, persegi 
panjang $AG$ dan $GE$ sama dengan dua kali persegi panjang yang dibentuk
$AC$ dan $CB$. Dan $HF$ dan $CK$ adalah bujur sangkar - bujur sangkar pada 
masing - masing $AC$ dan $CB$. Dengan demikian, keempat bidang $HF$, $CK$, 
$AG$, dan $GE$, sama besar dengan jumlah besar bujur sangkar - bujur sangkar
pada $AC$ dan $BC$, dan dua kali persegi panjang yang dibentuk $AC$ dan 
$CB$. Tetapi, bidang - bidang $HF$, $CK$, $AG$ dan $GE$ adalah sama dengan
keseluruhan $ADEB$, yang adalah bujur sangkar pada $AB$. Dengan demikian, 
bujur sangkar pada $AB$ sama besar dengan bujur sangkar - bujur sangkar pada 
$AC$ dan $CB$, dan dua kali persegi panjang yang dibentuk $AC$ dan $CB$.

Jadi jika sebuah garis lurus dipotong secara acak, maka bujur sangkar pada
garis tersebut secara utuh sama besar dengan jumlah dari besar bujur sangkar -
bujur sangkar dari potongan - potongannya, dan dua kali  besar persegi panjang
yang dibentuk potongan - potongan tersetbut. \footnote{Proposisi ini adalah
bentuk geomteri dari identitas aljabar $(a+b)^2 = a^2 + b^2 + 2ab$}
Hal yang perlu ditunjukkan.

\stepcounter{section}
\section*{\centering Proposisi \thesection} 
\addcontentsline{toc}{section}{Proposisi \thesection} 
Jika sebuah garis lurus dipotong menjadi potongan yang sama panjang dan yang
tidak sama panjang, maka besar persegi panjang yang dibentuk oleh potongan - potongan
yang tidak sama panjang dari garis tersebut, ditambah besar bujur sangkar
pada selisih antara potongan yang sama panjang dengan yang tidak sama panjang, 
sama besar dengan besar bujur sangkar dari setengah dari garis lurus tersebut.
\begin{center}
\begin{tikzpicture}
\draw (0, 0) node [anchor=north] {F} --++ (0:5) node [anchor=north east] {L}
	--++ (0:5) node [anchor=north west]{M} --++ (90:4) node [anchor=south west]{B}
	--++ (180:4) node [anchor=south]{D} --++ (180:1) node [anchor=south]{C}
	--++ (180:5) node [anchor=south]{A} --++ (270:4);
\draw (5, -1) node [anchor=north]{E} --++ (0:1) node [anchor=north]{G}
	--++ (0:4) node [anchor=north west]{F} --++(90:1);
\draw (5, -1) --++ (90:5) --++ (0:1) --++ (270:5);;
\draw (5, -1) --++ (45:{5*sqrt(2)});
\draw [dashed] (6, -.8) arc(-90:180:.8);
\draw (6, -.8) node [anchor=east]{P};
\draw (6, +.8) node [anchor=south west]{O};
\draw (5.2, -.3) node {N};
\draw (6, 0) node[anchor=south east] {H};
\end{tikzpicture}
\end{center}
Sebab misalkan garis lurus $AB$ sudah dipotong, sama panjang pada $C$, dan 
dengan tidak sama panjang pada $D$. Saya katakan bahwa persegi panjang yang
dibentuk oleh  $AD$ dan $DB$, ditambah bujur sangkar pada $CD$, adalah sama 
besar dengan bujur sangkar pada $CB$.  

Sebab misalkan bujur sangkar $CEFB$ sudah digambarkan pada $CB$ [Prop. 1.46], 
dan misalkan $BE$ sudah dihubungkan, dan misalkan $DG$ sudah ditarik melalui 
$D$, sejajar dengan $CE$ atau $BF$ [Prop. 1.31], dan lagi misalkan $KM$ sudah 
ditarik melalui $H$, sejajar dengan $AB$ atau $EF$ [Prop. 1.31], dan lagi
misalkan $AK$ sudah ditarik melalui $A$, sejajar dengan $CL$ atau $BM$ 
[Prop. 1.31].
Dan dikarenakan komplementer $CH$ sama besar dengan komplementer $HF$ 
[Prop. 43], 
misalkan
bujur sangkar $DM$ sudah ditambahkan pada keduanya. Dengan demikian persegi 
panjang
$CM$ sama besar dengan dengan persegi panjang $DF$. Tetapi, persegi panjang 
$CM$ 
sama besar dengan persegi panjang $AL$, karena $AC$ juga sama panjang $CB$ 
[Prop. 1.36].  Oleh sebab itu persegi panjang $AL$  juga sama besar dengan 
persegi panjang $DF$.
Misalkan persegi panjang $CH$ sudah ditambahkan kepada keduanya. Dengan 
demikian keseluruhan bujur sangkar $AH$ sama besar dengan gnomon $NOP$. 
Tetapi $AH$ adalah persegi panjang yang dibentuk $AD$ dan $DB$. Karena $DH$
sama panjang dengan $DB$. Dengan demikian gnomon $NOP$ juga sama besar dengan 
persegi panjang yang dibentuk $AD$ dan $DB$. Misalkan $LG$, yang sama besar
dengan bujur sangkar pada $CD$, sudah ditambahkan pada keduanya. Dengan 
demikian gnomon $NOP$ dan bujur sangkar $LG$ sama besar dengan persegi panjang
yang dibentuk $AD$ dan $DB$ dan bujur sangkar pada $CD$. Tetapi gnomon $NOP$
dan bujur sangkar $LG$ sama besar dengan keseluruhan bujur sangkar $CEFB$, yang
ada pada $CB$. Dengan demikian, persegi panjang yang dibentuk $AD$ dan $DB$
ditambah bujur sangkar pada $CD$, sama besar dengan bujur sangkar pada $CB$.

Dengan demikian, jika sebuah garis lurus dipotong menjadi potongan yang sama 
panjang dan yang tidak sama panjang, maka besar persegi panjang yang dibentuk oleh 
potongan - potongan yang tidak sama panjang dari garis tersebut, 
ditambah besar bujur sangkar pada selisih antara potongan yang sama panjang dengan 
yang sama tidak panjang, sama besar dengan besar bujur sangkar dari setengah dari 
garis lurus tersebut. Yang adalah hal yang perlu ditunjukkan.
\footnote{Proposisi ini adalah bentuk
geometri dari identitas aljabar $ab + [(a+b)/2-b]^2 = [(a+b)/2]^2$ }

\stepcounter{section}
\section*{\centering Proposisi \thesection} 
\addcontentsline{toc}{section}{Proposisi \thesection} 

Jika sebuah garis lurus dibagi menjadi dua bagian yang sama besar, 
dipanjangkan dengan sebuah garis lurus yang memiliki panjang tertentu, maka besar
persegi panjang yang dibentuk oleh keseluruhan garis setelah dipanjangkan 
 dengan garis yang digunakan untuk memanjangkannya, ditambah dengan besar bujur 
sangkar pada setengah garis lurus yang awal, sama besar dengan besar bujur sangkar 
pada setengah garis lurus yang awal ditambah garis lurus yang 
digunakan untuk memanjangkannya.

\begin{center}
\begin{tikzpicture}
\draw (5, 0) node [anchor=north]{E} --++ (0:5) node [anchor=north]{G} 
	--++ (0:2) node [anchor=north]{F} --++ (90:5) node [anchor=west]{M}
	--++ (90:2) node [anchor=south]{D} --++ (180:2) node [anchor=south]{B}
	--++ (180:5) node [anchor=south]{C} --++ (180:5) node [anchor=south] {A}
	--++ (270:2) node [anchor=north]{K} --++ (0:5) node [anchor=south east]{L}
	--++ (0:5) node [anchor=south east]{H} --++ (0:2);
\draw (5, 0) --++ (45:{7*sqrt(2)});
\draw (5, 0) --++ (90:7);
\draw (10,0) --++ (90:7);
\draw [dashed](10, 4.2) node [anchor=east]{P} arc(-90:90:.8) 
	node [anchor=south west]{O}
	arc (90:180:.8) node [anchor=north]{N};
\end{tikzpicture}
\end{center}

Misalkan sembarang garis $AB$ sudah dipotong menjadi dua bagian yang sama
besar pada $C$, dan misalkan garis lurus $BD$ sudah ditambahkan pada ujung
$B$. Saya katakan bahwa persegi panjang yang dibentuk oleh $AD$ dan $BD$, 
ditambah bujur sangkar pada $CB$, sama besar dengan bujur sangkar pada $CD$.

Sebab misalkan bujur sangkar $CEFD$ sudah digambarkan pada $CD$ [Prop. 1.46],
dan misalkan $DE$ sudah dipertemukan, dan misalkan $BG$ sudah ditarik melalui
$B$ sejajar dengan $EC$ atau $DF$ [Prop. 1.31], dan misalkan $KM$ sudah ditarik
melalui titik $H$, sejajar dengan $AB$ atau $EF$ [Prop. 1.31], dan akhirnya
misalkan $AK$ sudah ditarik melalui $A$, sejajar dengan $CL$ atau 
$DM$ [Prop. 1.31].

Dengan demikian, karena $AC$ sama panjang dengan $CB$, persegi panjang $AL$ 
juga sama panjang dengan persegi panjang $CH$ [Prop. 1.36]. Tetapi, persegi
panjang $CH$ sama panjang dengan persegi panjang $HF$ [Prop. 1.43]. Dengan 
demikian persegi panjang $AL$ juga sama panjang dengan persegi panjang $HF$.
Misalkan persegi panjang $CM$ sudah ditambahkan pada keduanya. Dengan demikian,
keseluruhan persegi panjang $AM$ sama panjang dengan gnomon $NOP$.  Tetapi $AM$
adalah persegi panjang yang dibentuk oleh $AD$ dan $DB$. Karena $DM$ sama 
panjang dengan $DB$. Dengan demikian, gnomon $NOP$ juga sama besar dengan 
persegi panjang yang dibentuk $AD$  dan $DB$. Misalkan $LG$, yang 
sama besar dengan bujur sangkar pada $BC$, sudah ditambahkan pada keduanya.
Dengan demikian, persegi panjang yang dibentuk $AD$ dan $DB$ ditambah bujur
sangkar pada $CB$, sama besar dengan gnomon $NOP$ dan bujur sangkar pada
$LG$. Tetapi gnomon $NOP$ dan bujur sangkar $LG$ sama besar dengan keseluruhan
bujur sangkar $CEFD$, yang adalah bujur sangkar pada $CB$. Dengan demikian,
persegi panjang yang dibentuk $AD$ dan $DB$, ditambah dengan bujur sangkar 
pada $CB$, sama besar dengan bujur sangkar pada $CD$

Dengan demikian jika sebuah garis lurus dibagi menjadi dua bagian yang sama 
panjang, dan dipanjangkan dengan sebuah garis lurus yang memiliki panjang tertentu, 
maka besar persegi panjang yang dibentuk oleh keseluruhan garis setelah dipanjangkan 
 dengan garis yang digunakan untuk memanjangkannya, ditambah dengan besar bujur 
sangkar pada setengah garis lurus yang awal, sama besar dengan bujur sangkar 
pada setengah garis lurus yang awal ditambah garis lurus yang 
digunakan untuk memanjangkannya. Yang adalah hal yang perlu ditunjukkan.
\footnote{Proposisi ini adalah bentuk
geometri dari identitas aljabar $(2a + b)b + a^2 = (a + b)^2 $ }

\stepcounter{section}
\section*{\centering Proposisi \thesection} 
\addcontentsline{toc}{section}{Proposisi \thesection} 

Jika sebuah garis lurus dibagi dua disatu titik tertentu secara acak, maka jumlah
dari besar bujur sangkar pada keseluruhan garis dengan besar bujur sangkar salah satu 
potongan, sama besar dengan jumlah dua kali besar persegi panjang yang dibentuk
keseluruhan garis dengan salah satu potongan tadi ditambah besar bujur sangkar pada potongan
sisanya.

\begin{center}
\begin{tikzpicture}
\draw(0, 0) node [anchor=north]{D} --++ (0:7) node [anchor=north]{N} 
	--++ (0:2) node [anchor=north]{E} --++ (90:7) node [anchor=west]{F} 
	--++ (90:2) node [anchor=south]{B} --++ (180:2) node [anchor=south]{C}
	--++ (180:7) node [anchor=south]{A} --++ (270:2) node [anchor=east] {H}
	--++ (270:7);
\draw(0, 0) --++ (45:{9*sqrt(2)});
\draw(0, 7) --++ (0:7) node[anchor=north west]{G} --++ (0:2);
\draw(7, 0) --++ (90:9);
\draw[dashed] (7, 6.2) node[anchor=east]{M} arc (-90:90:.8) node[anchor=south west]{L}
	arc(90:180:.8) node [anchor=north]{K};
\end{tikzpicture}
\end{center}

Sebab misalkan sembarang garis $AB$ sudah dibagi dua secara acak di satu titik 
$C$. Saya katakan bahwa jumlah dari besar bujur sangkar pada $AB$ dan besar bujur sangkar 
pada $BC$ sama besar dengan jumlah dua kali besar persegi panjang yang dibentuk 
$AB$ dan $BC$, dan besar bujur sangkar pada $CA$.

Sebab misalkan bujur sangkar $ADEB$ sudah digambarkan pada $AB$ [Prop. 1.46], dan
misalkan sisa dari bidang - bidang sudah digambarkan.

Selanjutnya, karena persegi panjang $AG$ sama besar dengan  persegi panjang $GE$
[Prop. 1.43], misalkan bujur sangkar $CF$ sudah ditambahkan pada keduanya. Dengan
demikian keseluruhan bujur sangkar $AF$ sama besar dengan keseluruhan bujur sangkar 
$CE$. Dengan demikian, persegi panjang $AF$ ditambah persegi panjang $CE$ adalah
dua kali persegi panjang $AF$. Tetapi persegi panjang $AF$ ditambah persegi panjang
$CE$ sama besar dengan gnomon $KLM$ ditambah bujur sangkar $CF$. Dengan demikian
gnomon $KLM$ dan bujur sangkar $CF$ adalah dua kali persegi panjang $AF$. Tetapi
dua kali bujur sangkar $AF$ juga adalah dua kali persegi panjang yang dibentuk 
oleh $AB$ dan $BC$. Karena $BF$ sama panjang dengan $BC$. Dengan demikian 
gnomon $KLM$ dan bujur sangkar $CF$ sama besar dengan dua kali persegi panjang 
yang dibentuk $AB$ dan $BC$. Misalkan $DG$, yang adalah bujur sangkar pada $AC$
, sudah ditambahkan pada keduanya. Dengan demikian gnomon $KLM$ dan bujur sangkar
- bujur sangkar $BG$ dan $GD$, sama besar dengan dua kali persegi panjang yang
dibentuk $AB$ dan $BC$ dan bujur sangkar pada $AC$. Tetapi, gnomon $KLM$ dan
bujur sangkar - bujur sangkar $BG$ dan $GD$ sama besar dengan keseluruhan dari
bujur sangkar $ADEB$ dan bujur sangkar $CF$, yang masing - masing adalah bujur 
sangkar - bujur sangkar pada $AB$ dan $BC$. Dengan demikian, jumlah 
bujur sangkar - bujur sangkar pada $AB$ dan $BC$ sama besar dengan dua kali
persegi panjang yang dibentuk $AB$ dan $BC$, dan bujur sangkar pada
$AC$.

Dengan demikian, jika sebuah garis lurus dibagi dua disatu titik tertentu 
secara acak, maka jumlah dari besar bujur sangkar dari keseluruhan garis dengan 
besar bujur sangkar salah satu potongan, sama besar dengan jumlah dua kali 
besar persegi panjang yang dibentuk
keseluruhan garis ditambah potongan tadi dengan besar bujur sangkar pada potongan
sisanya. Yang adalah hal yang perlu ditunjukkan. 
\footnote{Proposisi ini adalah bentuk
geometri dari identitas aljabar $(a + b)^2 + a^2 = 2(a + b)a + b^2 $ }

\stepcounter{section}
\section*{\centering Proposisi \thesection} 
\addcontentsline{toc}{section}{Proposisi \thesection} 

Jika sebuah garis lurus dipotong secara acak, maka empat kali dari besar
persegi panjang yang dibentuk oleh keseluruhan garis dengan salah satu
potongannya, ditambah dengan besar bujur sangkar pada potongan sisanya, 
sama besar dengan besar bujur sangkar pada keseluruhan garis lurus tersebut
setelah dipanjangkan dengan potongan pertama tadi.

\begin{center} 
\begin{tikzpicture}
\draw(0, 10) node [anchor=south]{A} --++ (0:5) node [anchor=south]{C} --++ (0:2) node [anchor=south]{B} 
	--++ (0:2) node[anchor=south]{D} --++ (270:2) node [anchor=west]{N} --++(270:2) node[anchor=west]{P} 
	--++(270:5) node[anchor=north]{F} --++(180:2) node[anchor=north]{L}
	--++ (180:2) node [anchor=north]{H} --++ (180:5) node[anchor=north]{E}
	--++ (90:5) node[anchor=east]{O} --++(90:2) node[anchor=east]{M} --++(90:2)
	--++ (0:5) 
	--++(270:2) node[anchor=south west]{G} --++(270:2)node[anchor=north west] {Q} --++ (270:5) 
	--++(0:2) --++(90:5) node [anchor=north west]{R} --++ (90:2) node[anchor=north west] {K} --++ (90:2)  --++(0:2);
\draw (0,6) --++(0:9) --++(90:2) --++ (180:9);
\draw [dashed] (5, 5) node [anchor = east] {U} arc (-90:45:1) node[anchor=south west]{T} arc (45:180:1) node[anchor=north] {U}; 
\draw (0, 1) --++ (9, 9);
\end{tikzpicture}
\end{center}

Sebab misalkan sembarang garis $AB$ sudah dipotong secara acak pada titik $C$.
Saya katakan empat kali besar persegi panjang yang dibentuk $AB$ dan $BC$, 
ditambah besar bujur sangkar pada $AC$, sama besar dengan besar bujur
sangkar pada $AB$ dipanjangkan dengan $BC$.

Sebab misalkan $BD$ sudah ditarik memanjangkan ($AB$ secara lurus)
, dan misalkan $BD$ sudah dibuat sama panjang dengan $CB$ [Prop. 1.3], dan
misalkan bujur sangkar $AEFD$ sudah digambarkan pada $AD$ [Prop. 1.46], dan 
misalkan bidang - bidang selebihnya sudah di gambarkan (Garis $ED$ dari $E$ 
ke $D$, $CH$ dan $BL$ sejajar dengan $AE$, Dan melalui $K$ tarik $MN$ dan 
melalui $Q$ tarik $OP$, keduanya sejajar dengan $AD$).

Dengan demikian, karena $CB$ sama panjang dengan $BD$, tetapi $CB$ juga 
sama panjang $GK$ [Prop. 1.34], dan $BD$ sama panjang dengan $KN$ [Prop. 1.34],
$GK$ dengan demikian juga sama panjang dengan $KN$. Jadi dengan alasan - alasan 
yang sama, $QR$ juga sama panjang dengan $RP$. Dan karena $BC$ sama panjang 
dengan $BD$, dan $GK$ dengan $KN$, besar bujur sangkar pada $CK$ juga sama besar
dengan besar bujur sangkar pada $KD$, dan besar bujur sangkar pada $GR$ sama
dengan besar bujur sangkar pada $RN$ [Prop. 1.36]. Tetapi besar bujur sangkar
pada $CK$ juga sama besar dengan besar bujur sangkar pada $RN$. Karena keduanya
adalah jajaran genjang kompelementer jajaran genjang $CP$ [Prop. 1.43].
Dengan demikian besar bujur sangkar $KD$ juga sama besar dengan besar bujur 
sangkar pada $GR$. Dengan demikian keempat bujur sangkar $DK$, $CK$, $GK$, dan
$RN$ adalah sama besar satu dengan yang lain. Dengan demikian, keempatnya 
sekaligus adalah empat kali lipat besar bujur sangkar $CK$. Dan lagi, karena 
$CB$ sama panjang dengan $BD$, tetapi $BD$ sama panjang dengan $BK$ 
(sudut $AED$ dan 
$ADE$ sama besar karena $AE$ sama panjang dengan $AD$ [Prop. 1.8], dan sudut 
$BKD$ juga sama besar dengan sudut $ADE$ karena $BK$ sejajar $AE$ dan $ED$ 
memotong keduanya, penerjemah [Prop 1.28])- dan juga $CG$ - dan
$CB$ sama panjang dengan $GK$ - dan juga $GQ$ - $CG$ dengan demikian juga 
sama panjang dengan $GQ$. Dan karena $CG$ sama panjang dengan $GQ$, dan $QR$
sama panjang dengan $RP$, persegi panjang $AG$ sama besar dengan persegi panjang 
$MQ$, dan persegi panjang $QL$ sama besar dengan persegi panjang $RF$[Prop. 1.36]. 
Tetapi persegi panjang $MQ$ sama besar dengan persegi panjang $QL$. Karena 
keduanya adalah komplementer pada jajaran genkang $ML$ [Prop. 1.43]. Dengan 
demikian, keempat persegi panjang $AG$, $MG$, $QL$m dan $RF$ adalah sama besar
satu dengan yang lain. Dengan demikian jumlah besar keempatnya sama dengan empat 
kali lipat besar persegi panjang $AG$. Dan juga sudah ditunjukkan keempat
bujur sangkar $CK$, $KD$, $GR$ dan $RN$ jika digabungkan besarnya adalah empat
kali besar bujur sangkar $CK$. Dengan demikian, jika kedelapan bidang digabungkan, 
yang membentuk gnomon $STU$, adalah empat kali besar bersegi panjang $AK$.
Dan karena $AK$ adalah persegi panjang yang dibentuk oleh $AB$ dan $BD$, karena
$BK$ sama panjang dengan $BD$, empat kali besar persegi panjang yang di bentuk
oleh $AB$ dan $BD$ adalah empat kali persegi panjang $AK$. Dengan demikian
empat kali persegi panjang yang dibentuk oleh $AB$ dan $BD$ sama besar dengan
gnomon $STU$. Misalkan $OH$, yang sama besar dengan bujur sangkar pada $AC$ 
sudah ditambahkan pada keduanya. Dengan demikian, empat kali besar persegi 
panjang yang dibentuk oleh $AB$ dan $BD$, ditambah dengan besar bujur sangkar
pada $AC$, sama besar dengan gnomon $STU$, ditambah dengan 
bujur sangkar $OH$. Tetapi gnomon $STU$ dan bujur sangkar $OH$ sama dengan keseluruhan
bujur sangkar $AEFD$, yang adalah bujur sangkar pada $AD$. Dengan demikian,
empat kali besar persegi panjang yang dibentuk $AB$ dan $BD$, ditambah dengan 
besar bujur sangkar pada $AC$, sama dengan besar bujur sangkar pada $AD$. Dan $BD$ 
sama panjang dengan $BC$. Dengan demikian, empat kali persegi panjang yang dibentuk 
$AB$ dan $BC$, ditambah dengan bujur sangkar pada $AC$, sama besar dengan bujur 
sangkar pada $AD$, yang adalah bujur sangkar digambarkan pada $AB$ dan $BC$. 

Dengan demikian, jika sebuah garis lurus dipotong secara acak, maka empat kali besar
persegi panjang yang dibentuk oleh keseluruhan garis dengan salah satu potongannya, 
ditambah dengan besar bujur sangkar dari potongan selebihnya, sama besar dengan besar
bujur sangkar pada keseluruhan garis tadi setelah dipanjangkan dengan potongan
pertama tadi \footnote{Proposisi ini adalah versi 
geometri dari identitas aljabar $4(a + b)a + b^2 = [(a + b) + a]^2$} .
Yang adalah hal perlu ditunjukkan.

\stepcounter{section}
\section*{\centering Proposisi \thesection} 
\addcontentsline{toc}{section}{Proposisi \thesection}

Jika sebuah garis lurus dipotong menjadi bagian - bagian yang sama panjang 
dan tidak sama panjang, maka jumlah dari besar bujur sangkar - bujur sangkar 
pada 
potongan - potongan yang
tidak sama panjang keseluruhan garis adalah dua kali jumlah dari jumlah besar
bujur sangkar pada setengah keseluruhan garis dan besar bujur sangkar pada
perbedaan garis yang sama panjang dan dengan yang tidak sama panjang.

\begin{center}
\begin{tikzpicture}
\draw (0, 0) node [anchor=north] {A} -- ++ (0:5) node[anchor=north]{C} 
	--++ (0:2) node [anchor=north]{D} --++ (0:3) node[anchor=north]{B} 
	--++ (135:{3*sqrt(2)}) node[anchor=west]{F} 
	--++ (135:{2*sqrt(2)}) node[anchor=south]{E} 
	--++(270:2) node[anchor=east]{G} --++ (0:2)
	--++({180 + atan(3/7)}:{sqrt(3*3+7*7)})
	--++(45:{5*sqrt(2)}) --++ (270:5) --++(0:2) --++ (90:3);
\end{tikzpicture}
\end{center}

Sebab misalkan sembarang garis $AB$ sudah dipotong sama panjang pada titik $C$,
dan sescara tidak sama panjang pada titik $D$. Saya katakan bahwa jumlah dari
besar bujur sangkar $AD$ dan besar bujur sangkar $DB$ adalah dua kali jumlah
besar bujur sangkar pada $AC$ dan besar bujur sangkar pada $CD$

Sebab misalkan $CE$ sudah ditarik dari titik $C$, secara siku - siku dari $AB$
[Prop. 1.11], dan misalkan garis tersebut sudah dibuat sama panjang dengan $AC$ 
dan $CB$ [Prop. 1.3], dan misalkan $EA$ dan $EB$ sudah dihubungkkan. Dan misalkan
$DF$ sudah ditarik sudah ditarik dari $D$ sejajar dengan $EC$ [Prop. 1.31], dan
misalkan $FG$ sudah ditarik melalui titik $F$, sejajar terhadap $AB$ [Prop. 1.31].
Dan misalkan $AF$ sudah dihubungkan. Dan karena $AC$ sama panjang dengan $CE$, 
sudut $EAC$ juga sama besar dengan sudut $AEC$ [Prop. 1.5]. Dan karena sudut 
pada $C$ sudut siku - siku, jumlah dari sudut - sudut selebihnya dari segitiga
$AEC$, yakni $EAC$ dan $AEC$, dengan demikian sama dengan sudut siku - siku
[Prop. 1.32]. Dan keduanya adalah sama besar. Dengan demikian sudut $CEA$ dan 
$CAE$ adalah masing - masing setengah sudut siku - siku. Jadi untuk alasan - 
alasan yang sama, sudut - sudut $CEB$ dan $EBC$ juga adalah sama dengan setengah
sudut siku - siku. Dengan demikian, keseluruhan sudut $AEB$ adalah sudut
siku - siku. Dan karena $GEF$ adalah setengah sudut siku - siku, dan $EGF$ 
adalah sudut  siku - siku - karena $EGF$ sama besar dengan sudut interior tidak
berseberangan $ECB$ [Prop. 1.29]- sudut - sudut selebihnya $EFG$ adalah sama
dengan setengan sudut siku - siku [Prop. 1.32[. Dengan demikiah sudut $GEF$
sama besar dengan sudut $EFG$. Dengan demikian sisi $EG$ juga sama panjang dengan
sisi $GF$ [Prop. 1.6]. Dan lagi karena sudut pada $B$ adalah setengah sudut
siku - siku, dan sudut $FDB$ adalah sudut siku - siku, - karena lagi $FDB$ 
sama besar dengan sudut interior tidak bersebelahan $ECB$ [Prop. 1.29] - sudut
selebihnya $BFD$ adalah setengan sudut siku - siku [Prop. 1.32]. Dengan demikian, sudut pada
$B$ sama besar dengan sudut $DFB$. Dengan demikian sisi $FD$ juga sama panjang 
dengan sisi $DB$ [Prop. 1.6]. Dan karena $AC$ sama panjang dengan $CE$, 
besar bujur sangkar pada $AC$ juga sama besar dengan besar bujur sangkar pada
$CE$. Dengan demikian, jumlah besar bujur sangkar pada $AC$ dan $CE$ adalah
dua kali besar bujur sangkar pada $AC$. Dan besar bujur sangkar pada $EA$ sama
besar dengan jumlah besar bujur sangkar pada $AC$ dan $CE$. Karena sudut $ACE$
sama adalah sudut siku - siku [Prop. 1.47]. Dengan demikian, besar bujur sangkar
pada $AE$ adalah dua kali besar bujur sangkar pada $AC$. Dan lagi, karena $EG$
sama panjang dengan $GF$, besar bujur sangkar pada $EG$ sama besar dengan besar
bujur sangkar pada $GF$. Dengan demikian, jumlah dari besar bujur sangkar
pada $EG$ dan besar bujur sangkar pada $GF$ sama dengan dua kali besar bujur
sangkar pada $GF$. Dan besar bujur sangkar pada $EF$ sama besar dengan jumlah
besar bujur sangkar pada $EG$ dan besar bujur sangkar pada $GF$ [Prop. 1.47].
Dengan demikian besar bujur sangkar pada $EF$ dua kali besar bujur sangkar 
pada $GF$. Dan $GF$ sama panjang dengan $CD$ [Prop. 1.34]. Dengan demikian,
besar bujur sangkar pada $EF$ dua kali besar bujur sangkar pada $CD$.
Dan besar bujur sangkar pada $EA$ adalah dua kali besar bujur sangkar pada $AC$.
Dengan demikian, jumlah dari besar bujur sangkar pada $AE$ dan besar bujur sangkar
pada $EF$ adalah dua kali jumlah besar bujur sangkar pada $AC$ dan $CD$. 
Dan besar bujur sangkar pada $AF$ sama dengan jumlah besar bujur sangkar
pada $AE$ dan $EF$. Karena sudut $AEF$ adalah sudut siku - siku [Prop. 1.47].
Dengan demikian, besar bujur sangkar pada $AF$ adalah dua kali jumlah besar
bujur sangkar pada $AC$ dan besar bujur sangkar pada $CD$. Dan jumlah besar
bujur sangkar pada $AD$ dan $DF$ sama dengan besar bujur sangkar pada $AF$.
Karena sudut $D$ adalah siku - siku [Prop. 1.47]. Dengan demikian, jumlah
dari besar bujur sangkar $AD$ dan besar bujur sangkar $DF$ adalah dua kali
bear bujur sangkar pada $AC$ dan $CD$. Dan $DF$ sama panjang dengan $DB$. 
Dengan demikian, jumlah dari besar bujur sangkar pada $AD$ dan besar bujur 
sangkar pada $DB$ adalah dua kali besar jumlah bujur sangkar pada $AC$ dan 
dan besar bujur sangkar pada $CD$.

Dengan demikian, jika sebuah garis lurus dipotong menjadi bagian - bagian 
yang sama panjang dan tidak
sama panjang, maka jumlah dari besar bujur sangkar - bujur sangkar pada 
potongan - potongan yang
tidak sama panjang keseluruhan garis adalah dua kali jumlah dari jumlah besar
bujur sangkar pada setengah keseluruhan garis dan besar bujur sangkar pada
perbedaan garis sama panjang dan garis yang tidak sama panjang. \footnote{Proposisi 
ini adalah bentuk geometri dari identitas aljabar: $a^2 + b^2 = 2[(\frac{a+b}{2})^2 + (\frac{a+b}{2} - b)^2]$}
Yang adalah hal yang perlu ditunjukkan.  

\stepcounter{section}
\section*{\centering Proposisi \thesection} 
\addcontentsline{toc}{section}{Proposisi \thesection}
Jika sebuah garis lurus dibagi dua sama panjang, dan dipanjangkan dengan
sembarang garis lurus lain, maka jumlah dari besar bujur sangkar pada 
keseluruhan garis lurus ketika sudah dipanjangkan dengan besar bujur
sangkar pada garis yang memanjangkannya, adalah dua kalau dari jumlah besar
bujur sangkar pada setengah garis mula - mula, dan besar bujur sangkar
pada setengah sisanya dengan perpanjangannya.

\begin{center}
\begin{tikzpicture}[scale=.9, every node/.style={scale=.9}]
\draw (0, 0) node [anchor=north]{A} --++(0:5) node[anchor=north]{C} 
	--++(0:5) node[anchor=north]{B} --++(0:2) node[anchor=west]{D}
	--++(90:5) node[anchor=south]{F} --++(180:7) node[anchor=south]{E}
	--++(225:{5*sqrt(2)}) 
	--++({atan(-2/12)}: {sqrt(12*12 + 2*2)}) node[anchor=north]{G}
	--++(90:2);
\draw (5, 0) --++ (90:5) --++ (315:{7*sqrt(2)});
\end{tikzpicture}
\end{center}

Sebab misalkan sembarang garis $AB$ sudah dipotong dua sama panjang pada titik
$C$, dan misalkan dipanjangkan dengan sembarang garis $BD$ di salah satu 
ujungnya. Saya katakan jumlah besar bujur sangkar pada $AD$ dan besar bujur 
sangkar pada $DB$ adalah
dua kali besar jumlah bujur sangkar pada $AC$ dan bujur sangkar pada $CD$.

Sebab misalkan $CE$ sudah ditarik dari titik $C$ secara tegak lurus terhadap
$AB$ [Prop. 1.11], dan misalkan panjangnya dibuat sama dengan masing - masing
$AC$ dan $CB$ [Prop. 1.3], dan misalkan $EA$ dan $EB$ sudah dihubungkan. Dan
misalkan $EF$ sudah ditarik melalui $E$, sejajar terhadap $AD$, dan misalkan
$FD$ sudah ditarik melalui $D$ sejajar terhadap $CE$ [Prop. 1.31]. Dan karena
garis lurus $EF$ memotong garis - garis sejajar $EC$ dan $FD$, sudut - sudut 
interior $CEF$ dan $EFD$ dengan demikian sama besar dengan dua buah sudut 
siku - siku [Prop. 1.29]. Dengan demikian $FEB$ dan $EFD$ lebih kecil dari
dua buah sudut siku - siku. Dan garis - garis lurus yang dipanjangkan dari
sudut - sudut internal yang besarnya lebih kecil dari dua buah sudut siku -
siku akan bertemua di satu titik [Post. 5]. Dengan demikian, jika dipanjangkan
di arah $B$ dan $D$, garis - garis lurus $EB$ dan $FD$ akan bertemu. Dan 
misalkan keduanya sudah dipanjangkan dan misalkan keduanya bertemu di titik 
$G$, dan misalkan $AG$ sudah dipertemukan. Dan karena $AC$ sama panjang dengan
$CE$, dan sudut $EAC$ juga sama besar dengan sudut $AEC$ [Prop. 1.5]. Dan 
sudut $C$ adalah siku - siku. Demgam demikian, $EAC$ dan $AEC$ masing - masing
adalah setengah sudut siku - siku [Prop. 1.32]. Jadi, dengan alasan - alasan
yang sama, $CEB$ dan $EBC$ juga masing - masing adalah setengah sudut siku - 
siku. Degan demikian, sudut $AEB$ adalah siku - siku. Dan karena $EBC$ adalah
setengah sudut siku - siku, $DBG$ dengan demikian juga adalah setengah sudut
siku - siku [Prop. 1.15]. Dan $BDG$ juga adalah sudut siku - siku. Karena 
besarnya sama dengan $DCE$. Karena keduanya adalah sudut yang berseberangan
[Prop. 1.29]. Dengan demikian, sudut selebihnya $DGB$ adalah setengan sudut
siku - siku. Dengan demikian, $DGB$ sama besar dengan $DBG$. Dengan demikian
panjang sisi $BD$ juga sama dengan panjang sisi $GD$ [Prop. 1.6]. Dan lagi,
karena $EGF$ adalah setengan sudut siku - siku, dan sudut $F$ adalah sudut
siku - siku, karena sama besar dengan sudut berhadapan $C$ [Prop. 1.34], 
sudut selebihnya $FEG$ dengan demikian sama dengan setengah sudut siku - siku.
Dengan demikian, sudut $EGF$ sama besar dengan sudut $FEG$. Dengan demikian
sisi $GF$ juga sama panjang dengan sisi $EF$ [Prop. 1.6]. Dan karena [$EC$ 
sama panjang $CA$] besar bujur sangkar di $EC$ [juga] sama besar dengan besar
bujur sangkar pada $CA$. Dengan demikian jumlah besar bujur sangkar pada $EC$ 
dan bujur sangkar pada $CA$ adalah dua kali besar bujur sangkar pada $CA$. Dan
besar bujur sangkar pada $EA$ sama dengan jumlah besar bujur sangkar pada $EC$
dan besar bujur sangkar pada $CA$ [Prop. 1.47]. Dengan demikian besar bujur
sangkar pada $EA$ adalah dua kali besar bujur sangkar pada $AC$. Dan lagi 
karena panjang $FG$ sama dengan panjang $EF$, besar bujur sangkar pada $FG$
juga sama besar dengan besar bujur sangkar pada $EF$. Dengan demikian besar
bujur sangkar pada $EG$ adalah dua kali besar bujur sangkar pada $EF$. Dan 
$EF$ sama panjang dengan $CD$ [Prop. 1.34]. Dengan demikian, besar bujur 
sangkar pada $EG$ adalah dua kali besar bujur sangkar pada $CD$. Tetapi juga
telah ditunjukkan bahwa besar bujur sangkar pada $EA$ adalah dua kali besar
bujur sangkar pada $AC$. Dengan demikian jumlah dari besar bujur sangkar
sangkar pada $AE$ dan besar bujur sangkar pada $EG$   adalah dua kali 
jumlah besar bujur sangkar pada $AC$ dan besar bujur sangkar pada $CD$. Dan 
besar bujur sangkar pada $AG$ juga sama besar dengan jumlah besar bujur 
sangkar pada $AE$ dan besar bujur sangkar pada $EG$ [Prop. 1.47]. Dengan 
demikian besar bujur sangkar pada $AG$ adalah dua kali jumlah besar bujur
sangkar pada pada $AC$ dan besar bujur sangkar pada $CD$. Dan jumlah dari
besar bujur sangkar pada $AD$ dan besar bujur sangkar pada $DG$ sama dengan 
besar bujur sangkar pada $AG$ [Prop 1.47]. Dengan demikian, jumlah dari besr
bujur sangkar pada $AD$ dan besar bujur sangkar pada $DG$ adalah dua kali 
besar bujur sangkar pada $AC$ dan besar bujur sangkar pada $CD$. Dan $DG$
sama panjang dengan $DB$. Dengan demikian, jumlah dari besar bujur sangkar
pada $AD$ dan besar bujur sangkar pada $DB$ adalah dua kali besar bujur 
sangkar pada $AC$ dan besar bujur sangkar pada $CD$.

Dengan demikian, jika sebuah garis lurus dibagi dua sama panjang, 
dan dipanjangkan dengan sembarang garis lurus lain, maka jumlah dari besar 
bujur sangkar pada keseluruhan garis lurus ketika sudah dipanjangkan dengan 
besar bujur sangkar pada garis yang memanjangkannya, 
adalah dua kalau dari jumlah besar bujur sangkar pada setengah garis 
mula - mula, dan besar bujur sangkar pada setengah sisanya dengan 
perpanjangannya \footnote{Proposisi ini adalah bentuk geometris dari identitas
aljabar $(2a + b)^2 + b^2 = 2[a^2 + (a + b)^2]$}. 
Yang adalah hal yang perlu ditunjukkan.

\stepcounter{section}
\section*{\centering Proposisi \thesection} 
\addcontentsline{toc}{section}{Proposisi \thesection}
Untuk memotong sebuah garis tertentu sedemikian rupa sehinga besar persegi 
panjang yang dibentuk keseseluruhan garis mula - mula dengan salah satu
potongannya adalah sama besar dengan besar bujur sangkar pada potongan sisanya.

\begin{center}
\begin{tikzpicture}
\draw(0, 0) node[anchor=east]{A} --++ (0:8) node [anchor=west]{B} 
	--++ (270:8) node [anchor=north]{D} --++ (180:8) node [anchor=north]{C}
	--++ (90:8);
\draw(0, -4) node [anchor=east] {E} --++ ({atan(4/8)}: {sqrt(8*8 + 4*4)});
\draw(0, -4) --++ (90: {sqrt(8*8 + 4*4)}) node [anchor=south]{F}
	--++ (0: {sqrt(8*8+ 4*4) - 4}) node [anchor=south]{G}
	--++ (270: {sqrt(8*8 + 4*4) - 4}) node [anchor=south west]{H}
	--++ (270: 8) node [anchor=north]{K};
\draw [dashed, gray] (0, -4) --++ (0: {sqrt(8*8 + 4*4)})
	--++ (90:{sqrt(8*8 + 4*4)}) --++ (180:4);
\draw [dash dot, gray] ({sqrt(8*8 + 4*4) - 4}, -4) --++ (0: 4)
	--++(90:4) --++(180:4);
\draw [dash dot, gray] (0, {sqrt(8*8 + 4*4) - 4}) --++ (315: {sqrt((8*8 + 4*4)*2)});
\end{tikzpicture}
\end{center}

Misalkan $AB$ adalah semabarang garis tertentu. Jadi dibutuhkan untuk memotong
$AB$ sedemikian rupa sehingga besar persegi panjang yang dibentuk oleh 
keseluruhan garis yang semula dengan salah satu potongannya adalah sama besar
dengan besar bujur sangkar pada potongan sisanya.

Misalkan bujur sangkar $ABCD$ sudah digambarkan pada $AB$ [Prop. 1.46], dan 
misalkan $AC$ sudah dipotong menjadi dua bagian pada titik $E$ [Prop. 1.10],
dan misalkan $BE$ sudah dipertemukan. Dan misalkan $CA$ sudah ditarik melalui
titik $F$ dan misalkan $EF$ dibuat sama panjang dengan $BE$ [Prop. 1.3]. Dan
misalkan bujur sangkar $FH$ sudah digambarkan pada $AF$ [Prop. 1.46], dan 
misalkan $GH$ sudah ditarik melalui titik $K$. Saya katakan bahwa $AB$
sudah dipotong pada $H$ sedemikian rupa sehingga membuat besar persegi panjang
yang dibentuk $AB$ dan $BH$ sama besar dengan besar bujur sangkar pada $AH$.

Disebabkan karena garis lurus $AC$ sudah dipotong dua sama besar pada titik
$E$, dan $FA$ sudah ditambahkan padanya, besar persegi panjang yang dibentuk
$CF$ dan $FA$, ditambah dengan besar bujur sangkar pada $AE$, dengan demikian
sama dengan besar bujur sangkar pada $EF$ [Prop. 2.6]. Dan $EF$ sama besar 
dengan $EB$. Dengan demikian, persegi panjang yang dibentuk oleh $CF$ dan $FA$
ditambah bujur sangkar pada $AE$ adalah sama besar dengan bujur sangkar pada 
$EB$. Tetapi, jumlah dari besar bujur sangkar pada $BA$ dan besar bujur 
sangkar pada $AE$ sama besar dengan bujur sangkar pada $EB$. Karena sudut $A$
adalah sudut siku - siku [Prop. 1.47]. Dengan demikian, besar persegi panjang
yang dibentuk $CF$ dan $FA$ , ditambah dengan besar bujur sangkar pada $AE$,
adalah sama vesar dengan jumlah dari besar bujur sangkar pada $BA$ dan bujur
sangkar pada $AE$. Misalakan bujur sangkar pada $AE$ sudah dikurangkan dari
keduanya. Dengan demikian, sisa persegi panjang yang dibentuk $CF$ dan $FA$ 
sama besar dengan bujur sangkar pada $AB$. Dan $FK$ adalah persegi panjang 
yang dibentuk oleh $CF$ dan $FA$. Karena $AF$ sama panjang dengan $FG$. Dan
$AD$ adalah bujur sangkar pada $AB$. Dengan demikian besar persegi panjang
$FK$ sama besar dengan besar bujur sangkar pada $AD$. Misalkan persegi 
panjang $AK$ sudah dikurangkan dari keduanya. Dengan demikian, sisa bujur
sangkar $FH$ sama besar dengan persegi panjang $HD$. Dan $HD$ adalah persegi
panjang yang dibentuk oleh $AB$ dan $BH$. Karena $AB$ sama panjang dengan $BD$.
Dan $FH$ adalah bujur sangkar pada $BH$. Dengan demikian, persegi panjang yang
dibentuk oleh $AB$ dan $BH$ sama besar dengan bujur sangkar pada $HA$.

Dengan demikian, garis lurus $AB$ sudah dipotong pada titik  $H$ sedemikian
sehingga persegi panjang yang dibentuk oleh $AB$ dan $BH$ sama besar dengan
bujur sangkar pada $HA$\footnote{Hal pembagian sebuah garis lurus sehingga 
rasio keseluruah garis dengan garis yang lebih besar sama besar 
dengan rasio yang lebih besar dengan garis yang lebih kecil disebut 
dengan "rasio emas"}.
Yang adalah hal yang perlu dilakukan.  

\stepcounter{section}
\section*{\centering Proposisi \thesection} 
\addcontentsline{toc}{section}{Proposisi \thesection} 

Pada segitiga tumpul, besar bujur sangkar dari sisi yang menghadap sudut tumpul
 adalah lebih besar dari jumlah besar bujur sangkar - bujur sangkar sisi - sisi 
yang mengapit sudut tumpul sebanyak dua kali besar persegi panjang yang 
dibentuk dari satu sisi yang yang mengapit sudut tumpul yang padanya garis lurus 
jatuh secara tegak lurus dari sudut segitiga yang didepannya, dengan garis lurus yang memotong dari luar 
(segitiga tersebut) dari titik jatuh garis tegak lurus kearah sudut tumpul.

\begin{center}
\begin{tikzpicture}
\draw [-, black] (-2, -2) -- (6, -2);
\draw (-2.25, -2.25) node {$D$};
\draw (6.25, -2.25) node {$C$};
\draw [-, black] (-2, -2) -- (-2, 3);
\draw (-2.25, 3.25) node {$B$};
\draw [-, black] (0, -2) -- (-2, 3);
\draw (0.0, -2.25) node {$A$};
\draw [-, black] (6, -2) -- (-2, 3);
\end{tikzpicture} 
\end{center}

Misalkan $ABC$ adalah segitiga tumpul, dengan sudut $BAC$ adalah sudut tumpul.
Misalkan $BD$ ditarik dari titik $B$ tegak lurus terhadap $CA$ [Prop 1.12]. Saya
katakan bahwa bujur sangkar $BC$ adalah lebih besar dari jumlah bujur sangkar $BA$
dan bujur sangkar $AC$, sebanyak dua kali persegi panjang yang dibentuk $CA$ dan $AD$.

Karena garis lurus $CD$ sudah dipotong pada sebarang titik, $A$, bujur sangkar 
pada $DC$ dengan demikian sama dengan jumlah dari jumlah bujur sangkar $CA$ 
dan $AD$, dan dua kali persegi panjang yang dibentuk $CA$ dan $AD$ [Prop.2.4]. 
Misalkan bujur sangkar pada $DB$ ditambahkan kepada keduanya. Dengan demikian, 
jumlah dari besar bujur sangkar $CD$ dan bujur sangkar $DB$ adalah sama besar
dengan jumlah besar bujur sangkar - bujur sangkar pada $AC$, $AD$, $DB$  ditambah
dua kali persegi panjang yang dibentuk $CA$ dan $AD$. Tetapi, besar bujur sangkar
pada $CB$ adalah sama dengan jumlah besar bujur sangkar - bujur sangkar pada $CD$
dan pada $DB$. Karena sudut $D$ adalah siku - siku [Prop. 1.47]. Dan bujur 
sangkar pada $AB$ sama dengan jumlah besar bujur sangkar pada $AD$ dan pada $DB$ 
[Prop. 1.47]. Dengan demikian, bujur sangkar pada $CB$ sama besar dengan jumlah
besar bujur sangkar pada $CA$ dan pada $AB$, dan dua kali besar persegi panjang
yang dibentuk $CA$ dan $AD$. Dengan demikian bujur sangkar pada $CB$ adalah
lebih besar dari jumlah besar bujur sangkar - bujur sangkar pada $CA$ dan $AB$
sebesar dua kali besar persegi panjang yang dibentuk $CA$ dan $AB$.

Dengan demikian, pada segitiga tumpul, besar bujur sangkar dari sisi yang 
menghadap sudut tumpul adalah lebih besar dari jumlah besar bujur sangkar - 
bujur sangkar sisi - sisi 
yang mengapit sudut tumpul sebanyak dua kali besar persegi panjang yang 
dibentuk dari satu sisi yang yang mengapit sudut tumpul yang padanya garis lurus 
jatuh secara tegak lurus dari sudut di depannya, dan garis lurus yang 
memotong dari luar 
(segitiga tersebut) dari titik jatuh garis tegak lurus kearah sudut 
tumpul\footnote{Proposisi
sama dengan	rumus kosinus yang terkenal 
$BC^2 = AB^2 + AC^2 - 2\ AB\  AC\cos BAC, \text{karena}  \cos BAC = -AD/AB$}. 
Yang adalah hal yang perlu ditunjukkan.

\stepcounter{section}
\section*{\centering Proposisi \thesection} 
\addcontentsline{toc}{section}{Proposisi \thesection} 

Pada segitiga lancip, besar bujur sangkar pada sisi yang menghadap sudut 
lancip, adalah lebih kecil dari jumlah besar bujur sangkar - bujur sangkar 
pada sisi yang mengapit sudut lancip sebesar dua kali besar persegi panjang
yang dibentuk salah satu sisi yang mengapit sudut lancip tadi, yang padanya sebuah
garis tegak lurus jatuh dari sudut dihadapannya, dengan garis lurus yang dipotong dari dalam segitiga oleh
garis tegak lurus tadi ke arah sudut lancip.

\begin{center}
\begin{tikzpicture}[scale=.8, every node/.style={scale=.8}]
\draw (0, 0) node [anchor=north east]{B} 
	--++ (0: 7) node [anchor=north west]{C}
	--++ (137:6) node [anchor=south]{A} 
	--++ (270:{6*sin(137)}) node [anchor=north east]{D};
\draw (0, 0) --++ 
	({atan(6*sin(137)/(7+6*cos(137)))}: 
	{sqrt(pow(6*sin(137), 2) + pow(7+6*cos(137), 2))});
\draw [dashed, gray](0, 0) --++ (270:7) --++ (0:7)
	--++ (90:7); 
\draw [dashed, gray](0, -7) --++ (90:{(7+6*cos(137))})
	--++ (0: {(7+6*cos(137))}) --++ (270: {(7+6*cos(137))}); 
\draw [dashed ,gray]({7+6*cos(137)}, 0) --++(270: {7 - (7+6*cos(137)})
	--++ (0: {7 - (7+6*cos(137))});
\draw [dashed, gray](0, -7) --++ (45: {7*sqrt(2)});
\draw [dash dot, gray] ({7 + 6 *cos(137)}, {6*sin(137)}) 
	--++ (180:{6*sin(137)}) --++ (270: {6 * sin(137)}) --(0, 0);
\end{tikzpicture}
\end{center}

Misalkan $ABC$ adalah segitiga lancip, yang salah sudut nya $B$ adalah sudut
lancip. Dan misalkan $D$ sudah ditarik dari titik $A$, tegak lurus terhadap 
$BC$ [Prop 1.12]. Saya katakan bahwa besar bujur sangkar pada $AC$ adalah 
lebih kecil dari jumlah besar bujur sangkar - bujur sangkar pada $CB$ dan 
$BA$ sebesar dua kali persegi panjang yang dibentuk $CB$ dan $BD$.

Sebab karena garis lurus $CB$ sudah dipotong secara acak pada $D$, jumlah 
besar bujur sangkar pada $CB$ dan bujur sangkar pada $BD$ dengan demikian
sama dengan dua kali besar persegi panjang $CB$ dan $BD$ ditambah  bujur 
sangkar pada $DC$ [Prop. 2.7]. Misalkan bujur sangkar $DA$ sudah ditambahkan
pada keduanya. Dengan demikian jumlah besar bujur sangkar - bujur sangkar pada
$CB$, $BD$ dan $DA$ adalah dua kali persegi panjang yang dibentuk $CB$ dan 
$BD$, dan jumlah besar bujur sangkar - bujur sangkar pada $AD$ dan $DC$. 
Tetapi, besar bujur sangkar pada $AB$ adalah sama dengan jumlah besar bujur
sangkar - bujur sangkar pada $BD$ dan $DA$. Karena sudut pada titik $D$ adalah
sudut siku - siku [Prop. 1.47]. Dan besar bujur sangkar pada $AC$ adalah
sama besar dengan jumlah besar bujur sangkar - bujur sangkar pada $AD$ dan 
$DC$ [Prop. 1.47]. Dengan demikian, jumlah besar bujur sangkar - bujur sangkar
pada $CB$ dan $BA$ adalah sama besar dengan besar bujur sangkar pada $AC$, dan
dua kali persegi panjang yang dibentuk $CB$ dan $BD$. Dengan demikian, besar
bujur sangkar pada $AC$ adalah lebih kecil dari jumlah besar bujur sangkar 
- bujur sangkar pada $CB$ dan $BA$ sebesar dua kali persegi panjang yang
dibentuk $CB$ dan $BD$.

Dengan demikian, pada segitiga lancip, besar bujur sangkar pada sisi yang
menghadap sudut lancip adalah lebih kecil dari jumlah besar bujur sangkar - 
bujur sangkar pada sisi - sisi yang mengapit sudut lancip sebesar dua kali
besar persegi panjang yang dibentuk salah satu sisi yang mengapit sudut
lancip tadi, yang padanya sebuah garis tegak lurus jatuh dari sudut 
dihadapannya, dengan garis lurus yang dipotong dari dalam segitiga oleh garis
tegak lurus tadi ke arah sudut lancip \footnote{Proposisi ini adalah sama 
dengan rumus consinus yang terkenal 
$AC^2 = AB^2 + BC^2 - 2\cdot AB\cdot BC\cdot cos(ABC)$}.


\stepcounter{section}
\section*{\centering Proposisi \thesection} 
\addcontentsline{toc}{section}{Proposisi \thesection} 
Untuk menggambarkan sebuah bujur sangkar yang  sama besar dengan bidang 
rektilinear tertentu.

\begin{center}
\begin{tikzpicture}
\draw(0, 0) node [anchor=north] {S}
	--++ (-10:4) node [anchor=north]{T}
	--++ (70:5)  node [anchor=west]{U}%%the inner angle is 100
	--++ (160:5) node [anchor=east]{V};%%the inner angle is 90 
	%--++ (%%{acos((50 - 2*5*sqrt(2)*4*cos(55))/(2*sqrt(66 - ( 40*sqrt(2)*cos(55) ) )*4))}: 5);	
	%	(260.6: {sqrt(25*2 + 4*4 - 2*4*5*sqrt(2)*cos(55))}); %% = 
			%%using cosine vs = sqrt(vt^2 + st^2 - 2vt*st*cos(55))
			%%8.20071206371003 
%%\pgfmathparse{acos((pow(5*sqrt(2),2) + 4*4 - 2*4*5*sqrt(2)*cos(55) + 4*4 - pow(5*sqrt(2), 2))/(2*5*sqrt(2)* 4))}
%%	\let\result\pgfmathresult
%%\pgfmathparse{(pow(5*sqrt(2),2) + 4*4 - 2*4*5*sqrt(2)*cos(55) + 4*4 - pow(5*sqrt(2), 2))/(2*5*sqrt(2)* 4)}
%%	\let\return\pgfmathresult
%%\pgfmathparse{(pow(5*sqrt(2),2) + 4*4 - 2*4*5*sqrt(2)*cos(55) + 4*4 - pow(5*sqrt(2), 2))}
%%	\let\numerator\pgfmathresult
\pgfmathparse {sqrt(25*2 + 4*4 - 2*4*5*sqrt(2)*cos(55))}
	\let\rectheight\pgfmathresult
\draw(0, 0) 
			%node [anchor=south west] {\result} %%90.45206 
			--++ ({acos( (pow(5*sqrt(2),2) + 4*4 - 2*4*5*sqrt(2)*cos(55) + 4*4 - pow(5*sqrt(2), 2))/(2*5*sqrt(2)* 4)) - 10}: 
			{sqrt(25*2 + 4*4 - 2*4*5*sqrt(2)*cos(55))}) ;
\draw (2.5, 2.5) node {A};
\draw[dashed, gray ](0, 0) --++ (-10:6) --++ (170:2) 
	--++ (250:2) --++ (70:9) --++ (250:2) --++ (340:2)
	--++ (160:7) --++ (295:{5*sqrt(2)}) --++ (115:8)
	--++ (295:{8 - 5*sqrt(2)}) --++ (160:1); 

\draw [dashed, gray](0, -8)
	--++ (0:{5*sqrt(2)*sin(55)})
	--++ (270:2)
	--++ (180:{5*sqrt(2)*sin(55)})
	--++ (90:2);
\draw [dashed, gray](0, -10) 
	--++ (270: {(2.5*5)/\rectheight}) 
	--++ (0:\rectheight) 
	--++ (90: {(2.5*5)/\rectheight});
\draw (0, -8) node[anchor=east]{B} 
	--++ (0:{5*sqrt(2)*sin(55)}) node [anchor=south west]{E};
\draw (0, -8) --++ (270: 2)
	--++ (270: {(2.5*5)/\rectheight}) node [anchor=east]{C}
	--++ (0:\rectheight) node [anchor=west]{D}
	--++ (90: {(2.5*5)/\rectheight})
	--++ (90:2);


\pgfmathparse{(2.5*5)/\rectheight}
	\let\recttwowidth\pgfmathresult
\draw (\rectheight, -8) 
	--++(0:{\recttwowidth + 2}) node [anchor=west]{F};

\pgfmathparse{2 + \recttwowidth}
	\let\rectwidth\pgfmathresult
\draw ({(\rectheight + \rectwidth)}, -8) arc(0:180:{(\rectheight + \rectwidth)/2});
\draw (\rectheight, -8) --++ (90:4.9) node [anchor=south]{H};

\pgfmathparse{\rectheight - ((\rectheight + \rectwidth)/2) }
	\let\triwidth\pgfmathresult

\draw ({(\rectheight + \rectwidth)/2}, -8) node [anchor=north]{G}
	--++ ({atan(4.9/\triwidth)}:{sqrt(4.9*4.9 + \triwidth*\triwidth)});

\draw [dash dot, gray] (\rectheight, -3.1) --++ (0: 4.9) --++(270:4.9)
	--++(180:4.9) --++ (90: 4.9);

\end{tikzpicture}
\end{center}

Misalkan ada $A$ adalah bidang rektilinear tertentu. Jadi diperlukan untuk
menggambar sebuah bujur sangkar yang sama besar dengan bidang rektilinear
$A$

Sebab misalkan persegi panjang $BD$ dengan sudut siku - siku, sama besar  
dengan bidang rektilinear $A$ sudah dibuat [Prop. 1.45]. Jadi,
jika $BE$  sama panjang dengan $ED$ maka bidang yang diperlukan sudah 
digambarkan. Karena bujur sangkar $BD$ sama besar dengan bidang $A$, sudah
digambarkan. Dan jika tidak, maka salah satu garis lurus tersebut $BE$ 
atau $ED$ lebih panjang dari yang lain. Misalkan $BE$ adalah yang lebih
panjang, dan misalkan sudah dipanjangkan sampai $F$, dan misalkan $EF$ 
sudah dibuat sama panjang dengan $ED$ [Prop. 1.3]. Dan misalkan $BF$ sudah
dibagi dua sama panjang pada titik $G$ [Prop. 1.10]. Dan dengan pusat $G$, 
dan jari - jari nya adalah salah satu garis lurus $GB$ atau $GF$, misalkan
setengan lingkaran $BHF$ sudah digambarkan. Dan misalkan $DE$ sudah 
dipanjangkan sampai $H$. Dan misalkan $GH$ sudah dihubungkan.

Dengan demikian, karena garis lurus $BF$ sudah dipotong sama panjang pada
$G$ dan secara tidak sama panjang pada $E$, persegi panjang yang dibentuk
$BE$ dan $EF$ ditambah bujur sangkar $EG$, dengan demikian sama besar denga
bujur sangkar pada $GF$ [Prop. 2.5]. Dan $GF$ sama panjang dengan $GH$. 
Dengan demikian, persegi panjang yang dibentuk oleh $BE$ dan dan $EF$, 
ditambah bujur sangkar pada $GE$, sama besar dengan bujur sangkar pada
$GH$. Dan jumlah dari besar bujur sangkar - bujur sangkar pada $HE$ dan $EG$
sama besar dengan bujur sangkar pada $GH$ [Prop[. 1.47]. Dengan demikian, 
persegi panjang yang dibentuk $BE$ dan $EF$, ditambah bujur sangkar pada $GE$,
sama besar dengan jumlah besar bujur sangkar - bujur sangkar pada $HE$ dan
$EG$. Misalkan bujur sangkar $GE$ sudah diambil dari keduanya. Dengan demikian, 
persegi panjang yang tersisa yang dibentuk $BE$ dan $EF$ sama besar dengan 
dengan bujur sangkar pada $EH$. Tetapi, $BD$ adalah persegi panjang yang 
dibentuk oleh $BE$ dan $EF$. Karena $EF$ sama panjang dengan $ED$. Dengan 
demikian, jajaran genjang $BD$ sama besar dengan  bujur sangkar pada $HE$.
Dan $BD$ sama besar dengan bidang rektilinear $A$. Dengan demikian, bidang
rektilinear $A$ juga sama besar dengan bujur sangkar pada $EH$.

Dengan demikian, bujur sangkar, yang digambarkan pada $EH$ sudah dibentuk, 
sama besar dengan bidang rektilinear $A$. Yang adalah hal yang perlu dilakukan.  

\stepcounter{chapter}
\chapter*{\centering \chaptername{ \thechapter} 
\addcontentsline{toc}{chapter}{Buku \thechapter}
Dasar - Dasar Geometri Bidang Datar Yang Melibatkan Lingkaran 
}

\begin{center}
\addcontentsline{toc}{section}{Definisi - definisi}
\textbf{Definisi - definisi}
\end{center} 
\begin{enumerate}
\item Beberapa lingkaran yang sama besar adalah yang diameter - diameter mereka
sama besar, atau jarak dari pusat ke garis keliling (busur) sama besar 
(yakni yang berjari - jari sama besar).
\item Garis lurus dikatakan menyinggung sebuah lingkaran jika garis lurus 
tersebut bertemu dengan lingkaran tersebut dan jika diperpanjang tidak 
memotong lingkaran tersebut.
\item Beberapa lingkaran dikatakan menyinggung  satu dengan yang lain jika
saling bertemu tetapi tidak saling memotong.
\item Dalam sebuah lingkaran, garis - garis lurus dikatakan sama jaraknya 
dari pusat, jika garis - garis tegak lurus yang ditarik pada mereka dari
pusat adalah sama panjang.
\item Dan dikatakan lebih jauh jika garis dari pusat lingkaran jatuh tegak
lurus padanya dengan jarak yang lebih besar.
\item Temberang dari sebuah lingkaran adalah bidang yang dibentuk oleh
sebuah garis lurus (disebut alas busur atau tali busur) dan sebuah busur 
lingkaran.
\item Dan sudut sebuah tembereng adalah sudut yang dibentuk oleh 
sebuah garis lurus dan sebuah busur.
\item Dan sudut di dalam sebuah tembereng adalah sudut yang dibentuk oleh
dua garis lurus pada satu titik pada busur tembereng, yang kedua garis tadi
masing - masing ditarik dari ujung - ujung tali busur tembereng.
\item Dan ketika garis - garis lurus membentuk sebuah sudut yang memotong 
beberapa busur, sudut tersebut dikatakan berdiri di atas busur tersebut.
\item Dan juring dari sebuah lingkaran adalah bidang yang dibentuk oleh
garis - garis lurus yang mengapit sebuah sudut di pusat lingkaran  dan busur 
tersebut dipotong oleh kedua garis tersebut.
\item Tembereng yang serupa dari sebuah lingkaran adalah yang memiliki sudut - 
sudut yang sama besar, atau yang pada busurnya sudut - sudut sama besar satu 
dengan yang lain.
\end{enumerate}

\stepcounter{section}
\section*{\centering Proposisi \thesection}
\addcontentsline{toc}{section}{Proposisi \thesection} 
Untuk menemukan pusat dari sebuah lingkaran.
\begin{center}
\begin{tikzpicture}
\pgfmathparse{sqrt(4*4-3*3)}
	\let\baselength\pgfmathresult
\draw (0, 0) node [anchor=east]{F} --++ (0:.1) --++ (180:.2) --++ (0:.1) 
	circle(4) 
	--++ (90:4) node[anchor=south]{C} --++ (270:8) node [anchor=north]{E};
\draw (0, -3) node [anchor=north east]{D} 
	--++(180:\baselength) node [anchor=east]{A}
	--(1, 0) node [anchor=south]{G} -- (0, -3)
	--++ (0:\baselength) node [anchor=west]{B} -- (1, 0);
\end{tikzpicture}
\end{center}
Misalkan $ABC$ lingkaran yang dimaksud. Jadi dibutuhkan untuk menemukan pusat
dari lingkaran $ABC$.

Misalkan sembarang garis $AB$ sudah ditarik melalui $ABC$, dan misalkan $AB$
sudah dibagi sama panjang di $D$ [Prop. 1.9]. Dan misalkan $DC$ sudah
ditarik dari $D$ dengan sudut siku - siku ke $AB$ [Prop. 1.11]. Dan misalkan
$CD$ sudah ditarik melalui $E$. Dan misalkan $CE$ sudah di potong ditengah pada
$F$ [Prop. 1.9]. Saya katakan bahwa titik $F$ adalah pusat dari lingkaran $ABC$.

Sebab jika tidak maka, jika memungkinkan, misalkan $G$ adalah pusat dari 
lingkaran, dan misalkan $GA$, $GD$ dan $GB$ sudah dihubungkan. Dan karena $AD$
sama panjang dengan $DB$, dan $DG$ milik bersama keduanya, kedua garis lurus
$AD$, $DG$ sama panjang dengan kedua garis lurus $BD$, $DG$. Dan sisi alas $GA$
sama panjang dengan $GB$. Karena keduanya adalah jari - jari. 
Dengan demikian sudut $ADG$ sama besar dengan sudut $GDB$
[Prop. 1.8]. Dan jika sebuah garis lurus berdiri diatas garis lurus yang lain
membentuk dua buah sudut bersisian yang sama besar, masing - masing sudut yang
sama besar tersebut adalah sudut siku - siku [Def. 1.10]. Dengan demikian, 
$GDB$ adalah sudut siku - siku. Dan $FDB$ juga adalah sudut siku - siku. Dengan
demikian $FDB$ sama besar dengan $GDB$, yang lebih besar sama besar dengan yang
lebih kecil. Hal tidak mungkin. Maka titik $G$ tidak berada pada pusat 
lingaran $ABC$. Dengan cara dapat ditunjukkan tidak ada  titik - titik yang 
lain selain $F$.


\begin{center}\textbf{Dalil}\end{center}

Dengan demikian, sudah diumumkan, bahwa jika sembarang garis lurus pada sebuah
lingkaran memotong garis lurus lain tepat ditengah dan pada sudut siku - siku,
maka pusat dari lingkaran ada pada  garis pertama tadi. Yang adalah hal yang
perlu dilakukan.

\stepcounter{section}
\section*{\centering Proposisi \thesection}
\addcontentsline{toc}{section}{Proposisi \thesection} 
Jika dua buah titik diambil secara acak pada busur sebuah lingkaran maka garis
lurus yang menghubungkan titik - titik tersebut akan berada di dalam lingkaran
tersebut.
\begin{center}
\begin{tikzpicture}
\draw (0, 0) node [anchor= south east]{D} circle (3);
\draw (0, 0) --++ (200: 3) node [anchor=east]{A} 
%%\draw (-1, -1) 
	arc(190:300:2.6) node [anchor=north] {B}-- (0, 0)
	--++ (250:3) node [anchor=south east]{F} 
	--++ (250:.235) node [anchor=north west]{E};
\draw ({3*sin(45)}, {3*cos(45)}) node [anchor = south west]{C};
\end{tikzpicture}
\end{center} 
Misalkan $ABC$ adalah sebuah lingkaran, dan misalkan dua buah titik $A$ dan $B$
sudah diambil secara acak pada busur nya. Saya katakan bahwa garis lurus yang 
menghubungkan $A$ dan $B$ berada di dalam lingkaran.

Sebab misalkan tidak, jika mungkin, misalkan garis tersebut berada di luar 
lingkaran tersebut, seperti $AEB$ pada gambar. Dan misalkan pusat lingkaran 
$ABC$ sudah ditemukan [Prop. 3.1], dan misalkan pusatnya adalah pada titik $D$.
Dan misalkan $DA$ dan $DB$ sudah dihubungkan, dan misalkan $DFE$ sudah ditarik.

Dengan demikian, karena $DA$ sama panjang dengan $DB$, sudut $DAE$ dengan 
demikian sama besar dengan sudut $DBE$ [Prop. 1.5]. Dan karena pada segitiga
$DAE$ satu sisi $AE$ dipanjangkan menjadi $AEB$, sudut $DEB$ dengan demikian 
lebih besar dari sudut $DAE$ [Prop. 1.16]. Dan $DAE$ sama besar dengan $DBE$ 
[Prop. 1.5]. Dengan demikian, $DEB$ lebih besar dari $DBE$, dan sudut yang
lebih besar akan menghadap sisi yang lebih panjang [Prop. 1.19]. Dengan
demikian, $DB$ lebih panjang dari $DE$. Dan $DB$ sama panjang dengan $DF$. 
Dengan demikian, $DF$ harusnya lebih besar dari $DE$, yang lebih panjang lebih
pendek dari yang seharusnya yang lebih pendek. Hal yang tidak mungkin. Dengan
demikian, garis lurus yang menghubungkan $A$ dan $B$ tidak akan berada di luar
lingkaran. Jadi, dengan cara yang sama, kita dapat buktikan juga garis tersebut
juga tidak berada pada busur. Dengan demikian, garis tersebut akan berada di 
dalam lingkaran.

Jadi, jika dua buah titik secara acak diambil pada busur sebuah lingkaran, maka
garis lurus yang menghubungkan kedua titik tersebut akan berada di dalam 
lingkaran. Yang adalah hal yang perlu ditunjukkan. [Selanjutnya garis
yang menghubungkan dua buah titik pada busur sebuah lingkaran disebut
tali busur].

\stepcounter{section}
\section*{\centering Proposisi \thesection}
\addcontentsline{toc}{section}{Proposisi \thesection} 
Pada sebuah lingkaran, jika sembarang tali busur yang melalui pusat lingkaran
memotong sama panjang sembarang tali busur lainnya yang tidak melalui 
pusat 
lingkaran, maka garis pertama memotong garis yang kedua secara tegak lurus.
Dan sebaliknya, jika ia memotongnya secara tegak lurus maka ia memotong garis
tersebut menjadi dua bagian yang sama besar.

\begin{center}
\begin{tikzpicture}
\draw (0, 0) node [anchor=south west]{E} circle (3) 
	--++ (220:3) node [anchor=east]{A}
	--++ (0: {3*cos(220)*-1}) node [anchor=north west]{F}
	--++ (0: {3*cos(220)*-1}) node [anchor=west]{B}
	--++ (140:3)
	--++ (90:3) node [anchor=south]{C}
	--++ (270:6) node [anchor=north]{D};
\end{tikzpicture}
\end{center}

Misalkan $ABC$ adalah sebuah lingkaran, dan di dalamnya, misalkan sembarang 
tali busur melalui pusat lingkaran, $CD$, memotong sama panjang 
sembarang tali
busur lain yang tidak melalui pusat lingkaran, $AB$, pada titik $F$. 
Saya katakan bahwa $CD$ juga memotong $AB$ secara tegak lurus.

Sebab misalkan pusat dari lingkaran $ABC$ sudah ditemukan [Prop. 3.1], dan 
misalkan berada pada titik $E$, dan misalkan $EA$ dan $EB$ sudah dihubungkan.

Dan karena $AF$ sama panjang dengan $FB$, dan $FE$ sama, kedua sisi
$AFE$ sama panjang dengan kedua sisi $BFE$. Dan sisi alas $EA$ sama panjang 
dengan sisi alas $EB$. Dengan demikian, sudut $AFE$ sama besar dengan sudut
$BFE$ [Prop. 1.8]. Dan bila sebuah garis lurus berdiri di atas garis lurus lain
dan membentuk sudut - sudut yang bersisian yang sama besar, kedua sudut yang
sama besar itu adalah sudut siku - siku [Def. 1.10]. Dengan demikian, $AFE$ dan
$BFE$ masing - masing adalah sudut siku - siku. Dengan demikian, garis lurus
$CD$, yang melalui pusat lingkaran dan memotong ditengah garis lurus $AB$, yang
tidak melalui pusah lingkaran, juga memotong $AB$ secara tegak lurus.

Dan juga misalkan $CD$ memotong $AB$ secara tegak lurus. Saya katakan bahwa
dia juga memotong $AB$ ditengah. Yang juga menyatakan, $AF$ sama panjang dengan
$FB$.

Karena, dengan susunan yang sama, karena $EA$ sama panjang dengan $EB$, sudut
$EAF$ juga sama besar dengan sudut $EBF$ [Prop. 1.5]. Dan sudut siku $AFE$ juga
sama besar dengan sudut siku $BFE$. Dengan demikian, $EAF$ dan $EFB$ adalah dua
buah segitiga dengan dua buah sudut pada yang satu sama besar dengan dua buah
sudut pada segitiga yang satunya, dan satu sisi pada yang satu sama panjang 
dengan satu sisi pada yang lainnya, yakni sisi yang adalah sama 
$EF$, yang menghadap sudut - sudut yang sama pada keduanya. Dengan demikian, 
sisi - sisi selebihnya juga sama panjang pada kedua segitiga [Prop. 1.26]. 
Dengan demikian, $AF$ sama panjang dengan $FB$.

Dengan demikian, pada sebuah lingkaran, jika sembarang garis lurus yang 
melalui pusat lingkaran memotong sama panjang sembarang garis lurus 
lainnya yang 
tidak melalui pusat lingkaran, maka garis pertama memotong garis yang kedua 
secara tegak lurus.  Dan sebaliknya, jika ia memotongnya secara tegak lurus 
maka ia memotong garis tersebut menjadi dua bagian yang sama besar. Yang adalah
hal yang perlu ditunjukkan.

\stepcounter{section}
\section*{\centering Proposisi \thesection}
\addcontentsline{toc}{section}{Proposisi \thesection} 
Pada sebuah lingkaran, jika dua buah tali busur, tidak melalui pusat lingkaran,
saling memotong satu dengan yang lain, maka keduanya tidak akan saling memotong 
yang lainnya tepat sama panjang. (Walaupun salah satu bisa memotong yang 
lainnya tepat sama panjang).

\begin{center}
\begin{tikzpicture}
\draw (0, 0) node [anchor=south]{F} circle(3)
	--++ (270: 1.5) node [anchor=north]{E}
	--++ (340:2.14) node [anchor=west]{C}
	--++ (160:5.3) node [anchor=east] {A}
	--++ (340:5.3 - 2.14)
	--++ (220:1.8) node [anchor=north east]{B}
	--++ (40:5.53) node [anchor=south west] {D};
\end{tikzpicture}
\end{center}

Misalkan $ABCD$ adalah sebuah lingkaran, dan di dalamnya misalkan dua buah 
garis lurus, $AC$ dan $BD$, yang tidak melalui pusat, saling memotong di titik
$E$. Saya katakan bahwa keduanya tidak saling memotong yang lainnya tepat
sama panjang.

Sebab, jika mungkin, misalkan keduanya saling memotong sama panjang , sehingga $AE$
sama panjang dengan $EC$, dan $BE$ sama panjang dengan $ED$. Dan misalkan 
pusat lingkaran $ABCD$ sudah ditemukan [Prop. 3.1], dan misalkan pusat yang
dimaksud adalah titik $F$, dan misalkan $FE$ sudah dihubungkan.

Dengan demikian, garis yang melalui pusat, $FE$, memotong garis lain yang
tidak melalui pusat, $AC$, maka akan memotongnya juga secara siku - siku 
[Prop. 3.3]. Dengan demikian $FEA$ adalah sudut siku - siku. Dan lagi, karena
garis $FE$ memotong sama panjang garis lainnya $BD$, maka ia memotongnya secara
siku - siku [Prop. 3.3]. Dengan demikian $FEB$ adalah sudut siku - siku. 
Tetapi $FEA$ sudah ditunjukkan tadi sebagai sudut siku - siku. Dengan demikian
$FEA$ sama besar dengan $FEB$, yang lebih kecil sama besar dengan yang
lebih besar atau bagian sama besar dengan keseluruhan
Yang adalah mustahil (Karena keduanya disebutkan berpotongan).
 Dengan demikian, $AC$ dan $BD$ tidak saling memotong sama panjang.

Dengan demikian, pada sebuah lingkaran, jika dua buah tali busur, tidak 
melalui pusat lingkaran,
saling memotong satu dengan yang lain, maka keduanya tidak akan saling memotong 
yang lainnya tepat sama panjang. (Walaupun salah satu bisa memotong yang lainnya 
tepat sama panjang).  Yang adalah hal yang perlu ditunjukkan.  

\begin{center}
\textbf{Dalil}
\end{center}
Jika dua tali busur lingkaran saling memotong sama panjang, maka keduanya adalah 
diameter lingkaran.

\stepcounter{section}
\section*{\centering Proposisi \thesection}
\addcontentsline{toc}{section}{Proposisi \thesection} 
Jika dua buah lingkaran saling berpotongan, maka titik pusat keduanya adalah  
tidak sama.
\begin{center}
\begin{tikzpicture}
\draw (0, 0) circle (3);
\draw (2, -1) circle (3.5); 
\draw (-2.85, .9) node [anchor=east]{A};
\draw (-1, -2.7) node [anchor=north east]{B};
\draw (1.4, 2.5) node[anchor=south west]{C};
\draw (-1.3, 0) node [anchor=east]{D};
\draw (-.5, -.5) node[anchor=east]{E}--++ (315:3) node [anchor=west]{F}
	--++(315:2.32) node [anchor=north west]{G};
\draw (-.5, -.5) -- (1.685, 2.49);
\end{tikzpicture}
\end{center}

Misalkan dua buah lingkaran $ABC$ dan $CDG$ saling memotong di titik $B$ dan 
titik $C$. Saya katakan bahwa pusat keduanya adalah titik yang berbeda.

Sebab jika mungkin, misalkan $E$ jadi pusat keduanya, dan misalkan $EC$ sudah 
dihubungkan, dan misalkan $EFG$ sudah ditarik melalui kedua lingkaran secara
acak. Dan karena titik $E$ adalah pusat lingkaran $ABC$, $EC$ sama panjang 
dengan $EF$. Lagi, karena titik $E$ adalah pusat lingkaran $CDG$, $EC$ sama
panjang dengan $EG$. Tetapi $EC$ sudah ditunjukkan sama panjang dengan $EF$.
Dengan demikian, $EF$ sama panjang dengan $EG$, yang lebih pendek sama panjang
dengan yang lebih panjang atau bagian sama besar dengan keseluruhan. 
Yang adalah hal mustahil. Dengan demikian, titik $E$
bukanlah pusat kedua lingkaran $ABC$ dan $CDG$.

Dengan demikian, jika dua buah lingkaran saling memotong maka keduanya tidak
memiliki titik pusat yang sama. Yang adalah hal yang perlu ditunjukkan.

\stepcounter{section}
\section*{\centering Proposisi \thesection}
\addcontentsline{toc}{section}{Proposisi \thesection}
Jika dua buah lingkaran saling bersinggungan , maka keduanya memiliki titik 
pusat yang berbeda.
\begin{center}
\begin{tikzpicture}
\draw (0, 0) circle (3) node [anchor=north east] {F};
\draw (1, -1) circle ({3+sqrt(2)});
\draw (0, 0) --++ (135:3) node [anchor=south east]{C}
	--++ (180+135:3) --++ (340:3.) node [anchor=south west]{E}
	--++(340:2.65) node[anchor=west]{B};
\draw (0, -3) node [anchor=north]{D};
\draw (0, -5.5) node {A};
\end{tikzpicture}
\end{center}
Sebab misalkan dua buah lingkaran $ABC$ dan $CDE$ bersinggungan di titik $C$.
Saya katakan bahwa titik pusat keduanya adalah berbeda.

Sebab jika memungkinkan, misalkan $F$ adalah pusat keduanya, dan $FC$ sudah 
dihubungkan, dan misalkan $FEB$  secara sembarang sudah ditarik melalui 
keduanya.

Karena itu, karena titik $F$ adalah pusat dari lingkaran $ABC$, $FC$ sama 
panjang dengan $FB$. Lagi, karena titik $F$ adalah pusat lingkaran $CDE$, $FC$
sama panjang dengan $FE$. Tetapi $FC$ sudah ditunjukkan sama panjang dengan
$FB$. Dengan demikian $FE$ sama panjang dengan $FB$, bagian sama panjang dengan
keseluruhan. Yang adalah hal yang tidak mungkin. Dengan demikian, titik $F$
bukanlah titik pusat milik kedua $ABC$ dan $CDE$.

Dengan demikian, jika dua buah lingkaran saling bersinggungan, maka keduanya
memiliki titik pusat yang berbeda. Yang adalah hal yang perlu ditunjukkan.

\stepcounter{section}
\section*{\centering Proposisi \thesection}
\addcontentsline{toc}{section}{Proposisi \thesection}
Jika sembarang garis, yang bukan pusat dari lingkaran, berada pada diameter
lingkaran, dan beberapa garis lurus memancar (sebagai garis lurus) dari titik 
tersebut ke arah busur lingkaran, maka garis (pancaran) yang terpanjang adalah 
yang padanya titik pusat lingkaran berada, dan yang terpendek adalah garis 
sisa dari diameter. Dan untuk garis - garis pancaran yang
lain\footnote{Kira - kira, dengan peninjauan sudut}, garis yang lebih dekat 
dengan garis yang melalui pusat lingkaran akan 
lebih panjang dari garis yang lebih jauh dari yang melalui pusat lingkaran. 
Dan hanya akan ada dua garis yang sama panjang yang memancar dari titik 
tersebut ke busur lingkaran, satu ke masing - masing sisi dari garis terpendek.  
\begin{center}
\begin{tikzpicture}
\draw(0, 0) node [anchor=north east]{E} circle(4) 
	--++(0:4) node[anchor=west]{D}
	--++(180:8) node[anchor=east]{A};
\draw(0, 0) --++(120:4) node[anchor=south east]{B} 
	--(1, 0) node[anchor=south west]{F};
\draw(0, 0) --++(100:4) node [anchor=south east]{C} -- (1, 0);
\draw(0, 0) --++(70:4) node [anchor=south]{G} -- (1, 0);
\draw(0, 0) --++(-70:4) node[anchor=north]{H} -- (1, 0);
\draw[dashed](0, 0) --++ (-80:4) node[anchor=north]{K}; 
\draw(1, 0) --+ (266: {sqrt(4*4 + 1*1 - 2*4*1*cos(-80))});

\end{tikzpicture}
\end{center}

Misalkan $ABCD$ adalah sebuah lingkaran, dan misalkan $AD$ adalah diameternya, 
dan misalkan sembarang titik $F$, yang bukan titik pusat lingkaran, sudah 
diletakkan pada $AD$. Misalkan $E$ adalah pusat lingkaran. Dan misalkan 
beberapa garis lurus, $FB$, $FC$, $FG$, memancar dari $F$ ke busur lingkaran.
Saya katakan bahwa $FA$ adalah garis lurus terpanjang, $FD$ garis terpendek,
dan yang lain,  $FB$ lebih panjang dari $FC$, dan $FC$ lebih panjang dari 
$FG$.

Sebab misalkan $BE$, $CE$, dan $GE$ sudah dihubungkan. Dan karena untuk setiap
segitiga, jumlah panjang dua buah sisi akan lebih besar dari panjang satu sisi 
selebihnya [Prop. 1.20], jumlah $EB$ dan $EF$ dengan demikian lebih panjang 
dari
$BF$. Dan $AE$ sama panjang dengan $BE$ [dengan demikian, $BE$ dan $EF$ sama
panjang dengan $AF$]. Dengan demikian, $AF$ lebih panjang dari $BF$. Lagi,
karena $BE$ sama panjang dengan $CE$, dan $FE$ ada pada keduanya, masing
kedua garis lurus $BE$, $EF$ sama panjang dengan kedua garis lurus $CE$, $EF$.
Tetapi, sudut $BEF$ juga lebih besar dari sudut $CEF$ \footnote{Hal ini tidak
dibuktikan, hanya dengan menggunakan gambar sebagai referensi}. Dengan 
demikian, sisi alas $BF$ juga lebih besar dari sisi alas $CF$ [Prop. 1.24]. 
Jadi, untuk alasan - alasan yang sama, $CF$ juga  lebih besar dari $FG$.

Lagi, karena $GF$ dan $FE$ lebih panjang dari $EG$ [Prop. 1.20], dan $EG$ sama
panjang dengan $ED$, $GF$ dan $FE$ dengan demikian lebih panjang dari $ED$.
Misalkan $EF$ sudah diambil dari keduanya. Dengan demikian $GF$ lebih panjang
dari $FD$. Dengan demikian, $FA$ adalah garis lurus terpanjang dan $FD$ adalah
yang terpendek, dan $FB$ lebih panjang dari $FC$ dan $FC$ lebih panjang dari 
$FG$.

Saya juga katakan dari titik $F$ hanya ada dua garis lurus sama panjang
yang memancar ke arah busur lingkaran $ABCD$, satu di masing - masing sisi
dari garis lurus terpendek $FD$. Sebab misalkan sudut $FEH$, sama besar dengan
sudut $GEF$, sudah dibuat pada garis lurus $EF$, di titik $E$ [Prop. 1.23]. 
Dan misalkan $FH$ sudah dihubungkan. Oleh karena itu, disebabkan $GE$ sama 
panjang dengan $EH$, dan $EF$ ada pada keduanya, masing - masing kedua garis 
lurus $GE$, $EF$ sama panjang dengan kedua garis lurus $HE$, $EF$. Dan sudut 
$GEF$ sama besar dengan sudut $HEF$. Dengan demikian sisi alas $FG$ sama 
panjang  dengan sisi alas $FH$ [Prop. 1.4]. Jadi saya katakan bahwa garis lain
sama panjang dengan $FG$ tidak akan  memancar ke busur lingkaran dari titik 
$F$. Sebab misalkan ada, misalkan $FK$ adalah garis tersebut. Dan karena $FK$
sama panjang dengan $FG$, tetapi $FH$ juga sama panjang dengan $FG$, $FK$ 
dengan demikian juga sama panjang dengan $FH$, yang lebih dekat dengan garis
yang melalui pusat lingkaran sama panjang dengan garis yang lebih jauh. Hal
yang tidak mungkin. Dengan demikian, garis lain yang sama panjang dengan $GF$
tidak akan memancar dari titik $F$ ke busur lingkaran. Jadi hanya ada satu
yang bisa.

Dengan demikian, jika sembarang garis, yang bukan pusat dari lingkaran, 
berada pada diameter lingkaran, dan beberapa garis lurus memancar 
(sebagai garis lurus) dari titik tersebut ke arah busur lingkaran, 
maka garis (pancaran) yang terpanjang adalah 
yang padanya titik pusat lingkaran berada, dan yang terpendek adalah garis 
sisa dari diameter. Dan untuk garis - garis 
pancaran yang lain, garis yang 
lebih dekat 
dengan garis yang melalui pusat lingkaran akan 
lebih panjang dari garis yang lebih jauh dari garis yang melalui pusat lingkaran. 
Dan hanya akan ada dua 
garis yang sama panjang yang memancar dari titik tersebut ke busur lingkaran, 
satu ke masing - masing sisi arah dari garis lurus terpendek.  

\stepcounter{section}
\section*{\centering Proposisi \thesection}
\addcontentsline{toc}{section}{Proposisi \thesection}
Jika sebuah titik diletakkan di luar sebuah lingkaran, dan beberapa garis lurus
ditarik dari titik tersebut ke beberapa titik di busur lingkaran, salah satu
nya melewati lingkaran dan melalui pusat lingkaran, maka untuk garis - garis
yang memancar ke arah bagian yang cekung dari busur lingkaran, yang terpanjang
adalah yang melalui titik pusat lingkaran. Untuk garis - garis yang memancar
ke arah bagian yang cembung dari lingkaran, yang terpendek adalah yang berada
diantara titik tersebut dari diameter.  Untuk yang lainya, garis yang lebih
dekat \footnote{Kira - kira, dengan peninjauan sudut} garis terpendek selalu
lebih pendek dari garis yang lebih jauh. Dan hanya akan ada dua garis lurus 
yang sama panjang yang memancar dari titik tersebut ke arah busur lingkaran,
satu ke masing - masing sisi garis terpendek.

\begin{center}
\begin{tikzpicture}
\draw (0, 0) node [anchor=north] {M} circle(4) 
	--++ (30:4) node [anchor=south west]{G} 
	-- ({8*cos(30)}, {8*sin(30)}) node[anchor=west]{D}
	--++ (210:12) node [anchor=east]{A};
\draw (0, 0) --++ (190: 4) node [anchor=east]{E}
	-- ({8*cos(30)}, {8*sin(30)});
\draw (0, 0) --++ (36.5: 4) node [anchor=south west]{K};
\draw (0, 0) --++ (170: 4) node [anchor=east]{F}
	-- ({8*cos(30)}, {8*sin(30)});
\draw (0, 0) --++ (43.8: 4) node [anchor=south west]{L};
\draw (0, 0) --++ (150: 4) node [anchor=east]{C}
	-- ({8*cos(30)}, {8*sin(30)});
\draw (0, 0) --++ (51.5: 4) node [anchor=south west]{H};
\draw (0, 0) --++ (23: 4) node [anchor=south west]{B}
	-- ({8*cos(30)}, {8*sin(30)}); 
\draw [dashed](0, 0) --++ (15: 4) node [anchor=south west]{N};
\draw ({4*cos(15)}, {4*sin(15)})
	-- ({8*cos(30)}, {8*sin(30)});



\end{tikzpicture}
\end{center}

Misalkan $ABC$ adalah sebuah lingkaran, dan misalkan sembarang titik $D$ sudah
ditempatkkan di luar lingkaran $ABC$, dan darinya misalkan beberapa garis lurus, $DA$, $DE$, $DF$, dan $DC$, sudah ditarik melalui lingkaran, dan misalkan 
$DA$ sudah melalui pusat lingkaran. Saya katakan bahwa untuk garis - garis 
yang memancar ke arah bagian cekung lingkaran, $AEFC$, yang terpanjang adalah
yang melalui pusat lingkaran, yakni $AD$, dan bahwa $DE$ lebih panjang dari
$DF$, dan $DF$ dari $DC$. Dan untuk garis - garis yang memancar ke bagian
cembung lingkaran $HLGK$, yang terpendek adalah yang berada di antara titik 
tersebut dan diameter $AG$, yakni $DG$, dan garis - garis yang lebih dekat 
dengan garis pancaran terpendek $DG$ ini akan selalui lebih pendek dari yang
lebih jauh dari $DG$, sehingga $DK$ lebih pendek dari $DL$, dan $DL$ dari $DH$.

Sebab misalkan pusat dari lingkaran sudah ditemukan [Prop. 3.1], dan misalakan
itu adalah titik $M$ [Prop. 3.1]. Dan misalkan $ME$, $MF$, $MC$, $MK$, $ML$
dan $MH$ sudah dihubungkan.

Dan karena $AM$ sama panjang dengan $EM$, misalkan $MD$ sudah ditambahkan
pada keduanya. Dengan demikian, $AD$ sama panjang dengan $EM$ dan $MD$. Tetapi
$EM$ dam $MD$ lebih panjang dari $ED$ [Prop. 1.20]. Dengan demikian $AD$ juga
lebih panjang dari $ED$. Lagi, karena $ME$ sama panjang dengan $MF$, dan $MD$
sama, masing - masing garis lurus $EM$, $MD$ sama panjang dengan
garis lurus $FM$, $MD$. Dan sudut $EMD$ lebih besar dari sudut $FMD$ 
\footnote{Hal ini tidak dibuktikan, hanya dengan pengamatan gambar}. Dengan
demikian, sisi alas $ED$ lebih panjang dari sisi alas $FD$ [Prop. 1.24]. 
Sehingga, dengan cara yang sama, dapat ditunjukkan $FD$ juga lebih panjang 
dari $CD$. Dengan demikian, $AD$ adalah garis yang terpanjang, dan $DE$
lebih panjang dari $DF$, dan $DF$ dari $DC$.

Dan karena $MK$ dan $KD$ lebih panjang dari $MD$ [Prop. 1.20], dan $MG$ sama
panjang dengan $MG$ sama panjang dengan $MK$, ruas garis selebihnya $KD$ 
dengan demikian lebih panjang dari ruas garis selebihnya $GD$. Dengan demikian
$GD$ lebih pendek dari $KD$. Dan karena pada segitiga $MLD$, kedua garis lurus
interior $MK$, $KD$ digambarkan pada salah satu sisi - sisi, $MD$, maka $MK$
dan $KD$ lebih pendek dari $ML$ dan $LD$ [Prop. 1.21]. Dan $MK$ sama panjang
dengan $ML$. Dengan demikian, ruas garis selebihnya $DK$ lebih pendek dari
ruas garis selebihnya $DL$. Jadi, dengan cara yang sama, dapat ditunjukkan
bahwa $DL$ juga lebih pendek dari $DH$. Dengan demikian, $DG$ adalah garis
lurus yang terpendek, dan $DK$ lebih pendek dari $DL$, dan $DL$ dari $DH$.

Saya juga katakan bahwa hanya ada dua garis lurus yang sama panjang yang 
memancar dari $D$ ke arah busur lingkaran, satu untuk masing - masing sisi
garis terpendek, $DG$. Misalkan sudut $DMB$, sama besar dengan sudut $KMD$,
sudah ditarik pada garis lurus $MD$, dengan titik $M$ padanya [Prop. 1.23],
dan misalkan $DB$ sudah dihubungkan. Dan karena $MK$ sama panjang dengan $MK$,
dan $MD$ sama, masing - masing kedua garis lurus $KM$, $MD$ sama panjang 
dengan kedua garis lurus $BM$, $MD$. Dan sudut $KMD$ sama besar dengan sudut
$BMD$. Dengan demikian, sisi alas $DK$ sama panjang dengan sisi alas $DB$ 
[Prop. 1.4]. Jadi, saya katakan bahwa tidak ada garis lurus lain yang akan sama
panjang dengan $DK$ yang akan memancar ke busur lingkaran dari titik $D$. 
Sebab jika mungkin, misalkan garis lurus tersebut memancar, dan misalkan itu
adalah $DN$. Karena itu, karena $DK$ sama panjang dengan $DN$, tetapi $DK$
sama panjang dengan $DB$, maka $DB$ demikian sama panjang dengan $DN$, 
sehingga garis lurus yang lebih dekat dengan garis lurus terpendek $DG$ sama
panjang dengan garis lain yang lebih jauh. Hal demikian sudah ditunjukkan 
tidak mustahil. Dengan demikian, tidak ada lebih daru dua garis lurus yang 
sama panjang yang memancar ke arah busur lingkaran $ABC$ dari titik $D$, satu
dari masing - masing sisi garis lurus terpendek $DG$.


Dengan demikian, jika sebuah titik diletakkan di luar sebuah lingkaran, 
dan beberapa garis lurus
ditarik dari titik tersebut ke beberapa titik di busur lingkaran, salah satu
nya melewati lingkaran dan melalui pusat lingkaran, maka untuk garis - garis
yang memancar ke arah bagian yang cekung dari busur lingkaran, yang terpanjang
adalah yang melalui titik pusat lingkaran. Untuk garis - garis yang memancar
ke arah bagian cembung dari lingkaran, yang terpendek adalah yang berada di
antara titik tersebut dan diameter.  Untuk yang lainya, garis yang lebih
dekat garis terpendek selalu
lebih pendek dari garis yang lebih jauh. Dan hanya akan ada dua garis lurus 
yang sama panjang yang memancar dari titik tersebut ke arah busur lingkaran,
satu ke masing - masing sisi garis terpendek. Yang adalah hal yang harus
ditunjukkan.

\stepcounter{section}
\section*{\centering Proposisi \thesection}
\addcontentsline{toc}{section}{Proposisi \thesection}
Jika sebuah titik diletakkan di dalam sebuah lingkaran, dan beberapa garis 
garis yang sama panjang memancar dari titik tersebut ke beberapa titik di
busur lingkaran, maka titik yang diletakkan tadi adalah pusat lingkaran.

Misalkan $ABC$ adalah sebuah lingkaran, dan $D$ adalah sebuah titik di 
dalamnya, dan misalkan lebih dari dua garis lurus yang sama panjang $DA$, 
$DB$, dan $DC$, memancar dari $D$ ke arah busur lingkaran $ABC$. Saya katakan 
titik $D$ adalah pusat dari lingkaran $ABC$.

\begin{center}
\begin{tikzpicture}
\pgfmathparse{sqrt(4*4+4*4 - 2*16*cos(220 - 150))}
	\let\ablength\pgfmathresult
\pgfmathparse{acos((4*4 + \ablength * \ablength - 4*4)/(2*4*\ablength))}
	\let\dabdir\pgfmathresult
\pgfmathparse{sqrt(4*4 + (\ablength/2) * (\ablength/2) - 2*4*(\ablength/2)*cos(\dabdir))}
	\let\aelength\pgfmathresult
\pgfmathparse{acos((\aelength*\aelength + (\ablength/2) * (\ablength/2) - 4*4)/(2*\aelength*(\ablength/2)))}
	\let\aeddir\pgfmathresult

\pgfmathparse{sqrt(4*4+4*4 - 2*16*cos(150 - 55))}
	\let\cblength\pgfmathresult
\pgfmathparse{acos((4*4 + \cblength * \cblength - 4*4)/(2*4*\cblength))}
	\let\dbfdir\pgfmathresult
\pgfmathparse{sqrt(4*4 + (\cblength/2) * (\cblength/2) - 2*4*(\cblength/2)*cos(\dbfdir))}
	\let\fdlength\pgfmathresult
\pgfmathparse{acos((\fdlength*\fdlength + (\cblength/2) * (\cblength/2) - 4*4)/(2*\fdlength*(\cblength/2)))}
	\let\bfddir\pgfmathresult

\draw (0, 0) node[anchor=north west]{D} circle(4);
\draw (0, 0) --++ (220:4) node [anchor=north east]{A}
	--++ ({220+180}:4)
	--++ (150:4) node [anchor=south east]{B}
	--++ ({150+180} :4)
	--++ (55:4) node [anchor=south west]{C};
%\draw (0, 0) --++ (220:4) --++ ({acos(4*4 + \ablength *\ablength - 4*4)/(2*4*8)}: {\ablength});
%\draw (0, 0) --++ (220:4) --++ ({acos((4*4 + \ablength *\ablength - 4*4)/(2*4*\ablength))+40 }: {\ablength});
\draw (0, 0) --++ (220:4) --++ ({\dabdir + 40 }: {\ablength});
\draw (0, 0) --++ (220:4) --++ ({\dabdir + 40 }: {\ablength/2}) node [anchor=south west]{E}
  --++ ({\aeddir + \dabdir + 40 + 180}:\aelength + 4) node [anchor=west]{G}
  --++ ({\aeddir + \dabdir + 40}:8) node [anchor=east]{K};
\draw (0, 0) --++ (150:4) --++ ({\dbfdir + 150 + 180} : \cblength);
\draw (0, 0) --++ (150:4) --++ ({\dbfdir + 150 + 180} : \cblength/2) node [anchor=south west]{F}
  --++ ({150 + \dbfdir + \bfddir}:\fdlength + 4) node [anchor=north]{H}
  --++ ({150 + \dbfdir + \bfddir - 180}:8) node [anchor=south]{L}; 
\end{tikzpicture}
\end{center}

Sebab misalkan $AB$ dan $BC$ sudah dihubungkan, dan masing - masing keduanya 
sudah di potong ditengah pada titik $E$ dan $F$ [Prop 1.10]. Dan $ED$ dan $FD$
sudah dihubungkan, dan misalkan keduanya sudah ditarik sampai ke titik - titik
$G$, $K$, $H$, dan $L$.

Dengan demikian, karena $AE$ sama panjang dengan $EB$, dan $ED$ milik bersama, masing - 
masing kedua garis lurus $AE$, $ED$ sama panjang dengan kedua garis lurus $BE$,
$ED$. Dan sisi alas $DA$ sama panjang dengan sisi alas $DB$. Dengan demikian,
sudut $AED$ sama besar dengan sudut $BED$ [Prop. 1.8]. Dengan demikian, masing - masing sudut $AED$ dan $BED$ adalah sudut siku - siku [Def. 1.10]. Dengan 
demikian, $GK$ membagi $AB$ sama panjang dan secara tegak lurus. Dan karena
jika sembarang garis lurus di dalam sebuah lingkaran, memotong garis lurus lain
sama panjang dan secara siku - siku, maka pusat lingkaran berada di garis
pertama tadi [Prop. 3.1], titik pusat lingkaran dengan demikian ada di garis
$GK$.  Dengan alasan - alasan yang sama, pusat dari lingkaran $ABC$ juga ada di
garis $HL$. Dan garis lurus $GK$ dan $HL$ tidak memiliki titik temu yang lain
selain di titik $D$. Dengan demikian, titik $D$ adalah pusat dari lingkaran
$ABC$.

Dengan demikian, jika sebuah titik diletakkan di dalam sebuah lingkaran, dan
lebih dari dua buah garis sama panjang memancar dari titik tersebut ke busur
lingkaran, maka titik tersebut sudah diletakkan di titik pusat lingkaran. Yang
adalah hal yang perlu ditunjukkan.

\stepcounter{section}
\section*{\centering Proposisi \thesection}
\addcontentsline{toc}{section}{Proposisi \thesection}
Sebuah lingkaran tidak akan memotong lingkaran lainnya di lebih dari
dua titik.

\begin{center}
\begin{tikzpicture}
\draw (0, 0) circle(4) node[anchor=north west]{P};
\draw [rotate=20](0, 0) circle(5 and 2.5);
\draw ({4*cos(20+25.8}, {4*sin(20+25.8)}) node[anchor=south]{B}
	--++ ({270+20}:{4*sin(25.8}) node[anchor=south east]{L}
	--++ ({270+20}:{4*sin(25.8}) node[anchor=north west]{G}
	--++ ({90+20}:{4*sin(25.8}) 
	--++ (20:{4 - 4*cos(25.8)}) node[anchor=north west]{O}
	--++ (20:1) node[anchor=west]{E}
	--(0, 0) --++ ({270+20}:4) node[anchor=north]{C}
	--++ ({90+20}:{4+4*(sin(25.8))}) node[anchor=north west]{K}
	--++ (20:{4*cos(25.8)})
	--++ ({180 + 20}:{2*4*cos(25.8)}) node [anchor=south east]{H}
	--++ ({20}:{4*cos(25.8)}) 
	--++ ({90+20}: {2.5 - 4*sin(25.8)}) node[anchor=south west]{D}
	--++ ({90+20}: {4 - 2.5}) node[anchor=south]{A}
	--++ ({270+20}: 4) 
	--++ ({180+20}: 4) node[anchor=south west]{N}
	--++ ({180+20}: 1) node[anchor=east]{M};
\draw (-{4*cos(20+25.8}, -{4*sin(20+25.8)}) node[anchor=north]{F};
\end{tikzpicture}
\end{center}

Sebab, jika mungkin, misalkan lingkaran $ABC$ memotong lingkaran $DEF$
pada lebih dari dua buah titik, $B$, $G$, $F$, dan $H$. Dan $BH$ dan 
$BG$ dihubungkan, dan misalkan masing - masing keduanya sudah dibagi 
sama panjang di titik $K$ dan $L$. Dan $KC$ dan $LM$ ditarik secara tegak
lurus terhadap $BH$ dan $BG$ dari masing - masing titik $K$ dan $L$ 
[Prop. 1.11], misalkan masing - masing sudah ditarik melalui titik $A$ 
dan $E$.

Dengan demikian, karena pada lingkaran $ABC$ sembarang garis lurus $AC$
memotong garis lainnya $BH$ sama panjang dan secara tegak lurus, 
pusat dari 
lingkaran $ABC$ dengan demikian ada di titik $AC$ [Prop. 3.1]. Dan 
lagi, karena pada lingkaran $ABC$ yang sama sembarang garis lurus $NO$
memotong garis lurus yang lain $BK$ sama panjang dan secara tegak lurus,
pusat dari dari lingkaran $ABC$ dengan demikian berada pada titik $NO$
[Prop. 3.1]. Dengan cara yang sama juga dapat ditunjukkan bahwa $P$ 
adalah pusat dari lingkaran $DEF$. Dengan demikian, dua lingkaran saling
berpotongan, $ABC$ dan $DEF$, memiliki titik pusat yang sama di titik $P$
Yang adalah hal yang mustahil $[Prop 3.5]$.

Dengan demikian, sebuah lingkaran tidak akan memotong lingkaran lainnya
di lebih dari dua titik. Yang adalah hal yang perlu ditunjukkan.  

\stepcounter{section}
\section*{\centering Proposisi \thesection}
\addcontentsline{toc}{section}{Proposisi \thesection}
Jika dua buah lingkaran saling bersinggungan dari dalam, dan pusat
keduanya sudah diketahui, maka garis lurus yang menghubungkan pusat - 
pusat lingkaran tersebut, jika dipanjangkan akan jatuh pada titik 
singgung kedua lingkaran.
\begin{center}
\begin{tikzpicture}
\pgfmathparse{(180 + (atan(1/-2))) - 110} \let\GFAangle\pgfmathresult
\pgfmathparse{sqrt(4*4 + 2*2 - 2*4*2*cos(\GFAangle))} 
	\let\GAlength\pgfmathresult 
\pgfmathparse{acos((\GAlength*\GAlength + 2*2 - 4*4)/(2*2*\GAlength))}
	\let\FGAangle\pgfmathresult 

\draw (0, 0) node [anchor=north]{F} circle(4);
\draw ({-4/3}, {2/3}) circle({4 - sqrt(20/9)});
\draw (0, 0) --++ ({180 + (atan(1/-2))}:4) node[anchor=north east]{A};
\draw (0, 0) --++ (110:2) node [anchor=west]{G} 
	--++ ({180 + 110 - \FGAangle}:\GAlength)
	--++ ({110 - \FGAangle}:\GAlength)
	--++ (110:1.1) node [anchor=east]{D}
	--++ (110:.9) node [anchor=south]{H}; 
\draw (4,0) node [anchor=west]{B};
\draw (-2,-3.5) node [anchor=north]{C};
\draw (-2,-1.8) node [anchor=north]{E};
\end{tikzpicture}
\end{center}
Sebab misalkan dua buah lingkaran, $ABC$ dan $ADE$, saling bersinggungan
dari dalam pada titik $A$, dan misalkan pusat dari lingkaran 
$ABC$ sudah diketahui [Prop. 3.1] di $F$, dan pusat dari lingkaran $ADE$
sudah diketahui di $G$ [Prop. 3.1]. Saya katakan bahwa garis lurus
yang menghubungkan $G$ dan $F$, jika dipanjangkan, akan jatuh pada 
titik $A$.

Sebab jika tidak maka, jika mungkin, misalkan garis tersebut jatuh seperti
$FGH$ di gambar, dan misalkan $AF$ dan $AG$ sudah dihubungkan. 

Oleh sebab itu, karena $AG$ dan $GF$ lebih panjang $AF$ [Prop. 1.20],
maka $AG$ dan $GF$ juga lebih panjang dari $FH$, misalkan $FG$ sudah
diambil dari keduanya. Dengan demikian, sisa $AG$ lebih besar dari
sisa $GH$. Dan $AG$ sama panjang dengan $GD$. Dengan demikian, $GD$
jusa lebih panjang dari $GH$, yang lebih pendek menjadi yang lebih
panjang, hal yang tidak mungkin. Dengan demikian, garis lurus yang 
menghubungkan $F$ dan $G$ tidak akan keluar dari lingkaran yang lebih
dalam. Dengan demikian, garis tersebut akan jatuh pada titik singung
kedua lingkaran.

Dengan demikian, jika dua buah lingkaran saling bersinggungan dari dalam, dan
titik pusat keduanya sudah diketahui, maka garis lurus yang menghubungkan
pusat kedua lingkaran akan jatuh pada titik singgung kedua lingkaran. 
Hal yang perlu ditunjukkan.

\stepcounter{section}
\section*{\centering Proposisi \thesection}
\addcontentsline{toc}{section}{Proposisi \thesection}
Jika dua buah lingkaran saling bersinggungan dari luar, maka
garis lurus yang menghubungkan pusat kedua lingkaran akan melalui
titik singgung kedua lingkaran.  
\begin{center}
\begin{tikzpicture}
\draw(0, 0) circle(3);
\draw(3, -6) circle({sqrt(3*3 + -6*-6)-3});
%%center to center gradient = -6/3 = -2
%%meeting point line length equal to radius of first circle 
%%which is  x^2+y^2 = 3^2
%%3^2 = x^2 + (-2x)^2
%%9 = 5x^2
\draw({sqrt(9/5)}, {-2*sqrt(9/5)}) node [anchor=south]{A}
--++({atan(-1)}:-3)node[anchor=east]{F}
--++({atan(-2)}:2.5) node [anchor=east]{C}
--++({atan(-2)}:1) node [anchor=east]{D}
--++({atan(-2)}:3.5) node [anchor=east]{G}
--({sqrt(9/5)}, {-2*sqrt(9/5)});
\draw(-2, 2.5) node[anchor=east]{B};
\draw(-0, -8.5) node[anchor=east]{E};
%%\draw[gray, dashed] (0, 0) --(3, -6);
\end{tikzpicture}
\end{center} 
Sebab misalkan dua buah lingkaran, $ABC$ dan $ADE$, saling bersinggungan
dari luar pada titik $A$, dan misalkan titik pusat $ABC$ sudah
diketahui di $F$ [Prop. 3.1], pusat lingkaran $ADE$ sudah diketahui 
di $G$ [Prop. 3.2]. Saya katakan bahwa garis lurus yang menghubungkan
$F$ dan $G$ akan melalui titik singgung $A$.

Sebab jiga tidak maka, jika mungkin, misalkan pertemuannya pada garis 
seperti $FCGD$ di gambar, dan misalkan $AF$ dan $AG$ sudah dihubungkan.

Dengan demikian, karena titik $F$ adalah pusat dari lingkaran $ABC$, 
$FA$  sama panjang dengan $FC$. Lagi, karena titik $G$ adalah pusat dari
lingkaran $ADE$, $GA$ sama panjang dengan $GD$. Dan $FA$ juga sudah
ditunjukkan sama panjang dengan $FC$. Dengan demikian, ruas garis $FA$
dan $AG$ sama panjang dengan ruas garis $FC$ dan $GD$. Dengan demikian
garis lurus $FG$ lebih panjang dari garis lurus $FA$ dan $AG$. Tetapi 
sudah dibuktikan bahwa $FG$ lebih pendek dari $FA$ dan $AG$ [Prop. 1.20].
Hal yang tidak mungkin. Dengan demikian, garis lurus  yang menghubungkan
$F$ dan $G$ tidak bisa tidak melalui titik singgung $A$. Yang artinya
harus melalui $A$.

Dengan demikian, juga dua buah lingkaran bersinggungan dari luar 
maka garis lurus yang menghubungkan pusat kedua lingkaran akan melalui
titik singgung kedua lingkaran. Yang adalah hal yang perlu ditunjukkan.

\stepcounter{section}
\section*{\centering Proposisi \thesection}
\addcontentsline{toc}{section}{Proposisi \thesection}
Sebuah lingkaran tidak akan menyinggung lingkaran lainnya pada lebih dari
satu titik, baik dari dalam maupun dari luar.  
\begin{center}
\begin{tikzpicture}
\draw (0, 0) circle(4);
\draw (0, 0) circle(4 and 3);
\draw (-4, 0) node [anchor=east]{B} --++(0:3) node [anchor=south]{G} 
	--++(0:2) node[anchor=south]{H}--++(0:3) node[anchor=west]{D};
\draw (0, 3) node [anchor=south]{E};
\draw (0, -3) node [anchor=south]{F};
\draw (0, 4) node [anchor=south]{A};
\draw (-1,-0.1) --++(90:0.2);
\draw (1,-0.1) --++(90:0.2);
\draw (2, 5) circle(2);
\draw (2.8, {2.8*5/2}) node [anchor=south]{K};
\draw (0.26, 4) --++ ({atan(5/2) + 270}:2.49) node [anchor=west]{C};
\end{tikzpicture}
\end{center}

Sebab jika mungkin, misalkan lingkaran $ABCD$ menyinggung lingkaran 
$EBFD$, pertama dari dalam, di lebih dari dua titik, $D$ dan $B$.

Dan misalkan pusat lingkaran $ABCD$ sudah diketahui di  $G$ [Prop. 3.1],
dan pusat dari lingkaran $EBFD$ sudah diketahui di titik $H$ [Prop. 3.1].

Dengan demikian, garis lurus yang menghubungkan $G$ dan $H$ akan jatuh 
pada titik $B$ dan $D$ [Prop. 3.11]. Misalkan garis tersebut seperti
garis $BGHD$ di gambar. Dan karena titik $G$ adalah pusat dari lingkaran
$ABCD$, $BG$ sama panjang dengan $GD$. Dengan demikian, $BG$ lebih
panjang dari $HD$. Dengan demikian, $BH$ jauh lebih besar dari $HD$.
Dan lagi karena titik $H$ adalah pusat dari lingkaran $EBFD$, $BH$
sama panjang dengan $HD$. Tapi juga sudah ditunjukkan bahwa $BH$ jauh
lebih besar dari $HD$. Hal yang tidak mungkin. Dengan demikian, 
sebuah lingkaran tidak akan menyinggung lingkaran lainnya dari dalam
pada lebih dari satu titik.

Juga saya katakan bahwa lingkaran tidak akan menyinggung dari luar 
lingkaran lainnya di lebih dari dua titik.

Sebab, jika mungkin, misalkan lingkaran $ACK$ menyinggung lingkaran
$ABCD$ dari luar di lebih dari dua titik, $A$ dan $C$. Dan misalkan
$A$ dan $C$ sudah dihubungkan.

Karena itu, karena dua titik, $A$ dan $C$, sudah diambil secara acak
pada busur kedua lingkaran $ABCD$ dan $ACK$, garis lurus yang 
menghubungkan kedua titik tersebut akan jatuh di dalam masing - masing
lingkaran [Prop. 3.2]. Tetapi, garis tersebut berada di dalam $ABCD$ 
tetapi di luar $ACK$ [Def. 3.3]. Hal yang mustahil. Dengan demikian,
sebuah lingkaran tidak akan menyinggung dari luar lingkaran lainnya 
di lebih dari satu titik. Dan seperti ditunjukkan juga tidak jika
menyinggung dari dalam.

Dengan demikian, sebuah lingkaran tidak menyinggung lingkaran lainnya
di lebih dari satu titik baik dari dalam maupun dari luar. Yang adalah
hal yang perlu ditunjukkan.

\stepcounter{section}
\section*{\centering Proposisi \thesection}
\addcontentsline{toc}{section}{Proposisi \thesection}
Dalam sebuah lingkaran, garis - garis yang sama panjang, memiliki
jarak yang sama jauh dari pusat, dan garis - garis yang sama jauh
dari titik pusat sama panjang satu dengan yang lain.
\begin{center}
\begin{tikzpicture} 
\pgfmathparse {20}
	\let\slant\pgfmathresult
\pgfmathparse {\slant + 60}
	\let\rightangle\pgfmathresult
\pgfmathparse {180 - 60 + \slant}
	\let\leftangle\pgfmathresult
\pgfmathparse {270 + \slant}
	\let\vertdown\pgfmathresult
\pgfmathparse {10}
	\let\displacement\pgfmathresult
\pgfmathparse {180 - (60 + 90)}
	\let\vdangle\pgfmathresult
\pgfmathparse {\vdangle - \displacement}
	\let\vddsngl\pgfmathresult
\pgfmathparse {2*5*cos(\vddsngl)}
	\let\vertlen\pgfmathresult
\pgfmathparse {sqrt(5*5 - (\vertlen*\vertlen/4))}
	\let\ulen\pgfmathresult


\draw (0, 0) circle(5) node [anchor = south east] {E};
\draw ({5*cos(\leftangle)}, {5*sin(\leftangle)}) node [anchor=south east]{B}
	--++ ({\vertdown + \displacement}:\vertlen) node [anchor = north]{A}
	-- (0, 0);
\draw ({5*cos(\rightangle)}, {5*sin(\rightangle)}) node [anchor=south]{D}
	--++ ({\vertdown - \displacement}:\vertlen) node [anchor = north west]{C}
	-- (0, 0);
\draw (0, 0) --++ ({\displacement + 180 + \slant}: \ulen) node [anchor=east]{F};
\draw (0, 0) --++ ({\slant - \displacement: \ulen}) node [anchor=west]{G};
%%\draw (0, 0) --++ ({asin(4/5) + 288.1 - 90}: 5);
%%\draw (0, 0) --++ (-{asin(4/5) + 291.9 + 90}: 5);

%\draw (0, 0) circle(5) node [anchor = south east] {E};
%\draw ({5*cos(145)}, {5*sin(145)}) node [anchor=south east]{B}
%	--++ (288.1:8) node [anchor = north]{A};
%\draw ({5*cos(75)}, {5*sin(75)}) node [anchor=south]{D}
%	--++ (291.9:8) node [anchor = north west]{C};
%\draw (0, 0) --++ ({288.1 - 90}: {sqrt(5*5- 4*4)}) node [anchor=east]{F};
%\draw (0, 0) --++ ({291.9 + 90}: {sqrt(5*5- 4*4)}) node [anchor=west]{G};
%\draw (0, 0) --++ ({asin(4/5) + 288.1 - 90}: 5);
%\draw (0, 0) --++ (-{asin(4/5) + 291.9 + 90}: 5);
\end{tikzpicture}
\end{center}

Misalkan $ABCD$ adalah sebuah lingkaran, dan misalkan $AB$ dan $CD$ 
adalah garis sama panjang yang ada di dalam lingkaran. Saya katakan
$AB$ dan $CD$ sama jauh dari pusat lingkaran.

Sebab misalkan pusat lingkaran $ABCD$ sudah ditemukan [Prop. 3.1] dan misalkan
itu adalah $E$. Dan misalkan $EF$ dan $EG$ sudah ditarik dari titik $E$, tegak
lurus terhadap masing - masing $AB$ dan $CD$ [Prop.1.12]. Dan misalkan $AE$ dan
$EC$ sudah dihubungkan.

Oleh karena garis lurus $EF$ melalui pusat lingkaran, memotong garis lurus 
,$AB$,  yang tidak melalui pusat lingkaran, secara tegak lurus, maka akan 
membaginya dua di tengan [Prop.3.3]. Dengan demikian, $AF$ sama panjang dengan
$FB$. Jadi, panjang $AB$ adalah dua kali $AF$. Juga untuk alasan yang sama, 
$CD$, dua kali panjang $AF$. Dan $AB$ sama panjang dengan $CD$. Dengan demikian,
$AF$ juga sama panjang dengan dengan $CG$. Dan karena $AE$ sama panjang dengan
$EC$, besar bujur sangkar pada (kuadrat) $AE$ juga sama besar dengan kuadrat 
(besar bujur sangkar pada) $EC$. Tetapi, jumlah besar bujur sangkar - bujur 
sangkar pada $AF$ dan $EF$ sama dengan besar bujur sangkar pada $AE$. Karena
sudut $F$ adalah sudut siku - siku [Prop. 1.47]. Dan jumlah besar bujur sangkar
pada $EG$ dan $EC$ sama dengan besar bujur sangkar pada $EC$. Karena sudut $G$
adalah sudut sikut - siku [Prop. 1.47]. Dengan demikian, jumlah besa bujur 
sangkar pada $AF$ dan $FE$ sama dengan besar jumlah besa bujur sangkar pada
$CG$ dan $GE$, yang mana besar bujur sangkar pada $AF$ sama besar dengan besar
bujur sangkar pada $CG$. Karena $AF$ sama panjang dengan $CG$. Dengan demikian, 
besar bujur sangkar - bujur sangkar selebihnya pada $AF$ dan pada $EG$ sama
besar. Dengan demikian, $EF$ sama panjang dengan $EG$, Dan garis lurus - garis
lurus pada sebuah lingkaran dikatakan sama jauh dari pusat lingkaran jika garis
lurus - garis lurus yang tegak lurus yang ditarik dari pusat lingkaran sama 
panjang [Def. 3.4]. Dengan demikian $AB$ dan $CD$ sama jauh dari pusat.

Jadi, misalkan $AB$ dan $CD$ (tali busur - tali busur) yang sama jauh dari pusat 
lingkaran. Yakni $EF$ sama panjang dengan $EG$. Saya katakan $AB$ juga sama 
panjang dengan $CD$.

Karena, dengan menggunakan gambar dan susunan yang sama, kita dapat tunjkukan
bahwa $AB$ dua kali panjang $AF$, dan $CD$ dua kali panjang $CG$. Dan karena
$AE$ sama panjang $CE$, besar bujur sangkar pada $AE$ sama besar dengan besar
bujur sangkar pada $CE$. Tetapi jumlah besar bujur sangkar - bujur sangkar pada
$EF$ dan $FA$ sama dengan besar bujur sangkar pada $AE$ [Prop. 1.47]. Dan jumlah
besar bujur sangkar - bujur sangkar pada $EG$ dan $GC$ sama dengan besar bujur
sangkar pada $CE$ [Prop. 1.47]. Dengan demikian jumlah besar bujur sangkar - 
bujur sangkar pada $EF$ dan $FA$ sama dengan jumlah besar bujur sangkar - bujur
sangkar pada $EG$ dan $CF$, yang mana bujur sangkar $EF$ sama besar dengan bujur
sangkar $EG$. Dengan demikian, $AF$ sama panjang dengan $CG$. Dan $AB$ dua kali
panjang $AF$, dan $CD$ dua kali panjang $CG$. Dengan demikian, $AB$ sama panjang
dengan $CD$.

Dengan demikian, Dalam sebuah lingkaran, garis - garis yang sama panjang, 
memiliki jarak yang sama jauh dari pusat, dan garis - garis yang sama jauh
 dari titik pusat sama panjang satu dengan yang lain. Yang adalah hal yang perlu
ditunjukkan.

\stepcounter{section}
\section*{\centering Proposisi \thesection}
\addcontentsline{toc}{section}{Proposisi \thesection}
Di dalam sebuah lingkaran, garis (tali busur) yang paling panjang adalah 
sebuah diameter, 
dan untuk garis - garis lainnya, garis yang lebih dekat dengan pusat lingkaran
adalah lebih panjang dari garis yang lebih jauh dari pusat lingkaran.
\begin{center}
\begin{tikzpicture}
\pgfmathparse {sqrt(5*5 - 2*2)}
	\let\bhlen\pgfmathresult
\pgfmathparse {acos(-2/5)}
	\let\lemangle\pgfmathresult 
\pgfmathparse {acos(-3.5/5)}
	\let\kefangle\pgfmathresult 
\draw (0, 0) circle(5) node[anchor=south west]{E} --++
	(-15:2) node[anchor=west]{H} --++
	({-15+90}: \bhlen) node [anchor=south]{B} --++
	({-15+270}:{2*\bhlen}) node[anchor=north]{C} --++ 
	({-15+90}: \bhlen) --++  (180-15:2) --++ 
	(90:5) node[anchor=south]{A} --++ 
	(270:10) node[anchor=north]{D} -- (0, 0) --++
	(\lemangle:5) node[anchor=south]{M} --++ 
	(270:{5 * sin(\lemangle)}) node [anchor=south east]{L} --++
	(270:{5 * sin(\lemangle)}) node [anchor=north]{N} -- (0, 0) --++
	(180:3.5) node [anchor=east]{K} --++ 
	(90: {5*sin(\kefangle)}) node[anchor=south]{F} -- (0, 0) --++
	({-\kefangle}: 5) node [anchor=north]{G} --++ 
	(90: {5*sin(\kefangle)});
\end{tikzpicture}
\end{center}

Misalkan $ABCD$ adalah sebuah lingkaran, dan misalkan $AD$ adalah diameter nya,
dan $E$ adalah pusat. Dan misalkan $BC$ yang lebih dekat dengan diameter $AD$ 
(ke pusat juga), dan $FG$ yang lebih jauh lagi. Saya katakan bahwa $AD$ adalah 
garis lurus yang paling panjang, dan $BC$ lebih panjang dari $FG$.

Sebab misalkan $EH$ dan $EK$ sudah ditarik dari pusat $E$, secara siku - siku 
terhadap masing - masing $BC$ dan $FG$ [Prop. 1.12]. Dan karena $BC$ lebih 
dekat dengan pusat lingkaran dan $FG$ sangat jauh, $EK$ dengan demikian 
lebih panjang dari $EH$ [Def. 3.5]. Misalkan $EL$ dibuat sama panjang dengan 
$EH$ [Prop. 1.3].  Dan $LM$ sudah ditarik dari $L$, tegak lurus terhadap 
$EK$ [Prop. 1.11], dan misalkan $LM$ sudah ditarik sampai $N$. Dan misalkan 
$ME$, $EN$, $FE$, dan $EG$ sudah dihubungkan.

Dan karena $EH$ sama panjang dengan $EL$, $BC$ juga sama panjang dengan $MN$
[Prop. 3.14]. Dan lagi, karena $AE$ sama panjang dengan $EM$, dan $ED$ sama 
panjang dengan $EN$, $AD$ dengan demikian sama panjang dengan $ME$ dan $EN$.
Tetapi $ME$ dan $EN$ lebih panjang dari $MN$ [Prop. 1.20] (jadi $AD$ lebih
panjang dari $MN$, dan $MN$ sama panjang dengan $BC$. Dengan demikian, $AD$
lebih panjang dari $BC$. Dan karena masing - masing kedua garis lurus $ME$, 
$EN$ sama panjang dengan kedua garis lurus $FE$, $EG$, dan sudut $MEN$ lebih
panjang dari $FEG$ \footnote{Hal ini tidak dibuktikan, hanya dari pengamatan
gambar}, sisi alas $MN$ dengan demikian lebih panjang dari sisi alas $FG$. Tetapi $MN$ sudah dibuktikan sama panjang dengan sisi alas $BC$ 
(jadi $BC$ dengan demikian lebih panjang dari $FG$). Dengan demikian, diameter
$AD$ adalah garis (tali busur) terpanjang, dan $BC$ lebih panjang dari $FG$.

Dengan demikian, di dalam sebuah lingkaran, sebuah diameter adalah garis lurus
terpanjang, dan untuk yang lain, garis lurus yang terdekat ke pusat lingkaran
selalu lebih panjang dari padah garis yang lebih jauh. Yang adalah hal yang
perlu dibuktikan.

\stepcounter{section}
\section*{\centering Proposisi \thesection}
\addcontentsline{toc}{section}{Proposisi \thesection}

Sebuah garis lurus yang ditarik secara tegak lurus dari ujung diameter sebuah 
lingkaran, akan berada di luar lingkaran. Dan garis lain tidak akan bisa
diselipkan dalam spasi  antara garis lurus tegak lurus tadi dengan busur 
lingkaran. Dan sudut dari setengah lingkaran dan garis lurus tadi lebih besar 
dari sudut lancip rektilinear apa pun, dan sudut selebihnya akan lebih kecil 
dari sudut lancip rektilinear apa pun.

\begin{center}
\begin{tikzpicture}
	\pgfmathparse{sqrt(3*3 + 3.5*3.5 - 2*3*3.5*cos(35))}
	\let\aglen\pgfmathresult
	\pgfmathparse{acos((3*3 - 3.5*3.5 - \aglen*\aglen)/(2*3.5*\aglen))}
	\let\dgaangle\pgfmathresult
	\draw(0, 0) node[anchor=west]{D} circle(3) 
	--++ (90:3) node[anchor=south]{B}
	--++ (270:6) node [anchor=north]{A} 
	--++ (180:4) node [anchor=east]{E};
	\draw(0, 0) --++ (200:3) node[anchor=east]{C}
	-- (0, -3);
	\draw(0, 0) --++ (235:2.9) node[anchor=south east]{H}
	--++(235:.6) node[anchor=south east]{G} --++ ({235+\dgaangle}: \aglen)
	--++({235+\dgaangle+180}:5) node[anchor=south]{F};
\end{tikzpicture}
\end{center}

Misalkan $ABC$ adalah lingkaran disekitar pusat $D$ dengan diameter $AB$. Saya 
katakan bahwa garis lurus yang ditarik dari ujung $A$ secara tegak lurus 
terhadap $AB$ [Prop. 1.11], akan jatuh di luar lingkaran.

Sebab jika tidak, jika memungkinkan, misalkan garis tersebut jatuh di dalam 
lingkaran, seperti $CA$ pada gambar, misalkan $DC$ sudah dihubungkan.

Karena $DA$ sama panjang dengan $DC$, sudut $DAC$ juga sama besar dengan sudut
$ACD$ [Prop. 1.5]. Dan $DAC$ adalah sudut sikut - siku. Dengan demikian, $ACD$
juga adalah sudut siku - siku. Jadi, dalam segitiga $ACD$, dua sudut $DAC$ dan
$ACD$ sama dengan besar dua sudut siku - siku. Yang adalah tidak mungkin [Prop.
1.17]. Dengan demikian, garis lurus yang ditarik dari titik $A$, secara tegak
lurus terhadap $BA$, tidak akan jatuh di dalam lingkaran. Jadi dengan cara yang
sama. kita dapat tunjukkan garis tersebut tidak akan jatuh pada busur lingkaran.
Dengan demikian, garis tersebut akan jatuh di luar lingkaran.

Misalkan garis tegak lurus yang kita maksud seperti garis $AE$ pada gambar. Jadi
, saya katakan bahwa gari lurus lainnya tidak akan bisa diselipkan di spasi
antara garis lurus $AE$ dan busur $CHA$.

Sebab, jika memungkinkan, misalkan garis tersebut sudah diselipkan seperti $FA$
pada gambar, dan misalkan $DG$ sudah ditarik dari titik $D$, tegak lurus 
terhadap $FA$ [Prop. 1.12]. Dan karena $AGD$ adalah sudut siku - siku, dan $DAG$
lebih kecil dari sudut siku - siku, $AD$ dengan demikian lebih besar dari $DG$
[Prop. 1.19]. Dan $DA$ sama panjang dengan $DH$. Dengan demikian, $DH$ lebih 
besar dari $DG$, yang lebih pendek menjadi lebih panjang. Yang adalah 
hal yang tidak mungkin. Dengan demikian, garis lurus lain tidak bisa diselipkan
di ruang antara garis lurus $AE$ dan busur lingkaran.

Dan juga saya katakan bahwa sudut setengah lingkaran yang diapit oleh garis 
lurus $BA$ dan busur $CHA$ adalah lebih besar dari sudut lancip rektilinear
manapun, dan sudut selebihnya yang diapit busur $CHA$ dan garis lurus $AE$ lebih
kecil dari sudut lancip rektilinear apa pun.

Sebab jika sembarang sudut lancip rektilinear lebih besar dari sudut yang diapit 
garis lurus $BA$ dan busur $CHA$, atau lebih kecil dari sudut yang diapit 
oleh busur $CHA$ dan garis lurus $AE$, maka sebuah garis lurus dapat diselipkan 
di spasi antara busur $CHA$ dan garis $AE$ - apapun yang akan membentuk sebuah 
sudut yang diapit oleh garis - garis lurus lebih besar dari sudut yang diapit 
garis lurus $BA$ dan busur $CHA$, atau lebih kecil dari sudut yang dibentuk
oleh busur $CHA$ dan garis lurus $AE$. Tetapi garis lurus tersebut pasti tidak 
bisa diselipkan seperti itu. Dengan demikian, sebuah sudut lancip yang dibentuk
oleh garis - garis lurus tidak bisa lebih besar dari sudut yang dibentuk oleh 
garis lurus $BA$ dan busur $CHA$, juga tidak bisa lebih kecil dari sudut yang
dibentuk oleh busur $CHA$ dan garis lurus $AE$.  

\begin{center}\textbf{Dalil}\end{center}
Jadi, dari sini, sudah dinyatakan, bahwa sebuah garis lurus yang ditarik secara
tegak lurus ke ujung diameter sebuah lingkaran, menyinggung 
lingkaran [dan bahwa garis lurus itu menyinggung lingkaran pada satu titik, 
karena sudah dibuktikan bahwa sebuah garis lurus yang bertemu sebuah lingkaran 
di dua titik jatuh di dalam lingkaran [Prop. 3.2]]. Yang adalah hal yang perlu 
ditunjukkan.

\stepcounter{section}
\section*{\centering Proposisi \thesection}
\addcontentsline{toc}{section}{Proposisi \thesection}
Untuk menarik sebuah garis lurus yang menyinggung sebuah lingkaran dari satu
titik tertentu.
\begin{center}
\begin{tikzpicture}
	\pgfmathparse{}
	\draw (0, 0) node[anchor=west]{E} circle(2) 
	--++ (90:2) node[anchor=north east]{D}
	--++ (90:2) node[anchor=south]{A};
	\draw(0, 0) circle(4); 
	\draw(0, 2) --++(180:{sqrt(4*4 - 2*2)}) node[anchor=east]{F}
	--++ ({asin(-2/4)}:2) node [anchor=east]{B}
	--++ ({asin(-2/4)}:2);
	\draw(0, 4) --++({asin(-2/4)-90}:{sqrt(4*4 - 2*2)});
	\draw(2, 0) node[anchor=west]{C};
	\draw(4, 0) node[anchor=west]{G};
\end{tikzpicture}
\end{center}
Misalkan $A$ adalah sebuah titik tertentu, dan $BCD$ adalah sebuah lingkaran 
tertentu. Jadi diperlukan untuk menarik sebuah garis lurus yang menyinggung 
lingkaran $BCD$ dari titik $A$.

Misalkan pusat lingkaran $BCD$ sudah ditemukan di $E$ [Prop. 3.1], dan misalkan
$AE$ sudah dihubungkan. Dan misalkan lingkaran $AFG$ sudah digambarkan dengan 
pusat di $E$ dengan $EA$ sebagai jari - jari. Dan misalkan $DF$ sudah ditarik
dari titik $D$ tegak lurus dengan $EA$ [Prop. 1.11]. Dan misalkan $EF$ dan
$AB$ sudah dihubungkan. Saya katakan garis lurus $AB$ sudah ditarik dari titik
$A$ menyinggung lingkaran $BCD$.

Karena $E$ adalah pusat dari lingkaran $BCD$ dan $AFG$, $EA$ dengan demikian 
sama panjang dengan $EF$, dan $ED$ sama panjang dengan $EB$. Dengan demikian
masing - masing kedua garis lurus $AE$, $EB$ sama panjang dengan kedua garis
lurus $FE$ dan $ED$. Dan keduanya memiliki sudut yang sama di $E$. Dengan 
demikian, sisi alas $DF$ sama panjang dengan sisi alas $AB$, dan segitiga $DEF$
sama dan sebangun dengan segitiga $EBA$, dan sudut - sudut selebihnya di satu
segitiga sama besar dengan sudut - sudut selebinya yang bersesuaian di segitiga 
yang satunya [Prop. 1.4]. Dengan demikian sudut $EDF$ sama besar dengan sudut 
$EBA$. Dan $EDF$ adalah sudut siku - siku. Dengan demikian, $EBA$ juga adalah
sudut siku - siku. Dan $EB$ adalah jari - jari. Dan sebuah garis lurus
yang ditarik secara tegak lurus terhadap ujung diameter sebuah lingkaran, 
menyinggung lingkaran tersebut [Prop. 3.16.dalil]. Dengan demikian, $AB$ 
menyinggung lingkaran $BCD$.

Dengan demikian, garis lurus $AB$ sudah ditarik menyinggung lingkaran $BCD$ dari
titik $A$. Yang adalah hal yang perlu dilakukan.

\stepcounter{section}
\section*{\centering Proposisi \thesection}
\addcontentsline{toc}{section}{Proposisi \thesection}
Jika suatu garis lurus menyinggung sebuah lingkaran, dan sebuah garis lurus 
lain dihubungkan dari pusat lingkaran ke titik persinggungan tadi, maka garis
yang dihubungkan dari lingkaran tadi akan tegak lurus terhadap garis singgung.
\begin{center}
\begin{tikzpicture}
	\pgfmathparse{3*tan(40)}
	\let\cg\pgfmathresult
	\pgfmathparse{sqrt(3*3 + \cg*\cg)}
	\let\gf\pgfmathresult
	\draw(0, 0) node[anchor=east]{F} circle(3) 
	--++ (320:3) node[anchor=west]{C} 
	--++({320+90}:\cg) node[anchor=west]{G}
	--++({320+90}:1) node[anchor=south west]{D};
	\draw(0, 0) --++(0:3) node[anchor=south east]{B}
	--++(0: {\gf-3}) --++ ({320-90}:5) node[anchor=north west]{E};
	\draw({3*cos(150)}, {3*sin(150)}) node[anchor=east]{A};
\end{tikzpicture}
\end{center}
Misalkan garis lurus $DE$ menyinggung lingkaran $ABC$ di titik $C$, dan 
misalkan 
pusat lingkaran $ABC$ sudah ditemukan di $F$ [Prop. 3.1], dan misalkan $FC$ 
sudah
dihubungkan dari titik $F$ ke $C$. Saya katakan bahwa $FC$ tegak lurus terhadap
$DE$.

Sebab jika tidak, misalkan $FG$ sudah ditarik dari $F$, tegak lurus terhadap 
$DE$ [Prop. 1.12].

Dengan demikian, karena $FGC$ siku - siku, maka sudut $FCG$ adalah sudut lancip 
[Prop. 1.17]. Dan sudut yang lebih besar akan menghadap sisi yang lebih panjang 
[Prop. 1.19]. Dengan demikian, $FC$ lebih panjang dari $FG$. Dan $FC$ sama 
panjang dengan $FB$ [Karena keduanya jari - jari]. Dengan demikian, $FB$ juga 
lebih panjang daripada $FG$, garis yang lebih pendek juga adalah yang lebih
panjang. Hal yang mustahil. Dengan demikian, $FG$ tidak tegak lurus terhadap 
$DE$.  Juga dengan cara yang sama dapat ditunjukkan tidak ada garis lain yang 
tegak lurus terhadap $DE$ dari $F$ selain $FC$. Dengan demikian, $FC$ tegak 
lurus terhadap $DE$.

Dengan demikian, jika sebuah garis lurus menyinggung sebuah lingkaran, dan garis
lurus yang lain dihubungkan dari pusat lingkaran ke titik persinggungan, maka
garis lurus yang dihubungkan tersebut akan tegak lurus terhadap garis singgung.
Yang adalah hal yang perlu ditunjukkan.

\stepcounter{section}
\section*{\centering Proposisi \thesection}
\addcontentsline{toc}{section}{Proposisi \thesection}
Jika suatu garis lurus menyinggung sebuah lingkaran, dan sebuah garis lurus 
ditarik dari titik persinggungan secara tegak lurus terhadap garis singgung [ke
dalam lingkaran], maka pusat lingkaran akan berada garis lurus yang ditarik 
tadi.
\begin{center}
\begin{tikzpicture}
	\draw(0, 0) circle(3);
	\draw(0, -3) node[anchor=north]{C} --++(90:6) node[anchor=south]{A};
	\draw(0, -3) --++(70:{(sin(90)/sin(70))*3}) node[anchor=south]{F};
	\draw(-3, -3) node[anchor=east]{D} --++ (0:6) node[anchor=west]{E};
	\draw(-3, 0) node[anchor=east]{B};
\end{tikzpicture}
\end{center}
Sebab misalkan suatu garis lurus $DE$ menyinggung lingkaran $ABC$ di titik $C$.
Dan misalkan $CA$ sudah ditarik dari $C$ dengan sudut siku - siku terhadap $DE$
[Prop. 1.11]. Saya katakan bahwa pusat dari lingkaran ada di garis $AC$.

Sebab jika tidak, jika mungkin, misalkan $F$ adalah pusat lingkaran dan 
misalkan $CF$ sudah dihubungkan.

Karenanya, disebabkan suatu garis lurus $DE$ menyinggung lingkaran $ABC$, 
dan $FC$
sudah dihubungkan dari pusat lingkaran ke titik persinggungan, $FC$ dengan 
demikian tegak lurus terhadap $DE$ [Prop. 3.18]. Dengan demikian, $FCE$ adalah 
sudut siku - siku. Dan $ACE$ juga siku - siku. Dengan demikian $FCE$ sama besar
dengan $ACE$, yang lebih kecil juga sama besar dengan yang lebih besar. Hal 
yang sungguh mustahil. Dengan demikian, $F$ tidak berada di pusat lingkaran 
$ABC$. Juga, dengan cara yang sama, dapat kita tunjukkan tidak ada garis lain
selain di garis $AC$.

Dengan demikian, jika suatu garis menyinggung sebuah lingkaran, dan sebuah 
garis lurus ditarik dari titik singgung, secara tegak lurus terhadap garis
singgung, maka pusat dari lingkaran akan berada pada garis lurus yang ditarik 
itu. Yang adalah hal yang perlu ditunjukkan.

\stepcounter{section}
\section*{\centering Proposisi \thesection}
\addcontentsline{toc}{section}{Proposisi \thesection}
Di dalam sebuah lingkaran, sudut di pusat lingkaran adalah dua kali besar sudut
di busur lingkaran, jika sudut - sudut tersebut memiliki alas busur yang sama.

\begin{center}
	\pgfmathparse{25} \let\slt\pgfmathresult
	\pgfmathparse{2.5}\let\rds\pgfmathresult
	\pgfmathparse{60} \let\bec\pgfmathresult
	\pgfmathparse{\bec/2} \let\bef\pgfmathresult
	\pgfmathparse{\bef/2} \let\aec\pgfmathresult
	\pgfmathparse{sqrt(2*(\rds*\rds) - 2*(\rds*\rds)*cos(180 - 2*\aec)}
	\let\ac\pgfmathresult
\begin{tikzpicture} 
%%	\draw(0, 0) node [anchor=south] {E} circle(\rds);
%%	\draw({\rds*cos(270-\bef+\slt)}, {\rds*sin(270-\bef+\slt)}) 
%%		node[anchor=north]{B}--++({180-(90+\bef)+\slt}:\rds) 
%%		--++ ({270+\slt}:\rds) node[anchor=north]{F}
%%		--++ ({90+\slt}:\rds)
%%		--++ ({(270+\bef)+\slt}:\rds) node [anchor=north]{C}
%%		--++ ({(180 + (270+\aec))+\slt}:\ac) node[anchor=south]{A}
%%		--++ ({(270 - \aec)+\slt}:\ac); 
%%	\draw(0, 0)--++(90+\slt:\rds);
%%	\draw({\rds*cos(270-\bef+\slt)}, {\rds*sin(270-\bef+\slt)}) --
%%		({\rds*(cos(20 + \slt))}, {\rds*(sin(20 + \slt))}) 
%%		node[anchor=west]{D} 
%%		--++ ({180 + (20+\slt)}:{2*\rds}) node[anchor=east]{G};
%%	\draw({\rds*cos(270+\bef+\slt)}, {\rds*sin(270+\bef+\slt)}) --
%%		({\rds*(cos(20 + \slt))}, {\rds*(sin(20 + \slt))}) ; 
%%
	\draw(0, 0) node [anchor=south] {E} circle(\rds);
	\draw({\rds*cos(270-\bef+\slt)}, {\rds*sin(270-\bef+\slt)}) 
		node[anchor=north]{B}--++({180-(90+\bef)+\slt}:\rds) 
		--++ ({270+\slt}:\rds) node[anchor=north]{F}
		--++ ({90+\slt}:{2*\rds}) --++ ({270+\slt}:\rds)
		--++ ({(270+\bef)+\slt}:\rds) node [anchor=north]{C}
		--++ ({(180 + (270+\aec))+\slt}:\ac) node[anchor=south]{A}
		--++ ({(270 - \aec)+\slt}:\ac); 

	\draw(6, 0) node [anchor=south] {E} circle(\rds);
	\draw({6+(\rds*cos(270-\bef+\slt))}, {\rds*sin(270-\bef+\slt)}) 
		node[anchor=north]{B}--++({180-(90+\bef)+\slt}:\rds) 
		--++ ({(270+\bef)+\slt}:\rds) node [anchor=north]{C};
	\draw({6+(\rds*cos(270-\bef+\slt))}, {\rds*sin(270-\bef+\slt)}) --
		({6+(\rds*(cos(20 + \slt)))}, {\rds*(sin(20 + \slt))}) 
		node[anchor=west]{D} 
		--++ ({180 + (20+\slt)}:{2*\rds}) node[anchor=east]{G};
	\draw({6+(\rds*cos(270+\bef+\slt))}, {\rds*sin(270+\bef+\slt)}) --
		({6+(\rds*(cos(20 + \slt)))}, {\rds*(sin(20 + \slt))}) ;
	\draw({6+(\rds*cos(90+\slt))}, {\rds*(sin(90+\slt))}) node[anchor=south]{A};

\end{tikzpicture}
\end{center} 

\begin{center}
\begin{tikzpicture} 
	\pgfmathparse{25} \let\slt\pgfmathresult
	\pgfmathparse{2.5}\let\rds\pgfmathresult
	\pgfmathparse{60} \let\bec\pgfmathresult
	\pgfmathparse{\bec/2} \let\bef\pgfmathresult
	\pgfmathparse{\bef/2} \let\aec\pgfmathresult
	\pgfmathparse{sqrt(2*(\rds*\rds) - 2*(\rds*\rds)*cos(180 - 2*\aec)}
	\let\ac\pgfmathresult
	\draw(0, 0) node [anchor=south] {E} circle(\rds);
	\draw({\rds*cos(270-\bef+\slt)}, {\rds*sin(270-\bef+\slt)}) 
		node[anchor=north]{B}--++({180-(90+\bef)+\slt}:\rds) 
		--++ ({270+\slt}:\rds) node[anchor=north]{F}
		--++ ({90+\slt}:\rds)
		--++ ({(270+\bef)+\slt}:\rds) node [anchor=north]{C}
		--++ ({(180 + (270+\aec))+\slt}:\ac) node[anchor=south]{A}
		--++ ({(270 - \aec)+\slt}:\ac); 
	\draw(0, 0)--++(90+\slt:\rds);
	\draw({\rds*cos(270-\bef+\slt)}, {\rds*sin(270-\bef+\slt)}) --
		({\rds*(cos(20 + \slt))}, {\rds*(sin(20 + \slt))}) 
		node[anchor=west]{D} 
		--++ ({180 + (20+\slt)}:{2*\rds}) node[anchor=east]{G};
	\draw({\rds*cos(270+\bef+\slt)}, {\rds*sin(270+\bef+\slt)}) --
		({\rds*(cos(20 + \slt))}, {\rds*(sin(20 + \slt))}) ; 

%%	\draw(0, 0) node [anchor=south] {E} circle(\rds);
%%	\draw({\rds*cos(270-\bef+\slt)}, {\rds*sin(270-\bef+\slt)}) 
%%		node[anchor=north]{B}--++({180-(90+\bef)+\slt}:\rds) 
%%		--++ ({270+\slt}:\rds) node[anchor=north]{F}
%%		--++ ({90+\slt}:{2*\rds}) --++ ({270+\slt}:\rds)
%%		--++ ({(270+\bef)+\slt}:\rds) node [anchor=north]{C}
%%		--++ ({(180 + (270+\aec))+\slt}:\ac) node[anchor=south]{A}
%%		--++ ({(270 - \aec)+\slt}:\ac); 
%%
%%	\draw(6, 0) node [anchor=south] {E} circle(\rds);
%%	\draw({6+(\rds*cos(270-\bef+\slt))}, {\rds*sin(270-\bef+\slt)}) 
%%		node[anchor=north]{B}--++({180-(90+\bef)+\slt}:\rds) 
%%		--++ ({(270+\bef)+\slt}:\rds) node [anchor=north]{C};
%%	\draw({6+(\rds*cos(270-\bef+\slt))}, {\rds*sin(270-\bef+\slt)}) --
%%		({6+(\rds*(cos(20 + \slt)))}, {\rds*(sin(20 + \slt))}) 
%%		node[anchor=west]{D} 
%%		--++ ({180 + (20+\slt)}:{2*\rds}) node[anchor=east]{G};
%%	\draw({6+(\rds*cos(270+\bef+\slt))}, {\rds*sin(270+\bef+\slt)}) --
%%		({6+(\rds*(cos(20 + \slt)))}, {\rds*(sin(20 + \slt))}) ;
%%
\end{tikzpicture}
\end{center}

\begin{center}
\begin{tikzpicture} 
	\pgfmathparse{2.5}\let\rds\pgfmathresult
	\pgfmathparse{50}\let\bec\pgfmathresult
	\draw (0, 0) node [anchor=south]{E} circle (\rds);
	\draw ({\rds*cos(270 - (\bec/2))}, {\rds*sin(270 - (\bec/2))}) 
		node[anchor=north]{B}
	--++ ({180 - (90 + (\bec/2))}: \rds)
	--++ ({270+(\bec/2)}:\rds) node[anchor=north]{C}
	--++ ({270+(\bec/2)+ 180}:{2*\rds}) node[anchor=south]{D}
	-- ({\rds*cos(270 - (\bec/2))}, {\rds*sin(270 - (\bec/2))});
\end{tikzpicture}
\end{center}
Misalkan $ABC$ adalah sebuah lingkaran, dan misalkan $BEC$ adalah sudut di pusat
lingkaran, dan $BAC$ adalah sudut di busur lingkaran. Dan misalkan keduanya
memiliki alas di busur yang sama seperti pada $BC$. Saya katakan bahwa sudut
$BEC$ adalah dua kali besar sudut $BAC$.

Sebab misalkan sudah dihubungkan, misalkan $AE$ sudah ditarik sampai $F$.

Jadi, karena $EA$ sama panjang dengan $EB$, sudut $EAB$ juga sama besar
dengan sudut $EBA$. Dengan demikian, sudut $EAB$ dan $EBA$ dua kali besar 
sudut $EAB$. Dan besar $BEF$ sama dengan jumlah besar $EAB$ dan $EBA$ 
[Prop. 1.32]. Dengan demikian, $BEF$ dua kali besar $EAB$. Jadi, dengan alasan 
- alasan yang sama, $FEC$ juga dua kali besar $EAC$. Dengan demikian, 
keseluruhan sudut $BEC$ dua kali sudut $BAC$.

Misalkan garis lurus lain sudah ditarik sedemikian, dan misalkan ada sudut lain,
 $BDC$. Dan $DE$ sudah dihubungkan, dan misalkan sudah ditarik sampai $G$.
Jadi, dengan cara yang sama, dapat kita tunjukkan $GEC$ dua kali besar $EDC$,
 dan juga $GEB$ dua kali $EDB$. Dengan demikian, jika $GEB$ diambil dari $GEC$
maka hasilnya adalah jumlah jika dua kali $EDB$ diambil dari dua kali $EDC$. 
Sedangkan jika $EDB$ diambil dari $EDC$ maka sisanya adalah sudut $BDC$. Maka
besar sudut sisa jika $GEB$ diambil dari $GEC$ adalah dua kali $BDC$. Tetapi 
sisa $GEB$ diambil $GEC$ adalah $BEC$. Jadi besar sudut $BEC$ adalah dua kali 
besar sudut $BDC$.

Dengan demikian, di dalam sebuah lingkaran, sudut di pusat lingkaran adalah
dua kali besar sudut di busur lingkaran, jika sudut - sudut tersebut memiliki
alas busur yang sama. Yang adalah hal yang perlu ditunjukkan.

\stepcounter{section}
\section*{\centering Proposisi \thesection}
\addcontentsline{toc}{section}{Proposisi \thesection}
Di dalam sebuah lingkaran, sudut yang berada dalam satu tembereng, sama besar
satu dengan yang lain. 
\begin{center}
\begin{tikzpicture}
	\draw(0, 0)circle(3) node [anchor=south]{F};
	\draw(0, 0) 
	--({3*cos(200)}, {3*sin(200}) node[anchor=east]{B}
	--({3*cos(340)}, {3*sin(340)}) node[anchor=west]{D}
	--(0, 0);
	\draw ({3*cos(200)}, {3*sin(200}) 
	-- ({3*cos(130)}, {3*sin(130)}) node[anchor=south]{A}
	--({3*cos(340)}, {3*sin(340)}); 
	\draw ({3*cos(200)}, {3*sin(200}) 
	-- ({3*cos(10)}, {3*sin(10)}) node[anchor=south west]{E}
	--({3*cos(340)}, {3*sin(340)}); 
	\draw (0, -3) node[anchor=north]{C};
\end{tikzpicture}
\end{center}

Misalkan $ABCD$ adalah sebuah lingkaran, dan $BAD$ dan $BED$ adalah sudut - 
sudut dalam satu tembereng $BAED$. Saya katakan sudut - sudut pada $BED$ dan 
$BAD$ sama besar satu dengan yang lain.

Sebab misalkan pusat lingkaran $ABCD$ sudah ditemukan [Prop. 3.1], dan misalkan
itu ada di titik $F$. Dan misalkan $BF$ dan $FD$ sudah dihubungkan.

Dan karena sudut $BFD$ ada di pusat, dan $BAD$ ada di busur lingkaran, dan 
keduanya memiliki alas busur yang sama di $BCD$, sudut $BFD$ dengan demikian
dua kali sudut $BAD$ [Prop. 3.20]. Jadi dengan alasan - alasan yang sama, $BFD$
juga dua kali besar $BED$. Dengan demikian, $BAD$ sama besar dengan $BED$.

Dengan demikian, di dalam sebuah lingkaran, sudut - sudut di dalam tembereng 
yang sama, sama besar satu dengan yang lain. Yang adalah hal yang perlu ditunjukkan.


\stepcounter{section}
\section*{\centering Proposisi \thesection}
\addcontentsline{toc}{section}{Proposisi \thesection}
Untuk semua persegi empat di dalam sebuah lingkaran, jumlah besar sudut - sudut
yang berhadapan sama dengan besar dua buah sudut siku - siku.

\begin{center}
\begin{tikzpicture}
	\pgfmathparse{2.5}\let\rds\pgfmathresult
	\pgfmathparse{50}\let\brad\pgfmathresult
	\pgfmathparse{150}\let\arad\pgfmathresult
	\pgfmathparse{220}\let\drad\pgfmathresult
	\pgfmathparse{290}\let\crad\pgfmathresult
	\draw(0, 0) circle(\rds);
	\draw({\rds*cos(\brad)}, {\rds*sin(\brad)}) node [anchor=south]{B} --
	({\rds*cos(\arad)}, {\rds*sin(\arad)}) node [anchor=south east]{A} --
	({\rds*cos(\drad)}, {\rds*sin(\drad)}) node [anchor=north east]{D} --
	({\rds*cos(\crad)}, {\rds*sin(\crad)}) node [anchor=north]{C} --
	({\rds*cos(\brad)}, {\rds*sin(\brad)}) --
	({\rds*cos(\drad)}, {\rds*sin(\drad)}) --
	({\rds*cos(\crad)}, {\rds*sin(\crad)}) --
	({\rds*cos(\arad)}, {\rds*sin(\arad)}) ;
\end{tikzpicture} 
\end{center}

Misalkan $ABCD$ adalah sebuah lingkaran, dan $ABCD$ adalah persegi empat yang
berada di dalam nya. Saya katakan bahwa jumlah besar dua sudut yang 
berhadapan sama besar dengan besar dua buah sudut siku - siku.

Misalkan $AC$ dan $BD$ sudah dihubungkan.

Maka, karena ketiga sudut dari sembarang segitiga sama besar dengan besar
dua buah sudut siku - siku [Prop. 1.32], ketiga sudut $CAB$, $ABC$, $BCA$ dari
segitiga $ABC$ dengan demikian sama besar dengan dua buah sudut siku - siku. 
Dan $CAB$ sama besar dengan $BDC$ karena keduanya ada di dalam tembereng yang
sama $BADC$ [Prop. 3.21]. Dan $ACB$ sama besar dengan $ADB$. Karena keduanya
ada di dalam tembereng yang sama $ADCB$ [Prop. 3.21]. Dengan demikian, 
keseluruhan $ADC$ sama besar dengan jumlah $BAC$ dan $ACB$. Misalkan $ABC$ 
sudah 
ditambahkan pada keduanya. Dengan demikian, jumlah $ABC$, $BAC$ dan $ACB$ sama
besar dengan jumlah $ABC$ dan $ADC$. Tetapi, $ABC$, $BAC$ dan $ACB$ sama besar 
dengan dua buah sudut siku - siku [Prop. 1.32]. Dengan demikian, 
jumlah sudut $ABC$ dan $ADC$ juga sama dengan besar dua buah sudut siku - siku.
Dengan cara yang sama, dapat kita tunjukkan bahwa sudut $BAD$ dan $DCB$ sama 
dengan besar dua buah sudut siku - siku.

Dengan demikian, untuk semua persegi empat di dalam lingkaran, jumlah dari dua
buah sudut yang berhadapan sama dengan besar dua buah sudut siku - siku. Yang 
adalah hal yang perlu ditunjukkan.

\stepcounter{section}
\section*{\centering Proposisi \thesection}
\addcontentsline{toc}{section}{Proposisi \thesection}
Dua temberang lingkaran yang serupa tetapi tidak berimpit tidak dapat
dibuat pada dan di sisi yang sama dari sebuah tali busur.
\begin{center}
\begin{tikzpicture}
	\pgfmathparse{10.1}\let\rdsa\pgfmathresult
	\pgfmathparse{5.21}\let\rdsb\pgfmathresult
	\pgfmathparse{50}\let\anglea\pgfmathresult
	\pgfmathparse{110}\let\angleb\pgfmathresult
	\pgfmathparse{90 - \anglea/2}\let\bpos\pgfmathresult
	\pgfmathparse{90 + \anglea/2}\let\apos\pgfmathresult
	\pgfmathparse{90 - \anglea/16}\let\cpos\pgfmathresult
	\pgfmathparse{atan((\rdsa*sin(\cpos)- \rdsa*sin(\apos))/
		(\rdsa*cos(\cpos) - \rdsa*cos(\apos)))}
	\let\tanca\pgfmathresult
	\pgfmathparse{\tanca + 180}\let\tanac\pgfmathresult

	\draw ({\rdsa*cos(\bpos)}, {\rdsa*sin(\bpos)}) node[anchor=north west]{B} arc ({\bpos}:{\apos}:\rdsa) node[anchor=north east]{A};
	\draw ({\rdsa*cos(\bpos)}, {\rdsa*sin(\bpos)}) arc ({90-\angleb/2}: {90+\angleb/2}: \rdsb); 
	\draw ({\rdsa*cos(\bpos)}, {\rdsa*sin(\bpos)}) --
	({\rdsa*cos(\cpos)}, {\rdsa*sin(\cpos)}) node[anchor=south]{C} --
	({\rdsa*cos(\apos)}, {\rdsa*sin(\apos)});
	\draw ({\rdsa*cos(\apos)}, {\rdsa*sin(\apos)}) --++
	(\tanca:7.25) node[anchor=south]{D} --
	({\rdsa*cos(\bpos)}, {\rdsa*sin(\bpos)})  --
	({\rdsa*cos(\apos)}, {\rdsa*sin(\apos)});

	
\end{tikzpicture}
\end{center}
Sebab, jika mungkin, misalkan dua buah tembereng yang serupa, $ACB$ dan 
$ADB$, sudah digambarkan pada dan di sisi yang sama dari 
tali busur $AB$. Dan misalkan $ACD$ sudah ditarik melalui tembereng - tembereng
tersebut, dan misalkan $CB$ dan $DB$ sudah dihubungkan.

Sebab itu, karena tembereng $ACB$ serupa dengan tembereng $ADB$, dan 
tembereng - tembereng yang serupa adalah tembereng - tembereng dengan sudut - 
sudut yang sama besar [Def. 3.11], sudut $ACB$ dengan demikian sama besar 
dengan sudut $ADB$, sudut eksterior sama besar dengan sudut interior. Yang
adalah hal yang mustahil [Prop. 1.16].

Dengan demikian, dua buah tembereng lingkaran yang serupa tetapi tidak berimpit
tidak dapat dibuat pada dan di sisi yang sama dari sebuah tali busur.

\stepcounter{section}
\section*{\centering Proposisi \thesection}
\addcontentsline{toc}{section}{Proposisi \thesection}
Tembereng - tembereng serupa sebuah lingkaran pada tali busur yang
sama panjang sama besar satu dengan yang lain
\begin{center}
\begin{tikzpicture}
	\draw (0, 0) node [anchor=north]{B} arc(30:90:5) 
	node [anchor=south]{E} arc(90:150:5) node [anchor=north]{A}
	--++(0:{2*5*cos(30)});

	\draw (0, -5) node [anchor=north]{D} arc(30:110:5) 
	node [anchor=south]{F} arc(110:150:5) node [anchor=north]{C}
	--++(0:{2*5*cos(30)});

	\draw (0, -5) arc(0:35:3) arc(35:90:2) arc(90:100:9.4) 
		node[anchor=south]{G}
	arc(100:133.5:9.4); 
\end{tikzpicture}
\end{center}
Misalkan $AEB$ dan $CFD$ adalah tembereng - tembereng yang serupa pada sebuah
lingkaran yang berada pada masing - masing tali busur yang sama panjang $AB$ 
dan $CD$. Saya katakan tembereng $AEB$ sama besar dengan tembereng $CFD$.

Sebab jika tembereng $AEB$ dihimpitkan pada tembereng $CFD$, dan titik $A$ di
letakkan tepat di titik $C$, dan tali busur $AB$ pada tali busur $CD$, maka 
titik $B$ akan juga akan berhimpit dengan titik $D$, dengan sarat $AB$ sama
panjang dengan $CD$. Dan jika $AB$ berhimpit dengan $CD$ maka tembereng $AEB$
juga akan berhimpit dengan tembereng $CFD$. Sebab jika tali busur $AB$ 
berhimpit dengan tali busur $CD$, dan tembereng $AEB$ tidak berhimpit dengan 
tembereng $CFD$, maka $CFD$ akan jatuh di dalam atau diluar nya\footnote{Pernyataan ini dihindari di proposisi 3.23}, atau sama sekali tidak bertemu seperti
gambar $CGD$ dan sebuah lingkaran akan memotong lingkaran lainnya di lebih
dari dua titik. Yang adalah hal yang tidak mungkin [Prop. 3.10]. Dengan 
demikian, jika tali busur $AB$ diimpitkan dengan tali busur $CD$, tembereng
$AEB$ tidak bisa tidak berimpit dengan $CFD$. Dengan demikian, akan berimpit,
dan akan sama besar dengan nya [G.U. 4].

Dengan demikian, tembereng - tembereng serupa dari sebuah lingkaran pada tali
busur yang sama panjang, sama besar satu dengan yang lain. Yang adalah hal
yang perlu dibuktikan.

\stepcounter{section}
\section*{\centering Proposisi \thesection}
\addcontentsline{toc}{section}{Proposisi \thesection}
Cara menggambarkan lingkaran utuh dari sebuah tembereng lingkaran.
\begin{center}
\begin{tikzpicture}
	\pgfmathparse{180 + atan((2*(sin(180) - sin(120)))/(2*(cos(180) - cos(120))))}\let\cabone\pgfmathresult 
	\pgfmathparse {sqrt((2*(sin(180) - sin(120)))^2 + (2*(cos(180) - cos(120)))^2 )}\let\abone\pgfmathresult
	\pgfmathparse {abs(2*(cos(180) - cos(120)))}\let\bdone\pgfmathresult
	\pgfmathparse {abs(2*(sin(180) - sin(120)))}\let\adone\pgfmathresult

	\draw(0, 0) node [anchor=south]{A} 
		arc(120:180:2) node[anchor=east]{B}
		arc(180:240:2) node[anchor=north]{C}
		--++(90:{2*2*sin(120)})
		--++(\cabone:\abone)
		--++(0:\bdone) node[anchor=south west]{D}
		--++(90:\adone)
		--++({180+120}:2) node[anchor=west]{E}
		--++(180:{2 - \bdone}) 
		--++(0:{2 - \bdone}) --++(240:2);
	\draw(4.5, 0) node[anchor=south]{A}
		arc(90:180:2) node[anchor=east]{B}
		arc(180:270:2) node[anchor=north]{C}
		--++(90:2) node[anchor=west]{D}
		--++(180:2) --++(0:2)
		--++(90:2) --++ (225:{2*sqrt(2)});

	\draw(9, 0) node [anchor=south]{A} 
		arc(70:180:2) node[anchor=east]{B}
		arc(180:290:2) node[anchor=north]{C}
		--++(90:{1*2*sin(70)}) node[anchor=west]{D}
		--++(90:{1*2*sin(70)})
		--++({180 + atan((2*sin(70))/(2*(1+cos(70)))}: 
		{sqrt(((2*sin(70)))^2 + (2*(1+cos(70)))^2)})
		--++(0:{2*(1+cos(70))})
		--++(180:{2*cos(70)}) node[anchor=north]{E}
		--++(70:2);
\end{tikzpicture}
\end{center}
Misalkan $ABC$ adalah tembereng lingkaran. Jadi diperlukan menggambarkan 
lingkaran utuh yang padanya tembereng $ABC$ berada.

Sebab misalkan $AC$ sudah dibagi dua sama besar di titik $D$ [Prop. 1.10], dan
misalkan $DB$ sudah ditarik dari titik $D$, tegak lurus terhadap $AC$ 
[Prop. 1.11]. Dan misalkan $AB$ sudah dihubungkan. Dengan demikian, sudut 
$ABD$ bisa lebih besar, sama besar atau lebih kecil dari sudut $BAD$.

Pertama - tama misalkan $ABD$ lebih besar dari $BAD$. Dan misalkan sudut 
$BAE$, sama besar dengan sudut $ABD$, sudah digambarkan pada garis lurus
$BA$, dengan titik $A$ terletak di garis itu [Prop. 1.23]. Dan misalkan
$DB$ sudah ditarik sampai $E$, dan misalkan $EC$ sudah dihubungkan. Dengan
demikian, karena sudut $ABE$ sama besar dengan sudut $BAE$, garis lurus $EB$
dengan demikian sama panjang dengan garis lurus $EA$ [Prop. 1.16]. Dan karena
$AD$ sama panjang dengan $CD$, dan $DE$ milik bersama, masing - masing kedua garis 
lurus $AD$, $DE$ sama panjang dengan $CD$, $DE$. Dan sudut $ADE$ sama besar
dengan sudut $CDE$. Karena masing - masing adalah sudut siku - siku. Dengan
demikian, sisi alas $AE$ sama panjang dengan sisi alas $CE$ [Prop. 1.4]. 
Tetapi $AE$ sudah ditunjukkan sama panjang dengan $BE$. Dengan demikian $BE$
juga sama panjang dengan $CE$. Dengan demikian, ketiga garis lurus $AE$, $EB$
dan $CE$ sama panjang satu dengan lain. Dengan demikian, jika sebuah lingkaran
digambarkan dengan pusat titik $E$, dan jari - jari  $AE$ atau $EB$ atau $EC$,
garis tersebut juga akan melalui titik - titik di busur tembereng tersebut, 
dan lingkaran yang dimaksud sudah utuh [Prop. 3.9]. Maka sebuah lingkaran
sudah dibuat utuh dari sebuah tembereng nya. Dan jelas bahwa tembereng $ABC$
lebih kecil dari separuh lingkaran, karena pusat lingkaran berada di luar
tembereng tersebut.

Dan, dengan cara yang sama, bahkan jika sudut $ABD$ sama besar dengan $BAD$,
karena $AD$ akan sama panjang dengan masing - masing $BD$ [Prop. 1.6] dan $DC$,
ketiga garis lurus $DA$, $DB$ dan $DC$ akan sama panjang satu dengan yang lain.
Dan titik $D$ adalah pusat dari lingkaran utuh. Dan $ABC$ akan berwujud separuh
lingkaran.

Dan jika $ABD$ lebih kecil dari $BAD$, dan kita menggambarkan sudut $BAE$, sama
besar dengan sudut $ABD$, pada garis lurus $BA$ dengan titik $A$ dilaluinya
[Prop. 1.23], maka pusat lingkaran akan ada di $DB$, di dalam tembereng $ABC$.
Dan tembereng $ABC$ akan berwujud lebih dari separuh lingkaran.

Dengan demikian, sebuah lingkaran sudah di gambarkan utuh dari sebuah tembereng.
Yang adalah hal yang perlu dilakukan.

\stepcounter{section}
\section*{\centering Proposisi \thesection}
\addcontentsline{toc}{section}{Proposisi \thesection}
Pada lingkaran - lingkaran yang sama besar, sudut - sudut yang sama besar 
berdiri pada alas busur yang sama besar, baik itu sudut - sudut pada busur atau pun 
sudut - sudut di titik pusat lingkaran.

\begin{center}
\begin{tikzpicture}
\draw(0, 0) node[anchor=south]{G} circle(2.5) 
	--++ (210: 2.5) node [anchor=east]{B}
	--++ (0:{2*2.5*cos(30)}) node[anchor=west]{C} --(0, 0);
\draw({2.5*cos(210)}, {2.5*sin(210)}) 
	-- ({2.5*cos(110)}, {2.5*sin(110)}) node[anchor=south]{A}
	-- ({2.5*cos(330)}, {2.5*sin(330});
\draw({2.5*cos(270}, {2.5*sin(270}) node[anchor=north]{K};
\draw (6, 0) node[anchor=south]{H} circle(2.5)
	--++ (210: 2.5)node[anchor=east]{E}
	--++ (0:{2*2.5*cos(30)}) node[anchor=west]{F} --(6, 0);
\draw({6 + 2.5*cos(210)}, {2.5*sin(210)}) 
	-- ({ 6 + 2.5*cos(20)}, {2.5*sin(20)}) node[anchor=west]{D}
	-- ({6 + 2.5*cos(330)}, {2.5*sin(330});
\draw({6 + 2.5*cos(270}, {2.5*sin(270)}) node[anchor=north]{L};
	
\end{tikzpicture}; 
\end{center}

Misalkan $ABC$ dan $DEF$ adalah lingkaran - lingkaran yang sama besar 
dan di dalam masing - masing lingkaran misalkan $BGC$ dan $EHF$ adalah 
sudut - sudut yang sama besar di pusat - pusat lingkaran tersebut, dan 
$BAC$ dan $EDF$ adalah sudut - sudut yang sama besar di busur lingkaran.
Saya katakan bahwa busur $BKC$ sama besar dengan busur $ELF$.

Sebab misalkan $BC$ dan $EF$ sudah dihubungkan.

Dan karena lingkaran $ABC$ dan $DEF$ sama besar, jari - jari keduanya
sama panjang. Maka masing - masing kedua garis lurus $BG$ dan $GC$ 
sama panjang dengan kedua garis lurus $EH$ dan $HF$. Dan sudut $G$
sama besar dengan sudut $H$. Dengan demikian, sisi alas $BC$ sama 
panjang dengan sisi alas $EF$ [Prop. 1.4]. Dan karena sudut $A$ sama
besar dengan sudut $D$, tembereng $BAC$ dengan demikian serupa dengan
tembereng $EDF$ [Def. 3.11]. Dan keduanya berada pada tali busur yang
sama panjang. Dan tembereng yang serupa yang berada pada garis lurus
yang sama panjang sama besar satu dengan yang lain [Prop. 3.24].
Dengan demikian, tembereng $BAC$ sama besar dengan tembereng
$EDF$. Dan keseluruhan lingkaran $ABC$ juga sama besar dengan
keseluruhan lingkaran $DEF$. Dengan demikian, busur selebihnya $BKC$ 
sama besar dengan busur selebihnya $ELF$.

Dengan demikian, pada lingkaran - lingkaran yang sama besar, sudut - 
sudut yang sama besar berdiri pada alas busur yang sama besar, baik itu 
di pusat maupun pada busur lingkaran.  Yang adalah hal yang 
perlu ditunjukkan.


\stepcounter{section}
\section*{\centering Proposisi \thesection}
\addcontentsline{toc}{section}{Proposisi \thesection}

Pada lingkaran - lingkaran yang sama besar, sudut - sudut yang berdiri
pada busur yang sama besar adalah sama besar satu dengan yang lain, baik
sudut tersebut berada di pusat maupun di busur lingkaran.

\begin{center}
\begin{tikzpicture}
	\draw(0, 0) node[anchor=south]{G} circle(2.5) --
	({2.5*cos(200)}, {2.5*sin(200)}) node[anchor=east]{B} --
	({2.5*cos(100)}, {2.5*sin(100)}) node[anchor=south]{A} --
	({2.5*cos(330)}, {2.5*sin(330)}) node[anchor=west]{C} --
	(0, 0) -- ({2.5*cos(300)}, {2.5*sin(300)}) node[anchor=north west]{K};

	\draw(6, 0) node[anchor=south]{H} circle(2.5) --
	({6 + 2.5*cos(200)}, {2.5*sin(200)}) node[anchor=east]{E} --
	({6 + 2.5*cos(100)}, {2.5*sin(100)}) node[anchor=south]{D} --
	({6 + 2.5*cos(330)}, {2.5*sin(330)}) node[anchor=west]{F} --
	(6, 0);
\end{tikzpicture} 
\end{center}

Sebab misalkan sudut $BGC$ dan $EHF$ berada pada masing - masing  pusat lingkaran di $G$ dan $H$, dan sudut - sudut $BAC$ dan $EDF$ ada di masing - masing busur
lingkaran, berdiri pada busur - busur yang sama besar $BC$ dan $EF$, pada 
lingkaran - lingkaran yang sama besar $ABC$ dan $DEF$. Saya katakan sudut
$BGC$ sama besar dengan sudut $EHF$, dan sudut $BAC$ sama besar dengan sudut
$EDF$.

Sebab jika $BGC$ tidak sama besar dengan $EHF$, maka salah satu akan lebih besar.
Misalkan $BGC$ adalah yang lebih besar, dan sudut $BGK$, sama besar dengan sudut
$EHF$, sudah dibuat di garis lurus $BG$, dengan titik $G$ sebagai titik sudut
[Prop. 1.23]. Tetapi sudut - sudut sama besar pada lingkaran - lingkaran yang
sama besar akan berdiri pada busur - busur yang sama besar, jika berada di pusat
lingkaran [Prop. 3.26]. Dengan demikian busur $BK$ sama besar dengan busur $EF$.
Tetapi $EF$ sama besar dengan $BC$. Dengan demikian, $BK$ sama besar dengan $BC$.
Yang berukuran kecil menjadi sama dengan yang berukuran lebih besar. Hal yang
tidak mungkin. Dengan demikian, sudut $BGC$ tidak tidak sama besar dengan $EHF$.
Dengan demikian keduanya sama besar. Dan sudut $A$ adalah setengan sudut $BFC$,
dan sudut $D$ setengan $EHF$ [Prop. 3.20]. Dengan demikian, sudut di titik $A$
juga sama besar dengan sudut di $D$.

Dengan demikian, pada lingkaran - lingkaran yang sama besar, sudut - sudut yang
berdiri pada busur - busur yang sama besar adalah sama besar satu dengan yang
lain, baik jika berada di pusat maupun di busur lingkaran. Yang adalah hal yang
perlu ditunjukkan.

\stepcounter{section}
\section*{\centering Proposisi \thesection}
\addcontentsline{toc}{section}{Proposisi \thesection}
Pada lingkaran - lingkaran yang sama besar, garis lurus - garis lurus yang sama
panjang akan memotong busur - busur yang sama besar, dan busur yang lebih
besar akan sama dengan busur yang lebih besar, dan yang lebih kecil akan sama
besar dengan yang lebih kecil.  

\begin{center}
\begin{tikzpicture}
	\draw(0, 0) node[anchor=south]{K} circle(2.5) --
	({2.5*cos(210)}, {2.5*sin(210)}) node[anchor=east]{A} --
	({2.5*cos(330)}, {2.5*sin(330)}) node[anchor=west]{B} --
	(0, 0);
	\draw ({2.5*cos(270)}, {2.5*sin(270)}) node[anchor=north]{G};
	\draw ({2.5*cos(90)}, {2.5*sin(90)}) node[anchor=south]{C};

	\draw(6, 0) node[anchor=south]{L} circle(2.5) --
	({6 + 2.5*cos(210)}, {2.5*sin(210)}) node[anchor=east]{D} --
	({6 + 2.5*cos(330)}, {2.5*sin(330)}) node[anchor=west]{E} --
	(6, 0);
	\draw ({6 + 2.5*cos(270)}, {2.5*sin(270)}) node[anchor=north]{H};
	\draw ({6 + 2.5*cos(90)}, {2.5*sin(90)}) node[anchor=south]{F};

\end{tikzpicture} 
\end{center}

Misalkan $ABC$ dan $DEF$ adalah lingkaran - lingkaran yang sama besar, dan 
misalkan $AB$ dan $DE$ adalah garis - garis lurus yang sama panjang pada
lingkaran - lingkaran ini, memotong masing - masing busur - busur yang lebih 
besar $ACB$ dan $DFE$ dan masing - masing busur - busur yang lebih kecil 
$AGB$ dan $DHE$. Saya katakan busur yang besar $ACB$ sama besar dengan busur
yang lebih besar $DFE$, dan busur yang lebih kecil $AGB$ sama besar dengan
$DHE$.

Sebab misalkan pusat - pusat lingkaran $K$ dan $L$ sudah ditemukan [Prop. 3.1],
dan misalkan $AK$, $KB$, $DL$ dan $LE$ sudah dihubungkan.

Dan karena $ABC$ dan $DEF$ adalah lingkaran - lingkaran yang sama besar, maka
jari - jari keduanya juga sama besar [Def. 3.1]. Dengan demikian, masing -
masing kedua garis lurus $AK$, $KB$ sama panjang dengan kedua gari lurus $DL$
dan $LE$. Dan alas $AB$ sama panjang dengan alas $DE$. Dengan demikian, sudut
$AKB$ sama besar dengan sudut $DLE$ [Prop. 1.8]. Dan sudut - sudut sama besar
berdiri pada busur - busur sama besar jika keduanya berada pada pusat - pusat
lingkaran [3.29]. Dengan demikian busur $AGB$ sama dengan $DHE$. Dan keseluruhan
lingkaran $ABC$ juga sama besar dengan keseluruhan lingkaran $DEF$. Dengan
demikian busur selebihnya $ACB$ juga sama besar dengan busur selebihnya $DFE$/


Dengan demikian, pada lingkaran - lingkaran sama besar, garis lurus yang sama
panjang memotong busur - busur yang sama besar, busur yang lebih besar sama besar
dengan busur yang lebih besar, dan busur yang lebih kecil sama besar dengan
busur yang lebih kecil. Yang adalah hal yang perlu ditunjukkan.


\begin{thebibliography}{9}
\bibitem{fitzpatrick}
Richard Fitzpatrick.
\textit{EUCLID’S ELEMENTS OF GEOMETRY , The Greek text of J.L Heiberg(1883 - 1885)}.
Revised and corrected - 2008 
\bibitem{jafareffendi}
M Jafar Effendi.
\textit{Geometri Euclid}.
Makalah mata kuliah Geometri Fakultas Pasca Sarjana Program Studi Pendidikan Matematika 2012.
\bibitem{clarkuni}
https://mathcs.clarku.edu/~djoyce/java/elements/elements.html 
\end{thebibliography}
\end{document} 
