\documentclass[a4paper, 12pt]{book}
  
\usepackage{algorithm}
\usepackage{amsmath}
\usepackage{amssymb}
\usepackage{multicol} 
\usepackage{color, colortbl}
\usepackage{tikz}
\usepackage[bahasa]{babel}

\definecolor{Gray}{gray}{0.9}
\definecolor{LightCyan}{rgb}{0.70,1,1}
\definecolor{OliveGreen}{rgb}{0,0.6,0}
\definecolor{ashgrey}{rgb}{0.7, 0.75, 0.71}
\definecolor{aliceblue}{rgb}{0.94, 0.97, 1.0}


\newcommand{\norm}[1]{\left\lVert#1\right\rVert}

\begin{document}
\title{\textbf{Element - Element dari Geometri Euklid 
(Cuplikan Dalam Indonesia)}}
\author{Euclid dari Alexandria}
\maketitle
\pagenumbering{gobble}
\pagestyle{empty}
\setcounter{chapter}{1}
\chapter*{\centering \chaptername{ \thechapter} \\Dasar - Dasar dari Geometri Bidang Datar Yang Melibatkan Garis Lurus}
%\author{Euclid of Alexandria}
%\begin{titlepage}
%\begin{center}
%{\huge \bfseries Element Element Geometri Euklid - Terjemahan Bahasa Indonesia}
%{\huge Element Element Geometri Euklid - Terjemahan Bahasa Indonesia}
%\end{center}
%\end{titlepage}
%comment comment comment
%\chaptername{\thechapter}

\pagenumbering{alph} 

\begin{center}
\textbf{Definisi - definisi}
\end{center}

\begin{enumerate}
\item Sebuah titik adalah yang darinya tidak ada bagian lagi.
\item Dan sebuah garis adalah panjang yang tidak memiliki lebar.
\item Dan ujung - ujung dari sebuah garis adalah titik - titik.
\item Sebuah garis lurus adalah garis yang terletak dengan rata dengan titik - titik padanya.
\item Dan sebuah bidang adalah yang memiliki panjang dan lebar saja.
\item Dan ujung - ujung sebuah bidang adalah garis - garis.
\item Dan bidang datar adalah bidang yang terletak dengan rata dengan garis - garis lurus padanya.
\item Dan sudut bidang datar adalah kemiringan garis - garis tersebut 
satu dengan lain, ketika dua garis dalam sebuah bidang datar bertemu satu dengan
yang lain, dan tidak terletak pada garis lurus.
\item Dan ketika garis - garis yang mengapit sudut tersebut adalah garis lurus maka sudut
tersebut disebut sebagai sudut \textit{rectilinear}.
\item Dan ketika sebuah garis lurus berdiri diatas garis lurus lain dan 
membentuk sudut - sudut yang bersebelahan yang sama besar satu dengan yang 
lain, masing - masing sudut yang sama besar tersebut adalah sudut siku - siku,
dan garis lurus yang pertama dikatakan tegak lurus terhadap garis lurus kedua
di mana dia berdiri.
\item Sebuah sudut tumpul adalah sudut yang lebih besar dari sudut siku - siku.
\item Sebuah sudut lancip adalah sudut yang lebih kecil dari sudut siku - siku.
\item Sebuah batas adalah ujung - ujung dari sesuatu.
\item Sebuah bangun adalah yang berada didalam batas atau batas - batas.
\item Sebuah lingkaran adalah bentuk bidang datar yang dikandung dalam sebuah 
sebuah garis [yang disebut keliling], (sedemikian rupa) sehingga semua 
garis - garis lurus menyebar kearah keliling dari satu titik di dalam bidang
tersebut sehingga garis - garis lurus tersebut sama panjang.
\item Dan titik tersebut dinamakan pusat lingkaran.
\item Dan diameter dari lingkaran adalah sembarang garis lurus yang melalui 
pusat lingkaran tadi dan berhenti di masing - masing arah pada
keliling lingkaran. Dan garis lurus seperti itu juga membagi lingkaran 
menjadi dua bagian sama besar.
\item Dan setengah lingkaran adalah bidang yang dikandung dari diameter dan 
keliling yang dipotong diameter. Dan pusat setengah lingkaran sama dengan 
pusat dari lingkaran.
\item Bangun - bangun rectilinear adalah bangun - bangun yang berada di dalam
garis - garis lurus: bangun segitiga adalah bangun yang berada di dalam tiga buah
garis lurus, bangun segiempat adalah bangun yang berada di dalam empat garis lurus dan 
segibanyak adalah bidang yang berada di dalam lebih dari empat garis lurus.
\item Dan pada bangun segitiga: sebuah segitiga sama sisi adalah yang memiliki tiga
sisi sama besar, segitiga sama kaki adalah yang memiliki dua sisi sama besar, dan 
segitiga sembarang adalah yang tidak memiliki sisi yang sama besar.
\item Dan selanjutnya pada bangun segitiga: segitiga siku - siku adalah yang memiliki
satu sudut siku - siku, segitiga tumpul adalah yang memiliki satu sudut tumpulm dan 
segitiga lancip adalah yang memiliki tiga sudut lancip.
\item Dan pada bangun segiempat: Bujur sangkar adalah yang segiempat yang
sudut - sudut nya siku - siku dan sisi - sisinya sama panjang, persegipanjang
adalah yang sudut - sudut nya siku dan tetapi tidak semua sisinya sama panjang, 
belah ketupat adalah yang memiliki sisi - sisi sama panjang tetapi
sudut - sudutnya tidak siku - siku, jajarang genjang adalah yang sudut - sudut 
dan sisi - sisi yang berhadapan sama besar tetapi tidak siku - siku dan tidak 
semua sisi sama panjang, dan persegiempat lainnya yang tidak dalam kategori tersebut
disebut dengan trapezium.
\item Garis - garis paralel adalah garis - garis lurus yang jika berada dalam 
bidang yang sama dan dipanjangkan hingga tak terhingga ke masing - masing arah
tidak bertemu satu dengan yang lain.  
\end{enumerate}

\begin{center}
\textbf{Postulat - postulat}
\end{center}
\begin{enumerate}
\item Anggap sudah didalilkan melalui dua titik sembarang dapat dibuat garis lurus.
\item Dan memanjangkan sebuah ruas garis dengan panjang tertentu tanpa terputus.
\item Dan dengan sembarang pusat dan jari -jari dapat dibuat sebuah lingkaran.
\item Dan bahwa semua sudut siku - siku sama besar.
\item Dan jika suatu garis lurus memotong dua buah garis lurus yang lain dan membentuk sudut - sudut dalam pada sisi yang sama yang jumlah besar sudutnya lebih kecil dari dua  sudut siku - siku, maka dua garis lurus yang di sisi lainnya, jika dipanjangkan sampai tak terhingga, akan bertemu di sisi tersebut, dan dua garis lurus yang berada di sisi sudut dalam yang lebih kecil dari dua sudut siku siku tidak akan pernah bertemu jika dipanjangkan di sisi tersebut.
\end{enumerate}

\begin{center}
\textbf{Gagasan - Gagasan Umum}
\end{center}
\begin{enumerate}
\item Hal - hal yang nilai nya sama dengan satu hal lain diluar keduanya adalah sama satu dengan yang lain. 
\item Dan jika hal - hal yang sama masing - masing ditambahkan ke hal yang lain yang sama, maka hasil nya sama satu dengan yang lain. 
\item Dan jika hal - hal yang sama masing - masing dikurangi dengan hal yang lain yang sama, maka hasil nya sama satu dengan yang lain. 
\item Dan hal - hal yang berhimpit satu dengan yang lain adalah sama.
\item Dan keseluruhan lebih besar dari bagian nya.
\end{enumerate}

\stepcounter{section}
\section*{\centering \thesection} 
Untuk menggambarkan segitiga sama sisi pada sebuah ruas garis.  

\begin{tikzpicture}
\draw (-2, 0) circle (4);
\draw (2, 0) circle (4);
\draw (-6, 0) node [anchor=east]{D} (-2, 0) node [anchor=east]{A} (-2, 0) 
-- (2, 0) node[anchor=west]{B} (6, 0) node[anchor=west]{E} 
(0, 4.0)node[anchor=north]{C};
\draw (-2, 0) -- (0, 3.464101) -- (2, 0) circle; 
\end{tikzpicture} 
Misalkan $AB$ adalah ruas garis.

Dibutuhkan untuk menggambarkan segitiga sama sisi pada ruas garis $AB$.

Misalkan $BCD$ adalah lingkaran dengan pusat $A$ dan jari jari $AB$ sudah 
digambarkan [Post. 3], dan lagi misalkan lingkaran $ACE$ dengan pusat $B$ 
dan jari- jari $BA$ sudah digambarkan [Post. 3]. Dan misalkan garis lurus 
$CA$ dan $CB$ dipertemukan dari titik $C$, dimana lingkaran - lingkaran 
tersebut saling memotong, masing - masing ke titik $A$ dan $B$ [Post. 1].  


Dan karena $A$ adalah pusat dari lingkaran $CDB$, maka $AC$ sama dengan $AB$
[Def. 1.15]. Dan karena titik $B$ adalah pusat dari lingkaran $CAE$, maka
$BC$ sama dengan $BA$ [Def. 1.15]. Tetapi $CA$ juga sudah dibuktikan sama 
dengan $AB$. Dengan demikian $CA$ dan $CB$ adalah sama dengan $AB$. Tetapi
hal - hal yang nilainya sama dengan satu hal lain di luar keduanya adalah sama
satu dengan yang lain [GU. 1]. Dengan demikian, $CA$ juga sama dengan $CB$.  
Dengan demikian ketika garis lurus $CA$, $AB$ dan $BC$ sama satu dengan 
yang lain.

Dengan demikian, segitiga $ABC$ adalah sama sisi, dan dibangun dari
satu ruas gari $AB$. Yang adalah hal yang kita butuhkan.  

\stepcounter{section}
\section*{\centering \thesection} 

Untuk menempatkan sebuah ruas garis yang sama dengan ruas garis tertentu
pada satu titik tertentu (sebagai salah satu ujungnya).

\begin{tikzpicture}
\draw [-] (0, 0) -- (2., 3) node [anchor=south west]{$C$};
\draw [-] (-2, -.0) -- (0, 0);
\draw (0, 0) node [anchor=west] {$B$}-- (120:2) node [anchor = south] {$D$}--+ (240:2) node [anchor=east] {$A$};
\draw (0, 0) -- (120:2) --+ (240:9)node [anchor = south east] {$E$};
\draw (0, 0) -- (-60:7) node [anchor = south west]{$F$};
\draw (0, 0) circle (3.606);
%\draw (-1, 1.73205) -- (240:8);
\draw (-1, 1.73205) circle (5.606) (-7, 0) node [anchor = west] {$K$};
\draw (-3.8, -3.8) node {$L$};
\draw (2.3, -3.4) node {$G$};
\draw (0.0, 4) node {$H$}; 
\end{tikzpicture} 

Misalkan $A$ adalah suatu titik tertentu, dan $BC$ adalah suatu ruas garis
tertentu. Dibutuhkan untuk menempatkan garis lurus di $A$ yang sama dengan
garis lurus $BC$.

Sebab, misalkan ruas garis $AB$ sudah terhubung dari titik $A$ ke titik $B$
[Post. 1], dan misalkan segitiga sama sisi $DAB$ sudah digambarkan 
padanya [Prop 1.1]. Dan misalkan garis lurus $AE$ dan $BF$ sudah 
ditarik lurus dengan masing masing $DA$ dan $DB$ [Post. 2]. Dan
misalkan lingkaran $CGH$ dengan pusat $B$ dan jari - jari $BC$ sudah 
digambarkan [Post. 3], dan lagi misalkan lingkaran $GKL$ dengan pusat $D$
dan jari - jari $DG$ sudah digambarkan [Post. 3].  

Dengan demikian, karena titik $B$ adalah pusat dari lingkaran $CGH$, $BC$ sama 
dengan $BG$ [Def. 1.15].  Dan lagi karena titik $D$ adalah pusat dari 
lingkaran $GKL$, $DL$ sama dengan $DG$ [Def. 1.15]. Dan di dalam ini, $DA$
sama dengan $DB$. Dengan demikian sisa dari $AL$ sama dengan sisa $BG$
[G.U. 3]. Tetapi $BC$ juga sudah dibuktikan sama dengan $BG$. Tetapi hal - hal
yang nilainya sama dengan satu hal lain di luar keduanya adalah sama satu 
dengan yang lain [G.U. 1]. Dengan demikian AL juga sama dengan BC.

Dengan demikian, ruas garis AL sama dengan ruas garis tertentu
$BC$, yang terletak pada garis tertentu $A$. Hal yang kita butuhkan.  

Ilustrasi lain
\begin{center}
\begin{tikzpicture}
\draw [-, ashgrey] (0, 0.0) node [anchor = north]{$A$} -- (2, 0.0) node [anchor = west] {$B$};
\draw [-, dashed, ashgrey] (0, 0) -- (120:3) --++ (240:3)--++(-0:3); 
\draw [-] (-1.5, 2.6) node [anchor = south] {$C$}-- (.5, 2.6);
\draw [-, dashed, ashgrey] (-1.5, 2.6) circle(2);
\draw [-, dashed, ashgrey] (-3.0, 0.0) circle(5);
\draw [-, dashed, ashgrey] (-3.0, 0.0) --++(0:6);
\draw [-, dashed, ashgrey] (-3.0, 0.0) --++(60:6);
\end{tikzpicture} 
\end{center}

\stepcounter{section}
\section*{\centering \thesection} 

Pada dua buah ruas garis yang tidak sama panjang, cara untuk
memotong dari ruas garis yang lebih besar ruas garis yang sama panjang dengan
ruas garis yang lebih pendek.

\begin{center}
\begin{tikzpicture}
\draw [-] (0, 0.0) node [anchor = north]{$A$} -- (7, 0.0) node [anchor = west] {$B$};
\draw [-, dashed, ashgrey] (0, 0)--(135 : 2.82842712474)--(195:2.82842712474)--(0, 0);
\draw [-, dashed, ashgrey] (0, 0)--(-45 : 5);
\draw [-] (-2, 2) -- (.5, 2);
\draw (-0.5, 2.9) node {$C$};
\draw [-, dashed, ashgrey] (-2, 2) circle(2.5);
\draw [-, dashed, ashgrey] (-2, 2) --++(75:2.5);
\draw [-, dashed, ashgrey] (0, 0)--(135 : 2.82842712474)--(195:2.82842712474)--(15: 5);
\draw [-, dashed, ashgrey] (-2.73205080756, -0.732050807567) circle(5.32842712474);
\draw [-] (0, 0.0) node [anchor = north]{$A$} -- (15:2.5) node [anchor = west] {$D$};
\draw [-] (0, 0.0) circle (2.5);
\draw (2.5, 0) node [anchor = north west] {$E$};
\draw (0, -2.5) node [anchor = north] {$F$};
\end{tikzpicture}
\end{center}

Misalkan $AB$ dan $C$ adalah dua buah garis tak sama panjang yang dimaksud,
dimana $AB$ adalah garis yang lebih panjang. Jadi dibutuhkan untuk memotong
dari ruas garis $AB$ sebuah ruas garis yang sama panjang dengan $C$.

Misalkan ruas garis $AD$ ,sama panjang dengan ruas garis $C$, sudah ditempatkan
pada titik $A$ sebagai salah satu ujungnya [Prop. 1.2]. Dan misalkan lingkaran
$DEF$ sudah digambarkan dengan titik $A$ dan jari - jari $AD$ [Post. 3].

Dan karena titik $A$ adalah pusat dari lingkaran $DEF$, $AE$ dengan demikian
sama dengan $AD$ [Def.1.14]. Tetapi $C$ juga sama panjang dengan $AD$. Dengan 
demikian $AE$ dan $C$ adalah sama dengan $AD$. Sehinggan $AE$ juga sama dengan
$C$ [G.U.1].

Dengan demikian , untuk dua buah ruas garis yang tidak sama panjang $AB$ 
dan $C$, ruas garis $AE$ yang sama panjang dengan $C$ telah dipotong dari 
$AB$. Hal yang mana yang perlu kita lakukan.

\stepcounter{section}
\section*{\centering \thesection} 

Jika pada dua buah segitiga, yang pertama  memiliki dua buah sisi yang sama 
panjang dengan dua buah sisi segitiga yang satunya, dan pada keduanya sudut
yang diapit oleh sisi - sisi yang sama panjang tadi adalah sama besar, 
maka alas (sisi ketiga) dari kedua segitiga tersebut adalah sama besar, dan kedua 
segitiga tersebut akan sama, dan sudut - sudut selebihnya yang menghadap dua 
sisi - sisi yang sama panjang yang pertama tadi adalah sama besar pada 
kedua segitiga.
\begin{center}
\begin{tikzpicture}
\draw(-3, -1) node[anchor=east] {$B$} -- (-1,  1) node[anchor=south]{A} 
     -- (0, -1) node[anchor=west]{$C$} -- (-3, -1); 
\draw(2, -1) node[anchor=east] {$E$} -- (4,  1) node[anchor=south]{D} 
     -- (5, -1) node[anchor=west]{$F$} -- (2, -1); 
\end{tikzpicture}
\end{center}

Misalkan $ABC$ dan $DEF$ adalah dua buah segitiga yang memiliki 
sisi - sisi $AB$ dan $AC$ yang sama panjang dengan sisi - sisi $DE$ dan $DF$.
Yakni $AB$ dengan $DE$  dan $AC$ dengan $DF$. Dan misalkan sudut $BAC$ sama
besar dengan sudut $EDF$. Saya katakan bahwa sisi alas $BC$ juga sama panjang
dengan  sisi alas $EF$, dan segitiga $ABC$ juga akan sama dengan segitiga 
$DEF$, dan sudut sudut selebihnya yang menghadap sisi - sisi yang sama yang 
akan sama satu dengan yang lain. Yakni $ABC$ dengan $DEF$ dan $ACB$ dengan 
$DFE$.

Karena jika segitiga $ABC$ di tempatkan ke segitiga $DEF$, titik $A$ 
ditempatkan pada titik $D$, dan ruas garis $AB$ pada $DE$, maka
titik $B$ juga akan berhimpit dengan $E$ dengan alasan $AB$ sama panjang
dengan $DE$. Jadi karena $AB$ berhimpit dengan $DE$, maka ruas garis $AC$ 
juga akan berhimpit dengan $DF$, dengan alasan sudut $BAC$ sama besar 
dengan $DEF$. Jadi titik $C$ juga akan berhimpit dengan titik $F$, lagi
dikarenakan alasan $AC$ sama panjang dengan $DF$. Tapi titik $B$ juga tepat
berhimpit dengan titik $E$, sehingga sisi alas $BC$ akan berhimpit
dengan sisi alan $EF$. Karena jika $B$ berhimpitan dengan $E$, dan $C$ dengan
$F$ , dan sisi alas $BC$ tidak berhimpin dengan $EF$, maka dua buah ruas
garis akan meliputi daerah tersebut. Hal yang sangat mustahil [Post. 1].
Dengan demikian sisi alas $BC$ juga berhimpit dengan $EF$ dan sama panjang
[G.U.4], dan sudut - sudut selebihnya akan sama besar [G.U.4]. Yakni
$ABC$ sama dengan $DEF$ dan $ABC$ sama dengan $DFE$ [G.U.4].

Dengan demikian, jika dua buah segitiga memiliki dua buah sisi yang sama
panjang dan sebuah sudut yang diapit oleh sisi - sisi tadi adalah sama besar,
maka kedua nya juga akan memiliki sisi alas yang sama panjang, dan kedua
segitiga tersebut adalah sama, dan sudut - sudut selebihnya yang menghadap
sisi - sisi panjang tadi akan sama besar. Hal yang mana perlu kita buktikan.

\stepcounter{section}
\section*{\centering \thesection} 

Pada segitiga sama sisi, sudut - sudut alas sama besar satu dengan yang lain, 
dan jika sisi - sisi yang sama panjang dipanjangkan lagi maka sudut - sudut 
yang ada di bawah sisi alas akan sama besar satu dengan yang lain.

\begin{center}
\begin{tikzpicture}
\draw [-, black] (-2, 0) node [anchor = north] {$D$} -- (0, 5) node [anchor = south]{$A$};
\draw [-, black] (2, 0) node [anchor = north] {$E$} -- (0, 5);
\draw [-, black] (-1.2, 2.0) node [anchor = east] {$B$} -- (1.2, 2.0) node [anchor = west]{$C$};
\draw [-, black] (-1.2, 2.0) -- (1.6, 1.0) node [anchor = west]{$G$};
\draw [-, black] (1.2, 2.0) -- (-1.6, 1.0) node [anchor = east]{$F$}; 
\end{tikzpicture}
\end{center}

Misalkan $ABC$ adalah segitiga sama sisi dengan sisi $AB$ sama panjang dengan 
sisi $AC$, dan misalkan ruas garis $BD$ and $CE$ sudah dipanjangkan 
masing - masing dari $AB$ dan $AC$[Post. 2].  Saya katakan bahwa sudut $ABC$ 
sama dengan $ACB$ dan sudut $CBD$ sama dengan $BCE$.

Sebab misalkan titik $F$ sudah diambil secara acak pada $BD$, dan misalkan 
$AG$ sudah dipotong dari  $AE$, sama panjang dengan $AF$ [Prop. 1.3]. Dan 
misalkan garis - garis $FC$ dan $GB$ sudah dihubungkan [Post. 1].

Kenyataanya, karena $AF$ sama panjang dengan $AG$, dan $AB$ dengan $AC$, kedua
ruas garis $FA$, $AC$ sama panjang dengan kedua ruas garis $GA$, $AB$. Keduanya
juga mengapit sudut yang sama yakni $FAG$. Dengan demikian, sisi alas $FC$ sama 
dengan sisi alas $GB$, dan segitiga $AFC$ sama dengan segitiga $AGB$, dan sudut
- sudut nya yang dihadapan sisi - sisi yang sama panjang sama besar satu sama
lain [Prop. 1.4]. Yakni $ACF$ dengan $ABG$ dan $AFC$ dengan $AGB$. Dan karena
panjang $AF$ sama dengan $AG$ dan di dalamnya $AB$ sama dengan $AC$, maka panjang
ruas garis sisanya $BF$ sama dengan $CG$ [G.U.3]. Tetapi $FC$ juga sudah kita
tunjukkan sama dengan $GB$. Jadi kedua garis lurus $BF$, $FC$ adalah sama dengan
dua garis lurus $CG$, $GB$, masing - masing, dan sudut $BFC$ sama besar dengan 
sudut $CGB$, dan sisi alas ada $BC$ adalah milik keduanya. Dengan demikian 
segitiga $BFC$ akan sama dengan segitiga $CGB$ dan sudut - sudut sisanya yang
menghadap sisi - sisi yang sama besar akan sama besar dengan 
sudut - sudut sisanya yang bersesuaian [Prop 1.4]. Dengan demikian $FBC$ sama
besar dengan $GCB$ dan $BCF$ dengan $CBG$. Dengan demikian, karena sudut 
keseluruhan $ABG$ sudah ditunjukkan sama dengan sudut $ACF$, yang didalamnya 
$CBG$ sama dengan $BCF$, sisa pengurangan $ABC$ dengan demikian sama dengan 
sisa pengurangan $ACB$ [G.U. 3]. Dan keduanya adalah alas dari segitiga $ABC$.
Dan $FBC$ juga sudah ditunjukkan sama dengan $GCB$. Dan keduanya ada dibawah
sisi alas.

Dengan demikian, untuk segitiga sama sisi, sudut - sudut yang pada alas sama
besar satu dengan yang lain, dan jika sisi - sisi yang sama panjang 
masing - masing dipanjangkan maka sudut - sudut dibawah sisi alas akan sama 
besar satu dengan yang lain. Yang adalah hal yang ingin kita buktikan.

\stepcounter{section}
\section*{\centering \thesection} 
Jika sebuah segitiga memiliki dua buah sudut yang sama besar satu dengan yang 
lain, maka sisi - sisi yang menghadap sudut - sudut yang sama besar tersebut
akan sama panjang satu dengan yang lain.
\begin{center}
\begin{tikzpicture}
\draw[-](0, 0)node[anchor=east] {B} --++(0:4) node [anchor=west] {$C$} --++
(110:5.84760880033) node [anchor=south] {$A$} --++(250:5.84760880033);
\draw[-](4, 0)--++(127:4.477) node [anchor=south east] {$D$};
\end{tikzpicture}
\end{center} 

Misalkan $ABC$ adalah segitiga dengan sudut $ABC$ sama dengan $ACB$. Saya 
katakan bahwa sisi $AB$ juga sama panjang dengan sisi $AC$.

Sebab jika $AB$ tidak sama panjang dengan $AC$ maka salah satu daripadanya
akan lebih panjang. Misalkan $AB$ adalah sisi yang lebih panjang. Dan 
misalkan $DB$ sama panjang dengan sisi yang lebih pendek $AC$, sudah dipotong
dari sisi yang lebih panjang $AB$ [Prop. 1.3]. Dan misalkan $DC$ sudah
terhubung [Post.1].

Dengan demikian, karena $DB$ sama panjang dengan $AC$, dan $BC$ adalah milik
keduanya, sisi - sisi yang sama $DB$, $BC$ sama panjang dengan sisi - sisi
$AC$, $CB$, masing - masing, dan sudut $DBC$ sama besar dengan sudut $ACB$.
Dengan demikian sisi alas $DC$ sama besar dengan sisi alas $AB$, dan segitiga
$DBC$ akan sama dengan segitiga $ACB$ [Prop. 1.4], yang lebih kecil sama
dengan yang lebih besar. Ide yang benar - benar tidak masuk akal [G.U. 5].
Dengan demikian keduanya sama.

Dengan demikian, jika sebuah segitiga memiliki dua buah sudut yang sama besar
satu dengan yang lain maka sisi - sisi yang menghadap sudut - sudut yang sama
besar akan juga sama panjang satu dengan yang lain. Yang adalah hal yang ingin
 kita buktikan

\setcounter{section}{13}
\section*{\centering \thesection} 
Jika sebuah garis lurus berdiri pada sebuah garis lain membentuk sudut - sudut, maka
sudut - sudut yang dibentuk adalah dua buah sudut sikut siku , atau sudut sudut yang
jumlahnya sama dengan dua buah sudut siku - siku.
\begin{tikzpicture}
\draw [-, black](-2,-2) -- (2, -2);
\draw (-2, -2.5) node {D} (0, -2.5) node {B} (2, -2.5) node {C};
\draw [-, black](-0,-2) -- (0, 2); 
\draw [-, black](-0,-2) -- (3, 2);
\draw (0, 2.5) node {E} (3, 2.5) node {A};
\end{tikzpicture}

Sebab misalkan garis lurus $AB$ berdiri pada garis lurus $CD$ membentuk 
sudut - sudut $CBA$ dan $ABD$. Saya katakan bahwa sudut - sudut $CBA$ dan
$ABD$ adalah masing - masing sudut siku - siku atau besar keduanya adalah 
sejumlah dua buah sudut siku - siku.

Pada kenyataannya, jika $CBA$ sama dengan $ABD$ maka keduanya adalah dua buah 
sudut siku - siku [Def 1.10]. Jika tidak, asumsikan $BE$
\setcounter{section}{29}
\section*{\centering \thesection} 
Sebuah ruas garis yang jatuh memotong ruas - ruas garis yang sejajar akan membuat
sudut - sudut yang berselang seling sama besar, sudut luar sama dengan sudut dalam
yang bertentangan, dan jumlah dari sudut - sudut pada sisi yang sama sama dengan
besar 2 buah sudut siku - siku.  

\begin{tikzpicture}
\draw(-2, 4) node[anchor=south]{E} (-2, 4) -- (3, -2) node[anchor=north]{F};
\draw (-6, 3) node[anchor=east]{A}(-6, 3) -- (6, 3) node [anchor=west]{B};
\draw (-6, -1) node[anchor=east]{C}(-6, -1) -- (6, -1) node [anchor=west]{D};
\end{tikzpicture} 

Misalkan ruas garis $EF$ jatuh memotong ruas - ruas gais $AB$ dan $CD$. 
Saya katakan bahwa $EF$ membuat sudut - sudut yang berselang seling 
$AGH$ dan $GHD$ sama besar, sudut luar $EGB$ sama besar dengan sudut dalam
yang bertentangan $GHD$, dan jumlah dari sudut - sudut dalam yang berada 
pada sisi yang sama $BGH$ dan $GHD$, sama dengan dua buah sudut siku - siku.

Karena jika $AGH$ tidak sama besar dengan $GHD$ maka salah satunya pasti
lebih besar. Misalkan $AGH$ adalah yang lebih besar. Misalkan $BGH$ 
ditambahkan kepada keduanya. Maka jumlah dari $AGH$ dan $BGH$ akan 
lebih besar dari jumlah dari $BGH$ dan $GHD$. Tetapi besar dari $AGH$
dan $BGD$ harus sama dengan dua buah sudut siku - siku [1.13]. Jadi
jumlah $BGH$ dan $GHD$ juga lebih kecil dari dua buah sudut siku - siku.

\setcounter{section}{31}
\section*{\centering \thesection} 
Untuk menggambarkan sebuah ruas garis yang sejajar terhadap sebuah ruas 
garis tertentu yang melalui sebuah titik tertentu.

\begin{tikzpicture}
\draw(2, 2) node[anchor=south]{A} (2, 2) -- (-2, -2) node[anchor=north]{D};
\draw (-6, 2) node[anchor=east]{E}(-6, 2) -- (6, 2) node [anchor=west]{F};
\draw (-6, -2) node[anchor=east]{B}(-6, -2) -- (6, -2) node [anchor=west]{C};
\end{tikzpicture}

Misalkan $A$ adalah sebuah titik tertentu, dan $BC$ adalah sebuah garis tertentu. Jadi
yang dimaksudkan adalah meggambarkan satu garis yang sejajar dengan garis $BC$ 
dan melalui titik $A$.

Misalkan titik $D$ adalah sembarang titik yang berada di $BC$, and $AD$ terhubung. 
Dan misalkan (sudut) $DAE$, yang sama besar dengan sudut $ADC$, telah dibuat 
pada ruas garis $SA$ dengan titik A barada diujungnya [1.23]. Dan misalkan garis
lurus $AF$ adalah perpanjangan $EA$.

Dan karena ruas garis $AD$, jatuh didalam  kedua ruas garis $BC$ dan $EF$, membuat 
sudut - sudut yang berseberangan $EAD$ dan $ADC$ sama besar, $EAF$ dengan demikian
sejajar dengan $BC$ [1.27]


\setcounter{section}{34}
\section*{\centering \thesection}

Pada bidang - bidang jajaran genjang, sisi - sisi dan sudut - sudut yang saling
bertentangan besarnya sama satu dengan yang lain , dan sebuah diagonal membaginya
menjadi dua bagian yang sama besar.

\begin{tikzpicture}
\draw (-4, 2) node[anchor=east]{A}(-4, 2) -- (6, 2) node [anchor=west]{B};
\draw (-6, -2) node[anchor=east]{C}(-6, -2) -- (4, -2) node [anchor=west]{D};
\draw (-6, -2) -- (-4, 2);
\draw (4, -2) -- (6, 2);
\draw (-6, -2) -- (6, 2); 
\end{tikzpicture} 

Misalkan $ABCD$ ada sebuah jajaran genjang, dan $BC$ adalah diagonalnya. Saya katakan
bahwa untuk jajaran genjang $ABCD$, sisi sisi yang bertentangan dan sudut - sudut yang bertentangan
sama besar satu dengan yang lain, dan diagonal $BC$ membaginya menjadi dua bagian yang sama besar.

Karena $AB$ sejajar dengan $CD$, dan ruas garis $BC$ jatuh memotong keduanya, sudut - sudut yang 
berhadapan $ABC$ dan $BCD$ sama besar satu dengan yang lain [1.29]. Dan karena $AC$ sejajar dengan
$BD$, dan $BC$ memotong keduanya, sudut - sudut yang berseberangan $ACB$ dan $CBD$ sama besar satu 
dengan yang lain [1.29]. Dengan demikian $ABC$ dan $BCD$ adalah dua buah segitiga yang memiliki
kedua sudut masing - masing $ABC$ dan $BCA$ sama besar dengan kedua sudut $BCD$ dan $CDB$, dan satu 
sisi sama besar, yaitu $BC$, yang di sini milik bersama kedua segitiga.
Dengan demikian sisi sisi lainnya akan sama besar satu dengan yang lain, 
dan satu sudut lainnya akan sama satu dengan yang lain [1.26]. Dengan demikian, sisi $AB$ 
sama besar dengan $CD$ , dan sisi $AC$ dengan $BD$. Lebih lanjut, sudut $BAC$ sama besar
dengan $CDB$. Dan karena $ABC$ sama besar dengan $BCD$, dan $CBD$ dengan $ACB$, keseluruhan
sudut segitiga $ABD$ dengan demikian sama dengan keseluruahn  sudut segitiga $ACD$

\stepcounter{section}

\section*{\centering \thesection}
\par Jajaran - jajaran genjang yang memiliki alas yang sama dan berada diantara dua
garis - garis sejajar yang sama , sama (besarnya) satu dengan yang lain.  

\begin{tikzpicture}
\draw (-8, 2) node[anchor=south]{A}(-8, 2) -- (-5, 2) 
node [anchor=south]{D} (-5, 2) -- (2, 2) 
node [anchor=south]{E} (2, 2) -- (5, 2) node[anchor=south]{F};
\draw (-6, -2) node[anchor=north]{B}(-6, -2) -- (-3, -2) node [anchor=north]{C};
\draw (-6, -2) -- (-8, 2);
\draw (-3, -2) -- (-5, 2);
\draw (-6, -2) -- (2, 2);
\draw (-3, -2) -- (5, 2);
\draw (-3.75, -0.5) node[anchor=west]{G};
\end{tikzpicture}

Misalkan $ABCD$ dan $EBCF$ adalah bidang - bidang jajaran genjang yang memiliki alas yang sama yaitu $BC$, 
dan berada diantara dua buah garis sejajar $AF$ dan $BC$. Saya katakan bahwa $ABCD$ sama dengan bidang
jajaran genjang $EBCF$.  

Karena $ABCD$ adalah sebauh jajaran genjang, $AD$ sama besar dengan $BC$ [1.34]. Dengan dengan alasan - alasan
yang sama, $EF$ juga sama dengan $BC$. Dengan demikian $AD$ juga sama dengan $EF$. Dan $DE$ adalah 

\setcounter{chapter}{2}
\chapter*{\centering \chaptername{ \thechapter} \\Dasar - Dasar Aljabar Geometri}
\setcounter{section}{12}
\section*{\centering \thesection} 

Pada segitiga tumpul, besar bujur sangkar dari sisi yang menghadap sudut tumpul
 adalah lebih besar dari jumlah besar bujur sangkar - bujur sangkar sisi - sisi yang 
mengapit sudut tumpul sebanyak dua kali besar persegi panjang yang dibentuk
dari satu sisi yang yang mengapit sudut tumpul yang padanya garis lurus jatuh
secara tegak lurus, dan garis lurus yang memotong sisi luar (segitiga tersebut) secara tegak lurus 
kearah sudut tumpul.

\begin{tikzpicture}
\draw [-, black] (-2, -2) -- (6, -2);
\draw (-2.25, -2.25) node {$D$};
\draw (6.25, -2.25) node {$C$};
\draw [-, black] (-2, -2) -- (-2, 3);
\draw (-2.25, 3.25) node {$B$};
\draw [-, black] (0, -2) -- (-2, 3);
\draw (0.0, -2.25) node {$A$};
\draw [-, black] (6, -2) -- (-2, 3);
\end{tikzpicture} 

Misalkan $ABC$ adalah segitiga tumpul, dengan sudut $BAC$ adalah sudut tumpul.
Misalkan $BD$ ditarik dari titik $B$ tegak lurus terhadap $CA$ [Prop 1.12]. Saya
katakan bahwa bujur sangkar $BC$ adalah lebih besar dari jumlah bujur sangkar $BA$
dan bujur sangkar $AC$, sebanyak dua kali persegi panjang yang dibentuk $CA$ dan $AD$.

Karena garis lurus $CD$ sudah dipotong pada sebarang titik, $A$, bujur sangkar para $DC$
dengan demikian sama dengan jumlah dari jumlah bujur sangkar $CA$ dan $AD$, dan dua kali
persegi panjang yang dibentuk $CA$ dan $AD$ [Prop.2.4]. Misalkan bujur  sangkar $DB$
ditambahkan kepada keduanya

\begin{thebibliography}{9}
\bibitem{fitzpatrick}
Richard Fitzpatrick.
\textit{EUCLID’S ELEMENTS OF GEOMETRY , The Greek text of J.L Heiberg(1883 - 1885)}.
Revised and corrected - 2008 
\bibitem{jafareffendi}
M Jafar Effendi.
\textit{Geometri Euclid}.
Makalah mata kuliah Geometri Fakultas Pasca Sarjana Program Studi Pendidikan Matematika 2012.
\bibitem{clarkuni}
https://mathcs.clarku.edu/~djoyce/java/elements/elements.html 
\end{thebibliography}
\end{document}

