\documentclass[a4paper, 12pt]{book}
  
\usepackage{algorithm}
\usepackage{amsmath}
\usepackage{amssymb}
\usepackage{multicol} 
\usepackage{color, colortbl}
\usepackage{tikz}
\usepackage[bahasa]{babel}

\definecolor{Gray}{gray}{0.9}
\definecolor{LightCyan}{rgb}{0.70,1,1}
\definecolor{OliveGreen}{rgb}{0,0.6,0}
\definecolor{ashgrey}{rgb}{0.7, 0.75, 0.71}
\definecolor{aliceblue}{rgb}{0.94, 0.97, 1.0}

\begin{document}
\title{\textbf{Kuliah - Kuliah Tentang Quaternion
(Cuplikan Dalam Indonesia)}}
\maketitle 

\chapter*{Preface}
\addcontentsline{toc}{chapter}{Preface}
\begin{enumerate}
\item Volume yang pada saat ini ditawarkan kepada khalayak dirancang sebagai 
bantuan bagi orang - orang yang mungkin memutuskan mempelajari dan menggunakan
satu metode matematis tertentu, yang telah, dalam beberapa waktu yang lalu, 
menyita banyak perhation saya, dan yang untuk nya saya berusaha mengajukan nama
Metode atau Kalkulus Quaternion - Quaternion. Meskipun banyak sekali petunjuk -
petunjuk analisa, dalam bentuk Daftar Isi, yang akan ditemukan telah 
ditambahkan di awal pekerjaannya, tetapi masih terlihat tepat untuk menawarkan
nya di sisni beberapa catatan dan pendahuluan umum: khususnya mengenai konsep 
darimana keseluruhan secara bertahap berkembang, dan motif - motif untuk 
memberikan pada metode yang menghasilkannya sebuah sebutan yang sebelumnya
tidak pernah digunakan.
\item Kesulitan - kesulitan yang dirasakan banyak orang dalam doktrin Negative
dan Kuantitas Imajiner dalam Aljabar memaksa mereka menjadi perhatian saya dari
waktu yang lampau sekali; dan meskipun saya sudah sejak awal sekali membentuk
beberapa perkenalan dengan berbagai jenis sudut pandang dan sara - saran yang
telah diajukan olah penulis - penulis terkenal, untuk maksud menghapus ataupun
menghindarkan kesulitan - kesulitan tersebut (seperti teori kuantitas langsung
dan kuantitas inverse, dan tentang bangun correlative secara tidak langsung, 
metode untuk membentuk hal - hal imajiner dengan garis yang digambarkan dari 
satu titik dengan berbagai arah pada sebuah bidang, dan sudut pandang yang 
seluruhnya mengacu pada hanya operasi - operasi aljabar, dan ciri - ciri dari
bahasa simbolik), meskipun demikian 

\end{enumerate}

\end{document}


