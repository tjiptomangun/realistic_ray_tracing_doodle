\documentclass[a4paper, 12pt]{book}
  
\usepackage{algorithm}
\usepackage{amsmath}
\usepackage{amssymb}
\usepackage{multicol} 
\usepackage{color, colortbl}
\usepackage{tikz}
\usepackage[bahasa]{babel}

\definecolor{Gray}{gray}{0.9}
\definecolor{LightCyan}{rgb}{0.70,1,1}
\definecolor{OliveGreen}{rgb}{0,0.6,0}
\definecolor{ashgrey}{rgb}{0.7, 0.75, 0.71}
\definecolor{aliceblue}{rgb}{0.94, 0.97, 1.0}

\begin{document}
\title{\textbf{Kuliah - Kuliah Tentang Quaternion
(Cuplikan Dalam Indonesia)}}
\maketitle 

\chapter*{Preface}
\addcontentsline{toc}{chapter}{Preface}
\begin{enumerate}
\item Volume yang pada saat ini ditawarkan kepada khalayak dirancang sebagai 
bantuan bagi orang - orang yang mungkin tertarik mempelajari dan menggunakan
satu metode matematis tertentu, yang telah, dalam beberapa waktu yang lalu, 
menyita banyak perhatian saya, dan yang untuknya saya berusaha mengajukan nama
Metode atau Kalkulus Quaternion - Quaternion. Meskipun banyak sekali petunjuk -
petunjuk analisa, dalam bentuk Daftar Isi, yang akan ditemukan telah 
ditambahkan di awal, tetapi masih terlihat tepat untuk menawarkannya di sini 
beberapa catatan dan pendahuluan umum: khususnya mengenai konsep darimana 
keseluruhan secara bertahap berkembang, dan motif - motif untuk memberikan pada 
metode hasilnya sebuah sebutan yang sebelumnya tidak pernah digunakan.
\item Kesulitan - kesulitan yang dirasakan banyak orang dalam doktrin Kuantitas
Negative dan Imajiner dalam Aljabar memaksakan mereka menjadi perhatian saya 
dari waktu yang lampau sekali; dan meskipun saya sudah sejak awal sekali 
membentuk beberapa perkenalan dengan berbagai jenis sudut pandang dan saran 
- saran yang telah diajukan olah penulis - penulis terkenal, untuk maksud 
menghapus ataupun menghindarkan kesulitan - kesulitan tersebut (seperti teori 
kuantitas langsung dan kuantitas inverse, dan tentang bangun correlative 
secara tidak langsung, metode untuk membentuk hal - hal imajiner dengan garis 
yang digambarkan dari 
satu titik dengan berbagai arah pada sebuah bidang, dan sudut pandang yang 
seluruhnya mengacu pada hanya operasi - operasi aljabar, dan ciri - ciri dari
bahasa simbolik), meskipun demikian seluruh subjek ini terlihat bagi saya masih
butuh penyelidikan tambahan, dan jadi lebih rentan terhadap penjelasan yang 
lebih lanjut. Dan sementara setuju dengan mereka yang menentang bahwa negatif 
dan imajiner bukanlah tepat disebut sebagai kuantitas, Saya masih merasa kurang
puas dengan pandangan - pandangan yang tidak seharusnya tidak di tujukan kepada
mereka, dari sejak permulaan, pengertian dan pemahaman yang jelas, dan berharap
hal ini seharusnya sudah dilakukan, karena akar dari bilangan negatif, tanpa
memperkenalkan pertimbangan - pertimbangan yang 
\textit{secara nyata secara geometris}, 
sebagaimana hal - hal yang terlibat dalam konsep dengan sudut.
\item Dari awal sekali saya memandang bahwa penyelesaian masalah ini dapat 
dicapai dengan kita menyetujui bahwa ALJABAR bukan sekedar seni, atau juga 
bukan bahasa, atau juga bukan yang terutama ilmu pengetahuan tentang kuantitas;
tetapi sebagai ilmu pengetahuan tentang urutan pertumbuhan. Bagaimanapun, hal
ini tadinya, bagian dari konsepsi ini, bahwa pertumbuhan yang dibicarakan 
di sini dipahami sebagai kontinu dan satu dimensi: berkembang secara tidak 
terdefenisi ke depan dan ke belakang, tetapi tidak ke arah menyamping manapun.
Dan meskipun status tersebut secara berturut - turut dari pertumbuhan tersebut
dapat tanpa diragukan dinyatakan dengan titik - titik dalam suatu garis, tetap
saja saya pikir bahwa keberurutan tersebut dipahami dengan membandingkan mereka
dengan saat - saat pada perjalanan waktu, dilepaskan dari segala referensi
sebab dan akibat; sehingga perjalanan waktu di sini dapat dianggap abstrak, 
ideal, atau murni, sebagai mana halnya ruang yang merupakan objek geometri.
Dengan cara seperti ini saya diarahakan, beberapa waktu yang lampau, 
menganggap Aljabar sebagai Ilmu pengetahuan tentang perjalanan Waktu yang Murni,
dan sebuah Esai, yang berisi pandangan saya tentang hal ini, dipublikasikan
pada tahun 1835. Jika saya pada saat ini mereproduksi ulang sedikit opini
- opini dari Esai awal tadi, hal ini karena mungkin akan membantu pembaca 
menempatkan diri nya sendiri dalam sudut pandang tersebut, sebagai element
pertama dari aljabar, yang daripadanya paragraf secara perlahan saya buat 
secara komparatif dengan  konsepsi geometrik yang adalah maksud dari buku
ini dibuat. Dan dengan memandang hal - hal yang tidak lazim dalam penafsiran
yang di sarankan, untuk beberapa penulisan yang dasar dan sederhana, adalah
harapan saya untuk dapat dipahami bahwa saya tidak memaksakan semuanya 
dipandang wajib, tetapi sekedar menunjukaan mereka konsisten pada diri mereka
sendiri , dan persiapan untuk pengkajian tentang quaternion, dalam setidaknya
satu aspek dari nya.
\item Dengan pandangan yang baru saja kita acu, jika huruf - huruf $A$ dan
$B$ dinyatakan sebagai tanggal, untuk menunjukkan dua buah saat dalam waktu,
yang bisa ya bisa tidak adalah berbeda, kasus kebetulan atau kedidentikan pada
kedua saat ini, atau kesamaan dari kedua tanggal ini, dinyatakan dengan 
persamaan 
\[
B = A;
\]
Yang mana merupakan pernyataan simbolis yang dapat ditafsirkan sebagai
tidak melibatkan referensi awal  apapun tentang kualitas, atau menyatakan
hasil dari pembandingan dua durasi yang dihitung. Ini berkaitan dengan 
konsep keserentakan atau sinkronisme; atau dengan bahasa yang lebih sederhana
, hal ini menyatakan pemikiran tentang sekarang dalam perjalanan waktu. Dari
semua jawaban yang mungkin terhadap pertanyaan Kapan, jawaban yang paling
sederhana adalah sekarang: Dan ini adalah sikap dari pikiran, yang berasumsi
dalam membuat jawaban ini, yang mana mungkin dikatakan secara orisinil 
menyimbolkan persamaan yang dituliskan di atas. Dan dengan cara yang mirip,
dua formula pertidaksamaan
\[
B > A, B < A,
\]
ditafsirkan, tanpa referensi awal terhadap kuantitas, untuk menunjukkan
dua hubungan yang sangat berbeda tentang sesudah dan sebelum, yang menjawab
gagasan tentang masa depan dan masa lalu dalam waktu; atau untuk secara 
sederhana menyatakan bahwa saat $B$ diatur supaya terjadi sesudah $A$, 
dan untuk yang satunya bahwa $B$ lebih awal dari $A$: tanpa lebih dulu 
memperkenalkan konsep tentang pengukuran, untuk menentukan seberapa banyak
lebih lama atau lebih awal satu saat dari saat yang lain.
\item Dengan demikian sudah diajukan pengertian - pengertian diberikan 
kepada tiga tanda dasar $=><$, kemudian disarankan lagi tanda $-$, dalam 
membentuk ilmu pengetahuan waktu murni, dapat dibayangkan sebagai
membentuk simbol $B - A$, untuk menyatakan perbedaan antara dua buah saat,
atau hubungan urutan antara saat $B$ terhadap saat $A$, apakah relasinya
adalah indentitas atau perbedaan; dan jika yang kedua; apakah itu hubungan
sesudah atau sebelum dalam derajat manapun. Dan di sini, tanpa diragukan;
mengikuti derajat dari perbedaan antara dua buah saat, konsep tentang
durasi sebagai kuantitas dalam waktu, diperkenalkan: makna penuh dari
simbol $B-A$, dalam penerapan tertentu manapun, tidak diketahui sampai
kita tahu berapa lama sesudah, atau berapa lama sebelum, $B$ dari $A$.
Tetapi juga jelas bahwa gagasan tentang kualitas (atau jenis) tertentu
dari perbedaan ini, jumlah atau durasi. Kontras antara masa depan dan
masa lampau muncul lebih awal dan lebih mendasar, dalam pikiran manusia,
dari pada gagasan tentang Besar atau Kecil.
\item Setelah membandingkan saat - saat, menjadi mudah untuk membandingkan
relasi; dan dalam pandangan ini, dengan sebuah perluasan dari pengartian[4]
dari tanda $=$, digunakan untuk menunjukkan persamaan dalam waktu; atau
lebih tepatnya , untuk menyatakan persamaan dua buah penanda yang memiliki
relasi keberurutan yang sama, antara dua pasangan saat. Dengan demikian
rumusan
\[
D - C = B - A
\]
ditafsirkan untuk menunjukkan persamaan antara dua selang di dalam waktu;
atau untuk menyatakan hubungan $D$ dengan $C$, sama dengan $B$ terhadap $A$,
dalam hal keindentikan dan perbedaaan: kuantitas dan kualitas dari perbedaan
(jika ada) dalam hal ini keduanya diperhitungkan. Rumusan sejenis ini muncul
untuk membenarkan inversi dan alternasi $(C - D = A - B, D - B = C - A)$; dan
secara umum padanya dapat dilakukan serangkaian transformasi dan kombinasi
persamaan seperti ini, yang semuanya diterima sebagai penafsiran dan 
penyesuaian oleh model memandang subjek begini, tetapi yang padanya diakui
segala segi aturan - aturan aljabar. Dengan rencana yang sama, dua formula
pertidaksamaan selisih
\[
D - C > B - A, D - C < B - A,
\]
ditafsirkan sebagai penunjuk, bahwa $D$ relatif terhadap $C$ lebih lama dari
$B$ relatif terhadap $A$, dan sisi yang satunya $D$ lebih awal.
\item Melanjutkan tanda $+$, Saya menggunakan tanda ini terutama sebagai 
penanda kombinasi antara sebuah simbol, seperti huruf kecil romawi $a$, sebagai
langkah dalam waktu, dan simbol seperti $A$, untuk saat yang darinya langkah
ini di mulai, secara berurut untuk membentuk simbol yang kompleks, $a + A$,
merekam konsep transisi ini, dan menunjukaan saat (misalkan $B$) yang padanya
langkah ini harusnya mengantar. Langkah atau transisi yang disebutkan
di sini dapat dianggap sebagai tindakan mental, yang dengan mudah dapat
diubah mengantarkan maju atau mundur dalam pertumbuhan waktu; atau bahkan 
langkah kosong, yang ditandai dengan $0$, yang tidak menghasilkan dampak
$(0 + A = A)$. Dengan demikian, dengan pengertian - pengertian tanda ini, 
penulisan
\[
B = a + A,
\]
menunjukkan konsep bahwa saat $B$ dapat dicapai, atau dihasilkan secara mental,
dengan membuat (dalam pikiran) langkah dari saat $A$. Dan ini terlihat bagi 
saya tanpa berhenti memandang $B - A$ menunjukkan, dalam satu sudut pandang 
[5], hubungan berurut antara dua saat, kita mungkin juga bisa menggunakannya
dalam pengertian terhubung untuk menunjukkan langkah ini dari satu dengan 
yang lain: yang akan memungkinkan kita (sebagaimana dalam aljabar biasa), untuk
menuliskan, dengan pengandaian yang barusan saja,
\[
B - A = a;
\]
kedua anggota dari persamaan yang baru disini melambangkan langkah satu yang 
sama.
\item Identitas yang biasa
\[
(B - A) + A = B,
\]
muncul dan ditafsirkan sebagai penanda yang terutama (dalam ilmu pengetahuan 
perjalanan waktu) sebagai hubungan yang terkandung antara beberapa operasi,
tentang  cara menentukan perbedaan antara dua saat sebagai relasi, dan 
penerapan perbedaan tersebut sebagai langkah. Dan dua lagi identitas yang 
berhubungan dan cukup dikenal
\[
C - A = (C - B) + (B - A), C - B = (C - A) - (B - A).
\]
diperlakukan dengan cara yang sama, pada awalnya sebagai penanda 
\textit{komposisi} dan \textit{dekomposisi} antara hubungan - hubungan berurut
atau langkah - langkah dalam waktu. Satu simbol tertentu untuk \textit{oposisi}
antara dua relasi atau langkah tersebut diajukan; tetapi ditemukan bahwa
notasi yang lebih biasa, $+a$ dan $-a$, untuk langkah $(a)$ itu sendiri, dan
untuk lawan dari langkah tersebut, mungkin, dengan konsistensi sepenuhnya 
terhadap pandangan umum, diterapkan, jika diperlakukan sebagai pembatasan
terhadap simbol yang lebih kompleks $0 + a$, $0 - a$: notasi yang lebih 
belakangan menghadirkan interpretasi yang tidak menyulitkan, atau membutuhkan
upaya untuk memahami pengurangan satu kuantitas dari ketiadaan, tetapi hanya
penguraian dari langkah kosong menjadi dua langkah yang berlawanan. Tetapi 
$operasi - operasi \; pada \; langkah - langkah$, yang dilakukan pada kerangka ini
, nampak sependapat dengan dalam hal dengan aturan - aturan aljabar biasa, 
dalam hal Penambahan dan Pengurangan.
\item Satu langkah-waktu $(b)$ dibandingkan dengan langkah-waktu yang lain
$(a)$, secara rasio aljabar, sedemikian untuk mengadakan konsep tentang
hubungan kompleks tertentu (atau $hasil\; bagi$), yang ditentukan sebagian oleh
$kebesaran\; relatif$, tetapi juga sebagian lagi dari $arah\; relatif$, sebagai
searah atau berlawanan; dan yang berhubungan dekat dengan konsep aljabar 
tentang $angka$ (atau $pengali$), yang beroperasi secara bersamaan pada
kuantitas dan arah dari satu langkah ($a$), untuk 
menghasilkan kuantitas dan arah dari langkah yang lain ($b$). Dengan kombinasi
kedua konsep ini, identitas yang biasa
\[
\frac{b}{a} \times a = b, \; \text{or} \;  b = a \times a, \;\text{if} \; \frac{b}{a}=a,
\] 
mendapat interpretasi, faktor $a$ \textit{positif} atau 
\textit{kontra-positif} (atau biasa disebut \textit{negatif}) sesuai dengan
apakah dia \textit{mempertahankan} atau \textit{membalik} \textit{arah} dari
langkah yang padanya dia dioperasikan. Keempat operasi utama, didapat dengan 
mengkombinasikan dua rasio atau angka atau faktor manapun, $a$ dan $b$, 
diantara mereka sendiri, didefinisikan dengan empat persamaan yang dengan
demikian dapat dituliskan, dan yang sebenarnya dipilih dari rumus - rumus
aljabar umum, tetapi yang diterapkan dengan tafsiran - tafsiran baru:
\begin{align*}
(b + a) \times a  &= (b \times a) + (a \times a); \\
(b \times a) \times a  &= b \times (a \times a); \\
(b - a) \times a  &= (b \times a) - (a \times a); \\
(b \div a) &= (b \times a) \div (a \times a); \\
\end{align*}
\item Operasi - operasi pada angka - angka aljabar (positive atau kontra
positive) dengan demikian dibuat tergantung

\end{enumerate}

\end{document}


